\documentclass[output=paper,colorlinks,citecolor=brown,modfonts,nonflat]{langsci/langscibook}
\ChapterDOI{10.5281/zenodo.3776561}
\author{M. Rita Manzini\affiliation{Università di Firenze}}
\markuptitle{Romance \textit{a}-phrases and their clitic counterparts: Agreement and mismatches}{Romance `a'-phrases and their clitic counterparts: Agreement and mismatches}

\renewcommand{\lsCollectionPaperFooterTitle}{Romance \noexpand\textit{a}-phrases and their clitic counterparts: Agreement and mismatches}

\abstract{DOM (Differential Object Marking) arguments in Romance are associated with the \textit{a}/dative morphology typical of goal arguments, because both have the same syntactic structure of embedding (\sectref{sec:manzini:1}). Clitics do not necessarily share the case alignment of full pronouns/lexical DPs. Indeed clitics and lexical DPs are separately merged each in their domain. The case array may therefore be set differently (\sectref{sec:manzini:2}). DOM objects give rise to a number of patterns under cliticization, including the standard Spanish one, \textit{leísmo} and \textit{loísmo/laísmo} (the latter typical also of South Italian). This variation depends on the fact that lexical DPs may be associated with DOM though clitics aren’t (standard Spanish, \textit{loísmo/laísmo}) or both may be associated with DOM (\textit{leísmo}) (\sectref{sec:manzini:3}). }

 \IfFileExists{../localcommands.tex}{
  % add all extra packages you need to load to this file  
\usepackage{tabularx} 
\usepackage{url} 
\urlstyle{same}

\usepackage{listings}
\lstset{basicstyle=\ttfamily,tabsize=2,breaklines=true}


%%%%%%%%%%%%%%%%%%%%%%%%%%%%%%%%%%%%%%%%%%%%%%%%%%%%
%%%                                              %%%
%%%           Examples                           %%%
%%%                                              %%%
%%%%%%%%%%%%%%%%%%%%%%%%%%%%%%%%%%%%%%%%%%%%%%%%%%%% 
%% to add additional information to the right of examples, uncomment the following line
% \usepackage{jambox}
%% if you want the source line of examples to be in italics, uncomment the following line
% \renewcommand{\exfont}{\itshape}
\usepackage{langsci-optional}
\usepackage{./langsci/styles/langsci-gb4e}
\usepackage{./langsci/styles/langsci-lgr}
\usepackage{pgfplots,pgfplotstable}

\definecolor{lsDOIGray}{cmyk}{0,0,0,0.45}

\usepackage{xassoccnt}
\newcounter{realpage}
\DeclareAssociatedCounters{page}{realpage}
\AtBeginDocument{%
  \stepcounter{realpage}
}


 



 

  \newcommand{\appref}[1]{Appendix \ref{#1}}
\newcommand{\fnref}[1]{Footnote \ref{#1}} 

\newenvironment{langscibars}{\begin{axis}[ybar,xtick=data, xticklabels from table={\mydata}{pos}, 
        width  = \textwidth,
	height = .3\textheight,
    	nodes near coords, 
	xtick=data,
	x tick label style={},  
	ymin=0,
	cycle list name=langscicolors
        ]}{\end{axis}}
        
\newcommand{\langscibar}[1]{\addplot+ table [x=i, y=#1] {\mydata};\addlegendentry{#1};}

\newcommand{\langscidata}[1]{\pgfplotstableread{#1}\mydata;}

\makeatletter
\let\thetitle\@title
\let\theauthor\@author 
\makeatother

\newcommand{\togglepaper}[1][0]{ 
%   \bibliography{../localbibliography}
  \papernote{\scriptsize\normalfont
    \theauthor.
    \thetitle. 
    To appear in: 
    Change Volume Editor \& in localcommands.tex 
    Change volume title in localcommands.tex
    Berlin: Language Science Press. [preliminary page numbering]
  }
  \pagenumbering{roman}
  \setcounter{chapter}{#1}
  \addtocounter{chapter}{-1}
}
\newcommand{\orcid}[1]{}

  %% hyphenation points for line breaks
%% Normally, automatic hyphenation in LaTeX is very good
%% If a word is mis-hyphenated, add it to this file
%%
%% add information to TeX file before \begin{document} with:
%% %% hyphenation points for line breaks
%% Normally, automatic hyphenation in LaTeX is very good
%% If a word is mis-hyphenated, add it to this file
%%
%% add information to TeX file before \begin{document} with:
%% %% hyphenation points for line breaks
%% Normally, automatic hyphenation in LaTeX is very good
%% If a word is mis-hyphenated, add it to this file
%%
%% add information to TeX file before \begin{document} with:
%% \include{localhyphenation}
\hyphenation{
affri-ca-te
affri-ca-tes
Tarra-go-na
Vio-le-ta
Jacken-doff
clit-ics
Giar-di-ni
Mor-fo-sin-tas-si
mi-ni-mis-ta
nor-ma-li-tza-ció
Caus-ees
an-a-phor-ic
caus-a-tive
caus-a-tives
Mar-antz
ac-cu-sa-tive
Ma-no-les-sou
phe-nom-e-non
Holm-berg
}

\hyphenation{
affri-ca-te
affri-ca-tes
Tarra-go-na
Vio-le-ta
Jacken-doff
clit-ics
Giar-di-ni
Mor-fo-sin-tas-si
mi-ni-mis-ta
nor-ma-li-tza-ció
Caus-ees
an-a-phor-ic
caus-a-tive
caus-a-tives
Mar-antz
ac-cu-sa-tive
Ma-no-les-sou
phe-nom-e-non
Holm-berg
}

\hyphenation{
affri-ca-te
affri-ca-tes
Tarra-go-na
Vio-le-ta
Jacken-doff
clit-ics
Giar-di-ni
Mor-fo-sin-tas-si
mi-ni-mis-ta
nor-ma-li-tza-ció
Caus-ees
an-a-phor-ic
caus-a-tive
caus-a-tives
Mar-antz
ac-cu-sa-tive
Ma-no-les-sou
phe-nom-e-non
Holm-berg
}

  \togglepaper[1]%%chapternumber
  \bibliography{../localbibliography}
}{}

\begin{document}
\maketitle
\shorttitlerunninghead{Romance \textit{a}-phrases and their clitic counterparts}



\section{DOM and inherent datives}\label{sec:manzini:1}

In examples like \REF{ex:manzini:1}, from a South Italian dialect, the \textit{a}-phrase \textit{a} \textit{iddu} ‘(to) him’ is traditionally described as instantiating Differential Object Marking (DOM) when co-occurring with \textit{viðinu} ‘they see’, and as instantiating a dative goal when co-occurring with \textit{parlanu} ‘they speak’.\footnote{DOM is a widespread phenomenon \citep{Bossong1985} whereby referentially high ranked objects and referentially low ranked objects have different morphosyntactic realizations. Ranking is determined by notions of animacy/definiteness, hence by a referentiality scale along the lines of the D-hierarchy of \citet{Kiparsky2008}.}  In other words, the morphological similarity is seen to conceal two different underlying syntactic structures.

\ea%1
    \label{ex:manzini:1}
    Celle di Bulgheria (\citealt{ManziniSavoia2005})\\
    \gll Parlanu/Viðinu a iddu.\\
        they.speak/they.see to/\textsc{dom} him\\
    \glt ‘They speak to/they see him.’
\z

A few recent generative works argue that the morphological similarity between DOM and dative arguments externalizes a deeper syntactic similarity, specifically in the Romance languages. \citet{Torrego1998} insists that the coincidence of dative and DOM \textit{a} cannot be accidental, given cross-linguistic evidence such as the coincidence of dative and DOM postpositions in Hindi. More explicitly, \citet{Torrego2010}, discussing sentences like \REF{ex:manzini:2}, provides the structural representation in \REF{ex:manzini:3}. Thus “agentive verbs such as Spanish \textit{contratar} ‘hire’ also have a hidden Appl selected by the light verb {\liv}\textsubscript{DO}” – in other words \textit{contratar} \textit{a} \textit{DP} is \textit{CAUSE} \textit{a} \textit{contract} \textit{to} \textit{DP}. Therefore, “the single animate object of a transitive accusative verb will \textit{always} be marked with dative morphology, simply because it is dative. The animate object will be in Spec, Appl, hence a Goal/Beneficiary receiving inherent Case from Appl” \citep[462]{Torrego2010}.

\ea%2
    \label{ex:manzini:2}
    Spanish \citep{Torrego2010}\\
    \gll {Han}     {contratado}  {*(a)} {una} {amiga/Julia/mi} {amiga.}\\
        they.have   hired      (to) a friend/Julia/my friend\\
    \glt ‘They hired a friend/Julia/my friend.’
\z

\ea%3
    \label{ex:manzini:3}{}
    [\textsubscript{{\liv}P} Agent [\textsubscript{v’} \textit{v}\textsubscript{DO} [\textsubscript{ApplP} \textit{a} DP [\textsubscript{Appl’} Appl [\textsubscript{N} \textit{contrato}]]]]]
\z

\citet[359--360]{Pineda2016} essentially adopts Torrego’s structure for Catalan \REF{ex:manzini:4}, where the verb can occur either with an \textit{a} argument (dative) or with a bare argument (accusative). According to Pineda, the case alternation is a parametric choice, independently needed to account for the difference between Romance ditransitives (with dative goals) and English ditransitives (with dative-shift objects)\footnote{This raises the question why Romance lacks dative shift in ditransitives (see \citealt{Lima-Salles2016} for what it might look like in some Brazilian Portuguese dialects). Note that no Appl need be involved in English dative shift, see \citet{Kayne1984, Pesetsky1995, Harley2002, BeckJohnson2004}.} (see also \citealt{Pineda2014}).

\ea%4
    \label{ex:manzini:4}
    Catalan \citep{Pineda2016}\\
    \gll {L’Anna}   {telefona}   {(a)} {l’Andreu}.\\
        {the Anna}   phones   (to) {the Andreu}\\
    \glt ‘Anna phones Andreu.’
\z

\ea%5
    \label{ex:manzini:5}{}
    [\textsubscript{VoiceP}  Agent [\textsubscript{{\liv}P} \textit{v}\textsubscript{DO} [\textsubscript{ApplP} (\textit{a}) DP [\textsubscript{Appl’} Appl [\textsubscript{N} \textit{telefonada}]]]]]\\
\z

\citet{ManziniSavoia2010, Manzini2012, ManziniFranco2016} reject the idea that Romance languages, or more in general Indo-European languages, have an Appl projection.\footnote{Their approach goes back to \citet[II: 517]{ManziniSavoia2005}, according to whom “prepositional accusatives, like locatives/datives introduced by \textit{a}, are interpreted in terms of denotational properties fundamentally of a locative type”. \citet{ManziniSavoia2005} take location to be primitive, while here location is taken to be a derived form of inclusion/possession, see below and especially fn. 6.}   Rather, in these languages relational content is carried directly by adpositions or by dative/oblique case inflections.\footnote{The theoretical point is that there is no advantage in enforcing what \citet{CulicoverJackendoff2005} call Interface Uniformity, namely that the same meaning always maps to the same syntactic structure. Interface Uniformity leads to the adoption of complex functional architectures of the cartographic type, which raise issues of evolvability and learnability in the sense of \citet{ChomskyGallegoOttTA}.} This line of work further individuates the fundamental relational content of \textit{a}/dative case in the inclusion or part/whole relation (cf. \citealt{BelvindenDikken1997}), which is taken to underlie inherent and material possession, possession of a mental state (experiencers) and also location (inclusion in location). Additionally, this approach makes use of the standard idea that transitive predicates are decomposable into two event layers, in the most typical instance a causation event and a result event, and adopts the standard minimalist structuring of the transitive predicate into a \textit{v} and a V layer.

On these grounds, Southern Italian examples like \REF{ex:manzini:1} are associated with the structure in \REF{ex:manzini:6}. The \textit{a} preposition/dative case, labelled  ${\subseteq}$, carries the inclusion/possession content (see also \citetv{chapters/franco}). The two arguments of ${\subseteq}$ are the pronoun \textit{him} and the VP event – so that the overall interpretation of \textit{They} \textit{speak} \textit{to} \textit{him} is ‘they cause him speech’; and \textit{They} \textit{see} \textit{him} is close to ‘they have sight of him’.

\ea%6
    \label{ex:manzini:6}
\begin{forest}
[{\liv}P
    [{\liv}\\CAUS]
    [VP
        [V\\{viði-}\\{parla-}]
        [${\subseteq}$P
            [${\subseteq}$\\{a}]
            [D\\{iddu}]
        ]
    ]
]
\end{forest}
%     v               VP
%      CAUS    wp
%         V          ${\subseteq}$P
%         {viði-}     3
%         {parla-}    ${\subseteq}$      D
%             {a}      {iddu}
\z

In turn, the structure of embedding of bare accusative objects is simply as shown in \REF{ex:manzini:7}.

\ea%7
    \label{ex:manzini:7}
    Celle di Bulgheria\\
    \ea\label{ex:manzini:7a}
    \gll {Camu} {na} {fimmina}.\\
        I.call a woman\\
    \glt ‘I call a woman.’
    \ex\label{ex:manzini:7b}
    … [\textsubscript{{\liv}P} CAUS  [\textsubscript{VP} \textit{camu} [\textsubscript{DP} \textit{na} \textit{fimmina}]]]
    \z
\z

The most serious problem for the unification of DOM and inherent datives is generally held to be passivization. Objects of \textit{call} in \REF{ex:manzini:7} or of \textit{see} in \REF{ex:manzini:1} passivize, independently of their referential ranking, hence independently of whether they are associated with structure \REF{ex:manzini:6} or with structure \REF{ex:manzini:7b} in the active. Objects of \textit{speak} in \REF{ex:manzini:1} never passivize. \citet{ManziniFranco2016} argue that the \textit{a} preposition/dative case with \textit{speak} in \REF{ex:manzini:6} is selected by the verb. Under passive, selected dative case must be preserved, barring raising to nominative position. In other words \textit{John} \textit{was} \textit{spoken} \textit{to} would be well-formed but is unavailable in Romance; *\textit{John} \textit{was} \textit{spoken} is ungrammatical exactly as in English and for the same reasons (violation of the selection properties of the verb).

On the contrary, the \textit{a} preposition/dative case with \textit{see} in \REF{ex:manzini:6} is structural, since it depends not on the selection properties of the verb, but on the DOM configuration. More explicitly, I assume that under DOM, a highly ranked referent cannot be embedded as a theme, but must be embedded with a role at least as high as that of possessor/locator of the VP subevent, as schematized in \REF{ex:manzini:8}.

\ea%8
    \label{ex:manzini:8}
    DOM\\{}
    [\textsubscript{VP} V   [*(P${\subseteq}$) DP]]\\
    where DP = 1/2P > pronoun > proper name etc.
\z

According to \citeauthor{ManziniFranco2016}, passive voids the context for the application of DOM, since the internal argument is raised out of its VP-internal position to [Spec, IP]. Therefore, no ${\subseteq}$ preposition or case need be present in the derivation, and passivization is well-formed.

An important point made by \citet{Pineda2016} is that given the identical structural realization of DOM and inherent datives, one may expect that some inherent datives are reanalyzed as DOM and end up being passivized. Apulian varieties like \REF{ex:manzini:9} are a case in point (see \citealt{Loporcaro1988, Ledgeway2000} for independent attestations). This kind of reanalysis further supports the unification of DOM and inherent datives.

\ea%9
    \label{ex:manzini:9}
    Minervino Murge\\
    \ea\label{ex:manzini:9a}
    \gll {Jaɟɟə}   {skrittə}   {a} {jiddə}.\\
        I.have  written  to him\\
    \glt ‘I have written to him.’
    \ex\label{ex:manzini:9b}
    \gll {Jiddə}   {jɛ}   {statə}   {skrittə}   (da la suərə).\\
        he  is  been   written (by the sister)\\
    \glt ‘He has been written to by his sister.’
    \z
\z

In what follows, I concentrate on a classical empirical aspect of the discussion of Romance DOM/dative arguments, namely the clitics that double or pronominalize them. Note that I will not be discussing the conditions under which clitic doubling is possible or necessary; my topic is just the morphological form of the clitics that double/pronominalize DOM and inherent datives. In addressing this matter, I adopt one of the two frameworks laid out above, namely the Relator P one, rather than the Appl one. One reason is that adopting an abstract Appl projection for languages that manifestly have no applicative morphology introduces additional structural complexity. Everything else equal, it is simpler to hold that the \textit{a} preposition, or the dative case, are elements endowed with semantic content, supporting the inclusion/possession predication. Whether there is a way of stating the conclusions of \sectref{sec:manzini:2}--\sectref{sec:manzini:3} in Appl terms or not remains an open question.

\section{Clitics and full DPs may be associated with different case arrays}\label{sec:manzini:2}

\citet{ManziniSavoia2014} find that in Albanian varieties, the case array of 1/2P pronouns does not match that of lexical DPs or 3P pronouns. Thus lexical DPs and 3P pronouns distinguish a nominative, an accusative and an oblique (dative/ablative) case. On the other hand, 1/2P pronouns distinguish the nominative case from an objective case that encompasses accusative and dative contexts, as well as an ablative case. In the examples in \REF{ex:manzini:10a}--\REF{ex:manzini:10b} the first object of \textit{see} and the second object of \textit{give} are lexicalized by the same 1/2P pronoun, while the prepositional object has a separate ablative form in \REF{ex:manzini:10c}. This contrasts with the two distinct forms of the 3P pronoun in \REF{ex:manzini:10a} and \REF{ex:manzini:10b}, accusative and oblique respectively; the latter also occurs in the prepositional object position in \REF{ex:manzini:10c}.

\largerpage[2]

\ea%10
    \label{ex:manzini:10}
    Shkodër, Albanian \citep{ManziniSavoia2014}\\
    \ea\label{ex:manzini:10a}
    \gll {ɛ/mə/na}      {ʃɔfin}    {atɛ/mu/ne}.\\
        \textsc{3sg.acc/1sg/1pl}    see.\textsc{3pl} \textsc{3sg.acc/1sg.obl/1pl.obl}\\
    \glt ‘They see him/me/us.’
    \ex\label{ex:manzini:10b}
    \gll {j/m/n}       {a}   {japin}    {atii/mu/ne}.\\
        \textsc{3sg.obl/1sg/1pl} \textsc{3sg.acc} give.\textsc{3pl}  \textsc{3sg.obl/1sg.obl/1pl.obl}\\
    \glt ‘They give it to him/me/us.’
    \ex\label{ex:manzini:10c}
    \gll {pɾei/poʃt/para}     {mejɛt/nɛʃ/atii}\\
        from/behind/before \textsc{1sg.loc/1pl.loc/3sg.obl}\\
    \glt ‘from/behind/before him/me/us’
    \z
\z

Data of the type in \REF{ex:manzini:10} are traditionally dismissed in descriptive accounts as instances of morphological irregularity. In these terms, Albanian has a four-case system (nominative, accusative, oblique, ablative) – and while 3P displays dative/ablative syncretism, 1/2P displays accusative/dative syncretism. However, still following \citet{ManziniSavoia2014}, there is a different way of looking at the pattern in \REF{ex:manzini:10}. Despite the fact that Albanian is not usually recognized as a DOM language, the 1/2P case system could depend on the fact that 1/2P pronouns are in fact subject to DOM.\footnote{This way of looking at things presupposes that 1/2P vs 3P is an independently attested referential cut for the application of DOM. Center-South Italian varieties provide  evidence that this is so, as in \REF{ex:manzini:i}.

\ea \label{ex:manzini:i}%i
    Colledimacine \citep{ManziniSavoia2005}\\
    \ea
    \gll {A} {camatə} {a} {mme/a} {nnu}.\\
        {he.has}  {called} \textsc{dom} {me/\textsc{dom}} {us}\\
    \glt ‘He called me/us.’
    \ex
    \gll {A}   {camatə} {frattə} {tiə/kwiʎʎə}.\\
        {he.has}  called   brother yours/him\\
    \glt ‘He called him/my brother.’
    \z
\zlast
}

\largerpage[2.5]
Recall that in the South Italian and Ibero-Romance languages in \sectref{sec:manzini:1}, or in Hindi as quoted by \citet{Torrego1998}, DOM takes the form of dativization. The fact that the context in \REF{ex:manzini:10a} displays dative forms, exactly like the context in \REF{ex:manzini:10b}, can then be construed as indicating that 1/2P pronouns are DOMed and that DOM takes the form of dativization/obliquization.\footnote{The dative realization of DOM is found not only in the Italic/Romance family and in the Indo-Aryan family, but also in the Iranian family, for instance in the Vafsi language, as well as in Armenian (\citealt{ManziniFranco2016}). Importantly, Romance \textit{a} also introduces location/direction, as does the Hindi dative/DOM postposition -\textit{ko}. This provides a bridge with the other major descriptive strategy of DOM marking in Indo-European, roughly a locative one. Thus in Eastern Romance (Romanian), where dative is inflectional, DOM takes the form of prepositional marking by locative \textit{pe}; Persian \textit{{}-ro} is also a directional.  The common lexicalization of dative and locative, as seen in Romance \textit{a}, Hindi -\textit{ko}, is accounted for by \citet{FrancoManzini2017Ins} by treating locative as a specialization of the inclusion relation, roughly inclusion in location. Following this line of argumentation to its logical conclusion, DOM in Indo-European languages would seem to involve the embedding of highly ranked referents under the \textrm{${\subseteq}$} relator without exception.}

Recall that I am interested in the reflexes of DOM and inherent dative on the clitic system. {Several properties distinguish 1/2P clitics from 3P clitics in Romance, which I will illustrate just with Italian. From the data in \tabref{extab:manzini:11} it is evident that 3P clitics are differentiated by gender (masculine/feminine) and by case (accusative/dative) – but 1/2P are insensitive to either distinction, as shown in \tabref{extab:manzini:12}.}\footnote{In \tabref{extab:manzini:11} the plural form \emph{loro} is parenthesized because it is not a clitic -- but a non-clitic oblique pronoun (with a special distribution) suppletive for the clitic.}

% \ea\label{ex:manzini:11}
% \glll {} \textsc{acc.m}   \textsc{acc.f}     \textsc{dat.m}   \textsc{dat.f}\\
% \textsc{3sg}    {lo}     {la}     {gli}     {le}\\
% \textsc{3pl}    {li}     {le}     {loro}\\
% \z

\begin{table}
\caption{3P Italian clitics}\label{extab:manzini:11}
\begin{tabular}{lllll}
\lsptoprule
 & \textsc{acc.m} & \textsc{acc.f} & \textsc{dat.m} & \textsc{dat.f}\\
 \midrule
\textsc{3sg} & {lo} & {la} & {gli} & {le}\\
\textsc{3pl} & {li} & {le} &  ({loro}) & ({loro})\\
\lspbottomrule
\end{tabular}
\end{table}

% \ea\label{ex:manzini:12}
% \gllll \textsc{1sg}    \textit{mi}\\
% \textsc{2sg}    {ti}\\
% \textsc{1pl}    {ci}\\
% \textsc{2pl}    {vi}\\
% \z


\begin{table}
\caption{1/2P Italian clitics}\label{extab:manzini:12}
\begin{tabular}{ll}
\lsptoprule
& \textsc{acc}/\textsc{dat}\\
\midrule
\textsc{1sg} & {mi} \\
\textsc{2sg} & {ti} \\
\textsc{1pl} & {ci} \\
\textsc{2pl} & {vi} \\
\lspbottomrule
\end{tabular}
\end{table}

\largerpage[2]
As already discussed for Albanian, the classical approach to asymmetries like those in \tabref{extab:manzini:12} between 1/2P and 3P clitics is to postulate a single underlying $\varphi ${}-features and case system, namely a system rich enough to be able to account for 3P, and to assume that morphological mechanisms are responsible for the surface syncretisms observed in 1/2P. Note, however, that the different morphological make-up in \tabref{extab:manzini:11}--\tabref{extab:manzini:12} correlates with a different positioning in the clitic string. Thus 1/2P clitics have the same position as 3P dative clitics in dative contexts such as \REF{ex:manzini:13a}. However, in \REF{ex:manzini:13b} it can be seen that the 3P accusative follows the locative clitic; the 1/2P clitic precedes it, as in \REF{ex:manzini:13c}. This means that dative and accusative 1/2P clitics are not distinguished syntactically.

\ea%13
    \label{ex:manzini:13}
    Italian\\
    \ea\label{ex:manzini:13a}
    \gll {(Sulla} {ferita)}   {gli/mi}    {ci}   {mette} {la} {pomata}.\\
        on.the wound   \textsc{3dat/1sg}   \textsc{loc}   put.\textsc{3sg} the pomade\\
    \glt ‘He put the pomade on my/his wound.’
    \ex\label{ex:manzini:13b}
    \gll {Mi}  {ci}   {mette}   {vicino}.\\
        \textsc{1sg}  \textsc{loc}  put.\textsc{3sg} close\\
    \glt ‘He puts me close to there.’
    \ex\label{ex:manzini:13c}
    \gll {Ce}  {lo}  {mette}   {vicino}.\\
        \textsc{loc}  \textsc{3sg}   put.\textsc{3sg} close\\
    \glt ‘He puts it/him close to there.’
    \z
\z

There is a third phenomenon with respect to which 1/2P and 3P clitics differ, besides different morphological make-up (\tabref{extab:manzini:11}--\tabref{extab:manzini:12}) and different positioning \REF{ex:manzini:13}. As shown by \citet{Kayne1989}, in Italian (French, etc.) perfect participles agree with D(P) complements placed to their left, hence with accusative clitics. This is illustrated in \REF{ex:manzini:14a}; the 3P feminine accusative clitic cannot co-occur with a masculine inflection on the participle. By contrast, 3P dative clitics do not agree with the perfect participle, as in \REF{ex:manzini:14b}; the latter must surface in the (default) masculine form, even if the clitic is overtly feminine.

\ea%14
    \label{ex:manzini:14}
    Italian\\
    \ea\label{ex:manzini:14a}
    \gll {La}     {ha}     {aiutata/*aiutato}.\\
        \textsc{3.mpl.acc}  has.\textsc{3sg}    helped-\textsc{fsg/}helped\textsc{{}-msg}\\
    \glt ‘He helped her.’
    \ex\label{ex:manzini:14b}
    \gll {Le}       {ha}     {parlato/*parlata}.\\
        \textsc{3.fsg.dat}  has.\textsc{3sg}    talked-\textsc{msg/}talked\textsc{{}-fsg}\\
    \glt ‘He talked to him/her.’
    \z
\z

Accusative 1/2P clitics may agree with the perfect participle, as illustrated in \REF{ex:manzini:15a}. However, lack of agreement is also possible in \REF{ex:manzini:15a}, leading to the masculine singular form of the participle. Agreement is impossible with 1/2P clitics in dative contexts in \REF{ex:manzini:14b}.\\

\ea%15
    \label{ex:manzini:15}
    \ea\label{ex:manzini:15a}
    \gll {Mi/ti}     {ha}     {aiutata/aiutato}.\\
        \textsc{1sg/2sg}  has\textsc{.3sg}    helped\textsc{{}-fsg/}helped\textsc{{}-msg}\\
    \glt ‘He helped me(f.)/you(f.).’
\ex\label{ex:manzini:15b}
    \gll {Mi/ti}        {ha}     {parlato/*parlata}.\\
        \textsc{1sg/2sg}  has\textsc{.3sg}    spoken\textsc{{}-msg/}spoken\textsc{{}-fsg}\\
    \glt ‘He spoke to me(f.)/you(f.).’
    \z
\z

In short, notionally accusative 1/2P clitics in \REF{ex:manzini:15a} may behave like dative clitics (irrespective of Person) in not triggering perfect participle agreement. Importantly, if the intrinsic features of 1/2P pronouns, such as the lack of overt gender, were at stake, we would expect them to always display optional agreement. However, agreement is obligatory in contexts where the 1/2P pronoun has been moved to subject position as in \REF{ex:manzini:16}. This supports the view that the optionality of 1/2P object agreement depends not on the intrinsic features of the 1/2P forms, but rather on their structure of embedding.

\ea%16
    \label{ex:manzini:16}
    Italian\\
    \gll {(Io)}  {sono}     {arrivata/*arrivato}.\\
        \textsc{1sg}  be.\textsc{1sg}   arrived\textsc{{}-fsg/-msg}\\
    \glt ‘I(f.) have arrived.’
\z

Let us then go back to what is traditionally construed as the accusative/dative syncretism of Italian 1/2P object clitics in \tabref{extab:manzini:12}.  Suppose that this syncretism is properly described as 1/2P clitics having an oblique (dative) but not an accusative form. The next step of the analysis is the observation that obliquization and specifically dativization of highly ranked referents characterizes DOM in Indo-European languages and specifically in Romance (cf. also fn. 5).  If so, one may reasonably surmise that what appear to be idiosyncratic morphological properties of 1/2P clitics in Italian are in reality due to the fact that Italian 1/2P clitics undergo DOM.

Under this analysis, the optionality of agreement with 1/2P clitics in \REF{ex:manzini:15} replicates at a smaller scale a well-known independent parameter concerning the optionality of agreement with DOM objects. Given a language where DP objects agree with the verb and inherent datives do not, in principle two configurations may arise with DOM datives, as indicated in \REF{ex:manzini:17}.

\ea%17
    \label{ex:manzini:17}
    Object agreement configurations with DOM arguments. \\
    \ea\label{ex:manzini:17a}
    DOM arguments, like object DPs, agree with the verb.
    \ex\label{ex:manzini:17b}
    DOM arguments, like inherent datives, do not agree with the verb.
    \z
\z

\largerpage[-1]
Indo-Aryan languages verify the existence of both patterns in \REF{ex:manzini:17}. These languages present agreement of the perfect participle with the internal argument, for instance in Punjabi \REF{ex:manzini:18a}, where the internal argument is absolutive and the external argument ergative. Furthermore, they are characterized by DOM, generally opposing animates/humans to inanimates/non-humans, realized by means of a postposition, which in Punjabi is -\textit{nu}, as in \REF{ex:manzini:18b}. What is relevant here is that the DOM object in \REF{ex:manzini:18b} does not agree with the perfect participle, which shows up in the masculine singular (similarly in Hindi).

\ea%18
    \label{ex:manzini:18}
    Punjabi \citep{ManziniSavoiaFranco2015}\\
    \ea\label{ex:manzini:18a}
    \gll {O-ne}     {kutt-e}     {peddʒ-e}.\\
        s/he-\textsc{erg}   dog-\textsc{mpl}   send.\textsc{prf-mpl}\\
    \glt ‘S/he sent the dogs.’
    \ex\label{ex:manzini:18b}
    \gll {Mɛː}   {o-nu/una-nu}       {dekkh-ea}.\\
        I   s/he\textsc{{}-dom}/they\textsc{{}-dom}    see.\textsc{prf-msg}\\
    \glt ‘I saw him/her/them.’
    \z
\z

In other Indo-Aryan languages, DOM objects, also realized by an oblique postposition, agree with the perfect participle exactly as absolutive objects do \emph{\textup{\citep[342]{Masica1991}}}. Thus in Marwari/Rajasthani the perfect participle always agrees with the object, whether it is DOM or not. In \REF{ex:manzini:19} I illustrate agreement of the perfect participle with DOM objects (-\textit{nai}).

\ea%19
    \label{ex:manzini:19}
    Rajasthani \citep{Khokhlova2002}\\
    \gll {Raawan}   {giitaa-nai}  {maarii}    {hai}.\\
        Rawan\textsc{.m}   Gita.\textsc{f-dom}   beat.\textsc{prf.f}    be.\textsc{prs.3sg}\\
    \glt ‘Rawan beat Gita.’
\z

Recall that if the present line of reasoning is correct, it is not possible to explain the 1/2P clitic paradigm in Italian in terms of morphological idiosyncrasy. Rather, 1/2P clitics are subject to DOM, hence they are externalized by oblique case. This in turn predicts two possible grammars for object agreement, given in \REF{ex:manzini:17}. In one grammar, object agreement characterizes direct and DOM objects; in the alternative grammar agreement is restricted to direct objects. Given the Indo-Aryan data, we can safely conclude that nothing stands in the way of analysing Romance 1/2P clitics as subject to the DOM constraint (rather than as displaying the accusative/dative morphological syncretism) – and that this treatment may actually be advantageous in understanding their optional agreement.

\newpage
In conclusion, clitics can be associated with a case array not matching that of lexical DPs/full pronouns, exactly like full pronouns may have a case alignment different from that of lexical DPs.\footnote{As an anonymous reviewer notes, the framework I adopt leads one to expect that there could be other case mismatches between DPs and the clitics that pronominalize (or perhaps double) them, within the boundaries imposed by UG. For instance, the anonymous reviewer notices that in French dative clitics are often reported to have a wider distribution than \textit{à}-phrases, in causatives (see \citetv{chapters/sheehan}) and in benefactives/malefactives. A banal example in Italian is (ia), where two datives are inadmissible within the embedded predicate, but a matrix dative clitic is a possible lexicalization for the embedded external argument in (ib).

\ea%i

    Italian\\
    \ea \gll {Ho} {fatto} {scrivere} (*a) {mio} {fratello} {alla} {sua} {fidanzata}.\\
        {I.have} made write (to) my brother to his fiancee\\
    \glt `I made my brother speak to his fiancee.'
    \ex \gll {Gli/lo} {ho} {fatto} {scrivere}   {alla} {sua} {fidanzata}.\\
        {3\textsc{dat}/\textsc{3msg.acc}} {I.have} made write   to his fiancee\\
    \glt ‘I made him speak to his fiancee.’
    \z
\z
}

Thus for instance 1/2P clitics in Italian undergo DOM, even though full pronouns/DPs do not. Note that if the clitic moved from a so-called ‘big DP’ hosted in the predicative domain, we would expect case uniformity. Therefore movement analyses of clitics are disfavoured by the present conclusions, and base generation analyses correspondingly favoured. In the rest of the article, I assume that clitics are base generated within their own field in the sentence \citep{Sportiche1996}.

\section{DOM and inherent datives under cliticization and clitic doubling}\label{sec:manzini:3}

In this section, I turn to the question of which clitics pronominalize DOM objects. The conditions under which clitic doubling is possible/required are outside the scope of the present work. Hence, in what follows, I will alternate doubling and non-doubling (simple cliticization) data without further comment.

Traditional approaches hold that DOM objects are syntactically accusatives, though they may be morphologically syncretic with datives/obliques; therefore, one may expect that they are doubled/pronominalized by accusative clitics, even though goal datives are doubled/pronominalized by dative clitics, as schematized in \REF{ex:manzini:20a}. However, if DOM arguments share the syntactic structure of inherent datives, as argued here in \sectref{sec:manzini:1}, we may expect that both are doubled (or more generally pronominalized) by the same clitics, as in \REF{ex:manzini:20b}. In turn, the option in \REF{ex:manzini:20b} may be taken to imply that both DOM and goal datives correspond to dative clitics, as in \REF{ex:manzini:20bi}. However one may also consider the possibility that both correspond to accusative clitics, as in \REF{ex:manzini:20bii}.

\ea%20
    \label{ex:manzini:20}
    Cliticization configurations with DOM and goal arguments\\
    \ea\label{ex:manzini:20a}
    Clitics doubling/ pronominalizing DOM arguments belong to the accusative series, clitics doubling/ pronominalizing goal datives belong to the dative series;
    \ex\label{ex:manzini:20b}
    Clitics doubling/pronominalizing DOM and goal datives belong to the same series:
        \ea\label{ex:manzini:20bi}
        both belong to the dative series
        \ex\label{ex:manzini:20bii}
        both belong to the accusative series.
        \z
    \z
\z

All three possibilities in \REF{ex:manzini:20} are attested by the data. Pattern \REF{ex:manzini:20bii}, in which accusative clitics lexicalize both theme and goal arguments, is known as \textit{loísmo/laísmo} in the Spanish descriptive tradition and is robustly attested in Central and Southern Italian varieties (\citealt[§633]{Rohlfs1969}), as exemplified in \REF{ex:manzini:21a}. Dialects like \REF{ex:manzini:21} do have a morphological dative clitic, but it regularly shows up only in ditransitive contexts, for instance \REF{ex:manzini:21c}, as opposed to \REF{ex:manzini:21b}. I agree with \citet{Pineda2016} that \textit{loísmo/laísmo} in the traditional sense of the term, i.e. an accusative 3P clitic doubling or pronominalizing a dative DP, must be kept separate from progressive varieties like Minervino in \REF{ex:manzini:9}, which allow the goal argument of \textit{write} to be passivized. Indeed in the corpus of \citet{ManziniSavoia2005}, dialects like Celle in \REF{ex:manzini:21} (or Tempio in \REF{ex:manzini:22} below) do not display passivization.

\ea%21
    \label{ex:manzini:21}
    Celle di Bulgheria \citep{ManziniSavoia2005}\\
    \ea\label{ex:manzini:21a}
    \gll {U}    {parlanu/viðinu}   {a}   {iddu}.\\
        \textsc{3msg.acc}  speak.\textsc{3pl}/see.\textsc{3pl} to/\textsc{dom} him\\
    \glt ‘They speak to/they see him.’
    \ex\label{ex:manzini:21b}
    \gll {U/a}       {ʃkrivu}     (a {idu/a} issa).\\
        \textsc{3msg.acc/3fsg.acc} write.\textsc{1sg}   to him/to her\\
    \glt ‘I write him/her.’
    \ex\label{ex:manzini:21c}
    \gll {Li}   {ʃkrivu}     {na} {littira}.\\
        \textsc{3.dat}   write.\textsc{1sg}   a letter\\
    \glt ‘I write him/her/them a letter.’
    \z
\z


Agreement with the perfect participle, which is absent from Ibero-Romance but present in Italo-Romance, shows that true accusative clitics are involved. Because of the phonological neutralization of final vowels to schwa, these data are difficult or impossible to find in Central and Southern varieties and are therefore briefly illustrated in \REF{ex:manzini:22} with a variety of Northern Sardinia.

\ea%22
    \label{ex:manzini:22}
    Tempio Pausania \citep{ManziniSavoia2005}\\
    \gll {l}    {aɟɟu}     {vaiɖɖat-u/-a/-i}  /  {vist-u/-a/-i}\\
        \textsc{3.acc}  have.\textsc{1sg}   spoken-\textsc{m.sg/-f.sg/-pl} / seen-\textsc{m.sg/-f.sg/-pl}\\
    \glt ‘I have seen/spoken to him/her/them.’
\z


Next, under the uniform treatment of DOM and inherent datives proposed here, one would normally expect the pattern \REF{ex:manzini:20bi} to be instantiated, whereby both inherent datives and DOM objects are lexicalized by dative clitics. This is robustly documented in Spanish dialects, under the traditional label of \textit{leísmo}, for instance in Basque varieties, as illustrated in \REF{ex:manzini:23}. Within the present analysis, it is natural to conclude that the clitics in \REF{ex:manzini:23} reflect the same case organization as their doubled DP counterparts – hence goal and DOM datives coincide in the dative clitic \textit{le}.

\ea%23
    \label{ex:manzini:23}
    Spanish, Basque dialect \citep{OrmazabalRomero2013Probus}\\
    \ea\label{ex:manzini:23a} \gll {Lo}     {vi}     (*el libro).\\
        \textsc{3m.sg.acc}  saw.\textsc{1sg}  {\db}the book\\
    \glt ‘I saw it/the book.’
    \ex\label{ex:manzini:23b} \gll {Le}   {vi}     (al niño/a la niña).\\
        \textsc{3dat}  saw.\textsc{1sg}  \textsc{dom}.the boy/\textsc{dom} the girl\\
    \glt ‘I saw him/her/the boy/the girl.’
    \z
\z

Going back to the schema in \REF{ex:manzini:20} once more, we must finally consider the possibility that DOM and goal arguments have different clitic counterparts, as in \REF{ex:manzini:20a}. This possibility is instantiated in some of the best-known varieties of Spanish, including the standard. In standard Spanish, animate internal arguments are pronominalized by an accusative clitic, as in \REF{ex:manzini:24a}. By contrast, a DP lexicalizing a goal dative is doubled by a dative clitic, as in \REF{ex:manzini:24b}. In the Rioplatense variety, though not in the standard one, \REF{ex:manzini:24a} is also grammatical as an instance of doubling.

\newpage
\ea%24
    \label{ex:manzini:24}
    Spanish (standard/Rioplatense)\\
    \ea\label{ex:manzini:24a}
    \gll {Lo}     {vio}    (*/\textsuperscript{OK} {a} Juan).\\
        \textsc{3m.sg.acc}  {}                     saw.\textsc{3sg}  \textsc{dom} Juan\\
    \glt ‘He saw him/Juan.’
    \ex\label{ex:manzini:24b}
    \gll {Le}   {dio}     {el} {libro}   (a Juan).\\
        \textsc{3dat}   gave.\textsc{3sg}  the book   to Juan\\
    \glt ‘He gave him the book (to John).’
    \z
\z


The pattern in \REF{ex:manzini:24} appears to favour the view that the \textit{a}-phrase in \REF{ex:manzini:24a} is an underlying accusative, determining doubling by an accusative clitic. But \textit{loísmo} and \textit{leísmo} dialects provide equally strong \textit{prima} \textit{facie} evidence in favour of the view that DOM and inherent datives have the same structure of embedding, so that they are treated alike under cliticization. As stated at the end of \sectref{sec:manzini:2}, I adopt the view that clitics and DPs are each separately merged in their relevant domains \citep{Sportiche1996}, and eventually connected by  Agree when cooccurring.\footnote{Adopting Agree as the operation that connects the clitic and doubled DP implies that all of the structural conditions on Agree, as defined by \citet{Chomsky2000}, hold in the doubling configuration. As for the Phase Impenetrability Condition (PIC), the simplest way to insure that it is met is to adopt the conclusion of \citet{Sportiche1996}, that clitics are Merged in a clitic field located in the periphery of {\liv}P, from where they move the short step to IP. On the other hand, if clitics are base generated in IP, additional or alternative assumptions may be needed. The other major condition is c-command (and in fact minimal c-command, i.e. Minimality). If clitics are heads adjoined to v/I, then c-command of {\liv}P/VP-internal arguments follows. Nevertheless, a delicate issue arises because we have adopted the view that object clitics may alternate between a \textrm{${\subseteq}$} and a D form, and so do object DPs/\textrm{${\subseteq}$}Ps, according to whether they do or not undergo DOM.  The simplest thing to say is that $\varphi $-features label the root node in any event. The anonymous reviewer raises more complex issues yet, such as Long Distance Agreement, which are beyond the scope of the present article.}{} At the same time, the clitic and the doubled DP do not necessarily agree in Case, which is again part of the conclusion of \sectref{sec:manzini:2}.

Consider then the \textit{leísmo} pattern again, as exemplified in \REF{ex:manzini:23} above. From the point of view of the analysis of DOM in \sectref{sec:manzini:1}, the clitic and the DP it doubles/pronominalizes actually agree in case, namely in dative case. More formally, the clitic and the DP share the ${\subseteq}$ property, lexicalized by P in front of the lexical DP and by dative case on the clitic, as schematized in \REF{ex:manzini:25}. In other words, varieties like \REF{ex:manzini:25} can be described simply by saying that the conditions attaching on VP-internal embeddings of full DPs, also hold for the insertion of D heads in the clitic domain.

\newpage

\ea%25
    \label{ex:manzini:25}{}
    [\textsubscript{IP} [\textsubscript{${\subseteq}$} \textit{le}]  [\textsubscript{I} \textit{vi}   [\textsubscript{ VP} \sout{\textit{vi}}  [\textsubscript{${\subseteq}$P} \textit{a la niña}]]    \hfill (cf. \REF{ex:manzini:23b})\\
\z

The \textit{loísmo} pattern is schematized in \REF{ex:manzini:26}. In present terms, there is a ${\subseteq}$ case mismatch in \REF{ex:manzini:26} both when the DP argument corresponds to an inherent dative (with the verb \textit{speak} \textit{(to)}) and when it corresponds to a DOM dative (with the verb \textit{see}). Recall that inherent dative and DOM arguments can be distinguished, among others, on the basis of passivization; DOM datives with \textit{see} passivize, while inherent datives with \textit{speak} \textit{(to)} do not passivize. Needless to say, the parallel behavior of goal and DOM \textit{a}-phrases under \textit{loísmo} tendentially supports their unification, though not as directly as the \textit{leísmo} pattern in \REF{ex:manzini:25}. Assume as before that DPs and clitics are each separately merged in their domains (predicative and inflectional, respectively), and that each domain may have its own case pattern. In the predicative domain in \REF{ex:manzini:26}, highly ranked DPs (including 3P full pronouns) are introduced by the oblique ${\subseteq}$ relator under DOM, as are goal arguments selected by the verb. By contrast, in the clitic domain, all 3P internal arguments are simply lexicalized as Ds, i.e. as accusative.

\ea%26
    \label{ex:manzini:26}{}
    [\textsubscript{IP} [\textsubscript{D} \textit{u}] [\textsubscript{I} \textit{parlanu/viðinu}   [\textsubscript{ VP} \sout{\textit{parlanu/viðinu}} [\textsubscript{${\subseteq}$P} \textit{a iddu}]]              \hfill (cf. \ref{ex:manzini:21a})
\z

In the ditransitive counterparts to \REF{ex:manzini:26}, goal clitics surface in the dative, as in structure \REF{ex:manzini:27}. In a functionalist vein, one could account for \REF{ex:manzini:27} by invoking the need for disambiguation. There are formal means to implement the same basic idea. In an Agree configuration, the clitic effectively acts as a probe for its DP argument goal. If the clitic was embedded as a bare D in \REF{ex:manzini:27}, its closest goal would be the direct object, namely \textit{na} \textit{littira} ‘a letter’, yielding a reading different from the intended one. In other words, the right reading is achieved only by having recourse to the specialized dative clitic. Economy considerations privilege the simpler lexicalization in \REF{ex:manzini:26} where possible.

\ea%27
    \label{ex:manzini:27}{}
    [\textsubscript{IP} [\textsubscript{${\subseteq}$} \textit{li}] [\textsubscript{I} \sout{\textit{ʃkrivu}}  [\textsubscript{ VP} \textit{ʃkrivu na littira}  [\textsubscript{${\subseteq}$P} \textit{a iddu}]]    \hfill (cf. \ref{ex:manzini:21c})
\z

\largerpage[-1]
The most problematic configuration from the present point of view of DOM arises in the standard variety of Spanish or in Rioplatense dialects, where DOM obliques are doubled by accusative clitics, while goal datives are doubled by dative clitics, along the lines of \REF{ex:manzini:28}. It is a fact that in languages like \REF{ex:manzini:28} cliticization distinguishes lexical datives and DOM objects, while the present approach says that they have the same structure. However, recall that I assume that the case array of clitics in the inflectional domain does not necessarily match the case array of lexical DPs in the predicative domain. If so, we can describe \REF{ex:manzini:28} by saying that in the clitic domain, themes and goals are assigned accusative and dative respectively, and no DOM applies – even though DOM applies to lexical DPs in the predicative domain.

\ea%28
    \label{ex:manzini:28}
    \ea\label{ex:manzini:28a}{}
    [\textsubscript{IP} [\textsubscript{D} \textit{lo}]  [\textsubscript{I} \textit{vio}   [\textsubscript{ VP} \sout{\textit{vio}}  [\textsubscript{PP${\subseteq}$} \textit{a Juan}]]\\
    \ex\label{ex:manzini:28b}{}
    [\textsubscript{IP} [\textsubscript{D${\subseteq}$} \textit{le}] [\textsubscript{I} \textit{dio} [\textsubscript{VP} \sout{\textit{dio}} \textit{el libro} [\textsubscript{PP${\subseteq}$} \textit{a Juan}]]
    \z
\z


Let us then take stock. In \sectref{sec:manzini:1} and~\ref{sec:manzini:2} I have briefly argued for two main conclusions, which form the basis of the discussion in this section, namely \REF{ex:manzini:29}.

\ea%29
    \label{ex:manzini:29}
    In Romance\\
    \ea\label{ex:manzini:29a}
    DOM and goal arguments are both embedded by ${\subseteq}$ (\sectref{sec:manzini:1})\\
    \ex\label{ex:manzini:29b}
    Clitic and full DP arguments are both first-merged in their respective domains, each with their own case alignment (\sectref{sec:manzini:2}).
    \z
\z


\largerpage[-1]
In this section, I only considered Romance varieties where arguments of the predicative domain display DOM. Let us call this the DOM=Dat case alignment. In the clitic domain, we can find the same alignment (\tabref{extab:manzini:30}bi). However, DOM may be missing, yielding the case pattern Acc${\neq}$Dat (\tabref{extab:manzini:30}a). Finally, the clitics may display a single accusative realization for all direct or indirect object, in \tabref{extab:manzini:30}bii. Obviously, the numbering of the schemas in \tabref{extab:manzini:30} is meant to match those in \REF{ex:manzini:20}.
%
%\ea%30
%    \label{ex:manzini:30}
%    Clitics DPs \\
%    \ea\label{ex:manzini:30a}
%    Acc${\neq}$Dat      DOM=Dat\\
%    \ex\label{ex:manzini:30b}
%        \ea\label{ex:manzini:30bi}
%        DOM=Dat    DOM=Dat\\
%        \ex\label{ex:manzini:30bii}
%        Acc        DOM=Dat \\
%        \z
%    \z
%\z

\begin{table}
 \caption{\label{extab:manzini:30}Case patterns of clitics and DPs}
 \begin{tabular}{llll}
 \lsptoprule
    &    & Clitics        & DPs     \\
 \midrule
 a. &    & \textsc{acc}${\neq}$\textsc{dat} & \textsc{dom}=\textsc{dat} \\
 b. & i  & \textsc{dom}=\textsc{dat}       & \textsc{dom}=\textsc{dat} \\
    & ii & \textsc{acc}            & \textsc{dom}=\textsc{dat} \\
 \lspbottomrule
 \end{tabular}
 \end{table}

Suppose DOM in the predicative domain results from embedding of the argument under the same elementary predicate as the dative \REF{ex:manzini:29a}. Suppose further that clitics can have their own independent case alignment \REF{ex:manzini:29b}. Then there are at least three logical possibilities – namely that clitics have the same DOM pattern as the predicative domain (\textit{leísmo}), or that they have a non-DOM pattern (standard Spanish) or that finally they have accusative for all internal arguments (\textit{loísmo}).{} \footnote{If both object clitics and object DPs can be ±DOM, and they freely mix and match, we expect a fourth configuration – namely that there may be languages where a DOM 3P clitic corresponds to non-DOM DP objects. The closest match to this fourth predicted possibility arises in Quiteño Spanish, where “the DO-CLs have been almost universally replaced by \textit{le(s)} … This replacement applies irrespective of the features [±animate] and [±masc] …Thus, it could be said that QS has carried \textit{leísmo} to conclusion” \citep[387--388]{Suñer1989}, cf. (i). Importantly, “if there is an IO phrase, the CL refers unambiguously to the IO argument, and the DO automatically goes to ø” \citep[389]{Suñer1989}, cf. (ii).

\ea%i

    \gll {Ya} {le}   {vendió}.   \hfill (where \textit{le} {=} \textit{el} \textit{carro} ‘the car’)\\
        already   it   sold\textsc{.3sg}\\
    \glt ‘He already sold it.’
\z

\ea%ii

    \gll {Al} {chófer}  {le}   {ø} {di}.   \hfill (where {ø} = \textit{los} \textit{papeles} ‘the papers’)\\
        to.the chauffeur   to.him   {} gave.\textsc{1sg}\\
    \glt ‘I gave them to the chauffeur.’
\zlast}

If the variation spread matches the logically possible outcomes, then parametri\-za\-tion is simply seen to correspond to the choices left open by Universal Grammar. In a sense, one might say that there is no explanation for the observed variation, but only descriptive statements, as \REF{ex:manzini:25}--\REF{ex:manzini:28} are. In another sense, the best of explanations actually turns out to hold, namely that variation, in this instance, does not require any additional statement. No parameter specifies the open choices, which simply follow from the structure of grammar and the lexicon.{} \footnote{In recent work, \citet{ManziniFranco2019} formalize the Object agreement parameter in \REF{ex:manzini:17} in terms of labelling. Thus, DOM objects can project both a D(P) label and a P(P)/K(P) label. Bare direct objects project only DP and inherent datives project only PP/KP (see \citetv{chapters/cornilescu} for similar ideas applied to partially overlapping data). The Cliticization parameter in \REF{ex:manzini:20} can perhaps be resolved in the same way. In any event, this is beyond the scope of the present paper, whose aim was solely to display the actual extent of variation in Romance and draw some conclusions about the free crossing of parameter values.}

\section{Conclusions}\label{sec:manzini:4}
\largerpage[-1]
DOM arguments are associated with the \textit{a}/dative morphology typical of goal arguments, because they share the same syntactic structure of embedding, namely the relational content ${\subseteq}$ associated with the preposition \textit{a} or with the dative case inflection (\sectref{sec:manzini:1}). Pronouns, especially 1/2P pronouns, do not necessarily share the same case alignment as lexical DPs. In \sectref{sec:manzini:2} I have concluded that though a language like Italian has no DOM with lexical DPs/full pronouns, the presence of a single object form of 1/2P clitics is to be interpreted as evidence of the presence of DOM in the clitic domain, and not as a mere morphological syncretism.

In \sectref{sec:manzini:3}, I assumed that clitics and lexical DPs are separately merged each in their domain. The case array may then be differently set. I interpreted standard/Rioplatense Spanish as instantiating the pattern in which 3P clitics are not sensitive to DOM, though DOM is enforced by lexical DPs and full pronouns. The traditional name of \textit{leísmo} describes configurations in which animate 3P clitics are always dative, whether DOMed or inherent goals. The equally traditional label of \textit{loísmo/laísmo} describes the pattern in which 3P object clitics corresponding to animate referents (subject to DOM in the predicative domain) or to inherent datives are both in the accusative.

\sloppy
\printbibliography[heading=subbibliography,notkeyword=this]
\end{document}

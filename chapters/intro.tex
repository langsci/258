\documentclass[output=paper,modfonts,nonflat,colorlinks,citecolor=brown]{langsci/langscibook} 
\author{Anna Pineda\affiliation{Universitat Pompeu Fabra}\orcid{}\lastand Jaume Mateu\affiliation{Universitat Autònoma de Barcelona}\orcid{}}
\title{Dative constructions across languages: An introduction}
\abstract{}
\IfFileExists{../localcommands.tex}{
  % add all extra packages you need to load to this file  
\usepackage{tabularx} 
\usepackage{url} 
\urlstyle{same}

\usepackage{listings}
\lstset{basicstyle=\ttfamily,tabsize=2,breaklines=true}


%%%%%%%%%%%%%%%%%%%%%%%%%%%%%%%%%%%%%%%%%%%%%%%%%%%%
%%%                                              %%%
%%%           Examples                           %%%
%%%                                              %%%
%%%%%%%%%%%%%%%%%%%%%%%%%%%%%%%%%%%%%%%%%%%%%%%%%%%% 
%% to add additional information to the right of examples, uncomment the following line
% \usepackage{jambox}
%% if you want the source line of examples to be in italics, uncomment the following line
% \renewcommand{\exfont}{\itshape}
\usepackage{langsci-optional}
\usepackage{./langsci/styles/langsci-gb4e}
\usepackage{./langsci/styles/langsci-lgr}
\usepackage{pgfplots,pgfplotstable}

\definecolor{lsDOIGray}{cmyk}{0,0,0,0.45}

\usepackage{xassoccnt}
\newcounter{realpage}
\DeclareAssociatedCounters{page}{realpage}
\AtBeginDocument{%
  \stepcounter{realpage}
}


 



 

  \newcommand{\appref}[1]{Appendix \ref{#1}}
\newcommand{\fnref}[1]{Footnote \ref{#1}} 

\newenvironment{langscibars}{\begin{axis}[ybar,xtick=data, xticklabels from table={\mydata}{pos}, 
        width  = \textwidth,
	height = .3\textheight,
    	nodes near coords, 
	xtick=data,
	x tick label style={},  
	ymin=0,
	cycle list name=langscicolors
        ]}{\end{axis}}
        
\newcommand{\langscibar}[1]{\addplot+ table [x=i, y=#1] {\mydata};\addlegendentry{#1};}

\newcommand{\langscidata}[1]{\pgfplotstableread{#1}\mydata;}

\makeatletter
\let\thetitle\@title
\let\theauthor\@author 
\makeatother

\newcommand{\togglepaper}[1][0]{ 
%   \bibliography{../localbibliography}
  \papernote{\scriptsize\normalfont
    \theauthor.
    \thetitle. 
    To appear in: 
    Change Volume Editor \& in localcommands.tex 
    Change volume title in localcommands.tex
    Berlin: Language Science Press. [preliminary page numbering]
  }
  \pagenumbering{roman}
  \setcounter{chapter}{#1}
  \addtocounter{chapter}{-1}
}
\newcommand{\orcid}[1]{}
 
  %% hyphenation points for line breaks
%% Normally, automatic hyphenation in LaTeX is very good
%% If a word is mis-hyphenated, add it to this file
%%
%% add information to TeX file before \begin{document} with:
%% %% hyphenation points for line breaks
%% Normally, automatic hyphenation in LaTeX is very good
%% If a word is mis-hyphenated, add it to this file
%%
%% add information to TeX file before \begin{document} with:
%% %% hyphenation points for line breaks
%% Normally, automatic hyphenation in LaTeX is very good
%% If a word is mis-hyphenated, add it to this file
%%
%% add information to TeX file before \begin{document} with:
%% \include{localhyphenation}
\hyphenation{
affri-ca-te
affri-ca-tes
Tarra-go-na
Vio-le-ta
Jacken-doff
clit-ics
Giar-di-ni
Mor-fo-sin-tas-si
mi-ni-mis-ta
nor-ma-li-tza-ció
Caus-ees
an-a-phor-ic
caus-a-tive
caus-a-tives
Mar-antz
ac-cu-sa-tive
Ma-no-les-sou
phe-nom-e-non
Holm-berg
}

\hyphenation{
affri-ca-te
affri-ca-tes
Tarra-go-na
Vio-le-ta
Jacken-doff
clit-ics
Giar-di-ni
Mor-fo-sin-tas-si
mi-ni-mis-ta
nor-ma-li-tza-ció
Caus-ees
an-a-phor-ic
caus-a-tive
caus-a-tives
Mar-antz
ac-cu-sa-tive
Ma-no-les-sou
phe-nom-e-non
Holm-berg
}

\hyphenation{
affri-ca-te
affri-ca-tes
Tarra-go-na
Vio-le-ta
Jacken-doff
clit-ics
Giar-di-ni
Mor-fo-sin-tas-si
mi-ni-mis-ta
nor-ma-li-tza-ció
Caus-ees
an-a-phor-ic
caus-a-tive
caus-a-tives
Mar-antz
ac-cu-sa-tive
Ma-no-les-sou
phe-nom-e-non
Holm-berg
}

  \bibliography{../localbibliography}
  \togglepaper[1]%%chapternumber
}{}

\begin{document}
\maketitle 
 


\section{Presentation}

\subsection{Interest of the volume}


The present volume offers a comprehensive account of dative structures across languages –with an important, though not exclusive, focus on the Romance family. As is well-known, datives play a central role in a variety of structures, ranging from ditransitive constructions to cliticization of IOs and DOM-marked DOs, and including also psychological predicates, possessor or causative constructions, among many others. As interest in all these topics has increased significantly over the past three decades, this volume provides an overdue update on the state of the art. Accordingly, the chapters in this volume account for both widely discussed patterns of dative constructions as well as some that are relatively unknown. 


\subsection{Structure of the volume}


The book is organized into four main parts, comprising 15 papers, preceded by an overview by M. \textsc{Cristina Cuervo}. This contribution offers a cross-linguistic perspective on applicative heads, which over the past years have been widely assumed to be licensers of dative arguments cross-linguistically.



\textsc{Part i} is dedicated to analyzing datives in the context of ditransitive constructions, with focus on identifying the well-known \textit{Double Object Construction}. 



The literature on Double Object Constructions (e.g. \textit{John gave Mary the book)}, which is typically focused on English, is very rich (\citealt{Larson1988,Oehrle1976,Kayne1984,Jackendoff1990structures,Jackendoff1990larson,Pesetsky1995,Harley2002}, among many others). The three main analyses found in the literature which account for constructions with dative arguments, particularly ditransitive constructions, stipulate: 


\begin{enumerate}
\item 
an extra structure above the lexical V (see \citealt{Baker1977, Baker1988}, \citegen{Marantz1993} Applicative Hypothesis for Bantu and English, \citet{Anagnostopoulou2003} for Greek, \citet{MiyagawaTsujioka2004} for Japanese, or \citet{MiyagawaJung2004} for Korean, a.o.); 

\item 
 an extra structure inside the lexical V (Small Clause, \citealt{Kayne1984}; Zero Morpheme, \citealt{Pesetsky1995}); and

\item 
a proposal reconciling the two approaches mentioned above by distinguishing Low and High Applicatives \citep{Pylkkänen2002}, which hypothesizes the existence of extra structure above the VP for High Applicatives (those for which the interpretation does not involve a Goal argument) and extra structure inside the VP for Low Applicatives (those for which the interpretation involves transfer of possession). 

\end{enumerate}
 
{Since Pylkkänen's work on Applicatives in English, Finnish and Japanese, the use of these syntactic heads has been further developed and has given rise to works on many languages (\citealt{McGinnis2001} for Albanian and Icelandic, \citealt{Cuervo2003} for Spanish, \citealt{McIntyre2006} for German, \citealt{Fournier2010} for French, \citealt{Pineda2013, Pineda2016, PinedaPress} for Catalan). Additionally, more types of Applicatives have been proposed (for example, \citegen{Cuervo2003} Affected Applicatives).~}


 
{One of the most important implementations of Applicatives involves a particular type of ditransitive construction, the aforementioned Double Object Construction (DOC), as in English} {\textit{John gave Mary the book}}{. Although DOCs have been traditionally considered to be absent in Romance languages (\citealt{Kayne1984,HolmbergPlatzack1995}), over the past decades several researchers have claimed that Spanish indeed has this construction (\citealt{Masullo1992,Demonte1995,Romero1997,Bleam2003}). On the basis of \citegen{Pylkkänen2002} aforementioned work on applicatives, the existence of DOCs in Spanish has again been argued to be correct \citep{Cuervo2003}. This proposal has been since extended to other Romance languages, such as French \citep{Fournier2010}, Portuguese (\citealt{TorresMoraisSalles2010}), Romanian (\citealt{DiaconescuRivero2007}) and Catalan (\citealt{Pineda2013, Pineda2016, PinedaPress}} 


 
However, while the existence of DOCs, usually assumed to be mediated by applicative heads, is widely established in the study of English ditransitive constructions (\citealt{Baker1988,Marantz1993,Pylkkänen2002,Pylkkänen2008}), their presence in other language families remains highly controversial, especially in the realm of Romance languages. Thus, it is generally assumed for English that an applicative head is the backbone of the DOC \REF{ex:intro:1}, introducing the IO in its specifier position and relating it to the DO, in its complement position  \REF{ex:intro:2}:



\ea%1
    \label{ex:intro:1} 
         John gives Mary the book.
\z


\ea%2
    \label{ex:intro:2}
    \begin{forest}
    [\textit{v}P
        [\textit{v}]
        [LowApplP
            [IO
            		[Mary]
            ]
            [LowAppl’
                [LowAppl]
                [DO
                		[the book]
                ]
            ]
        ]
    ]
\end{forest}
\z

For Romance languages, it has been argued that the DOC pattern, with an applicative head, is also attested. This gives rise to two different perspectives: those identifying the DOC with clitic-doubled ditransitives (see e.g. \citealt{Cuervo2003}) and those arguing that the presence or absence of dative clitic doubling is not structurally relevant for DOCs (see e.g. \citealt{Pineda2013, Pineda2016, PinedaPress}). That is, there is no consensus as to whether a doubling dative clitic is a {\textit{sine qua non}} condition for Romance DOC. Romance languages offer an interesting landscape from which to consider a doubling dative clitic in ditransitive constructions. While this construction is possible in Spanish, Catalan and Romanian, it is impossible in French, Portuguese and Standard Italian. Moreover, doubling is compulsory in some American varieties of Spanish (Río de la Plata / Chile / Caracas) (\citealt{Parodi1998,Senn2008,Pujalte2009}) and Trentino \citep{Cordin1993}. Another point of controversy has to do with the (non-)existence of an English-like dative alternation (\textit{John gave Mary the book, John gave the book to Mary}) in Romance. Most of the aforementioned authors defend the existence of two different ditransitive constructions, the double object one (with clitic doubling) and the prepositional one (without clitic doubling), featuring structural differences (opposite c-commanding relations between objects) and semantic differences (successful transfer of possession or not). However, Pineda (2013, 2016, in press a) challenged this claim by showing that the purported structural and semantic differences between clitic-doubled and non-clitic doubled ditransitives constructions are not as robust as suggested. This assertion brings Romance clitic-doubling languages such as Spanish, Catalan or Romanian (for the latter, see also Tigau \& von Heusinger in press) close to non-doubling languages, such as French, Italian and Portuguese, for which the existence of two structural relations between the objects of ditransitive sentences has been acknowledged in the literature (see \citealt{Harley2002,Anagnostopoulou2003,Fournier2010}, and \citealt{BonehNash2011} for French; and \citealt{GiorgiLongobardi1991,McGinnis2001,Harley2002} for Italian). 



In the present volume, this issue is tackled, with special attention extended to the situation in Portuguese, by {\textsc{Ana Calindro.}}  This author discusses whether a particular diachronic change in the expression of indirect objects (generalization of {\textit{para}} ‘to’ in ditransitive constructions) in Brazilian Portuguese distinguishes this language from other Romance languages. She treats the structural representation of ditransitives in this language by dispensing with applicative heads and instead making use of a {\textit{p}} head (\citealt{Svenonius2003, Svenonius20014, Wood2012}) and the {\textit{i*}} single argument introducer proposed by \citet{WoodMarantz2017}. 



The situation of Portuguese and Spanish ditransitives is also analyzed by {\textsc{Paula Cépeda \& Sonia Cyrino}}. These authors explore the causes and the consequences of the two linear orders (DO>IO and IO>DO) allowed for the DO and the IO in Spanish, European Portuguese and Brazilian Portuguese ditransitives. They conclude that arguments supporting a DOC analysis for ditransitive constructions in these languages are inconclusive on both semantic and structural grounds. They argue that the two previously mentioned orders are derivationally related via an information structure operation.



Romanian ditransitives are also discussed in detail in this volume. {\textsc{Alexandra Cornilescu}} provides an account of the binding relations between the DO and the IO in Romanian ditransitives, focusing on the grammaticality differences triggered by clitic doubled IOs, differentially marked DOs and clitic doubled DOs. The data discussed in her paper, which have otherwise received scant attention, lead the author to propose a derivational account for ditransitive constructions to explain these differences.



Finally, French, Italian and Catalan ditransitives are also considered in the volume. In the paper by {\textsc{Michelle Sheehan,}} the author argues that ditransitives in these languages have two underlying structures so that a DP introduced by ‘a/à’ can be either dative, akin to the English DOC, or locative, akin to the English {\textit{to}}{}-dative construction. {\textsc{Sheehan}} bases her claims on the relations between objects with a focus on Person Case Constraint (PCC) effects. The author contrasts PCC effects in ditransitives and in {\textit{faire-infinitive}} causatives, providing evidence that such effects are not limited to clitic clusters, as previously suggested for Spanish by \citet{OrmazabalRomero2013}. In causatives, clitics also trigger PCC effects because the {\textit{a/à}} is unambiguously dative.



The debate regarding the existence or absence of an English-like dative alternation, with a DOC and a {\textit{to-}}dative construction, has received interest outside Romance linguistics. Accordingly, the volume includes an exhaustive account of Russian ditransitives, by \textsc{Svitlana} {\textsc{Antonyuk.}} This author proposes that the well-known binary distinction between DOC and the prepositional {\textit{to-}}counterpart is insufficient for Russian and a ternary distinction is needed. She formulates her claim on the basis of Russian quantifier scope freezing data, which demonstrate that Russian ditransitive predicates are not a homogeneous group, but rather subdivide into three groups with distinct underlying structures. 



{\textsc{Part ii} is dedicated to} other{ dative constructions, including possessor and experiencer constructions and related structures. The study of possessor datives is tackled from three different perspectives. First, in \textsc{Egor Tsedryk’s} paper,} the focus is extended to predicative possession and possessive modality in Russian, which allows both the dative (‘Vanja\textsc{\textsubscript{DAT}} be\textsc{\textsubscript{EXIST}} this book’) and the locative (‘At Vanja\textsubscript{GEN} be\textsubscript{EXIST} this book’) to occur with the existential \textsc{be}. The dative has a directional meaning (possible possession), opposed to stative inclusion of the locative (actual possession). This construal of the dative is furthermore extended to modal necessity of imperfective infinitive constructions (‘Vanja\textsubscript{DAT} to get up early tomorrow’).  Finally, building on the part-whole relation (possessum {${\subseteq}$} possessor) described by dative ({\textit{give the books} ${\subseteq}$ \textit{to the woman}}) and genitive possessors ({\textit{the books} ${\subseteq}$ \textit{of the woman}}), as well as the reverse relation (possessor {${\supseteq}$} possessum) found with instrumentals {\textit{the woman} ${\supseteq}$ \textit{with the books}}, a discussion is offered by \textsc{Ludovico Franco \& Paolo Lorusso} on the instances of such inclusive relations in the aspectual domain, when continuous/progressive tenses are combined with dative ({\textit{Gianni is at hunt}} ‘Gianni is hunting’) or instrumental ({\textit{They eat with honey}} ‘They are eating honey’) morphemes in different languages, such as Italian or Baka. Additionally, experiencer{ constructions are analyzed by \textsc{Antonio Fábregas \& Rafael Marín}, with focus on the stative meaning that characterize dative experiencers with Spanish psychological verbs (compare \textit{A Juan le preocupan las cosas} ‘To John \textsc{cl\textsubscript{dat}} concern.\textsc{3pl} the things’ stative \textit{vs. Juan se preocupa por las cosas} ‘John \textsc{cl\textsubscript{refl}} concerns.\textsc{3sg} for the things’ dynamic). A semantic characterization of datives as not denoting a full transference relation, but only a boundary, allows one to account for the stativity associated with experiencer datives. This contrasts with other prototypical values of datives such as recipients or goals, which are claimed to denote a transfer and are therefore dynamic}.



{\textsc{Part iii}} contains two proposals regarding applicative heads, which recently have been considered a cross-linguistic licenser of dative arguments. Building on \citeauthor{Pylkkänen2002}'s \citeyear{Pylkkänen2002, Pylkkänen2008} analysis of high and low applicatives, two proposals are advanced. The first, based on Bantu data, is elaborated by {\textsc{Mattie Wechsler}}. This author proposes the existence of a ‘super high’ applicative, and argues that (at least in Bantu) applicative heads are underspecified regarding their height. In the second proposal, which is based on data from Chukchi, West Greenlandic and Salish, {\textsc{David Basilico}} advocates for a different syntax of the low applicative head, which permits one to account for the presence of an antipassive morpheme in applicative constructions. 



{\textsc{Part iv}} focuses on the study of case alternations involving dative case. A wide range of structures where case alternations occur are considered in this volume. Within the Romance family, alternations involving dative case are attested with agentive verbs whose single complement is dative or accusative-marked (see \citealt{FernándezOrdóñez1999} and \citealt{Sáez2009} for Spanish, \citealt{Ramos2005,Morant2008,PinedaRoyo2017} and Pineda (in press b) \todo{missing reference} for Catalan, \citealt{Ledgeway2000} for Neapolitan, \citealt{Troberg2008} for French (on a diachronic perspective), and \citealt{Pineda2016} for a comprehensive Romance view including Catalan, Spanish, Asturian and Italian varieties). In the present volume, a related case of variation is analyzed by {\textsc{Adam Ledgeway, Norma Schifano \& Giuseppina Silvestri}}, where dative in the marking of the IO with agentive verbs alternates with genitive case, in constructions such as {\textit{I told}} [{\textsubscript{GEN/DAT the boy}}]{ \textit{to go}} or {\textit{I spoke}} [{\textsubscript{GEN/DAT}\textit{the mayor}}]. The data discussed come from Southern Italian varieties, where the Romance-style dative marking ({\textit{a}} ‘to’) alternates with a Greek-style marking ({\textit{di}} ‘of’). 



Another instance of case alternation involving dative case involves psychological predicates ({\citealt{BellettiRizzi1988}}), where the experiencer may show dative or accusative case in several Romance languages (see for example \citealt{CabréMateu1998,PinedaRoyo2017} and \citealt{Royo2017} for Catalan, and \citealt{FernándezOrdóñez1999} for Spanish). In the present volume, {\textsc{Carles Royo}} offers an exhaustive account of dative/accusative alternations with psychological predicates in Catalan varieties, and analyses the connection between the case alternation and the causative vs. stative nature of the construction.



Variation involving dative structures in Catalan is further explored in  the contribution of {\textsc{Teresa Cabré \& Antonio Fábregas}} who examine the notion of dative from a morphological perspective. Catalan dialectal differences between Valencian and non-Valencian varieties suggest an analysis of the notion of dative as non-monolithic. Whereas the dative clitic exponent {\textit{li}} in Valencian Catalan is case-marked with dative, the corresponding {\textit{li}} in non-Valencian Catalan is claimed to correspond to a locative adverbial embedded under D (thus {\textit{l+i}}), the locative element being attested independently in these varieties as {\textit{hi}} (both in the plural dative clitic, {\textit{els hi}} ‘them{\textsc{\textsubscript{dat}}}’, and in strictly locative contexts, {\textit{Hi sóc}} ‘I am there’). The consequences of this dialectal divide for clitic combinations are also explored.



In the Romance context, dative/accusative alternations are also closely connected with the so-called {\textit{leísmo}}, the use of dative clitics for DOs, and {\textit{loísmo/laísmo}}, the use of accusative clitics for IOs. These phenomena are the object of a study by {\textsc{Rita Manzini,} who} compares the realization of Romance {\textit{a-}}DPs (including Goal arguments of (di)transitive, Goal arguments of unergative verbs, and differentially marked objects of transitive verbs) and their compatibility with a cliticized dative form. In {\textit{leísta}} varieties, a dative clitic is used not only for Goal arguments, but also for differentially marked objects. However, in {\textit{loísta/laísta}} varieties, accusative clitics are used not only for differentially marked objects but also for Goal arguments of unergative verbs. Both phenomena are exemplified using data from Spanish and Southern Italian varieties. {\textsc{Manzini}} offers a unified account of Standard Spanish, as well as {\textit{leísmo}} and {\textit{loísmo/laísmo}} patterns in Spanish and Italian varieties, arguing that the case array may be set differently for lexical DPs and for clitics, the latter being optionally associated with DOM (whose syntactic structure of embedding is the same as {\textit{typical}} dative arguments) and therefore giving rise to {\textit{leísmo}}.



Finally, beyond the Romance linguistic domain, a well-studied language with case variation involving the dative is Icelandic, where dative/accusative has been extensively analyzed (see for example B\citealt{Barðdal2001, Barðdal2008,Svenonius2002,Maling2002,JónssonEythórsson2005}). The present volume also offers a contribution in this~line of research, with particular attention extended to the degree of predictability of the use of dative case. {\textsc{Jóhannes Gísli Jónsson \& Rannveig Thórarinsdóttirónsson}} analyze Icelandic case alternations in marking the object of borrowings and neologisms, and assess the conditions that motivate the use of the dative case, at the expense of the default accusative case, in the context of these novel transitive verbs.  
 

\sloppy\printbibliography[heading=subbibliography,notkeyword=this]\end{document}

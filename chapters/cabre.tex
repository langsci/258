\documentclass[output=paper,modfonts,nonflat]{langsci/langscibook} 
\author{Teresa Cabré\affiliation{}\lastand
Antonio Fábregas\affiliation{}}
\title{Ways of being a dative across Romance varieties}
\abstract{}
\IfFileExists{../localcommands.tex}{
  % add all extra packages you need to load to this file  
\usepackage{tabularx} 
\usepackage{url} 
\urlstyle{same}

\usepackage{listings}
\lstset{basicstyle=\ttfamily,tabsize=2,breaklines=true}


%%%%%%%%%%%%%%%%%%%%%%%%%%%%%%%%%%%%%%%%%%%%%%%%%%%%
%%%                                              %%%
%%%           Examples                           %%%
%%%                                              %%%
%%%%%%%%%%%%%%%%%%%%%%%%%%%%%%%%%%%%%%%%%%%%%%%%%%%% 
%% to add additional information to the right of examples, uncomment the following line
% \usepackage{jambox}
%% if you want the source line of examples to be in italics, uncomment the following line
% \renewcommand{\exfont}{\itshape}
\usepackage{langsci-optional}
\usepackage{./langsci/styles/langsci-gb4e}
\usepackage{./langsci/styles/langsci-lgr}
\usepackage{pgfplots,pgfplotstable}

\definecolor{lsDOIGray}{cmyk}{0,0,0,0.45}

\usepackage{xassoccnt}
\newcounter{realpage}
\DeclareAssociatedCounters{page}{realpage}
\AtBeginDocument{%
  \stepcounter{realpage}
}


 



 

  \newcommand{\appref}[1]{Appendix \ref{#1}}
\newcommand{\fnref}[1]{Footnote \ref{#1}} 

\newenvironment{langscibars}{\begin{axis}[ybar,xtick=data, xticklabels from table={\mydata}{pos}, 
        width  = \textwidth,
	height = .3\textheight,
    	nodes near coords, 
	xtick=data,
	x tick label style={},  
	ymin=0,
	cycle list name=langscicolors
        ]}{\end{axis}}
        
\newcommand{\langscibar}[1]{\addplot+ table [x=i, y=#1] {\mydata};\addlegendentry{#1};}

\newcommand{\langscidata}[1]{\pgfplotstableread{#1}\mydata;}

\makeatletter
\let\thetitle\@title
\let\theauthor\@author 
\makeatother

\newcommand{\togglepaper}[1][0]{ 
%   \bibliography{../localbibliography}
  \papernote{\scriptsize\normalfont
    \theauthor.
    \thetitle. 
    To appear in: 
    Change Volume Editor \& in localcommands.tex 
    Change volume title in localcommands.tex
    Berlin: Language Science Press. [preliminary page numbering]
  }
  \pagenumbering{roman}
  \setcounter{chapter}{#1}
  \addtocounter{chapter}{-1}
}
\newcommand{\orcid}[1]{}
 
  %% hyphenation points for line breaks
%% Normally, automatic hyphenation in LaTeX is very good
%% If a word is mis-hyphenated, add it to this file
%%
%% add information to TeX file before \begin{document} with:
%% %% hyphenation points for line breaks
%% Normally, automatic hyphenation in LaTeX is very good
%% If a word is mis-hyphenated, add it to this file
%%
%% add information to TeX file before \begin{document} with:
%% %% hyphenation points for line breaks
%% Normally, automatic hyphenation in LaTeX is very good
%% If a word is mis-hyphenated, add it to this file
%%
%% add information to TeX file before \begin{document} with:
%% \include{localhyphenation}
\hyphenation{
affri-ca-te
affri-ca-tes
Tarra-go-na
Vio-le-ta
Jacken-doff
clit-ics
Giar-di-ni
Mor-fo-sin-tas-si
mi-ni-mis-ta
nor-ma-li-tza-ció
Caus-ees
an-a-phor-ic
caus-a-tive
caus-a-tives
Mar-antz
ac-cu-sa-tive
Ma-no-les-sou
phe-nom-e-non
Holm-berg
}

\hyphenation{
affri-ca-te
affri-ca-tes
Tarra-go-na
Vio-le-ta
Jacken-doff
clit-ics
Giar-di-ni
Mor-fo-sin-tas-si
mi-ni-mis-ta
nor-ma-li-tza-ció
Caus-ees
an-a-phor-ic
caus-a-tive
caus-a-tives
Mar-antz
ac-cu-sa-tive
Ma-no-les-sou
phe-nom-e-non
Holm-berg
}

\hyphenation{
affri-ca-te
affri-ca-tes
Tarra-go-na
Vio-le-ta
Jacken-doff
clit-ics
Giar-di-ni
Mor-fo-sin-tas-si
mi-ni-mis-ta
nor-ma-li-tza-ció
Caus-ees
an-a-phor-ic
caus-a-tive
caus-a-tives
Mar-antz
ac-cu-sa-tive
Ma-no-les-sou
phe-nom-e-non
Holm-berg
}

  \bibliography{../localbibliography}
  \togglepaper[1]%%chapternumber
}{}

\begin{document}
\maketitle 


\begin{styleHeading}
Ways of being a dative across Romance varieties
\end{styleHeading}

\sffamily\bfseries
Teresa Cabré

\sffamily\bfseries\itshape
CLT-Universitat Autònoma de Barcelona

\sffamily\bfseries
Antonio Fábregas

\sffamily\bfseries\itshape
University of Tromsø-Norway’s Arctic University

\begin{stylelsAbstract}
Abstract. In this article, we argue that the term dative can correspond to objects of a very different linguistic nature, even in typologically close languages. Specifically, in syntactic terms datives can be different from accusatives or identical to them at some point in the derivation; in the latter case, clashes between 3rd person clitics emerge. Our approach, then, argues that clitic incompatibilities are best explained through syntactic tools.
\end{stylelsAbstract}

\section{Introduction: the nature of datives} %1. /

In a non-trivial sense, advancing our understanding of language frequently involves learning that objects that we previously took to be primitive, underived units in fact are built through the combination of independent elements. The analysis of passives is a prime example of this, as noted by {\textbackslash}citet\{williams15\}: we have moved away from a view where passive is a type of construction to a view where passives emerge from the conspiracy of several factors, some of them independent of each other. Also very frequently, noting that a linguistic object is a derived notion also implies realising that a single, unified definition of that object is just wrong, and that in different languages there are distinct procedures to build it—and again, passives come to mind (see {\textbackslash}citet\{croft17\}), since they vary with respect to the availability of accusative objects, the types of predicates that can be subject to them or their relation with aspect. 

This chapter starts out with the observation that ‘dative’ is another notion that, like passive, is best understood merely as a traditional label which in reality can correspond to entities with very different properties across languages. This automatically predicts that in some languages what we call a dative is syntactically different from accusatives, while in other languages they will share some property, in derivational or representational terms. From here we will state the hypothesis that 3rd person clitic clashes between accusative and dative, in fact, depend on whether the dative in that particular language shares with accusatives a property that makes them compete to be licensed by one head or not (\sectref{sec:key:2}). We will then illustrate the situation with only a few languages, for reasons of space: Valencian Catalan (VC), non-Valencian Catalan (nVC) and Spanish (Sp.), ending the article with a short note on French and Italian (\sectref{sec:key:3}). But we will begin with some initial evidence that datives should be regarded as a derived object.

In her detailed cross-linguistic overview of datives, {\textbackslash}citet\{naess09\} adopts a functional or conceptual criterion to identify something as dative: something is called ‘dative’ in a language if it is the marking assigned to prototypical goals. This criterion is probably the only one that allows a systematic comparison across typologically unrelated languages, but note that one function can be performed by means of different devices. And if we look briefly at the literature, it does seem that researchers agree that what we call dative corresponds to different objects across languages. {\textbackslash}citet\{cuervo03\} treats dative as a structural case—one that only appears within a specific structural configuration, in her case spec, Applicative Phrase; by contrast, {\textbackslash}citet\{woolford06\} treats it as inherent case. Some languages allow dative-marked subjects {\textbackslash}citep\{zaenenetal85\}, while others reject them (German: {\textbackslash}citet\{bayer02\}) and others allow only some datives in subject position (Russian: {\textbackslash}citet\{mooreperlmutter00\}). Some treat datives as cases (Spanish: {\textbackslash}citet\{ormazabalromero13\}), while others treat datives as the spell-out of possessive structures (Swiss German: {\textbackslash}citet\{leu15\}). See also Ledgeway, this volume and Royo, this volume, for similar observations about the different natures of datives.

All these authors have compelling arguments to make their points. Our claim here is that they are all essentially right for the specific instances of dative they analyse. What is wrong, however, is the underlying assumption that the term ‘dative’, taken from descriptive traditional grammars, corresponds to a uniform phenomenon across languages. In the rest of this article we will argue that the idea that datives are derived objects, in fact, makes interesting predictions for the properties of clitic incompatibilities in Romance languages.

\section{Clitic wars and what it means to be too similar in syntax} %2. /

The example in \REF{ex:key:1} is one instance of a clitic clash within a cluster, a very frequent situation across Romance languages {\textbackslash}citep\{rezac10\}. The general format of the situation is that two clitic pronouns—and they must be clitics—are incompatible with each other when they are adjacent in the same sequence (see Cuervo, this volume, for a detailed analysis of different syntactic positions for datives; as far as we see, the dative clitic occupies the same structural position independently of the syntactic position of its related applicative phrase). 

\ea%1
    \label{ex:key:1}
    \gll\\
        \\
    \glt
    \z

          Spanish

    *María   le     lo   dio.       

      \textit{María}   \textit{him.dat}  \textit{it.acc}  \textit{gave3sg}

    Intended: ‘María gave it to him’ (cf. *\textit{María} \textit{gave} \textit{him} \textit{it})  

Languages react in different ways to this situation, sometimes with different solutions for different clashes in the same language: one of the two clitics may disappear, or one may be replaced by another clitic from the system. In the case of \REF{ex:key:1}, Spanish follows the latter pattern and replaces the dative with a reflexive form \textit{se}.

\ea%2
    \label{ex:key:2}
    \gll\\
        \\
    \glt
    \z

          Spanish

    María  se  lo    dio.

    \textit{María}  \textit{refl}  \textit{it.acc}  \textit{gave3sg}

    ‘María gave it to him’

 An overwhelming majority of the analyses of such clashes treats them as a morphological phenomenon, meaning that the clash and its repair are assumed to take place at a ‘surface’ level where the syntax and semantics of the structure are not affected by it—and by the same token not involved in triggering it. {\textbackslash}citet\{perlmutter71, bonet91, bonet93, bonet95, grimshaw97, pescarini07, nevins12\} are among the noteworthy authors that have adopted this view. Though their analyses differ from each other in very crucial details, they share several intuitions beyond their morphological treatment of the phenomena. 

First, they propose that the clash is due to a form of morphological Obligatory Contour Principle infraction (see especially {\textbackslash}citet\{nevins12\}): the clash is caused by the morphological shape of the two clitics being identical or semi-identical. Note that using an OCP violation in itself involves complicating the architecture of grammar by allowing it to contain filters that exclude well-formed representations in syntax or phonology.

Second, they agree that, when the OCP is resolved by replacing one clitic with another, the clitic used in the repair is always more underspecified than the infringing clitic—in their analyses, because the repair involves removing some features from the representation. In the case of \REF{ex:key:2}, the Spanish clitic \textit{se} is more featurally impoverished than the dative \textit{le}: as can be seen in \REF{ex:key:3}, \textit{le} contrasts in number—but not in gender—while \textit{se} contrasts neither in gender nor in number.

\ea%3
    \label{ex:key:3}
    \gll\\
        \\
    \glt
    \z

          a. le [dative singular masculine / feminine]  / les [dative   plural masculine / feminine]

  b. se [reflexive 3rd person, singular / plural, masculine /   feminine]

In Italian, {\textbackslash}citet\{pescarini07\} notes that a clash between two instances of reflexive \textit{si} is solved by substituting the first one with the form \textit{ci}, used in locative contexts \REF{ex:key:4}. This form \textit{ci} is less specified than \textit{si} because it can be used in locative contexts but also as a 1pl pronoun (see {\textbackslash}citet\{ferrazzano03\}).

\ea%4
    \label{ex:key:4}
    \gll\\
        \\
    \glt
    \z

          Italian 

  a. *Nel medioevo   si   si   lavava         raramente.

              in-the Middle-Ages refl    refl  washed3sg seldom  

  b. Nel medioevo  ci  si  lavava         raramente          in-the Middle-Ages  deic  refl  washed3sg seldom  

  ‘In the Middle Ages they [people] rarely washed’

Third, they also agree that when the clash involves two or more instances of the most underspecified clitic, the only repair possible is to erase one of them. This is illustrated for \textit{ci} in \REF{ex:key:5}.

\ea%5
    \label{ex:key:5}
    \gll\\
        \\
    \glt
    \z

          Italian 

    a. *A Roma  ci  ci   porta   Mario.

              to Rome  us.acc  there  brings  Mario

    b. A Roma   ci  porta   Mario.      

                   to Rome  us.acc  brings  Mario

    ‘To Rome, Mario brings us’

Our claim in this paper is that the second and third observation are right, but that they can be recast in a better way within a syntactic system where there is no need to propose a morphological OCP. Instead, the situations we see in the previous examples are an effect of standard syntactic competition for licensing by the same head or set of heads. In other words, the problem of \REF{ex:key:4} and \REF{ex:key:5} is the same as what we see in \REF{ex:key:6}, which is wrong according to every theory because there are two DPs that compete for the same position in the derivation.

\ea%6
    \label{ex:key:6}
    \gll\\
        \\
    \glt
    \z

            *[Mary] [our family] arrives today.

Specifically, we will assume that {\textbackslash}citet\{sportiche96\} is right in his claim that the functional structure of a clause (at least in the languages we are dealing with here) includes a Clitic Area above vP. The clitics must be licensed in this area, which in turn—as we will see—can be split into several regions that are determined by microparametric choices in each (variety of a) language. In this system, the clitic clash is due to the descriptive principle in \REF{ex:key:7}, which replaces the OCP.

\ea%7
    \label{ex:key:7}
    \gll\\
        \\
    \glt
    \z

          Two clitics produce a clash when they must occupy the same   position at some point of the syntactic derivation.

\ea%7
    \label{ex:key:7}
    \gll\\
        \\
    \glt
    \z

         covers two types of cases. The first type of case is a situation where the competition involves the base-generation position. In less technical terms, the first type of clash arises in situations where one clitic is built using the exact same pieces as a second clitic. We will illustrate this situation with non-Valencian Catalan (\sectref{sec:key:3.1}). 

The second type of case is the situation where, even though one clitic is not directly derived from the other, the grammatical properties of case assignment and checking of arguments in the specific language force them to compete to be licensed by the same head. We will illustrate this for Spanish (\sectref{sec:key:3.2}).

Let us now move on to the specific case studies.

\section{Different datives, different clashes} %3. /

Our main proposal is that Sportiche’s clitic area can be split into different regions, each one designated for a different type of clitic \REF{ex:key:8}. Specific languages make the split in different ways, following the general principle that within a (universal) domain languages have the freedom to select a subset of formal properties that they grammaticalise {\textbackslash}citep\{ramchandsvenonius14, wiltschko14\}. 

\ea%8
    \label{ex:key:8}
    \gll\\
        \\
    \glt
    \z

          [XP    X  [YP    Y  [ZP  Z  ...  ]]]

  clitic type 1    clitic type 2    clitic type 3...

The clash happens when the two clitics compete for a single region; the repair depends on how many other available regions—if any—there are in the language. Let us illustrate this with several cases.

\subsection{Valencian Catalan and non-Valencian Catalan} %3.1. /

The literature on the different clitic systems across Catalan varieties, and the corresponding clashes, is very abundant {\textbackslash}citep\{bonet93,martin12\}. For reasons of space we will concentrate here on the opposition between two general varieties, abstracting away from more fine-grained subvarieties. Consider \REF{ex:key:9}, which compares the VC system with the nVC system.

\ea%9
    \label{ex:key:9}
    \gll\\
        \\
    \glt
    \z

          a.   VC datives

  \{Li  /  els\}     portaré     un regal.

    3sg.dat  3pl.dat  will.bring1sg  a  present

  ‘I will give him/her/them a present’

  b.   nVC datives

  \{Li  /  els hi [ǝlzi]\}   portaré     un regal.

    3sg.dat  3pl.dat  will.bring1sg  a  present

  ‘I will give him/her/them a present’

One important difference between the two is that in nVC there is a component /i/ that is contained in both the singular and the plural dative. This component is identical to the locative clitic \textit{hi} /i/, which—crucially—nVC has but VC lacks. Following {\textbackslash}citep\{martin12\}, we propose that this is a sign that nVC builds the dative pronoun by combining the locative clitic with a DP layer identical to the accusative clitic (as represented in 10, abstracting away from morphophonological reordering). We part ways with {\textbackslash}citep\{martin12\} in that we consider this \textit{hi} a real locative (a noun denoting regions of space, \textit{pace} {\textbackslash}citet\{rigau78, rigau82\}), not an element expressing general deixis independent of any conceptual dimension.\footnote{Note that in view of the Catalan contrast between \textit{aqu-í} ‘here’ and \textit{aqu-est} ‘this’, it seems more plausible to propose that \textit{aqu}{}- corresponds to the deictic part of the word and -\textit{i} acts as a restrictor that provides place as the dimension where deixis applies. As one anonymous reviewer notes, probably the most controversial part of our analysis is to treat /i/ as the spell out of a locative N layer, given the existence of cases of \textit{hi} where it substitutes for predicates or is used with an apparent expletive function. See {\textbackslash}citet\{cabrefabregasinpress\} for a more detailed presentation of how we deal with cases where \textit{hi} behaves in Catalan like what seems to be a non-locative element: in short, we propose that the clitic still denotes space in a more metaphorical way, and is used to replace elements that correspond to the personal sphere of the subject, or to properties within whose set the subject is included.} 

\ea%10
    \label{ex:key:10}
    \gll\\
        \\
    \glt
    \z

          nVC dative clitics

  a. [DP  (ǝ)l  [NumP  ø  [NP  i]]]

  b. [DP  (ǝ)l  [NumP  z  [NP  i]]]

As VC lacks a locative, it follows that it cannot build its dative pronouns with a locative noun. This perforce means that accusative and dative clitics must be differentiated by a specific property (given that they are spelled out differently, see \REF{ex:key:11}). At this point, just for the sake of argument, we will assume that they are different through case marking in the form of a KP assigning them inherent case \REF{ex:key:12}, but that the specific property is irrelevant provided that one clitic is not derived from the other.

\ea%11
    \label{ex:key:11}
    \gll\\
        \\
    \glt
    \z

           a. el ‘accusative singular masculine’

   b. la ‘accusative singular feminine’

   c. li ‘dative singular, masculine / feminine’ 

\ea%12
    \label{ex:key:12}
    \gll\\
        \\
    \glt
    \z

         VC dative clitics

  a. [KP  Dative   [...NumP Sing]]   <-{}-> li

  b. [KP Dative   [...NumP Plural]]  <-{}-{}-> els    

Let us now consider the behaviour of accusative and dative third person clitics inside the cluster. The sequence of two 3rd person clitics in nVC is ungrammatical, and gets resolved by the forms /li/ and /ǝlzi/; that is, in practice the surface result is identical to a single dative (13-14).\footnote{Note that the orthography \textit{l’hi} (just like \textit{els} \textit{hi}) is pronounced identically to the singular clitic, /li/.} 

\ea%13
    \label{ex:key:13}
    \gll\\
        \\
    \glt
    \z

         a. *Li     (e)l     donaré.

          him.dat    it.acc    will.give1sg

   b. L’hi     donaré.

       L-HI    will.give1sg

  ‘I will give it to him’

\ea%14
    \label{ex:key:14}
    \gll\\
        \\
    \glt
    \z

         a. *Li    (e)ls     donaré.

        him.dat    them.acc  will.give1sg

   b. Els hi  donaré.

      ELS HI  will.give1sg

  ‘I will give them to him’

In contrast, VC does not display a clash in such sequences.

\ea%15
    \label{ex:key:15}
    \gll\\
        \\
    \glt
    \z

         a. Li’l     donaré. 

       him.dat-it.acc  will.give

  ‘I will give it to him’

  b. Li’ls    donaré.

      him.dat-them.acc  will.give

  ‘I will give them to him’

To put it simply, there is a correlation between building the dative from the locative and not allowing a sequence of dative + accusative third person clitics. This is precisely what we expect if the clash emerges in syntax, through the licensing by heads within Sportiche’s area. In nVC, the dative is in actuality an accusative containing a locative, so in a sequence that—using the traditional terminology—contains an accusative and a dative, there are in fact two accusative layers that will compete for licensing in the same position. The two cannot be licensed at the same time, so the resulting sequence is ‘impoverished’ on the surface \REF{ex:key:16b}. Our claim is that within ZP there is only syntactic space for one D layer, one Number layer and one NP layer. The D layer is occupied by /l/, the number layer by /z/ and the noun layer by /i/.\footnote{Interestingly, nVC uses a similar strategy to repair standard Person Case Constraint infractions: the dative is reduced to \textit{hi} (\textit{Al} \textit{director,} \textit{m’hi} \textit{ha} \textit{recomanat} \textit{la} \textit{Mireia} To-the director me-LOC has recommended Mireia ‘Mireia has recommended me to the director’). Although we will not develop the argument here, our suggestion is that YP is an area where all arguments interpreted as affected must be licensed. The dative clitic is interpreted as referring to an affected participant {\textbackslash}citep\{adgerharbour07\}, so ‘dative’ clitics must rise to YP. Person-marked clitics like \textit{me} ‘me’ are affected by default, so they are base{}-generated in YP—which, as an anonymous reviewer has pointed out to us, explains the connection between person-marked clitics and datives in, for instance, marking through a ‘to’ and the absence of gender contrasts. As at some point in the derivation datives and person clitics compete for the same area. This explains the clash, which is resolved in the same way as before.}  

\ea%16
    \label{ex:key:16}
    \gll\\
        \\
    \glt
    \z

           a. [XP  X  [YP    Y  [ZP  Z  ...  ]]]

     clitic type 1  clitic type 2    clitic type 3...

              *\textbf{li} \textbf{+} \textbf{(e)ls}

   b. [XP  X  [YP    Y  [ZP (e)l\textsuperscript{D}{}-s\textsuperscript{Num}{}-i\textsuperscript{N}     ]]]

By contrast, in VC, the dative is not derived from the accusative; each clitic is distinct, and therefore each one of them can be licensed in a different region of Sportiche’s area: the clash does not emerge in \REF{ex:key:17} because their datives are not derived from their accusatives.\footnote{At this point, we lack sufficient evidence of whether there is a correlation between carrying KP and being spelled out by a non-analytical form, as perhaps the contrast between VC and nVC suggests. We are forced at this point to treat it as a lexical accident which does not follow from independent principles. Further research is necessary to determine whether the correlation is real or an accident of VC.}

\ea%17
    \label{ex:key:17}
    \gll\\
        \\
    \glt
    \z

         [XP    X  [YP    Y  [ZP  Z  ...  ]]]

     clitic type 1  clitic type 2    clitic type 3...

        \textbf{li}      \textbf{(e)ls}

\subsection{3.2. Spanish spurious \textit{se}}

An\-\-\-other famous case of 3rd person clitic clash is provided by Spanish spurious \textit{se}. Here we will argue that this clash is due to the same type of competition, and that the repair follows from the regions that Spanish defines in its Sportiche’s area.

As can be seen in \REF{ex:key:18}, there is no obvious evidence that the Spanish dative and the Spanish accusative are built one from the other; in particular, Spanish lacks any clitic /e/ which would allow one to segment the dative into the accusative layer plus a morphosyntactically significant unit. At least at first glance, then, the situation is different from Catalan.

\ea%18
    \label{ex:key:18}
    \gll\\
        \\
    \glt
    \z

         Spanish clitics

  a. lo ‘accusative masculine singular’

  b. la ‘accusative feminine singular’

  c. le ‘dative singular, masculine / feminine’

However, there is an important sense in which datives and accusatives are syntactically non-distinct in Spanish: Differential Object Marking in some ‘accusatives’ is identical to dative marking \REF{ex:key:19}, and in fact within the same vP the two markings cannot occur at the same time \REF{ex:key:20} ({\textbackslash}citet\{ormazabalromero13\}; see also Cornilescu, this volume).

\ea%19
    \label{ex:key:19}
    \gll\\
        \\
    \glt
    \z

          a.   Le     di     un libro a María.

    \textit{her.dat}   \textit{gave1sg}  \textit{a.}  \textit{book} \textit{A} \textit{María} \textit{[dative]}

  b.   Vi      a María.

    \textit{saw.1sg} \textit{A} \textit{María} \textit{[accusative]}

\ea%20
    \label{ex:key:20}
    \gll\\
        \\
    \glt
    \z

         Entregué   (*a) los prisioneros   al     enemigo.

  delivered.1sg   A  the prisoners  A-the    enemy

  ‘I delivered the prisoners to the enemy’

This pattern has motivated analyses where Spanish vPs assign only one real case, ‘internal object case’ ({\textbackslash}citep\{romero12, ormazabalromero13\}), which is manifested through \textit{a}{}-marking. \REF{ex:key:20} is ungrammatical with double a-marking simply because the vP can only assign one case, and the two internal arguments compete with each other in order to get that case.\footnote{See {\textbackslash}citet\{romero12\} for the relation between the non \textit{a}{}-marked ‘accusative’ argument and case assignment.}  Our claim is, then, that \REF{ex:key:21} is ungrammatical for the same reason as \REF{ex:key:20}: in both cases, two elements compete to be licensed by the same head, the one that assigns DOM in \REF{ex:key:20} and the one that licenses 3rd person clitics in \REF{ex:key:21}.\footnote{We assume that the presence of an object clitic (\textit{lo} or \textit{le}) implies that the pronoun has been checked in the ZP area at some point in the derivation. Our approach implies that objects not carrying DOM are case-licensed in a different way from DOM-objects (see also {\textbackslash}citet\{lopez12\}), but the distinction dissolves from the perspective of the clitic, which requires them to be generated in ZP.} Irrespectively of whether they are generated in different regions, at some point they will have to establish a case-relation with a head, and given that accusatives and datives both compete for this—because the only real case in Spanish is ‘internal argument case’—only one of them will remain.

\ea%21
    \label{ex:key:21}
    \gll\\
        \\
    \glt
    \z

        *Le       lo       di.

    him.dat    it.acc    gave

In other words, like nVC datives, Spanish datives are ‘fake datives’, but for different reasons. To consider the repair strategy, and to be more explicit about the competition, let us say a bit more about the clitic regions in Spanish. We assume {\textbackslash}citet\{kayne10\} proposal about the relevant regions in Spanish \REF{ex:key:22}.\footnote{Among {\textbackslash}citet\{kayne10\} arguments for this ordering, he observes that it reproduces the natural ordering of clitics inside the cluster and that one can establish an implicational hierarchy in terms of which clitics can intervene across varieties between subject agreement morphology and the verbal stem.}   

\ea%22
    \label{ex:key:22}
    \gll\\
        \\
    \glt
    \z

         [WP  se  [XP  me/te  [YP  le  [ZP  lo ...[vP]]]]

    \textit{reflexive}  \textit{me/you}  \textit{3rd} \textit{dat}  \textit{3rd} \textit{acc}  

Interestingly, the sequence follows a logical ordering from two perspectives: first, the lowest type of clitics are those that are defined by the maximal amount of interpretable phi features: accusative pronouns contrast in gender and number; datives contrast just in number; and reflexive pronouns lack interpretable phi features {\textbackslash}citep\{reuland11\}.Second, person{}-marked pronouns are higher than third person pronouns—perhaps defined by absence of person features—thus matching the observation that speaker and addressee are defined high in the clausal structure as a form of deixis (see {\textbackslash}citet\{giorgi09\}, among many others). 

Once we assume this sequence, it is striking that the incompatibilities attested in Spanish always involve pronouns in adjacent areas, and the repair involves using a clitic that belongs to a higher area: 3rd accusative and 3rd dative occupy adjacent regions,\footnote{See Sheehan, this volume, for further details about PCC effects.} and the repair involves using a reflexive pronoun rather than a dative (see {\textbackslash}citet\{alcaraz17\} for syntactic arguments that spurious \textit{se} behaves as a real reflexive, which we do not reproduce for lack of space).

\ea%23
    \label{ex:key:23}
    \gll\\
        \\
    \glt
    \z

         [WP  se  [XP   [YP  le  [ZP  lo ...[vP]]]]

We have argued that \textit{le} and \textit{lo} are incompatible with each other due to case competition—in other words, because Spanish lacks a real dative case. To be more specific, we can assume that even though they end up in different regions, they are base-generated in the same position, which checks internal case (assume that is ZP). 

Crucially, the repair involves removing one of the two clitics. The first question is, which one? In a morphologically-oriented theory with OCP, in principle either of the two clitics could be the one replaced. In our proposal, we correctly predict that the one replaced is the first one, because that is the highest one: if the repair involves using a clitic that belongs to a higher region, replacing the lowest one would produce a standard intervention effect because the long distance relation established between WP and ZP has another clitic in YP that is closer to WP than ZP.

\ea%24
    \label{ex:key:24}
    \gll\\
        \\
    \glt
    \z

          *[WP   se  [XP  [YP  le  [ZP  lo ...[vP]]]]

Second, why should it be \textit{se}? On the assumption that a third person pronoun does not contain person features, the XP area—for 1st and 2nd person pronouns—is not available for 3rd person clitics; the closest available area is therefore the reflexive pronoun area. Given that the height correlates with the number of interpretable features, this also captures the intuition that the repairs always involve less specified clitics:. In our account, this is just an epiphenomenon of the fact that syntactic structure is built from bottom to top and not vice versa.

Third, our approach also explains why a clash between two adjacent \textit{se} pronouns cannot be resolved through substitution. \REF{ex:key:25} is ungrammatical; in our approach, it is so because the two \textit{se} pronouns want to be licensed in the WP region, and there is place for only one of them. Given that the clitic area finishes here in Spanish, there is no other region to license the reflexive not licensed in WP.

\ea%25
    \label{ex:key:25}
    \gll\\
        \\
    \glt
    \z

         *Se se   arrepiente mucho aquí.

     \textit{refl} \textit{refl}  \textit{repent}     \textit{a} \textit{lot}   \textit{here}

   Intended: ‘Here it is very common to regret something’

In Italian, on the other hand, such a repair strategy is possible, essentially because in this language the clitic area contains a region that precedes reflexives where \textit{ci} (as an underspecified deictic element) is located. \REF{ex:key:4}, repeated below as \REF{ex:key:26}, is explained through an area such as \REF{ex:key:27}; note that our approach also correctly predicts that \REF{ex:key:5}, repeated as \REF{ex:key:28}, can only be repaired through total erasure of one clitic.\footnote{An independent question is what specific features Italian \textit{ci} spells out so that it can be used for both the 1pl and the locative. In this respect, we follow {\textbackslash}citet\{ferrazzano03\}, who argues that Italian \textit{ci} in fact stands for proximal deixis, irrespective of whether it applies to participants—where the closest participant is the speaker—or place. This approach is also compatible in principle with Pescarini’s proposal that \textit{ci} is featurally underspecified, introduced as a default. Note that an account based on homophony begs the question of why first person and place happen to be spelled out by exactly the same sequence.} 

\ea%26
    \label{ex:key:26}
    \gll\\
        \\
    \glt
    \z

         Italian 

  a. *Nel medioevo   si  si   lavava. 

         in-the Middle-Ages  refl  refl  washed3sg    

  b. Nel medioevo  ci  si  lavava.

      in-the Middle-Ages  deic  refl  washed3sg    

  ‘In the Middle Ages they washed themselves’

\ea%27
    \label{ex:key:27}
    \gll\\
        \\
    \glt
    \z

        a. [HP  ci   [WP   si  [XP  mi  [YP  gli  [ZP lo  ...[vP]]]]

  b. [HP    [WP *si si  [XP    [YP    [ZP    ...[vP]]]]

  c. [HP  ci  [WP   si si  [XP    [YP    [ZP    ...[vP]]]]

\ea%28
    \label{ex:key:28}
    \gll\\
        \\
    \glt
    \z

        *A Roma  ci  ci   porta   Mario.

      to Rome  us.acc  there  brings  Mario

Again, note that in the case of \REF{ex:key:26b} our approach also correctly predicts that the clitic that will be replaced will be the first one in the series: on the assumption that structure is built bottom-up, necessarily the deictic clitic will precede the reflexive. As far as we can tell, the morphological approach cannot make this prediction.

\subsection{A short note on French} %3.3. /

At first sight, French is a surface counterexample to our proposal. It is plausible to speculate that at least the dative form \textit{lui} contains the locative \textit{y} /i/. This impression is confirmed, following our own logic, by the observation made by {\textbackslash}citet\{rezac10\} that \textit{lui} cannot occur in a sequence with \textit{y}, as illustrated in \REF{ex:key:29}.

\ea%29
    \label{ex:key:29}
    \gll\\
        \\
    \glt
    \z

        *Je   lui     y   parle.

    I  him.dat  loc  talk.1sg

  Intended: ‘I talk to him there’

We would then predict that a 3rd accusative should be incompatible with a 3rd dative in French, for the same reasons as in nVC: the dative is a fake dative involving the locative layer. And yet, French seems to allow the sequence seen in \REF{ex:key:30}.

\ea%30
    \label{ex:key:30}
    \gll\\
        \\
    \glt
    \z

         Je le     lui    ai     donné.

  I   it.acc  him.dat  have1sg  given

We argue that this is not a real counterexample, essentially because the apparent compatibility is purely orthographic, an artefact of the writing system. {\textbackslash}citet\{schwarze01\} notes that although the orthographic representation in French insists on keeping the two clitics, the natural pronunciation of the sequence written as \textit{je} \textit{le} \textit{lui} is [ʒɥi] or [ʒlɥi], where crucially what is preserved, at most, is the shape of the fake dative [lɥi], as in nVC. In our view, pending a deeper typological study in French, this language confirms our predictions. 

\section{Conclusions} %4. /

In this short contribution we have argued that what has been labelled ‘dative’ in typologically related languages corresponds to very different types of entities. In the set of languages considered, only Valencian Catalan can be argued to have a real dative integrated as a distinct case in the system. Fake datives can be obtained—perhaps among other ways—by building them with locatives and DP layers (as in non-Valencian Catalan, possibly French), or inside languages which only assign one internal argument case and therefore do not make a real distinction between dative and accusative case. 

We have furthermore argued that there is a correlation between whether the dative is fake or not, and whether it will display a clash with accusative 3rd clitics or not. Specifically, we have argued that whenever the dative is fake, it competes with the accusative for licensing at the syntactic level. 

We have thus sketched an account of ‘clitic wars’ which treats the phenomenon as syntactic rather than morphological. We have shown that our account, based on a standard notion of ‘competition for licensing by a syntactic head’, manages to capture the correct intuitions of the morphological approach (underspecification, alternation between substitution and erasure), while managing to make additional correct predictions not made by the morphological account (the highest clitic is substituted, erasure happens when there is no higher region in the clitic area). This makes the resort to generalised OCP solutions (or {\textbackslash}citet\{richards10\} Distinctness) redundant. We hope that even though our study is limited in empirical scope we have convincingly shown that the syntactic route is worth exploring in accounting for these facts. 

\textbf{Acknowledgements.} The authors would like to thank the audience at and organisers of the \textit{Dative} \textit{Structures} \textit{and} \textit{Beyond} conference (Barcelona, \citealt{January2017}) and the \textit{Romanistentag} (Zurich, \citealt{October2017}) for illuminating comments and suggestions. All disclaimers apply.

\bfseries
\begin{verbatim}%%move bib entries to  localbibliography.bib
\begin{stylepi}
@article{\textrm{Adger\textrm{Adger2007,
	author = {\textrm{Adger, David and Daniel Harbour},
	journal = {Syntax and syncretism of the PCC.} \textrm{\textit{Syntax}} \textrm{},
	note = {.}},
	pages = {2--37},
	sortname = {\textrm{Adger, David and Daniel Harbour},
	volume = {10},
	year = {2007}
}

\end{stylepi}

\begin{stylepi}
@misc{\textrm{Alcaraz2017,
	author = {\textrm{Alcaraz, Alejo},
	note = {} \textit{dative} \textit{problem}}\textrm{. Lingbuzz/003293}},
	title = {} \textrm{\textit{The} \textit{spurious} \textit{vs},
	year = {2017}
}

\end{stylepi}

\begin{stylepi}
\textrm{Bayer, Josef. 2004. Non-nominative subjects in comparison. In P. Bhashakarao \& K. V. Subbarao (eds.),} \textrm{\textit{Non} \textit{nominative} \textit{subjects}}\textrm{, 31.58. Amsterdam: John Benjamins.}
\end{stylepi}

\begin{stylepi}
@misc{\textrm{Bonet1991,
	author = {\textrm{Bonet, Eulàlia},
	note = {PhD dissertation, MIT.}},
	title = {} \textrm{\textit{Morphology} \textit{after} \textit{syntax}}\textrm{},
	year = {1991}
}

\end{stylepi}

\begin{stylepi}
@article{\textrm{Bonet1993,
	author = {\textrm{Bonet, Eulàlia},
	journal = {3rd person pronominal clitics in dialects of Catalan.} \textrm{\textit{Catalan} \textit{Working} \textit{Papers} \textit{in} \textit{Linguistics}} \textrm{},
	note = {.}},
	pages = {85--111},
	sortname = {\textrm{Bonet, Eulalia},
	volume = {3},
	year = {1993}
}

\end{stylepi}

\begin{stylepi}
@book{\textrm{Bonet1995,
	address = {Feature structure of Romance clitics.} \textrm{\textit{Natural} \textit{Language} \textit{and} \textit{Linguistic} \textit{Theory}} \textrm{13 \REF{ex:key},
	author = {\textrm{Bonet, Eulàlia},
	note = {607-647.}},
	publisher = {4}, pp},
	sortname = {\textrm{Bonet, Eulalia},
	year = {1995}
}

\end{stylepi}

\begin{stylepi}
\textrm{Cabré, Teresa \& Antonio Fábregas. In press. 3}\textrm{\textsuperscript{rd}} \textrm{person clitic combinations across Catalan varieties: consequences for the nature of the dative clitic.} \textrm{\textit{The} \textit{Linguistic} \textit{Review}}.
\end{stylepi}

\begin{stylepi}
@book{\textrm{Croft2017,
	address = {Aspect and causal structure. Oxford},
	author = {\textrm{Croft, William},
	note = {}},
	publisher = {Oxford University Press},
	sortname = {\textrm{Croft, William},
	title = {} \textrm{\textit{Verbs}}\textrm{},
	year = {2017}
}

\end{stylepi}

@misc{Cuervo2003,
	author = {Cuervo, María Cristina},
	note = {Ph.D. dissertation, MIT.},
	title = {\textit{Datives} \textit{at} \textit{large}},
	year = {2003}
}


@misc{Ferrazzano2003,
	author = {Ferrazzano, Lisa Reisig},
	note = {Unpublished ms., CUNY graduate center.},
	title = {\textit{The} \textit{morphology} \textit{of} \textit{ci} \textit{and} \textit{its} \textit{‘distal’} \textit{relative,} \textit{vi}},
	year = {2003}
}


@book{Giorgi2009,
	address = {Oxford},
	author = {Giorgi, Alessandra},
	publisher = {Oxford University Press},
	title = {\textit{About} \textit{the} \textit{speaker}},
	year = {2009}
}


@article{Grimshaw1997,
	author = {Grimshaw, Jane},
	journal = {\textit{Linguistic} \textit{Inquiry}},
	pages = {373--422},
	title = {Projection, heads, and optimality},
	volume = {28},
	year = {1997}
}


@incollection{Kayne2010,
	address = {Amsterdam and Philadelphia},
	author = {Kayne, Richard},
	booktitle = {\textit{{Romance}} \textit{languages} \textit{and} \textit{linguistic} \textit{theory} \textit{2008}},
	editor = {Brigitte Kampers-Manhe Reineke Bok-Bennema and Bart Hollebrandse},
	pages = {145–170},
	publisher = {John Benjamins},
	title = {Toward a syntactic reinterpretation of \citet{{HarrisH}alle2005}},
	year = {2010}
}


@book{Leu2015,
	address = {Oxford},
	author = {Leu, Tom},
	publisher = {Oxford University Press},
	title = {\textit{The} \textit{architecture} \textit{of} \textit{determiners}},
	year = {2015}
}


@book{López2012,
	address = {Cambridge (Mass.)},
	author = {López, Luis},
	publisher = {MIT Press},
	sortname = {Lopez, Luis},
	title = {\textit{Indefinite} \textit{Objects}},
	year = {2012}
}


@misc{Martin2012,
	author = {Martin, Jesús},
	note = {Ph. D. dissertation, New York University.},
	title = {\textit{Deconstructing} \textit{{Catalan}} \textit{object} \textit{clitics}},
	year = {2012}
}


@article{MooreMoore2000,
	author = {Moore, John and David Perlmutter},
	journal = {What does it take to be a dative subject? \textit{Natural} \textit{Language} \textit{and} \textit{Linguistic} \textit{Theory}},
	pages = {373--416},
	volume = {18},
	year = {2000}
}


@incollection{Næss2009,
	address = {Oxford},
	author = {Næss, Åshild},
	booktitle = {\textit{The} \textit{{Oxford}} \textit{Handbook} \textit{of} \textit{Case}},
	editor = {A. Spencer and A. Malchukov},
	pages = {572--581},
	publisher = {Oxford University Press},
	sortname = {Naess, Ashild},
	title = {Varieties of dative},
	year = {2009}
}


@incollection{Nevins2012,
	address = {Oxford},
	author = {Nevins, Andrew},
	booktitle = {\textit{The} \textit{morphology} \textit{and} \textit{phonology} \textit{of} \textit{exponence}},
	editor = {Jochen Trommer},
	pages = {84--117},
	publisher = {Oxford University Press},
	title = {Haplological dissimilation at distinct stages of exponence},
	year = {2012}
}


@article{OrmazábalOrmazábal2013,
	author = {Ormazábal, Javier and Juan \citet{Romero},
	journal = {Object agreement, clitics and dialectal variation. \textit{Probus}},
	pages = {301--344},
	sortname = {Ormazabal, Javier and Juan \citet{Romero},
	title = {}},
	volume = {25},
	year = {2013}
}


@book{Perlmutter1971,
	address = {New York},
	author = {Perlmutter, David M.},
	publisher = {Holt Rinehart and Winston},
	title = {\textit{Deep} \textit{and} \textit{surface} \textit{structure} \textit{constraints} \textit{in} \textit{syntax.} \textit{Transatlantic} \textit{series} \textit{in} \textit{linguistics}},
	year = {1971}
}


@article{Pescarini2007,
	author = {Pescarini, Diego},
	journal = {\textit{Anuario} \textit{del} \textit{seminario} \textit{de} \textit{filología} \textit{vasca} \textit{Julio} \textit{de} \textit{Urquijo}},
	pages = {285–300},
	title = {Types of syncretism in the clitic systems of romance},
	volume = {41},
	year = {2007}
}


@misc{RamchandRamchand2014,
	author = {Ramchand, Gillian and Peter Svenonius},
	note = {\textit{Language} \textit{Sciences} 46. 152 - 174.},
	title = {Deriving the functional hierarchy},
	year = {2014}
}


@book{Reuland2011,
	address = {Cambridge (Mass.)},
	author = {Reuland, Eric},
	publisher = {MIT Press},
	title = {\textit{Anaphora} \textit{and} \textit{language} \textit{design}},
	year = {2011}
}


@book{Richards2010,
	address = {Cambridge (Mass.)},
	author = {Richards, Norvin},
	publisher = {MIT Press},
	title = {\textit{Uttering} \textit{trees}},
	year = {2010}
}


@article{Rigau1978,
	author = {Rigau, Gemma},
	journal = {\textit{Els} \textit{Marges}},
	pages = {99--102},
	title = {‘Hi’ datiu inanimat},
	volume = {12},
	year = {1978}
}


@article{Rigau1982,
	author = {Rigau, Gemma},
	journal = {\textit{Linguistic} \textit{Inquiry}},
	pages = {146--150},
	title = {Inanimate indirect object in {Catalan}},
	volume = {13},
	year = {1982}
}


@incollection{Rezac2010,
	address = {Oxford, New York},
	author = {Rezac, Milan},
	booktitle = {\textit{Defective} \textit{paradigms:} \textit{missing} \textit{forms} \textit{and} \textit{what} \textit{they} \textit{tell} \textit{us}},
	editor = {Greville Corbett Matthew Baerman and Dunstan Brown},
	pages = {151–180},
	publisher = {Oxford University Press},
	title = {Ineffability through modularity: {{G}}aps in the french clitic cluster},
	year = {2010}
}


@incollection{Romero2012,
	address = {Oxford, New York},
	author = {Romero, Juan},
	booktitle = {\textit{Variation} \textit{in} \textit{datives:} \textit{a} \textit{microcomparative} \textit{perspective}},
	editor = {Beatriz Fernández and Ricardo Etxepare},
	pages = {283–300},
	publisher = {Oxford University Press},
	title = {Accusative datives in {Spanish}},
	year = {2012}
}


@incollection{Schwarze2001,
	address = {Chicago},
	author = {Schwarze, Christoph},
	booktitle = {\textit{Proceedings} \textit{of} \textit{the} \textit{{LFG}} \textit{01} \textit{conference}},
	editor = {Myriam Butt and Tracy Holloway King},
	pages = {1–24},
	publisher = {CSLI Publications},
	title = {On the representation of french and italian clitics},
	year = {2001}
}


@incollection{Sportiche1996,
	address = {Dordrecht},
	author = {Sportiche, Dominique},
	booktitle = {\textit{Phrase} \textit{structure} \textit{and} \textit{the} \textit{lexicon}},
	editor = {Johann. Rooryck and L. Zaring},
	pages = {213--276},
	publisher = {Kluwer},
	title = {Clitic constructions},
	year = {1996}
}


@book{Williams2015,
	address = {Cambridge},
	author = {Williams, Alexander},
	publisher = {Cambridge University Press},
	title = {\textit{Arguments} \textit{in} \textit{syntax} \textit{and} \textit{semantics}},
	year = {2015}
}


@book{Wiltschko2014,
	address = {Cambridge},
	author = {Wiltschko, Martina},
	publisher = {Cambridge University Press},
	title = {The universal structure of categories},
	year = {2014}
}


@article{Woolford2006,
	author = {Woolford, Ellen},
	journal = {\textit{Linguistic} \textit{Inquiry}},
	pages = {11--130},
	title = {Lexical case, inherent case and argument structure},
	volume = {37},
	year = {2006}
}


@article{ZaenenZaenen1985,
	author = {Zaenen, Annie, Joan Maling and Höskuldur Thráinsson},
	journal = {The Icelandic Passive. \textit{Natural} \textit{Language} \textit{and} \textit{Linguistic} \textit{Theory}},
	pages = {441--483},
	sortname = {Zaenen, Annie, Joan Maling and Hoskuldur Thrainsson},
	title = {Case and grammatical function},
	volume = {3},
	year = {1985}
}


\end{verbatim}
\sloppy\printbibliography[heading=subbibliography,notkeyword=this]\end{document}

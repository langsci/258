\documentclass[output=paper,modfonts,nonflat,colorlinks,citecolor=brown]{langsci/langscibook} 
\author{Adam Ledgeway\affiliation{}\and
Norma Schifano\affiliation{}\lastand
Giuseppina Silvestri\affiliation{}}
\title{Microvariation in dative-marking in the Romance and Greek varieties of Southern Italy}
\abstract{Greek and Romance have been spoken alongside of one another for centuries in southern Italy. Even though the Greek-speaking areas have been dramatically reduced over the centuries such that today Greek is now only spoken by a small number of increasingly elder speakers in a handful of villages of Calabria and southern Apulia (Salentino), the influence of Greek is still undeniable in that it has left its mark on the structures of the surrounding Romance dialects. Indeed, in this respect Rohlfs aptly coined the phrase spirito greco, materia romanza  (literally `Greek spirit, Romance material') to highlight the fact that in many respects the syntax of these so-called Romance dialects is underlying Greek, despite employing predominantly Romance lexis. In this paper we draw on two cases studies from the Romance and Greek varieties spoken in Calabria to illustrate how the syntax of argument-marking has variously been subject to contact-induced change, giving rise to significant variation in the marking and distribution of   \textsc{Recipient}  arguments in accordance with both pragmatic and structural factors. In both cases, it will be shown that contact-induced borrowing does not replicate the original structure of the lending language but, rather, produces hybrid structures which are ultimately neither Greek nor Romance in nature.}

 \IfFileExists{../localcommands.tex}{
  % add all extra packages you need to load to this file  
\usepackage{tabularx} 
\usepackage{url} 
\urlstyle{same}

\usepackage{listings}
\lstset{basicstyle=\ttfamily,tabsize=2,breaklines=true}


%%%%%%%%%%%%%%%%%%%%%%%%%%%%%%%%%%%%%%%%%%%%%%%%%%%%
%%%                                              %%%
%%%           Examples                           %%%
%%%                                              %%%
%%%%%%%%%%%%%%%%%%%%%%%%%%%%%%%%%%%%%%%%%%%%%%%%%%%% 
%% to add additional information to the right of examples, uncomment the following line
% \usepackage{jambox}
%% if you want the source line of examples to be in italics, uncomment the following line
% \renewcommand{\exfont}{\itshape}
\usepackage{langsci-optional}
\usepackage{./langsci/styles/langsci-gb4e}
\usepackage{./langsci/styles/langsci-lgr}
\usepackage{pgfplots,pgfplotstable}

\definecolor{lsDOIGray}{cmyk}{0,0,0,0.45}

\usepackage{xassoccnt}
\newcounter{realpage}
\DeclareAssociatedCounters{page}{realpage}
\AtBeginDocument{%
  \stepcounter{realpage}
}


 



 

  \newcommand{\appref}[1]{Appendix \ref{#1}}
\newcommand{\fnref}[1]{Footnote \ref{#1}} 

\newenvironment{langscibars}{\begin{axis}[ybar,xtick=data, xticklabels from table={\mydata}{pos}, 
        width  = \textwidth,
	height = .3\textheight,
    	nodes near coords, 
	xtick=data,
	x tick label style={},  
	ymin=0,
	cycle list name=langscicolors
        ]}{\end{axis}}
        
\newcommand{\langscibar}[1]{\addplot+ table [x=i, y=#1] {\mydata};\addlegendentry{#1};}

\newcommand{\langscidata}[1]{\pgfplotstableread{#1}\mydata;}

\makeatletter
\let\thetitle\@title
\let\theauthor\@author 
\makeatother

\newcommand{\togglepaper}[1][0]{ 
%   \bibliography{../localbibliography}
  \papernote{\scriptsize\normalfont
    \theauthor.
    \thetitle. 
    To appear in: 
    Change Volume Editor \& in localcommands.tex 
    Change volume title in localcommands.tex
    Berlin: Language Science Press. [preliminary page numbering]
  }
  \pagenumbering{roman}
  \setcounter{chapter}{#1}
  \addtocounter{chapter}{-1}
}
\newcommand{\orcid}[1]{}
 
  %% hyphenation points for line breaks
%% Normally, automatic hyphenation in LaTeX is very good
%% If a word is mis-hyphenated, add it to this file
%%
%% add information to TeX file before \begin{document} with:
%% %% hyphenation points for line breaks
%% Normally, automatic hyphenation in LaTeX is very good
%% If a word is mis-hyphenated, add it to this file
%%
%% add information to TeX file before \begin{document} with:
%% %% hyphenation points for line breaks
%% Normally, automatic hyphenation in LaTeX is very good
%% If a word is mis-hyphenated, add it to this file
%%
%% add information to TeX file before \begin{document} with:
%% \include{localhyphenation}
\hyphenation{
affri-ca-te
affri-ca-tes
Tarra-go-na
Vio-le-ta
Jacken-doff
clit-ics
Giar-di-ni
Mor-fo-sin-tas-si
mi-ni-mis-ta
nor-ma-li-tza-ció
Caus-ees
an-a-phor-ic
caus-a-tive
caus-a-tives
Mar-antz
ac-cu-sa-tive
Ma-no-les-sou
phe-nom-e-non
Holm-berg
}

\hyphenation{
affri-ca-te
affri-ca-tes
Tarra-go-na
Vio-le-ta
Jacken-doff
clit-ics
Giar-di-ni
Mor-fo-sin-tas-si
mi-ni-mis-ta
nor-ma-li-tza-ció
Caus-ees
an-a-phor-ic
caus-a-tive
caus-a-tives
Mar-antz
ac-cu-sa-tive
Ma-no-les-sou
phe-nom-e-non
Holm-berg
}

\hyphenation{
affri-ca-te
affri-ca-tes
Tarra-go-na
Vio-le-ta
Jacken-doff
clit-ics
Giar-di-ni
Mor-fo-sin-tas-si
mi-ni-mis-ta
nor-ma-li-tza-ció
Caus-ees
an-a-phor-ic
caus-a-tive
caus-a-tives
Mar-antz
ac-cu-sa-tive
Ma-no-les-sou
phe-nom-e-non
Holm-berg
}
 
  \bibliography{../localbibliography}
  \togglepaper[13]%%chapternumber
}{}

\begin{document}
\shorttitlerunninghead{Microvariation in dative-marking in Romance and Greek varieties}
\maketitle 
 
\section{Introduction: Greek-Romance contact in southern Italy}

As is well known, Greek has been spoken as an indigenous language in southern Italy since ancient times (\citealt[12-38]{Falcone1973}; \citealt[304-306]{Horrocks1997}; \citealt[112-121]{Manolessou2005}  \citealt[133]{Ralli2006}). According to one, albeit now unpopular, view championed most notably by \citet{Rohlfs1924, Rohlfs1933, Rohlfs1974, Rohlfs1977}, the Greek spoken in southern Italy, henceforth Italo-Greek, is to be considered a direct descendant of the ancient (mainly Doric) Greek varieties which were imported into \textit{Magna Graecia} as early as the eighth century BC with the establishment of numerous Greek colonies along the coasts of southern Italy. The opposing – and now widely accepted – view, argued most vehemently by \citet{Battisti1927} (cf. also \citealt{Morosi1870,Parlangèli1953}), sees the Greek of southern Italy as a more recent import dating from the Byzantine period of domination between the sixth and eleventh centuries (though see \citealt{Fanciullo2007}, for a conciliatory approach to these apparently two opposing views). Whatever the correct view, it is in any case clear that by the beginning of the second millennium AD Greek was still widely spoken as a native language in north-western Sicily, Calabria and Apulia.

Today, by contrast, Italo-Greek survives precariously only in a handful of villages of southern Calabria and Salento in the respective areas of Bovesía and Grecía Salentina. In Bovesía, where the local variety of Greek is known as \textit{Greko} (though usually known as \textit{grecanico} in Italian), the language is today confined to five remote villages of the Aspromonte mountains (namely, Bova (Marina), Chorío di Rochudi, Condofuri (Marina), Gallicianò and Roghudi (Nuovo)), where it is reputed (\citealt{Spano1965}; \citealt[308-313]{Martino1980}; \citealt[16-19]{Stamuli2007}; \citealt[126-127]{Remberger2011}), at least according to some of the most generous estimates (cf. \citealt[27-31]{Katsoyannou1995}; \citealt[8-9]{Katsoyannou2001}), to be spoken by as many as about 500 speakers (cf. however \citealt{SquillaciInpress}). In Grecía Salentina, on the other hand, the language, locally known as \textit{Griko}, appears to have fared somewhat better, in that it continues to be spoken in a pocket of seven villages of the Otranto peninsula (Calimera, Castrignano dei Greci, Corigliano d’Otranto, Martano, Martignano, Sternatia, Zollino) by as many as 20,000 speakers according to the most optimistic estimates (\citealt{Comi1989,SobreroMiglietta2005}; \citealt[105]{Manolessou2005}; \citealt[52-53]{Marra2008,Romano2008}; \citealt[3-4]{Baldissera2013}), though once again our recent investigations would indicate a considerably lower figure.

Now, although Greek was extensively spoken in southern Italy for centuries, following the gradual expansion first of Latin and then what were to become the local Romance varieties in this same area, Greek and Romance came to be used alongside of each other in a complex situation of diglossia with expanding bilingualism. As a consequence, the Romance dialects of these two areas, namely \textit{Calabrese} and \textit{Salentino}, display huge structural influences from Italo-Greek, since they first emerged among speakers whose mother tongue was Greek (the `substrate') and continued to develop and expand to the present day in the shadow of the surrounding, albeit shrinking, Italo-Greek dialects (the `adstrate'). In recent times these latter varieties also increasingly show some structural influences from the local Romance dialects and, in particular, from regional Italian which has also been thrown into the mix, at least among younger members of the speech community (cf. \citealt[338]{Martino1980,Profili1985,Marra2008,Romano2008}), as witnessed, for example, in causative constructions (\citealt{LedgewaySchifanoSilvestriInpress, LedgewaySchifanoSilvestriInPrep}).

Consequently, it has become commonplace in the literature to claim that once extensive Greek-Romance bilingualism throughout the extreme south of Italy has given rise to an exceptional Hellenization of the local Romance dialects or, as \citet[61]{Rohlfs1933} aptly put it, a case of \textit{spirito greco, materia romanza} `Greek soul, Romance (lexical) material'.\footnote{Cf. the distinction between PAT(tern) and MAT(erial) discussed in \citet{MatrasSakel2004,MatrasSakel2007}.} While accepting Rohlfs’ general thesis that the Romance dialects of this area superficially appear to be nothing more than Greek disguised as Romance, such broad-brush generalizations obscure many subtle differences between Italo-Greek and the local Romance varieties which have largely gone unnoticed (for an overview, see \citealt{Ledgeway2013}). In what follows we shall therefore consider two case studies in microvariation involving dative structures born of Greek-Romance contact in Calabria. More specifically, these case studies illustrate the influence of Grecanico on Calabrese involving the so-called Greek-style dative whereby the relevant Romance dialects have variously adopted and adapted an original Greek structure that highlights both significant diatopic and diachronic microvariation in the structural realization of dative marking within the DP, as well as in the structural positions in which dative-marked DPs are licensed. In both cases, the varieties in question marry together in still poorly explored and largely little understood ways facets of core Romance and Greek syntax to produce a number of innovative hybrid structures, the evidence of which can be profitably used to throw light on the nature of parametric variation and the proper formal characterization of convergence and divergence. Indeed, once we begin to peel back the layers, it soon becomes clear that convergence through grammars in contact does not necessarily lead to simple borrowing and transference through interference, but more frequently gives rise to new hybrid structures born of reanalysis of the original Italo-Greek structures within a Romance (or Italo-Greek) grammar instantiating `deeper' microparametric options.

\section{Greek-style dative}

Since at least \citet[§639]{Rohlfs1969},\footnote{Cf. \citet[§639]{Rohlfs1969}, \citet[232-233]{Trumper2003}, \citet[209]{Vincent1997}, \citet[243, 427-429]{Katsoyannou1995}, \citet[54-55]{Katsoyannou2001}, \citet[140-141]{Ralli2006}, \citet[192-196]{Ledgeway2013}.} it has been reported that many Romance dialects of southern Calabria, following an original Greek pattern (cf. \citealt[160]{Joseph1990}) now widespread within the Balkan Sprachbund (\citealt[187]{Sandfeld1930, Pompeo2012}), extended the distribution of the genitive preposition \textit{di} ‘of’ to mark many of the traditional uses of the dative (including benefactive and ethical datives in addition to core \textsc{Recipient} arguments), the so-called \textit{dativo greco} `Greek-style dative'. Consequently, on a par with the Grecanico pattern in (\ref{ex:ledgeway:1}a) in which the indirect object \textit{Ǵoséppi} is Case-marked genitive, witness the genitive form of the definite article \textit{tu}, in the Calabrese dialect of S. Ilario in (\ref{ex:ledgeway:1}b) the \textsc{Recipient} argument is marked with the genitive preposition \textit{d(i)} ‘of’. 

\ea\label{ex:ledgeway:1}
\ea  Bova  \\
  \gll Ordínettse  tu  Ǵoséppi  ná  ’ne  meθéto.\\  
    he.ordered  of.the  Giuseppe  that  he.be  with.them\\
    \glt `He ordered Giuseppe to stay with them.'
  \ex S. Ilario\\
    \gll Si  dissi  d-u  figghiòlu  ’u  si  ndi  vaci.\\ 
    \textsc{dat}.3=  I.said  of-the  boy  that  self=  therefrom=  he.goes \\
    \glt `I told the boy to go.'
    \z
    \z
    

This pattern of dative marking is attested in several dialects around Bova, witness the examples in (\ref{ex:ledgeway:2}a-c), although its use today in Bova itself can, at best, be described as moribund. By contrast, no such use of the genitive has been recorded for the Romance dialects of Salento, as further confirmed by our own fieldwork, witness \REF{ex:ledgeway:3} where the \textsc{Recipient} argument is marked by the typical Romance preposition \textit{a} ‘to’.

\ea\label{ex:ledgeway:2}
Calabrese  \\
\ea
	\gll Nci  lu  dissi  di  lu  párracu.\\
    \textsc{dat}.3=   it=  I.said  of  the  priest\\
    \glt `I told the priest.'
 \ex \gll Nci  u  mandai  d-u  nonnu.  \\
    \textsc{dat}.3=  it=  I.sent  of-the  grandfather\\
    \glt `I sent it to grandfather.'
    
\ex
	\gll Nci  u  muštrai  di  lu  mè  vicinu.\\
    \textsc{dat}.3=  it=  I.showed  of  the  my  neighbour\\
    \glt `I showed it to my neighbour.'
\z
\z

\ea\label{ex:ledgeway:3}
Scorrano, Salento\\
\gll Vene  cu  lli  face  lezione  alla  fija.\\
he.comes  that  \textsc{dat}.3=  does  lesson  to.the  daughter\\
\glt `He comes to teach their daughter.'
\z

Although there is undoubtedly some truth to these traditional descriptions of the Greek-style dative, they nonetheless conceal some non-trivial differences between Grecanico and Calabrese. In particular, a detailed examination of the distribution of the Greek-style dative highlights the need to distinguish between at least two varieties of Calabrese, henceforth Calabrese\textsubscript{1} and Calabrese\textsubscript{2}, in which the distribution of the Greek-style dative not only displays some important differences with respect to Grecanico, but also in relation to each other.

\subsection{Case study 1: Calabrese\textsubscript{1}}

From our fieldwork and investigations the varieties that come under the label of Calabrese\textsubscript{1} include, at least, the dialects of Bagaladi, San Lorenzo, Brancaleone, Palizzi, Bovalino, \textsuperscript{(†)}Bova, Chorío, Roccaforte, Africo, Natile di Careri, San Pantaleone and S. Ilario.\footnote{For full details about the authors’ fieldwork, see the project’s website at \url{https://greekromanceproject.wordpress.com/the-project}.} In contrast to the traditional description of the Greek-style dative reviewed in §2 above, the distribution of the Greek-style dative in these varieties shows some major differences (cf. \citealt{Trumper2003}; \citealt[193-196]{Ledgeway2013}). First, Greek-style genitive marking of indirect objects is not obligatory in Calabrese. Indeed, in accordance with the typical Romance pattern,\textsc{Recipient} arguments surface much more frequently in the dative marked by the preposition \textit{a} ‘to’ (\textit{a} ‘to’ + \textit{u} ‘the.\textsc{msg}’ > \textit{ô} ‘to the’), witness (\ref{ex:ledgeway:4}a) which forms a minimal pair with (\ref{ex:ledgeway:4}b).

\ea\label{ex:ledgeway:4}
  Africo\\
\ea
	\gll Nci  dissi  ô  figghiòlu  ’i  ccatta  u  latti.\\
    \textsc{dat}.3=  I.said  to.the  boy  that  he.buys  the  milk\\

\ex
	\gll Nci  dissi  d-u  figghiòlu  ’i  accatta  u  latti.\\
    \textsc{dat}.3=  I.said  of-the  boy  that  he.buys  the  milk\\
    \glt `I told the boy to buy the milk.' 
    \z
    \z

Second, in structures such as (\ref{ex:ledgeway:4}b) the genitive-marked indirect object DP is always obligatorily doubled by a dative clitic, witness the grammaticality judgments reported in (\ref{ex:ledgeway:5}a-c). 

\ea\label{ex:ledgeway:5}
\ea Africo\\
\gll        *(Nci)  dissi  d-u  figghiòlu  ’i  accatta  u  latti.\\
      \textsc{dat}.3=  I.said  of-the  boy  that  he.buys  the  milk\\
      \glt `I told the boy to buy the milk.'

    \ex   Bagaladi\\
          \gll *(Nci)  lu  scrissi  di  mè  frati.\\
      \textsc{dat}.3=  it=  I.wrote  of  my  brother\\
      \glt `I wrote it to my brother.'
    
    \ex  Bagaladi\\
          \gll *(Nci)  lu  vindia  di  Don  Pippinu.\\
      \textsc{dat}.3=  it=  I.sold  of  Don  Peppino\\
      \glt `I was selling it to Don Peppino.'
\z
\z

It would appear then that we are not dealing with an autonomous genitive structure as in (Italo-)Greek, but, rather, with a hybrid structure in which the indirect object is referenced in part through dative marking on the verbal head and in part through genitive marking on the nominal dependent. This observation is even more striking when we consider that many of the same dialects have an independent genitive clitic (\textsc{inde} >) \textit{ndi} ‘of it; thereof/-from’ which, despite providing a perfect match for the genitive case of the nominal dependent, cannot double the indirect object in such examples:

\ea\label{ex:ledgeway:6}
\ea Africo\\
	\gll Ndi  dissi  d-u  figghiòlu  ’i  accatta  u  latti.\\
      \textsc{gen}=  I.said  of-the  boy  that  he.buys  the  milk\\

\ex Bagaladi\\
	\gll Ndi  lu  scrissi  di  mè  frati.\\
      \textsc{gen}=  it=    I.wrote  of  my  brother\\

\ex  Bagaladi\\
	\gll Ndi  lu  vindia  di  Don  Pippinu.\\
      \textsc{gen}=  it=  I.sold  of  Don  Peppino\\
      \z
      \z
      

Finally, the use of the so-called Greek-style dativeis not indiscriminate, but carries a marked pragmatic interpretation. Thus, despite appearances, (\ref{ex:ledgeway:4}a-b) are not entirely synonymous. By way of comparison, consider the English minimal pair in (\ref{ex:ledgeway:7}a-b), where the indirect object of the first example (\textit{to someone}) has undergone so-called \textit{dative shift} in the second example, instantiating the double object construction where it now appears without the dative marker \textit{to} and comes to precede the underlying direct object (see the contributions in \textsc{Part I} of this volume for further detailed discussions of the double object construction).

\ea\label{ex:ledgeway:7}
\ea
 I promised to rent every apartment in the building to someone.\\
\ex
I promised to rent someone every apartment in the building.\\
\z
\z

As is well known, one of the pragmatico-semantic consequences of \textit{dative shift} in English is to force a known or given interpretation of the \textsc{recipient} argument, as can be clearly seen by the contrast in (\ref{ex:ledgeway:7}a-b):\footnote{For full discussion, see \citet{Larson1988, Larson1990,Jackendoff1990larson,Torrego1998} and references cited there.} whereas the quantifier \textit{to someone} in (\ref{ex:ledgeway:7}a) typically refers to an unknown individual or group of individuals (e.g. whoever I can find who is willing to pay the rent), dative-shifted \textit{someone} in (\ref{ex:ledgeway:7}b) typically, though not necessarily unambiguously for all speakers, refers to a particular individual already known to the speaker (e.g. my father’s best friend), but whom the speaker simply chooses not to name in this particular utterance (for discussion, see \citealt{AounLi1993}). By the same token, it is this same presuppositional reading of the \textsc{recipient} that is licensed by the Greek-style dative in Calabrese\textsubscript{1}, witness the implied specific reading of \textit{studenti} in (\ref{ex:ledgeway:8}b) when marked by the genitive \textit{di} in contrast to its non-specific reading in (\ref{ex:ledgeway:8}a) when it surfaces with the dative \textit{a}; similarly, the identity of `the boy' in (\ref{ex:ledgeway:4}b) is assumed to be known to the addressee.\footnote{An anonymous reviewer points out that the alternation between the analytic prepositional construction with σε ‘to’ and the synthetic genitive is not necessarily free in Standard Modern Greek where the difference between the non-specific and specific readings in (\ref{ex:ledgeway:8}a-b) finds an exact parallel (cf. \citealt{Dimitriadis1999,Michelioudakis2012}). Nonetheless, there still remains a significant difference between Calabrese1 and Standard Modern Greek, in that the use of the genitive in Calabrese1 is only ever employed as a marked strategy to signal the presuppositional reading, whereas in Standard Modern Greek the synthetic genitive can also mark non-presuppositional readings just like the analytic prepositional construction.}

\ea\label{ex:ledgeway:8}
  Bova\\
\ea
	\gll La  machina,  nci  la  vindu  a  nu  studenti.\\
      the  car  \textsc{dat}.3=  it=  I.sell  to  a  student\\
      \glt `I’ll sell the car to a student (= not known to me, any gullible student I can find).'

\ex
	\gll La  machina,  nci  la  vindu  di  nu  studenti.\\
      the  car  \textsc{dat}.3=  it=  I.sell  of  a  student\\
      \glt `I’m selling a student the car (= specific student known to me).'
      \z
      \z

Integrating these observations with the results of the investigation of indirect object marking across Greek dialects carried out by \citet{ManolessouBeis2004} (cf. also \citealt[160]{Joseph1990}; \citealt[125-126]{Horrocks1997}; \citealt[628-629]{Horrocks2007}; \citealt[140-141]{Ralli2006}), \citet[194-195]{Ledgeway2013} proposes a partial parameter hierarchy based on the marking of indirect objects (IOs) along the lines of \REF{ex:ledgeway:9} with representative examples in (\ref{ex:ledgeway:10}a-d), ultimately to be understood as part of a larger hierarchy related to argument marking and alignments (cf. \citealt{Sheehan2014}).\todo{check tree}

\ea\label{ex:ledgeway:9}
\small
  \begin{forest}
 [{Are all internal arguments Case-marked accusative?}
    [Yes
        [{nth. Gk dialects,\\ Asia Minor,\\ Tsak.,\\ Dodec.  (\ref{ex:ledgeway:10}a)}, text width=2cm, align=left]
    ]
    [No
        [{Are all IOs\\ Case-marked dative?}, text width=3cm
            [Yes
                [{AG, Sal. (\ref{ex:ledgeway:10}b)}, text width=2cm]
            ]
            [No
                [{Are all IOs\\ Case-marked genitive?}, text width=3cm
                    [Yes
                        [{SMG, sth. dialects,\\ Italo-Gk (\ref{ex:ledgeway:10}c)}, text width=3cm]
                    ]
                    [No
                        [{Are a subset of IOs\\ Case-marked genitive\\ (= hybrid Case)?}, text width=3cm
                            [Yes
                                [{Calabrese (\ref{ex:ledgeway:10}d)\\{}[+presup. ${\Rightarrow}$ dative-genitive]}, text width=5cm]
                            ]
                        ]
                    ]
                ]
            ]
        ]        
    ]
 ]
 \end{forest}
 \z

 \newpage 
\ea\label{ex:ledgeway:10}

  \ea  Tsakonian (\citealt{ManolessouBeis2004})\\
\gll      επ\'{ε}τσε  \textbf{τoν} \textbf{óνε} \\
      he.said  the.\textsc{acc}  donkey.\textsc{acc} \\
      \glt `he said to the donkey.'
      
\ex Ancient Greek (Xenophon, \textit{Anabasis} 3.1.7)  \\
\gll  λ\'{ε}γει τὴν μαντείαν \textbf{τῷ} \textbf{Σωκράτει}.\\
      he.says  the.\textsc{acc}  oracle.\textsc{acc}  the.\textsc{dat}  Socrates.\textsc{dat}\\
      \glt `I am a servant to the gods.'

\ex Martano, Griko\\
      \gll Ce  t’  adrèffiatu  \textbf{tù} ’pane.\\
      and  the  brothers=his  him.\textsc{gen}  said\\
      \glt `And his brothers said to him.'

\ex Africo\\
      \gll Nci  dissi  \textbf{ô} figghiòlu  ’i  ccatta  u  latti.\\
      \textsc{dat}.3=  I.said  to.the  boy  that  he.buys  the  milk\\
      \glt `I told the boy to buy the milk.'
      \z
      \z

The first option in \REF{ex:ledgeway:9} represents the least marked question that we can ask about the marking of indirect objects, namely whether they are formally distinguished at all from other internal arguments (cf. also the contribution by Manzini, this volume). The negative reply to this question thus isolates a group of northern Greek dialects, Asia Minor dialects, Tsakonian and Dodecanese which, in contrast to all other Greek varieties, fail to mark a formal distinction between direct and indirect objects, witness the accusative-marking of the \textsc{recipient} in (\ref{ex:ledgeway:10}a). We are thus dealing with a case of mesoparametric variation, in that in these varieties accusative, arguably the core object Case crosslinguistically and licensed by \textit{v}, hence situated at the top of our hierarchy, indiscriminately marks all DP objects, a naturally definable class (namely, [-\textsc{nom}] Ds). The next option is that exhibited by varieties such as ancient Greek and Salentino which, by contrast, unambiguously distinguish indirect objects by marking them dative (10b; cf. also \REF{ex:ledgeway:3} above), in contrast to varieties such as standard modern Greek, southern Greek dialects and Italo-Greek which are situated further down the hierarchy in that they conflate this category with the genitive (\ref{ex:ledgeway:10}c). The greater and increasing markedness of these latter two options follows from the observation that crosslinguistically dative, generally taken to be licensed by an Appl(icative) functional head (see, for example, Cuervo, this volume; for an opposing view, see however Manzini, this volume), represents the least marked distinctive Case for indirect objects, whereas genitive, at least in those languages with rich case systems, typically displays all the hallmarks of an inherent Case whose distribution is largely defined by not entirely predictable lexical factors, hence taken here to be assigned by a lexical V head. These two options reflect, respectively, micro- and nanoparametric variation. In the former case dative serves to uniquely mark a small, lexically definable subclass of functional heads, namely all Ds bearing the \textsc{Recipient} feature (for arguments in favour of treating theta roles as formal features, see \citealt{Hornstein1999}). In the latter case, by contrast, genitive is associated with a class of predicates whose membership can only be established on purely lexical grounds, inasmuch as the \textsc{Recipient} feature is just one of many semantic roles associated with genitive marking. 

The final option in \REF{ex:ledgeway:9} is represented by the \textit{dativo greco} in Calabrese (\ref{ex:ledgeway:10}d), clearly the most marked option of all, insofar as the marking of \textsc{Recipient} arguments in this variety is strictly context-sensitive, with the \textit{dativo greco} serving to narrowly delimit individual \textsc{Recipient} arguments in accordance with their [±presuppositional] reading. This more complex and non-uniform behaviour is further reflected in the surface form of the so-called \textit{dativo greco} which, we have observed, involves a composite Case structure combining dative clitic marking on the verbal head with genitive prepositional marking on the nominal dependent, presumably reflecting the simultaneous intervention of Appl\textsc{\textsubscript{dat}} and V\textsc{\textsubscript{gen}} heads in the licensing of such indirect objects. These facts highlight how convergence through grammars in contact does not necessarily lead to simple borrowing, but frequently yields new hybrid structures born of reanalysis. Below we shall explore the syntax of this instantiation of the Greek-style dative in greater detail to ascertain its significance for theoretical issues about argument structure and especially the mapping between morphological marking and syntactic configurations.
 

 
\subsection{Case study 2: Calabrese\textsubscript{2}}

The second variety of Calabrese identified through our fieldwork that we must consider, henceforth Calabrese\textsubscript{2}, is found in the villages of Gioiosa Ionica and San Luca. In contrast to Calabrese\textsubscript{1}, the Greek-style dative in Calabrese\textsubscript{2} displays a much more restricted distribution subject to lexico-structural factors. In particular, the Greek-style dative in this variety only surfaces when the \textsc{Recipient} argument is introduced by a definite article (\ref{ex:ledgeway:11}a), with the typically Romance prepositional marker \textit{a} ‘to’ surfacing in all other contexts, witness (\ref{ex:ledgeway:11}b) where the \textsc{Recipient} is headed by the indefinite article.\footnote{The
    variety of San Luca dialect investigated by \citet{Chilà2017} – henceforth San Luca2 – appears to represent a more conservative variety in which all dative arguments are marked by \textit{di} ‘of’, and not just those introduced by the definite article as shown by the examples in (i.a-b):
    
    \ea
    San Luca2 \\
    \ea
    \gll Telefonanzi  ’i  zzìuta!   \\
        telephone.\textsc{imp.2sg=dat.3}  of  uncle=your  \\
        \glt `Ring your uncle!' 
    \ex  ’A  torta  si  piacìu  ’i  tutti.\\
        the  cake  \textsc{dat}.3  pleased  of  all  \\
        \glt `Everyone liked the cake.'
        \z
        \z        
        Although \citet[4-5]{Chilà2017} argues that presuppositionality – or, in her terms, the feature [±known] – plays no role in the licensing of the Greek-style dative in San Luca2, all her examples involve specific and definite referents, including those such as (ii.a-c) which she claims are [–known] but which are clearly presupposed (note that Chilà does not provide any examples with nominals introduced by the indefinite article).  
    
    \ea
    San Luca2 \\
    \ea 
    \gll Si  dissi  d  ’u  postinu  ’u  si  ndi  vai.\\
    \textsc{dat}.3  I.said  of  the  postman  that  self=  therefrom=  he.goes\\
    \glt `I told the postman to leave.'  
    
    \ex
    \gll Si  fici  ’na  telefunata  d  ’u  funtaneri.\\
    \textsc{dat}.3  I/made  a  telephone.call  of  the  plumber \\
    \glt `I gave the plumber a call.' 
    
    \ex
    \gll Si  telefonai  d  ’a  putìca /  ’u  bonchettu.\\
    \textsc{dat}.3  I.telephoned  of  the  shop  the  restaurant\\
    \glt `I rang the shop / the restaurant.'
    \z
    \z
    Pending further investigation, it might then be that San Luca2 is not necessarily the most conservative Calabrian variety replicating the generalized distribution of the Greek-style genitive of Grecanico, but, rather, represents another variety to be included among those grouped under the label of Calabrese1.
}

\ea\label{ex:ledgeway:11}
  Gioiosa Ionica\\
\ea
	\gll Nci  detti  nu  libbru  d-u  figghjiolu.\\
        \textsc{dat}.3=  I.gave   a  book  of-the   kid  \\
        \glt `I gave a book to the kid.'
\ex
	\gll Nci  detti  nu  libbru  a  nu  figghjiolu.\\
        \textsc{dat}.3=  I.gave   a  book  to  a   kid  \\
        \glt `I gave a book to a kid.'
        \z
        \z

This contrast can be seen even more clearly through a comparison of the dialects of Gioiosa Ionica and San Luca in relation to the behaviour of proper names. As in many Romance varieties (cf. \citealt[103-104]{Ledgeway2012}; \citealt[111-112]{Ledgeway2015}), proper names do not co-occur with a definite article in the dialect of Gioiosa Ionica, whereas in the dialect of San Luca proper names are introduced by an expletive definite article just as in Greek (\citealt[198]{Mackridge1985}; \citealt[276-278]{HoltonMackridgePhilippaki-Warburton1997}; \citealt[208-209]{Ledgeway2013}). As a consequence, whenever a \textsc{Recipient} is lexicalized by a proper name it is marked by \textit{a} ‘to’ in Gioiosa Ionica (\ref{ex:ledgeway:12}a), but by \textit{di} ‘of’ in San Luca since the presence of the definite article in this variety automatically triggers and licenses the use of the Greek-style dative (\ref{ex:ledgeway:12}b).

\ea\label{ex:ledgeway:12}

  \ea  Gioiosa Ionica\\
 \gll Nci      detti   nu   libbru    a   Maria.\\
               \textsc{dat}.3=  I.gave  a  book  to   Maria\\
               \glt `I gave a book to Maria.'
               
\ex San Luca\\
 \gll Stamatina        si          detti  nu  pocu   i   pani   d-u   Petru.\\
        this.morning   \textsc{dat}.3=  I.gave   a  little   of   bread   of-the   Petru\\
        \glt `This morning I gave a bit of bread to Petru.'
        \z
        \z

To sum up, we note then that in Calabrese\textsubscript{2} dative is marked by \textit{a} ‘to’, and not by the Greek-style dative with \textit{di} ‘of’, whenever the \textsc{Recipient} surfaces as: (a) a proper name (12a; but cf. 12b), singular kinship term (\ref{ex:ledgeway:13}a) or tonic pronoun (\ref{ex:ledgeway:13}b); (b) an indefinite DP (\ref{ex:ledgeway:13}c); (c) a nominal introduced by a demonstrative (\ref{ex:ledgeway:13}d) or a bare quantifier (\ref{ex:ledgeway:13}e). In structural terms, what all three contexts have in common is that the D position is either not available to the definite article, since this position is already directly lexicalized by the nominal (e.g. pronoun) or through N-to-D movement (e.g. proper name, kinship term), or the D position is simply not lexicalized, as happens with indefinite DPs, where the cardinal lexicalizes the head of a lower functional projection (variously termed CardP/NumP), and with demonstratives and bare quantifiers where the DP is embedded within a DemP and a QP, respectively. 

\ea\label{ex:ledgeway:13}
  Gioiosa Ionica\\
\ea
	\gll Non  nci  telefonari  a  ziuma!\\
      not  \textsc{dat}.3=  phone.INF   to  uncle=my\\
      \glt `Do not phone my uncle!'

\ex
	\gll Maria  m’  u  detti  a  mia.\\
    Maria  me=  it=  gave.3SG   to  me\\
    \glt `Maria gave it to me.'

\ex
	\gll Ajeri  nci  telefonau  a  nu   previte.\\
      yesterday  \textsc{dat}.3=  I.phoned  to  a  priest \\
      \glt `Yesterday I phoned a priest.'

\ex
	\gll Ajeri  nci   telefonava  a  iju    previte   i   Messina.\\
      yesterday  \textsc{dat}.3=  I.phoned  to  that   priest   of   Messina\\
      \glt `Yesterday I phoned that priest from Messina.'

\ex  
    \gll Non  telefonari  a  nuju!\\
    not  phone.INF   to  nobody\\
    \glt `Don’t phone anybody!'
\z
\z

\subsection{Interim conclusions and questions}

We have seen that the Romance dialects of Calabria have been in centuries-old contact with Grecanico as the sub- and adstrate contact language. As a consequence, Calabrese has adopted and, in turn, adapted a number of original Greek structural traits, including the Greek-style genitive-dative syncretism which has led to a certain degree of competition in the marking of \textsc{Recipient} arguments which may variously surface in conjunction either with \textit{di} ‘of’ or \textit{a} ‘to’ in accordance with the competing Greek and Romance patterns, respectively. In particular, in Calabrese\textsubscript{1} the Greek-style dative is pragmatically restricted, in that it has been shown to be intimately linked to presuppositionality, marking just those \textsc{Recipients} which are interpreted as being highly individuated and specific. By contrast, in Calabrese\textsubscript{2} the Greek-style dative is structurally restricted, in that its distribution has been shown to be strictly linked to the availability of the definite article and, by implication, the lexicalization or otherwise of the D position, with the Greek-style dative occurring just in those contexts in which the D position is realized by the definite article.

Against these considerations, we must consider a number of related questions. First, are the distributions of the Greek-style dative witnessed in Calabrese\textsubscript{1} and Calabrese\textsubscript{2} related, or should they be seen as separate developments arising from the reanalysis of the original underlying Greek pattern? Second, if they are related, as we shall argue below, how then does one develop from the other and, what is their diachronic relationship? Third, we have superficially observed how in both varieties of Calabrese the Greek-style dative (\textit{di}) variously alternates with a Romance-style dative (\textit{a}), but it remains to be understood how this alternation is to be interpreted in structural terms. Finally, we must also ask what these competing structures tell us about the structural positions in which dative DPs are licensed, and about the locus of dative-marking in DPs.

\section{Calabrese\textsubscript{1} revisited}

With these considerations in mind, we now return to Calabrese\textsubscript{1}. The basic facts which need to be accounted for include why: (i) the use of the Greek-style dative gives rise to a presuppositional reading of the \textsc{Recipient}; (ii) the DP has to be clitic-doubled; (iii) the doubling clitic has to be marked dative, rather than genitive; (iv) there is an apparent Case mismatch between the dative-marked clitic on the verbal head and the genitive-marked DP dependent, giving rise to an apparently hybrid Case structure; and (v) canonical datives are marked with (\textsc{ad} >) \textit{a} ‘to’. Superficially, then, one might be tempted to propose a double object analysis for the Calabrese\textsubscript{1} facts,\footnote{Cf. a.o. \citet{BarssLasnik1986,Larson1988, Larson1990,Jackendoff1990larson,CollinsThráinsson1993,Marantz1993,Demonte1995,Pesetsky1995,Collins1997,Torrego1998,Harley2002,Pyllkänen2008,Anagnostopoulou2003,Cuervo2003,Jeong2007,Bruening2010DOC, Bruening2010Ditrans,OrmazabalRomero2010,HarleyJung2015,Pineda2016}..} since, on a par with the double object construction reported in many languages, the \textsc{Recipient} necessarily receives a presuppositional reading, is animate, and is clitic-doubled (for futher discussion, see also Cuervo this volume). Furthermore, double object constructions have previously been independently reported for other dialects of southern Italy (cf. \citealt[ch.2]{Ledgeway2000}; \citealt[844-847]{Ledgeway2009}), witness the representative examples in (\ref{ex:ledgeway:14}a-d). 

\ea\label{ex:ledgeway:14}
\ea Naples\\

\gll ’A  purtaie  a  Maria  o  rialo  (*a  Maria).\\
    her.\textsc{acc}=  I.brought  \textsc{dom}  Maria  the.\textsc{m}  gift.\textsc{m}    \textsc{dom}  Maria\\
    \glt `I took Maria the present.'

\ex Curti, Caserta\\
    \gll ’O  facettero  n’  ata  paliata.\\
    him.\textsc{acc}=  they.did  an  other.\textsc{f}  thrashing.\textsc{f}\\
    \glt `They gave him another thrashing.'

\ex Calvello, Potenza\\
        \gll la  {}'rakǝ  nu  ka'vaddǝ.\\
    her.\textsc{acc}=  I.give  a.\textsc{m}  horse.\textsc{m}\\
    \glt `I’ll give her a horse.'

\ex Mattinata, Foggia\\
    \gll lu  tur'ʧi  lu  {}'kuǝddǝ.\\
    him.\textsc{acc}=  he.wrung  the.\textsc{m}  neck.\textsc{m}\\
    \glt `He wrung its neck.'
    \z
    \z

In the Neapolitan example (\ref{ex:ledgeway:14}a), for instance, the \textsc{Recipient} argument \textit{a Maria} has been `shifted' such that it obligatorily surfaces, as in the corresponding English sentence, to the left of the \textsc{Theme} argument marked by the prepositional accusative \textit{a} ({\textless} \textsc{ad}) and doubled by the accusative clitic \textit{’a}. Similarly, in examples (\ref{ex:ledgeway:14}b-d) the \textsc{Recipient} surfaces as a pronominal clitic, but is marked accusative, not dative (for further discussion, see also the chapter by Manzini, this volume) .

Although the parallels between the Greek-style dative in Calabrese\textsubscript{1} and the double object construction initially appear quite compelling, a closer look at the relevant facts reveals a number of problems with such an analysis. First, the \textsc{Recipient} in the Greek-style dative is not, at least superficially, `shifted' to a position in front of the \textsc{Theme} (cf. 14a), although this does not necessarily appear to be a precondition for the \textsc{Recipient} in the double object construction, witness, for example, the position of the \textsc{Recipient} in the Spanish construction (\citealt{Demonte1995}; see also the discussions in the chapters by Cuervo, by Calindro, and by Cépeda \& Cyrino, this volume). Second, there is no requirement that the subject in a Greek-style dative construction be interpreted as a causer (cf. 13a-d), a reading which is standardly taken to be characteristic of the subject in the double object construction. Third, an analysis in terms of a double object construction fails to offer any explanation for the apparent mismatch between the dative and genitive Case-marking borne by the clitic and coreferent DP, respectively. Fourth, unlike what happens in the double object construction (cf. \citealt{BarssLasnik1986,Larson1988}), where the asymmetrical binding of the dative-marked \textsc{Recipient} by the accusative-marked \textsc{Theme} in the prepositional dative construction (cf. 15a) is reversed allowing the accusative-marked \textsc{Recipient} to bind into the \textsc{Theme} (cf. 15b), the use of the Greek-style dative does not engender a reversal in the asymmetrical c-command relations between the \textsc{Theme} and \textsc{Recipient} (cf. 16a-b; see Cornilescu, this volume, for discussion of the binding facts in Romanian ditransitives).

\ea\label{ex:ledgeway:15}
\ea I sent every book to its author.  
\ex I sent every author his book.
\z
\z

\ea\label{ex:ledgeway:16}
  Africo\\
\ea
	\gll A  sarta  (nci)  mandau  ogni  vesta  â  so patruna.\\
      the  dressmaker  \textsc{dat}.3  sent  each  dress  to.the  its  owner\\
\ex
	\gll A  sarta  nci  mandau  ogni  vesta  d-a  so patruna.\\
      the  dressmaker  \textsc{dat}.3  sent  each  dress  of-the  its     owner\\
      \glt `The dressmaker sent each dress to its owner.'
      \z
      \z

Finally, a very clear piece of evidence that the Greek-style dative in Calabrese\textsubscript{1} is not amenable to a double object analysis comes from the observation that the Greek-style dative is not limited to ditransitive clauses, but is also found with monotransitives (cf. 17a-b) that otherwise canonically select for dative arguments.

\ea\label{ex:ledgeway:17}
\ea Natile di Careri\\
 \gll Non  si  gridari  d-u  figghiolu!\\
    not  \textsc{dat}.3=  shout.\textsc{inf}  of-the  son\\
     \glt `Don’t shout at the child!'

\ex Palizzi  \\
    \gll Nci  parrai / scrivia / telefunai d-u sindacu.\\
    \textsc{dat}.3= I.spoke / I.wrote / I.phoned of-the mayor\\
    \glt `I spoke to / wrote to / rang the mayor.'
    \z
    \z

In what follows, we thus exclude the possibility of a double object analysis for the Greek-style dative in Calabrese\textsubscript{1}. Instead we adopt the view here that, on a par with other Romance varieties (though not Romanian), dative is canonically marked in Calabrese\textsubscript{1} with the preposition \textit{a} ‘to’, giving rise to a structure like that in (\ref{ex:ledgeway:18}a) and exemplified in (\ref{ex:ledgeway:19}a). The \textsc{Recipient} DP thus constitutes a core argument which in Calabrese\textsubscript{1} is very frequently, though not obligatorily, doubled by a dative clitic. By contrast, we analyse the Greek-style dative exemplified in (\ref{ex:ledgeway:19}b) along the lines of (\ref{ex:ledgeway:18}b), where we take dative once again to be assigned to a core argument, here instantiated by pro and obligatorily referenced by a dative clitic on the verb. Consequently, we interpret the DP introduced by \textit{di} ‘of’ to be an adjunct, albeit coreferential with the clitic-pro argument chain.\footnote{Observe that this analysis comes very close to, and indeed is compatible with, the idea in many analyses of the double object construction that the \textsc{Recipient} argument is not a core argument but, rather, is an adjunct licensed by an Appl head.} 

\ea\label{ex:ledgeway:18}
\ea  (Cl\textsubscript{i})…T-V…(DP\textsc{\textsubscript{acc}}\textsc{)}…[a  DP\textsc{\textsubscript{dat}}]\textsubscript{i}
\ex *(Cl\textsubscript{i})…T-V…(DP\textsc{\textsubscript{acc}}\textsc{)}…[pro\textsc{\textsubscript{dat}}]\textsubscript{i}, [di DP]\textsubscript{i}
\z
\z

\ea\label{ex:ledgeway:19}
\ea
	\gll (Nci)  la  vindu  a  nu  studenti.\\
    \textsc{dat}.3=  it=  I.sell  to  a  student\\
    \glt `I’ll sell it to a student.'

\ex  *(Nci\textsubscript{i}) la vindu pro\textsubscript{i} di  nu  studenti.\\
      \textsc{dat}.3= it= I.sell { } of a student\\
    \glt `I’ll sell a student it.' (lit. `I’ll sell it to him, a student.')
    \z
    \z

Under this analysis we can now capture the principal characteristics of the Greek-style dative. First, the obligatory presuppositional reading of the \textsc{Recipient} argument follows immediately from the fact that the dative argument is instantiated by a pro licensed and referenced by a dative clitic, inasmuch as clitic-pro chains invariably yield presuppositional readings of their pronominal referents which are interpreted as known, specific and highly salient in the discourse. This is not the case in the canonical Romance-style dative construction (18a, 19a), where the dative argument is realized by a lexical DP and hence not pragmatically restricted.

Second, we now have a straightforward explanation for the obligatory presence of the dative clitic in the so-called Greek-style dative construction, since the clitic is part of a clitic-pro argument chain and is therefore necessary to reference and license pro. Despite appearances, there is then no doubling as such involved, inasmuch as the clitic licenses pro rather than doubling the coreferential DP adjunct. 

Third, and by the same token, the observed Case mismatch between the clitic, marked dative, and the full DP, marked genitive, is only apparent, since dative Case is exhausted by the clitic-pro argument chain, whereas the coreferential DP represents an adjunct licensed by the canonical marker of obliques/non-arguments, namely the genitive preposition \textit{di} ‘of’. 

Fourth, the rightmost position of the DP in examples such as (\ref{ex:ledgeway:19}b) now follows without further stipulation, since the DP is an adjunct and hence occurs in extra-sentential positions (whether to the right or to the left) outside of the sentential core, thereby also excluding any form of `dative shift'. Indeed, when the \textsc{Recipient} is marked by \textit{a} ‘to’ it can bind into the \textsc{Theme} in examples such as (\ref{ex:ledgeway:20}a), where the latter presumably involves a case of marginalization occupying its in situ position within the \textit{v}{}-VP, witness the absence of a resumptive accusative clitic, and the \textsc{Recipient} has been raised to a focus position within the lower left periphery crossing the \textsc{Theme} (\citealt{Frascarelli2000s,Cardinaletti2002}; \citealt[42-47]{Cruschina2012}). However, when the \textsc{Recipient} occurs in the so-called Greek-style dative (\ref{ex:ledgeway:20}b), such binding is not possible. Given our (topical) adjunct interpretation of DPs marked by the Greek-style dative, the ungrammaticality of (\ref{ex:ledgeway:20}b) is fully expected since the \textsc{Recipient} is merged in an extra-sentential right-peripheral position from where it cannot precede the \textsc{Theme} in its in situ position within the \textit{v}{}-VP.

\ea\label{ex:ledgeway:20}
  Africo\\
\ea
	\gll A  sarta  nci  mandau  a  ogni  patruna  a  so  vesta.\\
    the  dressmaker  \textsc{dat}.3=  sent  to  each  owner  the  her  dress\\

\ex
    \gll *A  sarta  nci  mandau  di  ogni  patruna  a  so  vesta.\\
    the  dressmaker  \textsc{dat}.3=  sent  of  each  owner  the  her  dress\\
    \glt `The dressmaker sent each owner her dress.'
\z
\z

  Fifth, the stability of the binding facts observed in \REF{ex:ledgeway:16} now follows straightforwardly since, even when the Greek-style dative is employed (cf. 16b), the \textsc{Recipient} is still realized by a core DP argument (viz. pro) Case-marked dative and licensed in the same argument position as a lexical DP in the so-called Romance-style dative (cf. 16a) from where it can be bound by the c-commanding \textsc{Theme} argument. The presence or otherwise of a coreferential topic adjunct introduced by \textit{di} ‘of’ therefore proves irrelevant to the basic binding facts, which are invariably determined within the sentential core by the two internal arguments whose licensing positions, and hence also their binding relations, remain unchanged. However, one respect in which the two sentences in (\ref{ex:ledgeway:16}a-b) differ concerns the availability of the individual and distributive scopal readings of \textit{patruna}. Whereas both readings of \textit{patruna} are available in (\ref{ex:ledgeway:16}a) where both scope relations can be reconstructed within the \textit{v}{}-VP between the QP \textit{ogni} and the possessive anaphor \textit{so}, only the individual reading is possible in (\ref{ex:ledgeway:16}b) in accordance with the characteristic presuppositional reading of the so-called Greek-style dative noted above. The absence of this distributive reading in (\ref{ex:ledgeway:16}b) highlights how the adjunct \textit{d-a so patruna} takes scope over the \textsc{Theme} \textit{ogni vesta}, but not vice versa, providing further proof for the fact that right-peripheral (familiar) topics like \textit{d-a so patruna} are merged in extra-sentential positions from where quantifiers like \textit{ogni} ‘each’ cannot scope over them at LF (cf. \citealt{Cardinaletti2002, Frascarelli2004, FrascarelliHinterhölzl2007}).\footnote{In fact, in (\ref{ex:ledgeway:16}a) when the dative clitic is absent both the individual and distributive readings are possible, although the distributive interpretation is strongly preferred, whereas only the individual reading is possible when the clitic is present. Thus, just as in (\ref{ex:ledgeway:16}b), it would appear that the presence of the clitic in (\ref{ex:ledgeway:16}a) forces a right-dislocated topical interpretation of the \textsc{Recipient} DP which takes scope over the \textsc{Theme} licensing the individual reading.}

  Finally, the analysis outlined in (\ref{ex:ledgeway:18}b) correctly predicts that the distribution of the Greek-style dative should equally occur in monotransitives as in ditransitives, inasmuch as its distribution is not linked to the presence of a \textsc{Theme} argument. Furthermore, the use of the Greek-style dative with monotransitives also highlights the weakness of functionalist accounts which take the obligatory use of the dative clitic as a means of distinguishing between the dative-\textsc{Recipient} and genitive-\textsc{Possessor} readings of the lexical DP in examples such as (\ref{ex:ledgeway:21}a). However, the evidence of monotransitives such as (\ref{ex:ledgeway:21}b), where there is no ambiguity regarding the dative-\textsc{Recipient} interpretation of the lexical DP but the dative clitic continues to be obligatory, excludes any such functionalist interpretation of the facts.

\ea\label{ex:ledgeway:21}
\ea  Bova\\
\gll Nci  vindu  la  machina  di  lu  studenti.\\
    \textsc{dat}.3=  I.sell  the  car  of  the  student\\
    \glt `I’ll sell the student the car.' (*'I’ll sell the student’s car.')

\ex Natile di Careri\\
    \gll Non  *(si)  gridamu  d-i  nostri  figghioli!\\
    not  \textsc{dat}.3=  we.shout  of-the  our  children\\
    \glt `Let’s not shout at our children!'
    \z
    \z

  To conclude, we have established that in Calabrese\textsubscript{1} the dative is invariably marked by \textit{a} ‘to’ as in most other varieties of Romance. By contrast, the use of \textit{di} ‘of’ in the so-called Greek-style dative has been shown to mark right-peripheral adjuncts, with the dative-marked \textsc{Recipient} still licensed as a core argument (pro) in association with a coreferential dative clitic (cf. pronominal argument hypothesis developed in \citealt{Jelinek1984}). It thus appears that Greek-Romance contact in the case of Calabrese\textsubscript{1} has given rise to an imperfect replication of the corresponding Grecanico genitive-marked \textsc{Recipient} structure. In particular, in Calabrese\textsubscript{1} dative is canonically marked by \textit{a} ‘to’, with Greek-style marking of \textsc{Recipients} by means of \textit{di} ‘of’ having been reanalysed as a marked structure pressed into service as a last resort option to Case-mark adjunct DPs whenever dative Case has been otherwise exhausted within the sentential core. Indeed, as argued in \citet{Ledgeway2013}, such exaptive outcomes are far from infrequent in the Greek-Romance contact situation of southern Italy where contact-induced borrowing typically does not replicate the original structure of the lending language but, rather, produces hybrid structures which are ultimately neither Greek nor Romance in nature.

\section{Calabrese\textsubscript{2} revisited}

We now return to Calabrese\textsubscript{2} where the facts to be accounted for include: (i) why the Greek-style dative only occurs in conjunction with the definite article (cf. 22a); (ii) why the dative is marked by \textit{a} ‘to’ (cf. 22b) if the definite article is absent; and (iii) what the relationship, if any, is between the distribution of the Greek-style dative in Calabrese\textsubscript{2} and Calabrese\textsubscript{1} (cf. 19a-b). 

\ea\label{ex:ledgeway:22}
  Gioiosa Ionica\\
\ea  \gll Ajeri  nci  telefonau  d-u  previte.\\
    yesterday  \textsc{dat}.3=  I.phoned  of-the  priest\\
    \glt `Yesterday I phoned the priest.'

\ex
	\gll Ajeri  nci  telefonau  a   nu   previte.\\
      yesterday  \textsc{dat}.3=  I.phoned   to   a  priest \\
      \glt `Yesterday I phoned a priest.'
      \z
      \z

  We begin with the last question regarding the diachronic relationship between the distribution of the Greek-style dative in Calabrese\textsubscript{1} and Calabrese\textsubscript{2}, which, we will see, also provides an answer to our first question regarding the restriction of the Greek-style dative to nominals introduced by the definite article. In particular, we argue that the use of the Greek-style dative in Calabrese\textsc{\textsubscript{2}} represents a development from the more conservative distribution observed in Calabrese\textsubscript{1} where it was seen to license a presuppositional reading of the DP adjunct. In such cases, the DP is typically headed by the definite article, the archetypal marker of presuppositionality, thereby creating a strong association between the definite article and the Greek-style genitive. It is therefore entirely plausible to suppose that this frequent pairing of the definite article with the Greek-style dative under the presuppositional reading eventually led in Calabrese\textsubscript{2} to a distributional reanalysis of the Greek-style dative which came to be restricted to the definite article. We thus also have a highly natural explanation for our first question regarding the distributional restriction of the Greek-style dative to nominals headed by the definite article. 

  Further proof for this diachronic development comes from the observation that while most speakers of Calabrese\textsubscript{2} today restrict the Greek-style genitive to definite DPs introduced by the definite article, some speakers of the dialect of San Luca (but not the dialect of Gioiosa Ionica) are less restrictive in that they optionally extend the use of the Greek-style genitive to definite DPs situated higher up the animacy/definiteness hierarchy (\citealt{Silverstein1976,Aissen2003}) to also include, for example, kinship terms \REF{ex:ledgeway:23}.

\ea\label{ex:ledgeway:23}
  San Luca\\
\gll Aieri  u  Petru  si  talefunau  ’i  /  a  fratima.\\
  yesterday  the  Peter  3.\textsc{dat}=  telephoned  / of to brother=my\\
    \glt `Yesterday Pietro rang my brother.'
    \z 

In this respect, it is not coincidental that San Luca is also the variety that employs the definite article with proper names, hence also systematically marked by the Greek-style dative (cf. 12b) and therefore extending its distribution higher up the animacy/definiteness hierarchy. Evidence like this highlights how the pragmatico-semantic category of presuppositionality has been subject to formal reinterpretation and reanalysis in the passage from Calabrese\textsubscript{1} to Calabrese\textsubscript{2}, such that today the distribution of the Greek-style dative variously maps onto different subgroupings of nominals characterized by differing degrees of animacy and definiteness, but ultimately all interpreted in some sense as presupposed.

Unlike in Calabrese\textsubscript{1} where \textit{di-}marked DPs were shown to be adjuncts that occur in extra-sentential positions (\ref{ex:ledgeway:24}a), in Calabrese\textsubscript{2} \textsc{Recipient} DPs introduced by \textit{di} ‘of’ therefore represent genuine dative arguments integrated and licensed within the sentential core, as further witnessed by the optionality of the doubling dative clitic \textit{nci} on the verb in (\ref{ex:ledgeway:24}b), although it proves extremely common.

\ea\label{ex:ledgeway:24}
\ea  Calabrese\textsubscript{1}\\
	\gll (Nci\textsubscript{i}) lu dissi pro\textsubscript{i}, di lu párracu.\\
       \textsc{dat}.3= it= I.told { } of the priest\\

\ex Calabrese\textsubscript{2}\\
    \gll (Nci)  u  dissi  d-u  previte.\\
    \textsc{dat}.3=  it=  I.told  of-the  priest\\
    \glt `I told the priest.'
    \z
    \z

We turn finally to consider the formal alternation between \textit{a} ‘to’ and \textit{di} ‘of’ in the marking of \textsc{Recipient} arguments in Calabrese\textsubscript{2}. Above we noted that \textit{a} ‘to’ surfaces whenever Dº is lexicalised by a pronominal D (\ref{ex:ledgeway:25}a) or a raised N (\ref{ex:ledgeway:25}b) and whenever Dº is not lexicalised (\ref{ex:ledgeway:25}c-e).

\ea\label{ex:ledgeway:25}
  Gioiosa Ionica\\
\ea
	\gll Maria  m’  u  detti  a  mia.\\
    Maria  to.me=  it=  gave.3SG   to  me\\
    \glt `Maria gave it to me.'

\ex
	\gll Non  nci  telefonari  a  ziuma!\\
      not  \textsc{dat}.3=  phone.INF   to  uncle=my\\
      \glt `Do not phone my uncle!'

\ex
	\gll Ajeri  nci  telefonau  a   nu   previte.\\
      yesterday  \textsc{dat}.3=  I.phoned    to   a    priest \\
      \glt `Yesterday I phoned a priest.'

\ex
	\gll Ajeri  nci   telefonava  a  iju    previte   i   Messina.\\
      yesterday  \textsc{dat}.3=  I.phoned  to  that   priest   of   Messina\\
      \glt `Yesterday I phoned that priest from Messina.'

\ex
	\gll Non  telefonari  a  nuju!\\
    not  phone.INF   to  nobody\\
    \glt `Don’t phone anybody!'
    \z
    \z

Consequently, we concluded that \textit{di} ‘of’ surfaces uniquely in conjunction with nominals introduced by the definite article (\ref{ex:ledgeway:26}a). While this descriptive generalization captures the core distributional facts of the Greek-style dative in Calabrese\textsubscript{2}, it is not entirely correct and needs to be revised in the light of evidence such as (\ref{ex:ledgeway:26}b).

\ea\label{ex:ledgeway:26}
  Gioiosa Ionica\\
\ea
	\gll Ajeri  nci  telefonau  d-i     cuggini   mei.\\
      yesterday  \textsc{dat}.3=  I.phoned  of-the  cousins   my\\
      \glt `Yesterday I phoned my cousins.'

\ex
	\gll Ajeri  nci  telefonau  a /*di  tutti   i     cuggini   mei.\\
      yesterday  \textsc{dat}.3=  I.phoned  to   of  all   the   cousins   my\\
      \glt `Yesterday I phoned all my cousins.'
      \z
      \z

Although example (\ref{ex:ledgeway:26}b) involves a nominal introduced by the definite article, just as in (\ref{ex:ledgeway:26}a), it is also preceded by the universal quantifier \textit{tutti} ‘all’ and dative is marked by the preposition \textit{a} ‘to’ rather than \textit{di} ‘of’. This seems to suggest that the correct descriptive generalization is that the Greek-style dative in Calabrese\textsubscript{2} only occurs in conjunction with the definite article (cf. 26a), but that it does not necessarily always occur whenever the definite article is employed (cf. 26b). Indeed, the contrast witnessed in (\ref{ex:ledgeway:26}a-b) highlights how morphosyntactic variation in dative-marking through the formal alternation between \textit{a} ‘to’ and \textit{di} ‘of’ crucially depends on whether K(ase) is realized in a scattered or syncretic fashion (cf. \citealt{GiorgiPianesi1997}). In particular, as illustrated structurally in (\ref{ex:ledgeway:27}a) and exemplified in (\ref{ex:ledgeway:28}a-c) we see that whenever lexical material intervenes between the K° and D° positions, whether the latter is lexicalized (cf. 28a) or not (cf. 28b-c), then these two positions are independently projected and the two heads are realized in a scattered fashion with the K° head lexicalized by \textit{a} ‘to’. When, however, the two heads are adjacent and the D° position is lexicalized, as in examples (\ref{ex:ledgeway:29}a-b), then a syncretic K/D head obtains in which both Case and definiteness are inextricably bound together and morphologically spelt out as a single head \textit{d-u/-a/-i} ‘of-the.\textsc{msg/fsg/pl}’.



\ea\label{ex:ledgeway:27} 
\ea  
\begin{forest}
[KP
    [{K\\\textit{a}\\to}, text width=1cm]
    [QP
        [{Q\\\textit{tutti}\\all}, text width=1cm]
        [DemP
            [{Dem\\\textit{ijji}\\those}, text width=1cm]
            [DP
                [{D\\\textit{(i)}\\the}, text width=1cm]
                [NumP
                    [{Num\\\textit{du}\\two}, text width=1cm]
                    [NP
                        [{\textit{cuggini (di) mei}\\cousins (of) my}, text width=2.5cm, roof]
                    ]
                ]
            ]
        ]
    ]
]
\end{forest}

\ex
\begin{forest}
[K/DP
    [{K+D\\\textit{d-i}\\of-the}, text width=1cm]
    [NumP
        [{Num\\\textit{du}\\two}, text width=1cm]
        [NP
            [{\textit{cuggini (di) mei}\\cousins (of) my}, text width=2.5cm, roof]
        ]
    ]
]
\end{forest}
\todo{node Num has no label}

\z
\z

\ea\label{ex:ledgeway:28}
  Gioiosa Ionica\\ 
  \ea
  \gll Ajeri  nci  telefonau  a  tutti  i  cuggini  mei.\\
    yesterday  \textsc{dat}.3=  I.phoned  to  all  the  cousins  my\\
    \glt `Yesterday I rang all my cousins.'

\ex
	\gll Ajeri  nci  telefonau  a  (tutti)  ijji  cuggini  mei.\\
    yesterday  \textsc{dat}.3=  I.phoned  to  (all)  those  cousins  my\\
    \glt `Yesterday I rang (all) those cousins of mine.'

\ex
	\gll Ajeri  nci  telefonau  a  du  cuggini  di  mei.\\
    yesterday  \textsc{dat}.3=  I.phoned  to  two  cousins  of.the  my\\
    \glt `Yesterday I rang two of my cousins.'
\z
\z

\ea\label{ex:ledgeway:29}
  Gioiosa Ionica\\
\ea
	\gll Ajeri  nci  telefonau  d-i  cuggini  mei.\\
    yesterday  \textsc{dat}.3=  I.phoned  of.the  cousins  my\\
    \glt `Yesterday I rang my cousins.'

\ex
	\gll Ajeri  nci  telefonau  d-i  du  cuggini  di  mei.\\
    yesterday  \textsc{dat}.3=  I.phoned  of.the  two  cousins  of.the  my\\
    \glt `Yesterday I rang my two cousins.'
    \z
    \z
    
In conclusion, we have seen that in Calabrese\textsubscript{2} the Romance-style dative \textit{a} represents the scattered spell-out of a single [Kase] feature in contrast to the Greek-style dative \textit{d-u/-a/-i} which instantiates the syncretic spell-out of a feature bundle [Kase, Definite]. As argued above, this latter development represents the outcome in Calabrese\textsubscript{2} of a progressive association of the definite article with the Greek-style genitive under its original presuppositional reading (as still preserved in Calabrese\textsubscript{1}), yielding the portemanteau morphs \textit{du/da/di}. In this respect, it is revealing to note that in conjunction with the genitive preposition \textit{di} ‘of’ we find in Calabrese\textsubscript{1}, where there is no necessary structural association of the definite article with the Greek-style dative, both full forms of the definite article preserving the initial lateral as well as aphaeresized forms, namely bisyllabic \textit{di lu/la/li} and monosyllabic \textit{du/da/di}. In Calabrese\textsubscript{2}, by contrast, the definite article forms a syncretic head with the genitive preposition and only the aphaeresized forms \textit{du/da/di} are found.

\section{Conclusion}

The detailed discussion of Grecanico and Calabrese argument marking above has shown how, at least on the surface, the grammars of the these two linguistic groups are in many respects very similar, to the extent that the observed structural parallels are far too striking for them to be dismissed as accidental but, rather, must be considered the result of centuries-old structural contact between Greek and Romance, ultimately to be placed towards the upper end of the five-point scale of contact intensity proposed by \citet{ThomasonKaufman1988}. The direction of such contact has consistently been shown to be unidirectional, involving the transfer and extension of original Greek structural features into the surrounding Romance varieties. At the same time, however, we have seen that a detailed examination of the Greek-style dative reveals how the finer details of such structural parallels often differ in subtle and unexpected ways once adopted in Romance: this highlights how speakers have not so much borrowed actual Greek forms but, rather, reshaped and reanalysed, often in a process of replication (\citealt{HeineKuteva2003, HeineKuteva2005}), already existing Romance categories (e.g. dative and genitive marking) to approximate the superficial Greek models and patterns. Indeed, data from argument marking highlight how the varieties in question marry together in still poorly explored and largely little understood ways facets of core Romance and Greek syntax to produce a number of innovative hybrid structures, the evidence of which can be profitably used to throw light on parametric variation and, in our particular case, on the nature and licensing of dative, as well as the proper formal characterization of convergence and divergence. 

In the case of Grecanico and Calabrese, which it must not be forgotten independently share a common Indo-European ancestry that is in large part responsible for their shared macro- and mesoparametric settings (e.g. head-initial, nominative-accusative alignment, pro-drop), observed Greek-biased convergence between the two can typically be reduced to a surface effect of shared microparametric settings. By way of illustration, consider once again the case of the Greek-style dative. Specifically, we saw that Calabrese patterns not with standard Romance varieties such as Italian, but, rather, with Grecanico in exhibiting varying degrees of syncretism in the marking of dative and genitive, the manifestation of which was argued to be ultimately understood as a case of microparametric variation in the marking of \textsc{Recipient} arguments (cf. \ref{ex:ledgeway:9}). On the other hand, the more subtle nature of divergence between Calabrese and Grecanico can be reduced to the surface effect of different settings in relation to hierarchically `deeper' microparametric options and, above all, in relation to nanoparametric differences. Returning again to the Greek-style dative, although Grecanico and Calabrese share the same parametric setting in relation to dative and genitive syncretism, we have seen how only distinct deeper microparametric settings can provide the key to understanding the more restricted distribution of the syncretism in Calabrese licensed by specific structural and pragmatic features associated with different functional heads (namely, K° and D°).

Finally, the preceding discussion has provided and reviewed significant evidence to demonstrate that ultimately the local Romance varieties of southern Calabria cannot be regarded as Greek disguised as Romance. Although such a view has traditionally enjoyed a great deal of acceptance since Rohlfs’ now classic slogan \textit{spirito greco, materia romanza}, it is based on rather superficial structural similarities deriving from retained macro- and mesoparametric settings and, above all, from shared `shallow' microparametric settings. However, as soon as one begins to peel back the layers, it soon becomes clear that convergence through grammars in contact does not necessarily lead to simple borrowing and transference through interference, but more frequently gives rise to new hybrid structures born of reanalysis of the original Greek structures within a Romance grammar instantiating `deeper' microparametric options. This observation goes against the general prediction (cf. \citealt{BiberauerRoberts2012hierarchy}) that, all things being equal, syntactic change should proceed `upwards' within parametric hierarchies as acquirers strip away features in their attempt to postulate the simplest featural analyses compatible with the PLD (\citealt{RobertsRoussou2003}). In the particular cases at hand, however, we are dealing with convergence where speakers are not so much trying to provide the best fit with the PLD, but, rather, are striving to accommodate fully acquired structures from an increasingly less native/attrited L1 (viz. Grecanico) in a native L2 (viz. Calabrese), frequently introducing competing and additional options within the contact grammar. Within this scenario, one possibility that presents itself to speakers is to reanalyse such optionality as meaningful variation, thereby enriching the contact grammar with new choices and concomitant distinctions. This appears to have been the case with the Greek-style dative, where the introduction of Greek-style genitive marking of R\textsc{ecipient} arguments does not replace Romance-style dative marking wholesale, but, rather, emerges in Calabrese\textsubscript{1} as a marked context-sensitive option that is specialized in the marking of individual \textsc{Recipient} arguments in accordance with their [±presuppositional] reading,\footnote{The [±presuppositional] distinction also plays a crucial role in the licensing of differential object marking in Calabrese, as fully demonstrated in \citet{LedgewaySchifanoSilvestriInpress}.} a development, in turn, reanalysed in Calabrese\textsubscript{2} as a structurally-conditioned alternation in accordance with the syncretic realization or otherwise of Case and definiteness.

\section*{Acknowledgments}
This work is part of the Leverhulme Research Project RPG-2015-283 \textit{Fading voices in southern Italy: investigating language contact in Magna Graecia.}

\sloppy
\printbibliography[heading=subbibliography,notkeyword=this] 
\end{document}

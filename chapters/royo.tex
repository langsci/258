\documentclass[output=paper,colorlinks,citecolor=brown,modfonts,nonflat]{langsci/langscibook}

\author{Carles Royo\affiliation{Universitat Rovira i Virgili}}
\title{The accusative/dative alternation in Catalan verbs with experiencer object}

\abstract{Various Catalan psychological verbs that are part of causative sentences with an accusative experiencer (\textit{Els nens van molestar la Maria} or \textit{La van molestar} ‘The kids annoyed Maria or They annoyed her’) alternate with stative sentences that change the sentence order and have a dative experiencer (\textit{A la Maria li molesta el teu caràcter} ‘lit. To Maria your character is annoying’). Other psychological verbs, however, can form both types of sentence without changing the accusative morphology of the experiencer (\textit{Els nens van atabalar la Maria} or \textit{La van atabalar} ‘The kids overwhelmed Maria or They overwhelmed her’; \textit{A la Maria l’atabala el teu caràcter} ‘lit. To Maria your character is overwhelming’). I argue that in stative sentences of all these verbs the experiencer is a real dative, regardless of its morphology (dative or accusative). Differential indirect object marking (DIOM) explains why accusative morphology is possible in these constructions.}

\IfFileExists{../localcommands.tex}{
 % add all extra packages you need to load to this file  
\usepackage{tabularx} 
\usepackage{url} 
\urlstyle{same}

\usepackage{listings}
\lstset{basicstyle=\ttfamily,tabsize=2,breaklines=true}


%%%%%%%%%%%%%%%%%%%%%%%%%%%%%%%%%%%%%%%%%%%%%%%%%%%%
%%%                                              %%%
%%%           Examples                           %%%
%%%                                              %%%
%%%%%%%%%%%%%%%%%%%%%%%%%%%%%%%%%%%%%%%%%%%%%%%%%%%% 
%% to add additional information to the right of examples, uncomment the following line
% \usepackage{jambox}
%% if you want the source line of examples to be in italics, uncomment the following line
% \renewcommand{\exfont}{\itshape}
\usepackage{langsci-optional}
\usepackage{./langsci/styles/langsci-gb4e}
\usepackage{./langsci/styles/langsci-lgr}
\usepackage{pgfplots,pgfplotstable}

\definecolor{lsDOIGray}{cmyk}{0,0,0,0.45}

\usepackage{xassoccnt}
\newcounter{realpage}
\DeclareAssociatedCounters{page}{realpage}
\AtBeginDocument{%
  \stepcounter{realpage}
}


 



 

 \newcommand{\appref}[1]{Appendix \ref{#1}}
\newcommand{\fnref}[1]{Footnote \ref{#1}} 

\newenvironment{langscibars}{\begin{axis}[ybar,xtick=data, xticklabels from table={\mydata}{pos}, 
        width  = \textwidth,
	height = .3\textheight,
    	nodes near coords, 
	xtick=data,
	x tick label style={},  
	ymin=0,
	cycle list name=langscicolors
        ]}{\end{axis}}
        
\newcommand{\langscibar}[1]{\addplot+ table [x=i, y=#1] {\mydata};\addlegendentry{#1};}

\newcommand{\langscidata}[1]{\pgfplotstableread{#1}\mydata;}

\makeatletter
\let\thetitle\@title
\let\theauthor\@author 
\makeatother

\newcommand{\togglepaper}[1][0]{ 
%   \bibliography{../localbibliography}
  \papernote{\scriptsize\normalfont
    \theauthor.
    \thetitle. 
    To appear in: 
    Change Volume Editor \& in localcommands.tex 
    Change volume title in localcommands.tex
    Berlin: Language Science Press. [preliminary page numbering]
  }
  \pagenumbering{roman}
  \setcounter{chapter}{#1}
  \addtocounter{chapter}{-1}
}
\newcommand{\orcid}[1]{}

 %% hyphenation points for line breaks
%% Normally, automatic hyphenation in LaTeX is very good
%% If a word is mis-hyphenated, add it to this file
%%
%% add information to TeX file before \begin{document} with:
%% %% hyphenation points for line breaks
%% Normally, automatic hyphenation in LaTeX is very good
%% If a word is mis-hyphenated, add it to this file
%%
%% add information to TeX file before \begin{document} with:
%% %% hyphenation points for line breaks
%% Normally, automatic hyphenation in LaTeX is very good
%% If a word is mis-hyphenated, add it to this file
%%
%% add information to TeX file before \begin{document} with:
%% \include{localhyphenation}
\hyphenation{
affri-ca-te
affri-ca-tes
Tarra-go-na
Vio-le-ta
Jacken-doff
clit-ics
Giar-di-ni
Mor-fo-sin-tas-si
mi-ni-mis-ta
nor-ma-li-tza-ció
Caus-ees
an-a-phor-ic
caus-a-tive
caus-a-tives
Mar-antz
ac-cu-sa-tive
Ma-no-les-sou
phe-nom-e-non
Holm-berg
}

\hyphenation{
affri-ca-te
affri-ca-tes
Tarra-go-na
Vio-le-ta
Jacken-doff
clit-ics
Giar-di-ni
Mor-fo-sin-tas-si
mi-ni-mis-ta
nor-ma-li-tza-ció
Caus-ees
an-a-phor-ic
caus-a-tive
caus-a-tives
Mar-antz
ac-cu-sa-tive
Ma-no-les-sou
phe-nom-e-non
Holm-berg
}

\hyphenation{
affri-ca-te
affri-ca-tes
Tarra-go-na
Vio-le-ta
Jacken-doff
clit-ics
Giar-di-ni
Mor-fo-sin-tas-si
mi-ni-mis-ta
nor-ma-li-tza-ció
Caus-ees
an-a-phor-ic
caus-a-tive
caus-a-tives
Mar-antz
ac-cu-sa-tive
Ma-no-les-sou
phe-nom-e-non
Holm-berg
}

 \bibliography{../localbibliography}
 \togglepaper[1]%%chapternumber
}{}

\begin{document}
\maketitle
\nocite{DIEC2007}

\todo[inline]{please revise according to guidelines:\\
provide translations\\
provide glosses for all source language words\\
do not use alternatives in the source line\\
do not use alternatives in gloss lines (/). Every morpheme has exactly one gloss. Instead of `He/She', use `3.SG'\\
do not use zeros in source lines\\
do not use `+'; use '.' for several meanings in a morpheme. \\
do not use () in the gloss line\\
}
\section{Introduction}\label{sec:royo:1}

Since the first half of the 20\textsuperscript{th} century (cf. \citealt[16]{Ginebra2003}, \citealt[147]{Ginebra2015}), some Catalan psychological verbs belonging to \citet{BellettiRizzi1988}'s type II – which make sentences with an accusative experiencer or AcExp \REF{ex:royo:1a}/\REF{ex:royo:2a} – have appeared with some frequency in both the written and spoken language with a change in sentence order and a dative experiencer \REF{ex:royo:1b}/\REF{ex:royo:2b}. This accusative/dative alternation has generated considerable academic debate. In most instances, the rules of the Institute of Catalan Studies (IEC) governing the Catalan language do not countenance this change in case marking, although the IEC’s new normative grammar \citep{GIEC2016} and the changes introduced on 5 April 2017 to its online normative dictionary \citep{DIEC2007} accept the dative case marking – as well as the accusative – in some particular predicates: including the verbs \textit{encantar} ‘delight’, \textit{estranyar} ‘surprise’, \textit{molestar} ‘annoy’ and \textit{preocupar} ‘worry’.\footnote{Before publication of the GIEC, the IEC accepted the intransitive nature of the verb \textit{interessar} ‘interest’ as well as an accusative case marking.}\todo{many ex. dont have proper glosses-translations. please revise.}

\ea%1
 \label{ex:royo:1}
 \ea \label{ex:royo:1a}
 \gll Els nens van molestar la Maria (\emph{o} la van molestar).\\
The kids  \textsc{aux.3pl} annoy.\textsc{inf} the Maria.\textsc{acc} or 3\textsc{fsg.acc} \textsc{aux.3pl} annoy.\textsc{inf}\\
 \glt ‘The kids annoyed Maria (or They annoyed her).’

 \ex \label{ex:royo:1b}
 \gll A la Maria li molesten els nens.\\
 to the Maria.\textsc{dat} \textsc{3sg.dat} annoy.\textsc{3pl} the kids\\
 \glt  (lit.) ‘To Maria kids are annoying.’
 \z
 \z

\ea%2
 \label{ex:royo:2} \citealt[77]{CabréMateu1998}
 \ea \label{ex:royo:2a}
 \gll Les teves paraules la van \emph{sorprendre}, \emph{preocupar}, \emph{molestar} molt.\\
the  your words     \textsc{3fsg.acc} \textsc{aux.3pl} surprise.\textsc{inf}, worry.\textsc{inf},  annoy.\textsc{inf} a\_lot\\
 \glt ‘Your words surprised, worried, annoyed her a lot.’

 \ex \label{ex:royo:2b}
 \gll Li \emph{sorprèn}, \emph{preocupa}, \emph{molesta} que la joventut d’ avui fumi tant.\\
\textsc{3sg.dat} surprises, worries,     annoys  that the youth     of today smoke so\_much\\
 \glt	(lit.) ‘To him/her that the youth of today smoke so much is surprising, worrying, annoying.’

 \z
 \z


This change has not had a uniform impact on Catalan dialects. Moreover, notable differences often occur within each dialect and even in the use that a specific speaker makes of these predicates (cf. \citealt[70]{CabréMateu1998}). Indeed, some predicates have become more entrenched than others, something that is irregularly reflected in several lexicographical collections in the Catalan language. It is common for AcExp verbs in Spanish to present this argument alternation (cf. \citealt{MendivilGiro2005, MarínMcNally2011}, among others). For this reason, psychological verbs that are used with dative constructions in Catalan, when they have traditionally been used with accusative constructions (AcExp), have often been regarded as syntactic calques of the Spanish; yet, some studies describe the change as being inherent to the Catalan language.

This paper argues that in a stative sentence containing these verbs the experiencer is a real dative, not only when it presents the dative morphology, but also when it presents the accusative form (see also \citetv{chapters/cabre} and \citetv{chapters/ledgeway}, about the different natures of datives). I also argue that the accusative morphology of such stative sentences is facilitated by a mechanism of differential indirect object marking (DIOM).

\section{Syntactico-semantic configuration of sentences with accusative and dative}\label{sec:royo:2}

\citet{Ynglès1991} and \citet{CabréMateu1998} point out that the syntactico-semantic configuration differs when some AcExp verbs are used with the accusative and when they are used with the dative: see the contrast in \REF{ex:royo:3}.\footnote{For further information on the proof and examples that show that sentences such as that in  \REF{ex:royo:1a}/\REF{ex:royo:2a} are configured differently from those illustrated in \REF{ex:royo:1b}/\REF{ex:royo:2b}, see \citet[Section 4.1]{Royo2017}.} In \REF{ex:royo:1a} and \REF{ex:royo:2a}, three components of causative verbs imply a change of state: cause + process (change) + resulting state (cf. \citealt{LevinRappaportHovav1995, CabréMateu1998, Rossello2008}). The verb needs to be followed by an accusative in an eventive sentence of external causation and a neutral subject-verb-object (SVO) order. On the other hand, \REF{ex:royo:1b} and \REF{ex:royo:2b} do not have these three components, and the verb requires the dative in a stative sentence and a neutral object-verb-subject (OVS) order and clitic doubling (see also \citetv{chapters/fabregas}).

\ea%3
 \label{ex:royo:3}
 \ea \label{ex:royo:3a}
 \gll Els nens van molestar la Maria \emph{expressament} i els mestres també \emph{ho} \emph{van} \emph{fer}. \\
The kids \textsc{aux.3pl} annoy.\textsc{inf} the Maria.\textsc{acc} on\_purpose     and the teachers also     it  \textsc{aux.3pl} do.\textsc{inf}\\
 \glt ‘The kids annoyed Maria on purpose and the teachers also did it.’

 \ex \label{ex:royo:3b}
 \gll *A la Maria li molesten els nens \emph{expressament} i els mestres també \emph{ho} \emph{fan}.\\
  to the Maria.\textsc{dat} \textsc{3sg.dat} annoy.\textsc{3pl} the kids on\_purpose     and the teachers also    it   do\\
\glt

 \z
 \z

Two mechanisms help differentiate the causative structure in \REF{ex:royo:1a}/\REF{ex:royo:2a} from the stative structure in \REF{ex:royo:1b}/\REF{ex:royo:2b}. On the one hand, their verbal aspect: the perfective aspect contributes to a causative interpretation while the imperfective aspect contributes to a stative interpretation; hence, there is a relation between the lexical aspect of the sentence (eventive or stative) and the verbal aspect of the predicate (perfective or imperfective). And, on the other, the sentence order: a neutral SVO order gives a causative interpretation and a neutral OVS order gives a stative interpretation.

In line with \citet{Ynglès1991, CabréMateu1998, Rossello2008} and \citet[Section 21.5]{GIEC2016} for Catalan, \citet{Pesetsky1995} for English, \citet{Bouchard1995} for French and \citet{Acedo-MatellánMateu2015} for Spanish, I consider that Catalan psychological verbs with an accusative experiencer (AcExp) generally cause a change of state:\footnote{According to other authors, the characterization of these sentences is different or allows different structures: cf. \citet{vanVoorst1992, Arad1999, Landau2010} and \citet{MarínMcNally2011} and \citet{Fabregas2015experimentante}. Several authors, including \citet{FabregasMarín2012,FábregasMarínMcNally2012,MarínSanchezMarco2012,Ganeshan2014} and \citet{Viñas-de-Puig2014}, study these constructions in their general analyses of the stative and eventive nature of Spanish sentences with psychological verbs (note Viñas-de-Puig do the same also with Catalan psychological verbs). \citet[83 (4)]{Acedo-MatellánMateu2015}) also accept that these verbs cause a change of state in Spanish but point out that there is a less common construction of AcExp verbs with the accusative, that is, stative causative transitive (\textit{Este problema la ha preocupado desde siempre}).} in these sentences subjects are agents or inanimate causes and accusative experiencers are strictly speaking patients, even though conceptually they can be regarded as experiencers. I also concur with several authors who point out that the OVS stative construction of some AcExp Catalan verbs is the same as that of psychological verbs with a dative experiencer (DatExp, for example \textit{agradar} ‘to like’; cf. \citealt{CabréMateu1998, Ramos2004, Rossello2008, Cuervo2010Cuestiones}, among others): the subject is a stimulus or source of the psychological experience and the dative experiencer is not a patient, it does not undergo a change of state. What is more, clitic doubling occurs when the experiencer phrase appears in preverbal position.\footnote{\citet{Acedo-MatellánMateu2015} have questioned this assumption in psychological verbs in Spanish and draw a distinction between DatExp verbs (unaccusative statives) and AcExp verbs that are constructed with the dative (unergative statives). For a discussion of this issue, see \citet[Section 6.2.4.1]{Royo2017}.}

These data suggest that many speakers need to change both the syntactical pattern of AcExp verbs and the sentence order when they use these verbs in a stative construction: the different semantic or lexical-aspectual interpretation of these sentences is reflected in the different syntactic configuration of constructions that contain Catalan AcExp verbs.\footnote{Several authors claim that the change between causative and stative interpretation implies a change in the Spanish case marking, between accusative and dative respectively: cf. \citet{Fabregas2015experimentante,Viñas-de-Puig2017} and \citet{Ganeshan2019}.} According to \citep[14, 29--30]{Ginebra2003}, however, the examples in \REF{ex:royo:4} show that Catalan can also denote a stative OVS construction without changing from the accusative to the dative with some predicates. These can be AcExp verbs \REF{ex:royo:4a} or non-psychological causative verbs that become psychological by means of a metaphorical expansion of the meaning \REF{ex:royo:4b} (the \textit{psych constructions} described by \citealt{Bouchard1995}). Therefore, the lexical nature of the verb plays an important role in the alternation since some verbs tend not to construct stative sentences with the dative.

\ea%4
	\citealt[29--30]{Ginebra2003}
 \label{ex:royo:4}
 \ea \label{ex:royo:4a}
 \gll Al seu germà l’ atabala la nova responsabilitat.\\
to.the his brother \textsc{3msg.acc} overwhelms the new  responsibility\\
\glt (lit.) ‘To his brother the new responsibility is overwhelming.’ \\

 \ex \label{ex:royo:4b}
 \gll Al Xavier el destrossa aquesta tensió contínua.\\
to.the Xavier \textsc{3msg.acc} destroys   this       tension constant\\
 \glt (lit.) ‘To Xavier this constant tension is destroying.’
 \z
 \z
 \todo{the gloss ADJ is probably not necessary (also elsewhere)}

What is more, with AcExp verbs such as those identified by \citet{CabréMateu1998} – \textit{molestar}, \textit{preocupar}, \textit{sorprendre} (see \REF{ex:royo:2}) – speakers may hesitate between accusative and dative case marking in OVS stative sentences. Some examples of this hesitation in a Catalan/Spanish bilingual newspaper are shown in \REF{ex:royo:5}. The print edition of the paper includes an OVS sentence with the verb \textit{preocupar} ‘worry’ that governs the accusative in Catalan \REF{ex:royo:5a} and the dative in Spanish \REF{ex:royo:5b}; on the other hand, in the Catalan online edition the same sentence appears with a dative \REF{ex:royo:5c}. Examples \REF{ex:royo:6} and \REF{ex:royo:7} show the same hesitation with the verb \textit{molestar} ‘annoy’, in the same news item reported by six media in Catalan on 5 December 2012: three use the accusative \REF{ex:royo:6} and three the dative \REF{ex:royo:7}.\footnote{The three sentences in the accusative use direct speech while the three in the dative use indirect speech, which may indicate that the person making the statement conceptualizes the verb differently from the journalists who report it.}

\ea%5
 \emph{La} \emph{Vanguardia}, 15 May 2015, p. 15 (headline),
 \label{ex:royo:5}\\
 \ea Catalan, printed version \label{ex:royo:5a}\\
 \gll Per\_què a Ciu la preocupa Ciutadans \\
why      to Ciu.\textsc{f} 3\textsc{fsg.acc} worries   Ciutadans.\textsc{sg} \\


 \ex Spanish, printed version \label{ex:royo:5b}\\
 \gll Por\_qué a Ciu le preocupa Ciutadans\\
why      to Ciu.\textsc{f} \textsc{3sg.dat} worries    Ciutadans.\textsc{sg}\\


\ex Catalan, online version \label{ex:royo:5c}\\
 \gll Per\_que a Ciu li preocupa Ciutadans \\
	why      to Ciu.\textsc{f} 3\textsc{sg.dat} worries    Ciutadans.\textsc{sg}\\
 \glt (lit.) ‘Why to Ciu Ciutadans is worrying.’

 \z
 \z


\ea%6
 \label{ex:royo:6}
 \ea VilaWeb (headline)\\
 \gll Rigau: ‘A Wert el molesta l’ èxit del model d’ immersió’\\
Rigau   to Wert \textsc{3msg.acc} annoys   the success of.the model of immersion\\
 \glt (lit.) ‘Rigau: ‘To Wert the model of immersion’s success is annoying’.’

 \ex \textit{El Periódico de Catalunya} (headline)\\
 \gll Rigau: “A Wert el molesta l’ èxit de la immersió lingüística”\\
 Rigau   to Wert 3\textsc{msg.acc} annoys  the success of the immersion language\\
 \glt (lit.) ‘Rigau: “To Wert the language immersion’s success is annoying”.’

 \ex \textit{Ara} (headline)\\
 \gll Rigau: “A Wert, el\_que el molesta és l’ èxit del model educatiu català”\\
Rigau    to Wert what   \textsc{3msg.acc} annoys  is the success of.the model educational Catalan\\
 \glt (lit.) ‘Rigau: “What is annoying to Wert is the Catalan educational model’s success”.’
 \z
 \z
\todo{examples adapted up to here}


\ea%7
 \label{ex:royo:7}
 \ea 3/24, www.ccma.cat (headline)\\
 \gll Rigau creu que a Wert li molesta “l' èxit” del model català\\
  Rigau believes that to Wert \textsc{3sg.dat} annoys the success of.the model Catalan\\

 \ex diaridegirona.cat (headline)\\
 \gll Rigau creu que a Wert li molesta “l’ èxit” del model català\\
Rigau believes that to Wert \textsc{3sg.dat} annoys   the success of.the model Catalan \\
 \glt (lit.) ‘Rigau believes that to Wert the Catalan model’s “success” is annoying.’

 \ex \emph{El} \emph{Punt} \emph{Avui}\\
 \gll La titular d’ Ensenyament, creu que a Wert li “molesta” el model “d’ èxit” de l’ escola catalana.\\
 the minister of Education believes that to Wert \textsc{3sg.dat} annoys the model of success of the school Catalan\\
 \glt (lit.) ‘The minister of Education believes that to Wert the model “of success” of the Catalan school
is annoying.’
 \z
 \z


In fact, if in \REF{ex:royo:1b} and \REF{ex:royo:2b} we replace the dative clitic with the accusative clitic – \textit{A la Maria la molesten els nens}; \textit{(A ella) La sorprèn, preocupa, molesta que la joventut d’avui fumi tant} –- our discussion above about distinguishing these sentences from those in \REF{ex:royo:1a} and \REF{ex:royo:2a} is still valid: they are useful ways of characterizing both constructions differently, but they do not help determine the case marking.

The ability of Catalan to construct a stative sentence with an AcExp verb and an accusative experiencer makes it necessary to analyse this accusative in those cases of hesitation with the dative (that is, in OVS stative sentences). We need to know whether the order of the sentences and clitic doubling in Catalan are sufficient to denote a lexical-aspectual change in the sentence or whether a change in case marking is also required.


\section{Nature of the accusative and dative experiencer in OVS stative sentences}\label{sec:royo:3}

In the sentences in \REF{ex:royo:1b}/\REF{ex:royo:2b} and \REF{ex:royo:4}-\REF{ex:royo:7}, whether the verb governs the accusative or the dative, the subject is a stimulus of the emotion and the object is not a patient but an experiencer of the whole event in a more prominent structural position than that occupied by the stimulus. It can be shown that this experiencer argument, regardless of whether it is accusative or dative, is not a topicalized element and that it has properties of a subject: cf. examples {a} and {b} in \REF{ex:royo:8}-\REF{ex:royo:13}. It behaves just like the experiencer in sentences with DatExp verbs such as \textit{agradar} ‘like’ (see the {c} examples in \REF{ex:royo:8}-\REF{ex:royo:13}) and other canonical subjects (see the {d} examples in \REF{ex:royo:8} and \REF{ex:royo:12} and example \REF{ex:royo:10e}): it behaves quite differently from topicalized objects (see the {d} examples in \REF{ex:royo:9}-\REF{ex:royo:11} and \REF{ex:royo:13}).\footnote{In examples \REF{ex:royo:8}-\REF{ex:royo:13}, as in the other examples employed in this paper, I conduct a descriptive rather than a prescriptive assessment.}

The experiencer can link an anaphora in the subject (cf. \citealt{Demonte1995, EgurenFernándezSoriano2004}) \REF{ex:royo:8}, be modified by the adverb \textit{només} ‘only’ (cf. \citealt{Cuervo1999}) \REF{ex:royo:9}, allow \textit{Wh-}extraction (cf. \citealt{BellettiRizzi1988}) \REF{ex:royo:10}, be an indefinite generalized quantifier in initial position (cf. \citealt{BellettiRizzi1988, Masullo1992quirky, Cuervo1999}) \REF{ex:royo:11}, control the subject of an infinitive clause (cf. \citealt{Campos1999, Alsina2008}) \REF{ex:royo:12} and it cannot be separated, in Catalan, by a comma from the rest of the sentence (cf. \citealt{Ginebra2003, Ginebra2005}) \REF{ex:royo:13}.
\textbf{}
\ea%8
 \label{ex:royo:8}
 \ea OVS AcExp, \textbf{dative/accusative} \label{ex:royo:8a}\\
 \gll A l’ Albert\textsubscript{i} \{\textbf{li\textsubscript{i}} / \textbf{el\textsubscript{i}}\} molesta aquesta fotografia de si\_mateix\textsubscript{i}.\\
 to the Albert {\textsc{3sg.dat}} / {\textsc{3msg.acc}} annoys this photo       of himself\\
\glt (lit.) ‘To Albert this photo of himself is annoying.’


 \ex OVS AcExp, \textbf{accusative}\label{ex:royo:8c}\\
 \gll A l’ Albert\textsubscript{i} \textbf{el\textsubscript{i}} neguiteja aquesta fotografia de si\_mateix\textsubscript{i}.\\
 to the Albert  {\textsc{3msg.acc}} disturbs this photo of himself\\
\glt (lit.) ‘To Albert this photo of himself is disturbing.’

 \ex DatExp, \textbf{dative}\label{ex:royo:8b}\\
 \gll A l’ Albert\textsubscript{i} \textbf{li\textsubscript{i}} agrada aquesta fotografia de si\_mateix\textsubscript{i}.\\
 to the Albert {\textsc{3sg.dat}} likes this photo of himself\\
\glt ‘Albert likes this photo of himself.’


  \ex Subject, dative\label{ex:royo:8d}\\
 \gll L’ Albert\textsubscript{i} envia una fotografia de si\_mateix\textsubscript{i/*j} a la Núria\textsubscript{j}.\\
 the Albert sends a photo of himself to the Nuria.\textsc{dat}\\
\glt ‘Albert sends a photo of himself to Nuria.’

 \z
 \z

\ea%9
 \label{ex:royo:9}
 \ea OVS AcExp, \textbf{dative/accusative} \label{ex:royo:9a}\\
 \gll Només a l’ Albert \{\textbf{li} / \textbf{el}\} molesta aquesta situació. \\
 only to the Albert \textbf{\textsc{3sg.dat}} / \textbf{\textsc{3msg.acc}} annoys this situation \\
\glt (lit.) ‘Only to Albert this situation is annoying.’

 \ex OVS AcExp, \textbf{accusative} \label{ex:royo:9b}\\
 \gll Només a l’ Albert \textbf{el} neguiteja aquesta situació.\\
 Only to the Albert \textbf{\textsc{3msg.acc}} disturbs this situation\\
 \glt (lit.) ‘Only to Albert this situation is disturbing.’

  \ex DatExp, \textbf{dative} \label{ex:royo:9c}\\
 \gll   Només a l’ Albert \textbf{li} agrada la cervesa. \\
 Only to the Albert \textbf{\textsc{3sg.dat}} likes the beer\\
\glt ‘Only Albert likes beer.’

 \ex Topicalized dative \label{ex:royo:9d}\\
\gll \textsuperscript{?}Només a l’ Albert \textbf{li} vaig prendre el bolígraf.\footnotemark{}\\
 Only to the Albert \textbf{\textsc{3sg.dat}} \textsc{aux.1sg} take.\textsc{inf} the pen\\
\glt
 \footnotetext{This sentence is acceptable with a stressed intonation: \textit{Només} \textit{A} \textit{L’ALBERT}…}
 \z
 \z


\ea%10
 \label{ex:royo:10}
 \ea  OVS AcExp, \textbf{dative/accusative} \label{ex:royo:10a}\\
 \gll  La situació que a l’ Albert \{\textbf{li} / \textbf{el}\} molesta és aquesta.\\
	 the situation that to the Albert \textsc{3sg.dat} / \textsc{3msg.acc} annoys is this\\
 \glt (lit.) ‘The situation that to Albert is annoying is this.’

 \ex OVS AcExp, \textbf{accusative}\label{ex:royo:10b}\\
 \gll La  situació  que a  l’ Albert (\textbf{el}) neguiteja és aquesta.\\
  the situation that to the Albert \textsc{3msg.acc} disturbs is this\\
 \glt (lit.) ‘The situation that to Albert is disturbing is this.’

 \ex  DatExp, \textbf{dative} \label{ex:royo:10c}\\
 \gll Els llibres que a l’ Albert (\textbf{li}) han agradat són aquests.\\
  the books that to the Albert \textbf{\textsc{3sg.dat}} have.\textsc{3pl} liked are these\\
 \glt ‘The books that Albert liked are these.’


 \ex{Topicalized dative \label{ex:royo:10d}\\
 \gll \textsuperscript{??}Els llibres que a l’ Albert (\textbf{li}) he donat són aquests.\\
 the books that to the Albert \textsc{3sg.dat} have.\textsc{1sg} given are these\\
 \glt


 \ex  Preverbal subject\label{ex:royo:10e}\\
 \gll  Els llibres que l’ Albert m’ ha donat són aquests.\\
	 the books that the Albert.\textsc{sbj} \textsc{1sg.dat} has given are these\\
 \glt ‘The books that Albert gave me are these.’

 \z
 \z


\ea%11
 \label{ex:royo:11}
 \ea OVS AcExp, \textbf{dative/accusative}\label{ex:royo:11a}\\
 \gll  A ningú (no) \{{li} / \textbf{el}\} molesta aquesta situació.\\
    to nobody (\textsc{neg}) \textbf{\textsc{3sg.dat}} / \textbf{\textsc{3msg.acc}} annoys this situation\\
\glt (lit.) ‘To nobody this situation is annoying.’


 \ex OVS AcExp, \textbf{accusative} \label{ex:royo:11b}\\
 \gll A ningú (no) (\textbf{el}) neguiteja aquesta situació.\\
  to nobody (\textsc{neg}) \textbf{\textsc{3msg.acc}} disturbs this situation\\
\glt (lit.) ‘To nobody this situation is disturbing.’

 \ex  DatExp, \textbf{dative} \label{ex:royo:11c}\\
 \gll  A ningú (no) \textbf{li} va agradar la pel·lícula.\\
 to nobody (\textsc{neg}) \textbf{\textsc{3sg.dat}} \textsc{aux.3sg} like.\textsc{inf} the film\\
\glt ‘Nobody likes the film.’


 \ex{Topicalized dative \label{ex:royo:11d}\\
 \gll  *A ningú (no) \textbf{li} vaig donar el quadre.\\
 to nobody (\textsc{neg}) \textbf{\textsc{3sg.dat}} \textsc{aux.1sg} give.\textsc{inf} the painting\\
\glt

  \z
  \z


\ea%12
 \label{ex:royo:12}
 \ea OVS AcExp, \textbf{dative/accusative} \label{ex:royo:12a}\\
 \gll  A l’ Albert\textsubscript{i} \{\textbf{li\textsubscript{i}} / \textbf{el\textsubscript{i}}\} molesta PRO\textsubscript{i} parlar en públic.\\
 to the Albert \textbf{\textsc{3sg.dat}} / \textbf{\textsc{3msg.acc}} annoys PRO speak.\textsc{inf} in public\\
 \glt (lit.) ‘To Albert speaking in public is annoying.’

 \ex OVS AcExp, \textbf{accusative}\label{ex:royo:12b}\\
 \gll A l’ Albert\textsubscript{i} \textbf{el\textsubscript{i}} neguiteja PRO\textsubscript{i} parlar en públic.\\
  to the Albert \textbf{\textsc{3msg.acc}} disturbs PRO speak.\textsc{inf} in public\\
\glt (lit.) ‘To Albert speaking in public is disturbing.’

 \ex DatExp, \textbf{dative}\label{ex:royo:12c}\\
 \gll A l’ Albert\textsubscript{i} \textbf{li\textsubscript{i}} agrada PRO\textsubscript{i} parlar en públic.\\
 to the Albert \textbf{\textsc{3sg.dat}} likes PRO speak.\textsc{inf} in public\\
\glt ‘Albert likes speaking in public.’

 \ex Subject\label{ex:royo:12d}\\
 \gll L’ Albert\textsubscript{i} vol PRO\textsubscript{i} arribar aviat.\\
 the Albert.\textsc{sbj} wants PRO arrive.\textsc{inf} early\\
\glt ‘Albert wants to arrive early.’

 \z
 \z



\ea%13
 \label{ex:royo:13}
 \ea OVS AcExp, \textbf{dative/accusative}\label{ex:royo:13a}\\
 \gll A l’ Albert\textsubscript{(*},\textsubscript{)} \{\textbf{li} / \textbf{el}\} molesta aquesta situació.\\
 to the Albert \textbf{\textsc{3sg.dat}} / \textbf{\textsc{3msg.acc}} annoys this situation\\
\glt (lit.) ‘To Albert this situation is annoying.’

 \ex OVS AcExp, \textbf{accusative}\label{ex:royo:13b}\\
 \gll A l’ Albert\textsubscript{(*},\textsubscript{)} \textbf{el} neguiteja aquesta situació.\\
 to the Albert \textbf{\textsc{3msg.acc}} disturbs this situation \\
\glt (lit.) ‘To Albert this situation is disturbing.’

 \ex DatExp, \textbf{dative}\label{ex:royo:13c}\\
 \gll A l’ Albert\textsubscript{(*},\textsubscript{)} \textbf{li} agrada aquesta situació.\\
 to the Albert \textbf{\textsc{3sg.dat}} likes this situation\\
\glt ‘Albert likes this situation.’

 \ex Topicalized object\label{ex:royo:13d}\\
 \gll (A) L’ Albert\textsubscript{(},\textsubscript{)} \textbf{l’} he vist que plorava.\\
\textsc{dom}  the Albert \textbf{\textsc{3msg.acc}} have.\textsc{1sg} seen that cried.\textsc{3sg}\\
\glt ‘Albert, I saw that he cried.’
 \z
 \z


\section{OVS sentences with AcExp verbs and an accusative experiencer}\label{sec:royo:4}

The analysis conducted in section \sectref{sec:royo:3} highlights the similarity between the dative experiencer in sentences with DatExp verbs and the experiencer object in OVS stative sentences with AcExp verbs, whether the morphology is dative or accusative. When the experiencer has accusative morphology, there is evidence to show that it is in fact a dative if we place it in sentence-initial position by using a relative pronoun (\REF{ex:royo:14a}-\REF{ex:royo:14b}) (adjectival relative clause and noun relative clause),\footnote{In the examples, I do not consider the use of the relative often referred to as the \textit{relatiu popular} (cf. \citealt[154--155]{Ginebra2005}), which is always marked with an asterisk.} an interrogative pronoun (\REF{ex:royo:14c}-\REF{ex:royo:14d}) (direct and indirect interrogative) or a determiner phrase \REF{ex:royo:14e}. In this context, the experiencer can optionally take either the accusative or dative morphology in the corresponding agentive sentences with AcExp verbs \REF{ex:royo:16}, which is similar to how the person semantic object behaves in transitive sentences of non-psychological verbs, whether they are causative or not \REF{ex:royo:17}. But in stative sentences with AcExp verbs \REF{ex:royo:14}, the experiencer in initial position behaves like the dative experiencer in the corresponding sentences with DatExp verbs \REF{ex:royo:15}: it can only be dative, even though in \REF{ex:royo:14} the morphology is still accusative clitic within the sentence (cf. \citealt[Section 4.3.4]{Royo2017}.

To illustrate this contrast, the examples below are of stative sentences with an imperfective verbal aspect \REF{ex:royo:14}-\REF{ex:royo:15} and causatives and non-causative transitives with a perfective aspect \REF{ex:royo:16}-\REF{ex:royo:17}. What is more, in \REF{ex:royo:14} and \REF{ex:royo:16} I use an AcExp verb that can easily be conceived as causative of change of state, such as \textit{atabalar} ‘overwhelm’, unlike other AcExp verbs such as \textit{molestar} ‘annoy’, which in some contexts can have the meaning of  \textit{desagradar molt} (‘displease a lot’).

\ea%14
 \label{ex:royo:14}
 \ea \label{ex:royo:14a}
 \gll És una persona \{\textbf{a} \textbf{qui} / *que\} (\textbf{l’}) atabala el record d’ aquell fracàs.\\
  is.\textsc{3sg} a.\textsc{f} person.\textsc{f} to whom.\textsc{dat} / who.\textsc{acc} \textsc{3fsg.acc} overwhelms the memory of that failure\\
\glt (lit.) ‘He/She is a person to whom the memory of that failure is overwhelming.’

 \ex \label{ex:royo:14b}
 \gll \{\textbf{A} \textbf{qui} / *Qui\} (\textbf{l’}) atabala el record d’ aquell fracàs és *(\textbf{a}) la Maria.\\
 to whom.\textsc{dat} / who.\textsc{acc} \textsc{3sg.acc} overwhelms the memory of that failure is to the Maria.\textsc{dat}\\
\glt (lit.) ‘To whom the memory of that failure is overwhelming it is to Maria.’

 \ex \label{ex:royo:14c}
 \gll \{\textbf{A} \textbf{qui} / *Qui\} (\textbf{l’}) atabala el record d' aquell fracàs?\\
 to whom.\textsc{dat} / who.\textsc{acc} \textsc{3sg.acc} overwhelms the memory of that failure\\
\glt (lit.) ‘To whom the memory of that failure is overwhelming?’

 \ex \label{ex:royo:14d}
 \gll Voldria saber \{\textbf{a} \textbf{qui} / *qui\} (\textbf{l’}) atabala el record d’ aquell fracàs.\\
 would\_like.\textsc{1sg} know.\textsc{inf}  to whom.\textsc{dat} / who.\textsc{acc} \textsc{3sg.acc} overwhelms the memory of that failure\\
\glt (lit.) ‘I would like to know to whom the memory of that failure is overwhelming.’

 \ex \label{ex:royo:14e}
 \gll *(\textbf{A}) la Maria\textsubscript{(*},\textsubscript{)} \textbf{l’} atabala el record d’ aquell fracàs.\footnotemark{}\\
 to the Maria.\textsc{dat} \textsc{3fsg.acc} overwhelms the memory of that failure\\
\glt (lit.) ‘To Maria, the memory of that failure is overwhelming.’
 \footnotetext{In examples \REF{ex:royo:14e} and \REF{ex:royo:15e} the asterisk indicates that these sentences cannot be constructed without the preposition \textit{a} at the beginning of the sentence. With the preposition \textit{a}, they are fully acceptable sentences.}
 \z
 \z




\ea%15
 \label{ex:royo:15}

 \ea\label{ex:royo:15a}
 \gll És una persona \{a qui / *que\} no (li) agrada el record d’ aquell fracàs.\\
 is.\textsc{3sg}  a person to whom.\textsc{dat} / who.\textsc{acc} \textsc{neg} \textsc{3sg.dat} likes the memory of that failure\\
\glt ‘He/She is a person who doesn’t like the memory of that failure.’

 \ex \label{ex:royo:15b}
 \gll  \{A qui / *Qui\} no (li) agrada el record d’ aquell fracàs és *(a) la Maria.\\
  to whom.\textsc{dat} / who.\textsc{acc} \textsc{neg} \textsc{3sg.dat} likes the memory of that failure is to the Maria.\textsc{dat}\\
\glt ‘Maria is the one who doesn’t like the memory of that failure.’



 \ex \label{ex:royo:15c}
 \gll \{A qui / *Qui\} no (li) agrada el record d’ aquell fracàs?\\
 to whom.\textsc{dat} / who.\textsc{acc} \textsc{neg} \textsc{3sg.dat} likes the memory of that failure\\
\glt ‘Who doesn’t like the memory of that failure?’

 \ex \label{ex:royo:15d}
 \gll Voldria saber \{a qui / *qui\} no (li) agrada el record d’ aquell fracàs.\\
 would\_like.\textsc{1sg} know.\textsc{inf} to whom.\textsc{dat} / who.\textsc{acc} \textsc{neg} \textsc{3sg.dat} likes the memory of that failure\\
\glt ‘I would like to know who doesn’t like the memory of that failure.’

 \ex \label{ex:royo:15e}
 \gll *(A) la Maria\textsubscript{(*},\textsubscript{)} no li agrada el record d’ aquell fracàs.\\
 to the Maria.\textsc{dat} \textsc{neg} \textsc{3sg.dat} likes the memory of that failure\\
\glt ‘Maria doesn’t like the memory of that failure.’

 \z
 \z


\ea%16
 \label{ex:royo:16}
 \ea \label{ex:royo:16a}
 \gll És una persona \{a qui (l’) / que\} han atabalat contínuament amb insídies.\\
 is.\textsc{3sg} a.\textsc{f}  person.\textsc{f} to whom.\textsc{dat} \textsc{3fsg.acc} / who.\textsc{acc} have.\textsc{3pl} overwhelmed continuously with malicious\_acts\\
\glt ‘He/She is a person who somebody has overwhelmed continuously with malicious acts.’

 \ex \label{ex:royo:16b}
	\gll \{A qui (l’) / Qui\} han atabalat contínuament amb insídies és (a) la Maria.\\
	 to whom.\textsc{dat} \textsc{3sg.acc} / who.\textsc{acc} have.\textsc{3pl} overwhelmed continuously with malicious\_acts is \textsc{dom} the Maria.\textsc{acc}\\
	\glt ‘Maria is the one who somebody has overwhelmed continuously with malicious acts.’

% 	\footnote{DOM = Differential object marking (see \citetv{chapters/manzini}).}%moved to list of abbreviations


 \ex \label{ex:royo:16c}
 \gll \{A qui (l’) / Qui\} han atabalat amb aquestes insídies?\\
to whom.\textsc{dat} \textsc{3sg.acc} / who.\textsc{acc} have.\textsc{3pl} overwhelmed with these malicious\_acts\\
\glt ‘Who has somebody overwhelmed with these malicious acts?’

 \ex \label{ex:royo:16d}
 \gll Voldria saber \{a qui (l’) / qui\} han atabalat amb aquestes insídies.\\
 would\_like.\textsc{1sg} know.\textsc{inf}  to whom.\textsc{dat} \textsc{3sg.acc} / who.\textsc{acc} have.\textsc{3pl} overwhelmed with these malicious\_acts\\
\glt ‘I would like to know who somebody has overwhelmed with these malicious acts.’

\ex \label{ex:royo:16e}
 \gll (A) la Maria\textsubscript{(},\textsubscript{)}  l’ han atabalat contínuament amb insídies.\\
 \textsc{dom} the Maria.\textsc{acc} \textsc{3fsg.acc} have.\textsc{3pl} overwhelmed continuously with malicious\_acts\\
\glt ‘Somebody has overwhelmed Maria continuously with malicious acts.’

 \z
 \z


\ea%17
 \label{ex:royo:17}
 \ea \label{ex:royo:17a}
 \gll És una persona \{a qui (l’) / que\} han \{mullat / vist\} amb una mànega.\\
 is.\textsc{3sg} a.\textsc{f} person.\textsc{f} to whom.\textsc{dat} \textsc{3fsg.acc} / who.\textsc{acc} have.\textsc{3pl} wet / seen with a hose\\
\glt ‘He/She is a person who somebody has \{wet / seen\} with a hose.’

 \ex \label{ex:royo:17b}
 \gll \{A qui (l’) / Qui\} han \{mullat / vist\} amb una mànega és (a) la Maria.\\
 to whom.\textsc{dat} \textsc{3sg.acc} / who.\textsc{acc} have.\textsc{3pl} wet / seen with a hose is \textsc{dom} the Maria.\textsc{acc}\\
\glt ‘Maria is the one who somebody has \{wet / seen\} with a hose.’

 \ex \label{ex:royo:17c}
 \gll \{A qui (l’) / Qui\} han \{mullat / vist\} amb una mànega?\\
 to whom.\textsc{dat} \textsc{3sg.acc} / who.\textsc{acc} have.\textsc{3pl} wet / seen with a hose\\
\glt ‘Who has somebody \{wet / seen\} with a hose?’

 \ex \label{ex:royo:17d}
 \gll Voldria saber \{a qui (l’) / qui\} han \{mullat / vist\} amb una mànega.\\
 would\_like.\textsc{1sg} know.\textsc{inf} to whom.\textsc{dat} \textsc{3sg.acc} / who.\textsc{acc} have.\textsc{3pl} wet / seen with a hose\\
\glt ‘I would like to know who somebody has \{wet / seen\} with a hose.’

 \ex \label{ex:royo:17e}
 \gll (A) la Maria\textsubscript{(},\textsubscript{)} l’ han \{mullat / vist\} amb una mànega.\\
 \textsc{dom} the Maria.\textsc{acc} \textsc{3fsg.acc} have.\textsc{3pl} wet / seen with a hose\\
\glt ‘Somebody has \{wet / seen\} Maria with a hose.’

 \z
 \z

Bearing in mind that stative sentences of AcExp verbs are constructed with a real dative, regardless of the morphology of the experiencer clitic, I use the abbreviation Dat(>{\textbar}<Ac)Exp to differentiate these constructions from both AcExp causatives and DatExp statives. The abbreviation can be used in cases of hesitation between the accusative and the dative form and, at the same time, to differentiate Dat(>Ac)Exp when the morphology is dative and Dat(<Ac)Exp when the morphology is accusative.

\section{Argument structure of stative sentences with AcExp verbs}\label{sec:royo:5}

According to \citet[Sections 13.3.6.2a-b and 13.3.7.2b]{Rossello2008} and \citet[21.2.2b and 21.5a]{GIEC2016}, one characteristic of Catalan psychological verbs with an experiencer object (AcExp and DatExp) is that they can elide their object in the absolute use of the verb. Sentences with the absolute use of these predicates can express the property of a stimulus to affect a hypothetical experiencer, a stative construction with both DatExp verbs \REF{ex:royo:18a} and AcExp verbs \REF{ex:royo:18b}, which in this case does not express an action.\footnote{The GIEC (Section 21.2.2c) points out that in absolute use those verbs that have an instrumental value (\textit{tallar} ‘cut’, \textit{obrir} ‘open’, \textit{tancar} ‘close’, \textit{tapar} ‘cover’, etc.), which like AcExp verbs are generally causative of change of state, express a property of the subject rather than a particular action.}

\ea%18
 \label{ex:royo:18}
 \ea \label{ex:royo:18a}
 \gll La xocolata agrada (‘és agradable’); La família importa (‘és important’). \\
 the chocolate likes is pleasant the family matters is important\\
 \glt (lit.) ‘Chocolate is pleasant.’ (lit.) ‘Family is important.’

 \ex \label{ex:royo:18b}
 \gll Els nens molesten (‘són molestos’); El teu caràcter atabala (‘és atabalador’).\\
  the kids annoy are annoying the your character overwhelms is overwhelming\\
 \glt  (lit.) ‘Kids are annoying.’ (lit.) ‘Your character is overwhelming.’

 \z
 \z

Following \citeauthor{Cuervo2003}'s proposal (\citeyear[Section 1.3.3.2]{Cuervo2003}) for verbs that she calls \textit{predicational statives}, all the sentences in \REF{ex:royo:18} have an underlying stative unaccusative structure. For sentences with an experiencer, we need a functional head that introduces a dative with experiencer semantics and the characteristics of a subject in a hierarchically superior position and which relates it to the whole event that indicates a property of the stimulus: a high applicative head (external argument), with the dative in the position of specifier (cf. \citealt{Pylkkänen2008, Cuervo2003, Cuervo2010Cuestiones}; see also \citetv{chapters/cuervo}) \REF{ex:royo:19}.\footnote{Other authors explain the variability between the stative and the causative reading of these verbs without a high applicative head that introduces the experiencer in the stative construction (see \citealt{Viñas-de-Puig2014, Viñas-de-Puig2017}, and references therein). For example, Viñas-de-Puig proposes that in both readings the experiencer is licensed for a S\textit{v}\textsc{\textsubscript{exp}} head above the root, in a basic stative structure, which will take a causative reading by adding a S\textit{v}\textsc{\textsubscript{caus}} above the S\textit{v}\textsc{\textsubscript{exp}}.}

\ea%19
 \label{ex:royo:19}
 \ea DatExp\label{ex:royo:19a}\\
 \gll A la Maria li agrada la xocolata. \\
 to the Maria \textsc{3sg.dat} likes the chocolate\\
 \glt (lit.) ‘To Maria chocolate is pleasant.’

 \ex  Dat(>{\textbar}<Ac)Exp\label{ex:royo:19b}\\
 \gll A la Maria \{li / la\} molesten els nens.\\
 to the Maria \textsc{3sg.dat} / \textsc{3fsg.acc} annoy.\textsc{3pl} the kids\\
 \glt (lit.) ‘To Maria kids are annoying.’

 \z
 \z


\begin{figure}
	\begin{forest}
		[ApplP
			[DP
				[a la Maria, name=dp]
			]
			[
				[Appl
					[li, name = clitic]
				]
				[\liv P
					[DP
						[la xocolata]
					]
					[
						[\liv\textsubscript{\textsc{be}}]
						[Root
							[√agrad-]
						]
					]
				]
			]
		]
		\draw[->] (clitic) to[out=south west,in=south] (dp);
	\end{forest}
	\caption{\label{fig:royo:1}Structure of DatExp verb sentence}
\end{figure}

\begin{figure}
	\begin{forest}
		[ApplP
			[DP
				[a la Maria, name=dp]
			]
			[
				[Appl
					[li/la, name= clitic]
				]
				[\liv P
					[DP
						[els nens]
					]
					[
						[\liv\textsubscript{\textsc{be}}
						]
						[Root
							[√molest-]
						]
					]
				]
			]
		]
		\draw[->] (clitic) to[out=south west,in=south] (dp);
	\end{forest}
	\caption{\label{fig:royo:2}Structure of Dat(>|<Ac)Exp verb sentence}
\end{figure}

The unaccusative structure of \REF{ex:royo:19a} for DatExp verbs matches \citegen{BellettiRizzi1988} characterisation of type-III predicates. The construction of \REF{ex:royo:19b}, however, requires some additional clarifications. Apparently, we should reject an unaccusative structure with an accusative experiencer – and in Catalan we do not expect an accusative to be an external argument – but if we bear in mind that it is a superficial accusative and that it is really a dative (cf. \sectref{sec:royo:3} and  \sectref{sec:royo:4}), this objection disappears. We also need to be explain how some verbs can optionally use the accusative and dative forms \REF{ex:royo:5}-\REF{ex:royo:7}, and other verbs the accusative form in OVS stative sentences, whether they are AcExp \REF{ex:royo:4a} or causative predicates with a metaphorical psychological meaning \REF{ex:royo:4b}.

In these sentences, the experiencer is a non-topicalized element with subject properties and a real dative, regardless of the form it takes. The syntactic mechanism that can explain sentences in which the experiencer has an apparent accusative morphology \REF{ex:royo:20b} is differential indirect object marking or DIOM (cf. \citealt{Bilous2011, Pineda2016}, \citeyear{Pineda2019}; \citealt{PinedaRoyo2017}), which is not necessary when the clitic takes a dative morphology \REF{ex:royo:20a}.

\ea%20
 \label{ex:royo:20}
 \ea Dat(>)Exp\label{ex:royo:20a}\\
 \gll A la Maria \textbf{li} molesten els nens.\\
 to the Maria.\textsc{dat} \textsc{3sg.dat} annoy.\textsc{3pl} the kids\\
 \glt (lit.) ‘To Maria kids are annoying.’

 \ex Dat(<Ac)Exp\label{ex:royo:20b}\\
 \gll A la Maria \textbf{l}’ atabala el teu caràcter.\\
 to the Maria.\textsc{dat} \textsc{3fsg.acc.diom} overwhelm the your character\\
 \glt (lit.) ‘To Maria your character is overwhelming.’
\z
\z

  \begin{figure}
	\begin{forest}
		[ApplP
			[DP
				[a la Maria, name=dp]
			]
			[
				[Appl
					[li, name=clitic]
				]
				[\liv P
					[DP
						[els nens]
					]
					[
						[\liv\textsubscript{\textsc{be}}]
						[Root
							[√molest-]
						]
					]
				]
			]
		]
	\draw[->] (clitic) to[out=south west,in=south] (dp);
	\end{forest}
	\caption{\label{fig:royo:3}Structure of Dat(>Ac)Exp verb sentence}
\end{figure}

\begin{figure}
	\begin{forest}
		[ApplP
			[DP
				[a la Maria, name=dp]
			]
			[
				[Appl
					[l'\textsubscript{[\textsc{diom}]}, name=clitic]
				]
				[\liv P
					[DP
						[el teu caràcter]
					]
					[
						[\liv\textsubscript{\textsc{be}}]
						[Root
							[√atabal-]
						]
					]
				]
			]
		]
	\draw[->] (clitic) to[out=south west,in=south] (dp);
	\end{forest}
	\caption{\label{fig:royo:4}tructure of Dat(<Ac)Exp verb sentence}
\end{figure}



The dative case marking of these sentences is congruent with the semantic and syntactic characteristics of the experiencer and with the function of the high applicative heads in a Romance language like Catalan. A DIOM accusative morphology would allow speakers to use these constructions with verbs that are difficult to conceive as stative, because in the minds of speakers they are closely related to verbs that cause a change of state \REF{ex:royo:4}. The morphological aspect of the experiencer depends on the lexical characteristics of the verb: even though the sentence is always stative, we can regard DIOM as being an anti-stativization mechanism in the minds of speakers. In this sense, it is significant that non-psychological causative verbs with a metaphorical psychological meaning present the superficial accusative form in OVS stative sentences (\textit{destrossar} ‘destroy’, \textit{enfonsar} ‘sink’). Like some psychological verbs (\textit{commoure} ‘move, touch’, \textit{esparverar} ‘terrify’),\footnote{\citet[14, 29--30]{Ginebra2003} offers more examples of OVS stative sentences of this type with a superficial accusative in both verb types, that is, psychological and non-psychological verbs with metaphorical psychological meaning.} they are verbs that speakers conceptualize habitually as being causative of change of state, unlike other verbs that more readily permit a stative conceptualization in certain contexts: for example, \textit{molestar} ‘annoy’, which can sometimes have the meaning of \textit{desagradar molt} (‘displease a lot').\footnote{For an explanation of other factors that intervene so that an AcExp verb can participate in sentences such as Dat(<Ac)Exp or Dat(>Ac)Exp, see \citet[Section 5]{Royo2017}.}

This explanation takes into account the conceptual mechanisms that can, according to several authors, affect the construction of sentences and syntactic change: the speakers’ conception of the world (cf. \citealt{Ramos2002}), the linguistic conception of particular communicative contexts (cf. \citealt{Rossello2008}) and the different conceptualisation of transitivity (cf. \citealt{Ynglès2011, Pineda2012}).

\section{Conclusions}\label{sec:royo:6}

The main argument presented in this article is that in stative sentences of Catalan AcExp predicates, the experiencer is a real dative. In stative sentences of some AcExp verbs and other non-psychological causative verbs with metaphorical psychological semantics, the experiencer may present an external accusative morphology by means of differential indirect object marking (DIOM). DIOM is the manifestation in the minds of speakers of their difficulty to conceive certain verbs as being stative or, in other words, of their tendency to conceive them as being causative of change of state.

\section*{Acknowledgments} This study has been supported by research project FFI2014-56258-P ({Ministerio de Economía y Competitividad}). I would like to thank Jaume Mateu for specific comments made in relation to this paper and Anna Pineda for encouraging me to present this research in public and to have it published.

\section*{Abbreviations}
\textsc{dom} differential object marking

\sloppy\printbibliography[heading=subbibliography,notkeyword=this]
\end{document}

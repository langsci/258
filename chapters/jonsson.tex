\documentclass[output=paper,modfonts,nonflat,colorlinks,citecolor=brown]{langsci/langscibook} 
\author{Jóhannes Gísli Jónsson\affiliation{Unversity of Iceland}\lastand
Rannveig Thórarinsdóttir\affiliation{Unversity of Iceland}}
\title{Dative objects with novel verbs in Icelandic}
\abstract{This paper discusses the results of two online surveys testing object case with novel verbs in Icelandic. The results show that a novel transitive verb takes a dative direct object if the verb (a) encodes some kind of motion of the object referent, or (b) has a translational substitute that takes a dative object. If neither (a) nor (b) holds, the object gets the default accusative case. Thus, caused motion plays a major role in the licensing of dative case with direct objects in Icelandic.}
\IfFileExists{../localcommands.tex}{
  % add all extra packages you need to load to this file  
\usepackage{tabularx} 
\usepackage{url} 
\urlstyle{same}

\usepackage{listings}
\lstset{basicstyle=\ttfamily,tabsize=2,breaklines=true}


%%%%%%%%%%%%%%%%%%%%%%%%%%%%%%%%%%%%%%%%%%%%%%%%%%%%
%%%                                              %%%
%%%           Examples                           %%%
%%%                                              %%%
%%%%%%%%%%%%%%%%%%%%%%%%%%%%%%%%%%%%%%%%%%%%%%%%%%%% 
%% to add additional information to the right of examples, uncomment the following line
% \usepackage{jambox}
%% if you want the source line of examples to be in italics, uncomment the following line
% \renewcommand{\exfont}{\itshape}
\usepackage{langsci-optional}
\usepackage{./langsci/styles/langsci-gb4e}
\usepackage{./langsci/styles/langsci-lgr}
\usepackage{pgfplots,pgfplotstable}

\definecolor{lsDOIGray}{cmyk}{0,0,0,0.45}

\usepackage{xassoccnt}
\newcounter{realpage}
\DeclareAssociatedCounters{page}{realpage}
\AtBeginDocument{%
  \stepcounter{realpage}
}


 



 

  \newcommand{\appref}[1]{Appendix \ref{#1}}
\newcommand{\fnref}[1]{Footnote \ref{#1}} 

\newenvironment{langscibars}{\begin{axis}[ybar,xtick=data, xticklabels from table={\mydata}{pos}, 
        width  = \textwidth,
	height = .3\textheight,
    	nodes near coords, 
	xtick=data,
	x tick label style={},  
	ymin=0,
	cycle list name=langscicolors
        ]}{\end{axis}}
        
\newcommand{\langscibar}[1]{\addplot+ table [x=i, y=#1] {\mydata};\addlegendentry{#1};}

\newcommand{\langscidata}[1]{\pgfplotstableread{#1}\mydata;}

\makeatletter
\let\thetitle\@title
\let\theauthor\@author 
\makeatother

\newcommand{\togglepaper}[1][0]{ 
%   \bibliography{../localbibliography}
  \papernote{\scriptsize\normalfont
    \theauthor.
    \thetitle. 
    To appear in: 
    Change Volume Editor \& in localcommands.tex 
    Change volume title in localcommands.tex
    Berlin: Language Science Press. [preliminary page numbering]
  }
  \pagenumbering{roman}
  \setcounter{chapter}{#1}
  \addtocounter{chapter}{-1}
}
\newcommand{\orcid}[1]{}
 
  %% hyphenation points for line breaks
%% Normally, automatic hyphenation in LaTeX is very good
%% If a word is mis-hyphenated, add it to this file
%%
%% add information to TeX file before \begin{document} with:
%% %% hyphenation points for line breaks
%% Normally, automatic hyphenation in LaTeX is very good
%% If a word is mis-hyphenated, add it to this file
%%
%% add information to TeX file before \begin{document} with:
%% %% hyphenation points for line breaks
%% Normally, automatic hyphenation in LaTeX is very good
%% If a word is mis-hyphenated, add it to this file
%%
%% add information to TeX file before \begin{document} with:
%% \include{localhyphenation}
\hyphenation{
affri-ca-te
affri-ca-tes
Tarra-go-na
Vio-le-ta
Jacken-doff
clit-ics
Giar-di-ni
Mor-fo-sin-tas-si
mi-ni-mis-ta
nor-ma-li-tza-ció
Caus-ees
an-a-phor-ic
caus-a-tive
caus-a-tives
Mar-antz
ac-cu-sa-tive
Ma-no-les-sou
phe-nom-e-non
Holm-berg
}

\hyphenation{
affri-ca-te
affri-ca-tes
Tarra-go-na
Vio-le-ta
Jacken-doff
clit-ics
Giar-di-ni
Mor-fo-sin-tas-si
mi-ni-mis-ta
nor-ma-li-tza-ció
Caus-ees
an-a-phor-ic
caus-a-tive
caus-a-tives
Mar-antz
ac-cu-sa-tive
Ma-no-les-sou
phe-nom-e-non
Holm-berg
}

\hyphenation{
affri-ca-te
affri-ca-tes
Tarra-go-na
Vio-le-ta
Jacken-doff
clit-ics
Giar-di-ni
Mor-fo-sin-tas-si
mi-ni-mis-ta
nor-ma-li-tza-ció
Caus-ees
an-a-phor-ic
caus-a-tive
caus-a-tives
Mar-antz
ac-cu-sa-tive
Ma-no-les-sou
phe-nom-e-non
Holm-berg
}

  \bibliography{../localbibliography}
  \togglepaper[12]%%chapternumber
}{}

\begin{document}
\maketitle 
 

 

\section{Introduction} %1. /
\label{sec:jonsson:1}

Dative case with direct objects in Icelandic has been widely discussed in the linguistic literature (see e.g. \citealt{YipEtAl1987,Barðdal2001,Barðdal2008,Svenonius2002,Maling2002}, and \citealt{Jónsson2013a}). The central issue is the degree to which the dative is semantically predictable. As discussed by \citet{Maling2002}, verbs with dative objects are found in various verb classes in Icelandic, most of which also include verbs with accusative objects. Thus, it appears that dative is predictable only in a broad sense. However, it can be shown that dative objects are fully predictable in at least three closely related classes, verbs of ballistic motion \citep{Svenonius2002} verbs of emission (\citealt{Maling2002}, \citealt{Jónsson2013a}) and pour verbs \citep{Jónsson2013a}.

One way of probing the semantics of dative objects in Icelandic is to examine novel transitive verbs since these verbs should reflect the regular aspects of dative case assignment. Indeed, the fact that new verbs never take genitive objects (\citealt{JónssonEythórsson2011}) suggests that they cannot assign truly idiosyncratic case. However, with the exception of \citet{Barðdal2001,Barðdal2008}, new verbs with dative objects have not been a central concern in the literature on the Icelandic case system. 

We report here on the results of two online surveys testing object case with verbs that have become part of colloquial Icelandic in the last few decades (see \citealt{Thórarinsdóttir2015}). The results show that a novel transitive verb takes a dative direct object if the verb (a) encodes motion of the object referent, or (b) has a translational substitute that takes a dative object. We will refer to (b) as isolate attraction, following \citet{Barðdal2001}, and take the term translational substitute to mean an established verb taking a dative object that can replace the new verb semantically. If neither (a) nor (b) applies, the object gets accusative case, the default case for direct objects in Icelandic. This holds not only for verbs selecting one object case but also for verbs displaying variation between dative and accusative. This means that some verbs may be ambiguous in whether they encode caused motion or not. Note, however, that case variation in Icelandic may also be purely formal and not reflect any semantic distinction between the variants (see \citealt{Jónsson2013b}); for discussion of formal case variation in Romance, see \citetv{chapter/ledgeway} and \citetv{chapters/royo}).

The strong link between caused motion and dative objects in Icelandic has often been discussed, e.g. by \citet{Barðdal2001, Barðdal2008, Svenonius2002, Maling2002}; and \citet{Jónsson2013a}. Our proposal is new in that caused motion is argued to be the crucial meaning component of new dative verbs in Icelandic that are not licensed by isolate attraction. That isolate attraction plays a role independent of caused motion is shown by novel verbs like \textit{dílíta} ʽdelete electronicallyʼ, which does not express any motion of the object. This verb takes a dative object just like its translational substitute \textit{eyða} ʽdelete, spend, wasteʼ, which has a broader meaning than \textit{dílíta}. Further examples of isolate attraction will be discussed in \sectref{sec:jonsson:3.2} below.

Since there are only two ways in which a novel verb can get a dative object and both of them are quite restricted, our proposal makes strong predictions about dative objects with novel verbs in Icelandic. As discussed in \sectref{sec:jonsson:3} and \sectref{sec:jonsson:4}, these predictions are borne out by the data from the two online surveys. Importantly and in clear contrast to \citet{Barðdal2001, Barðdal2008}, we do not allow for the possibility that novel verbs take a dative object if they are attracted to specific classes of dative verbs with a similar meaning. Thus, the data from the two surveys will be accounted for without any recourse to this possibility although various subclasses of verbs taking the same object case will be mentioned in our discussion.

\section{Background} %2. /
\label{sec:jonsson:2}

There is a fundamental unity to all dative objects in Icelandic in that dative is preserved under passivization. In this respect, dative differs sharply from accusative (\citealt{ZaenenMalingThráinsson1985}). Case preservation in passives applies equally to datives that are completely predictable, such as dative recipients or benefactives with ditransitive verbs \citep{Jónsson2000}, and datives that are idiosyncratically associated with some monotransitive verbs. This latter type is exemplified by verbs like \textit{anna} ʽmeet (demand), have time forʼ, \textit{gleyma} ‘forget’, \textit{stríða} ‘tease’, \textit{treysta} ʽtrustʼ and \textit{unna} ‘love’. This contrast between dative and accusative shows that dative is not a structural case in Icelandic, at least not in the same sense as accusative (see \citealt[181--192]{Thráinsson2007} and references cited there).

A further difference is that accusative is not associated with any specific semantics as transitive verbs of all kinds take accusative objects in Icelandic. In fact, as shown the by the ECM construction, accusative can even be assigned to a DP that is not an argument of the relevant verb. Although it has been observed that certain sublasses of transitive verbs in Icelandic only allow accusative objects \citep{Jónsson2013a}, this is best understood as a constraint on the assignment of dative (and genitive) case. In \sectref{sec:jonsson:3} and \sectref{sec:jonsson:4}, some verb classes that systematically exclude dative or genitive objects will be mentioned but this should not be taken to mean that accusative is semantically determined in these classes.

Despite the differences between accusative and dative discussed above, it has become fairly common in recent years to link both these cases to functional heads in the extended \liv P. Thus, the Icelandic dative is often associated with an applicative head inside VoiceP/\liv P. \citet[128--138]{Wood2015} argues that this is correct for indirect objects but generally not for direct objects. His arguments are based e.g. on the fact that dative is preserved with indirect objects but not direct objects under suffixation of the “middle” suffix -\textit{st} in Icelandic. His proposal is that direct object datives are licensed by a functional head that he labels \liv\textsc{\textsubscript{dat}}, following \citet{Svenonius2006alternations}. The results discussed in \sectref{sec:jonsson:3} and \sectref{sec:jonsson:4} suggest that there is a functional head that licenses dative objects with verbs that express caused motion. Diachronic evidence from Faroese points in the same direction since dative has systematically disappeared with all such verbs but is preserved with various other monotransitive verbs in Faroese \citep{Jónsson2009}. This diachronic development can be interpreted as the loss of the relevant functional head in the history of Faroese.

\citet{Svenonius2002} shows that verbs of ballistic motion like \textit{kasta} ʽthrowʼ always take a dative object in Icelandic. With these verbs, the agent applies force to cause an object to move but the motion of the object continues after the agent has done his/her part. \citet{Svenonius2002} claims that dative objects in Icelandic are found with verbs where the subevent associated with the agent does not completely overlap temporally with the subevent associated with the theme object. This is correct as a one-way generalization as every verb that complies with it takes a dative object in Icelandic but it is not immediately obvious how far this generalization extends beyond verbs of ballistic motion. We cannot discuss this issue fully here, but it seems to us that it also comprises emission verbs, pour verbs and many of the verbs tested in the online surveys relating to information technology and expressing motion from one electronic location to another. 

Another complicating factor is that dative case is found with various verbs of motion that involve complete temporal overlap of the two subevents associated with the agent and the theme. Thus, verbs of accompanied or directed motion may take a dative object (cf. \textit{drösla} ‘move (with difficulty)’, \textit{lyfta} ‘lift, raise’, \textit{smeygja} ‘slip, slide’, and \textit{ýta} ‘push’) or an accusative object (cf. \textit{bera} ‘carry’, \textit{draga} ‘pull’, \textit{hækka} ‘raise’, and \textit{lækka} ‘lower’). However, the data discussed in \sectref{sec:jonsson:3} and \sectref{sec:jonsson:4} suggest that dative objects with novel verbs are licensed by caused motion, irrespective of any subclassification of the relevant verbs. Hence, it appears that dative is in the process of being generalized to all transitive motion verbs in Icelandic \citep{Barðdal2008}.

The theoretical literature on motion verbs across languages is very much focused on intransitive verbs like \textit{run} and \textit{dance} and there is no standard definition of caused motion that we are aware of. Still, this does not turn out to be much of a problem for our purposes. As we will see, the crucial issue is to distinguish verbs that encode caused motion of the direct object from verbs where caused motion is not encoded but rather inferred from world knowledge. It is only novel verbs in the former class that take a dative object, i.e. if isolate attraction does not play a role.

\section{The results} %3. /
\label{sec:jonsson:3}

In the following two sections, the results of a large-scale study of direct object case with 40 novel verbs in Icelandic will be discussed. These verbs have become part of the Icelandic lexicon in the last few decades, mainly as borrowings from English or Danish but some as native neologisms. Most of these verbs are highly colloquial and not often found in writing, especially the loanverbs, but as far as we know this has no effect on object case. 

\subsection{The two surveys} %3.1 /
\label{sec:jonsson:3.1}

The study to be discussed here consisted of two online surveys, with 393 and 402 participants, respectively (see \citealt{Thórarinsdóttir2015} for details). Each survey featured 50 sentences, 20 sentences testing object case with novel verbs and 30 fillers. For every sentence, the participants were asked to select four options presented to them in this order: (a) the accusative form of the direct object, (b) the dative form of the direct object, (c) both forms accepted, (d) neither form accepted. Option (d) was selected quite often, especially with verbs of low frequency, presumably because some of the participants were not familiar with these verbs. By contrast, very few opted for (c), even with verbs where we suspect that many speakers allow both accusative and dative.

The verbs tested in the two surveys are listed below. The glosses are based on the relevant test sentences in the surveys. 


\ea
\ea Verbs in the first survey: \\
brodkasta ʽbroadcastʼ, dánlóda ʽdownloadʼ, droppa ʽquit, dropʼ, drulla ʽget, putʼ, dömpa ʽdumpʼ, farta ʽdrive fastʼ, flexa ʽshow off with, throw aroundʼ, gúgla ʽgoogleʼ, hannesa ʽsteal (a text)ʼ, installa ʽinstall (a program)ʼ, jáa ʽsearch for on ja.isʼ, jinxa ʽput a curse onʼ, krakka ʽunlock, crackʼ, krassa ʽcause to crashʼ, offa ʽturn off, shockʼ, rippa ʽcopy (illegally)ʼ, slaka ʽpassʼ, slumma ʽkick (a ball)ʼ, smessa ʽsend by smsʼ, sneika ʽsneakʼ
\ex Verbs in the second survey: \\
átsorsa ʽoutsourceʼ, bekka ʽlift in bench pressʼ, blasta ʽplay loudly, blastʼ, bleima ʽblameʼ, domma ʽdominateʼ, fiffa ʽfix (illegally)ʼ, gólfa ʽpress (the pedal) to the floorʼ, gramma ʽput on Instagramʼ, græja ʽprocureʼ, gúffa ʽeat greedilyʼ, kikka ʽkick, hitʼ, meila ʽe-mailʼ, mæna ʽcollect, mineʼ, neimdroppa ʽnamedropʼ, peista ʽpasteʼ, pósta ʽpost (online)ʼ, sjera ʽshare (online)ʼ, skrína ʽscreen, keep an eye onʼ, skúbba ʽbe the first to tell (a piece of news)ʼ, syngja ʽtell (a secret)ʼ\\
\z
\z

Five verbs are not included in the following discussion here, either because the relevant test sentences allowed for too many possibilities for their semantic interpretation (\textit{jinxa}, \textit{kikka}) or because it can be argued that they are not really new  (\textit{drulla}, \textit{slaka}, \textit{skúbba}). 


The two surveys were designed to test our hypothesis that dative case with novel transitive verbs in Icelandic is licensed by two factors: (a) caused motion of the object referent, or (b) a translational substitute taking a dative object (isolate attraction). The verbs were selected so that they would fall into three groups of roughly the same size: (a) verbs taking a dative object, (b) verbs taking an accusative object, and (c) verbs displaying variation between dative and accusative. A random selection of novel verbs would have produced a less balanced sample in view of \citeauthor{Barðdal2008}’s (2008:78--79) study of 107 novel verbs in Icelandic where accusative outscored dative by a ratio of approximately 2:1. Note that no effort was made to include verbs from all the subclasses of the dative verbs discussed by \citet{Maling2002} as the right verbs would have been hard to find and this would have required a much bigger study.


The novel verbs tested in the study can be divided into three classes: (a) verbs that strongly favour dative, (b) verbs that strongly favor accusative, and (c) verbs that vary between dative and accusative object. For concreteness, classes (a) and (b) were defined such that the preferred case was selected at least five times more often than the other case. Verbs from the first two classes are discussed in \sectref{sec:jonsson:3.2} and \sectref{sec:jonsson:3.3} below but variation between dative and accusastive is the topic of \sectref{sec:jonsson:4}.

\subsection{Dative objects} %3.2 /
\label{sec:jonsson:3.2}

Many verbs in the current study showed a strong preference for a dative object. This is true of the following verbs in the first survey: 

\begin{table}
\caption{\label{tab:jonsson:1}Verbs taking a dative object in survey 1}
\begin{tabularx}{\textwidth}{XXrrrr}
\lsptoprule
Verb & Gloss & \textbf{DAT} & ACC & Both & Neither\\
\midrule
dánlóda & ʽdownloadʼ & \textbf{93,1} & 4,1 & 0,5 & 2,3\\
droppa & ʽquit, dropʼ & \textbf{90,6} & 1,5 & 0,3 & 7,6\\
dömpa & ʽdumpʼ & \textbf{87,8} & 0,5 & 1,3 & 10,4\\
installa & ʽinstallʼ & \textbf{85,8} & 6,6 & 2,5 & 5,1\\
brodkasta & ʽbroadcastʼ & \textbf{85,2} & 2,8 & 1,0 & 11,0\\
sneika & ʽsneakʼ & \textbf{55,0} & 5,1 & 1,5 & 38,4\\
flexa & ʽthrow aroundʼ & \textbf{54,7} & 8,7 & 1,0 & 35,6\\
slumma & ‘kick (a ball)’ & \textbf{47,8} & 8,4 & 3,8 & 40,0\\
\lspbottomrule
\end{tabularx}
\end{table}

Although the acceptance rate for dative ranges from 47,8\% to 93,1\%, the dative was chosen at least five times more often than the accusative for every verb here. There were also significant differences with respect to the last option (neither), with high frequency verbs like \textit{dánlóda}, \textit{droppa}, and \textit{installa} scoring below 8\% but verbs of low frequency like \textit{sneika}, \textit{flexa} and \textit{slumma} scoring above 35\%. We take this to show that the lowest scoring verbs were the most familiar to the participants and vice versa. The same trend was also evident in other tables in this paper.

As discussed in more detail below, all the verbs listed in \tabref{tab:jonsson:1} encode some kind of motion of the object referent. This is also true of all the verbs in the second survey that showed a clear preference for a dative object:

\begin{table}
{\caption{\label{tab:jonsson:2}Verbs taking a dative object in survey 2}}
\begin{tabularx}{\textwidth}{XXrrrr}
\lsptoprule
Verb & Gloss & \textbf{DAT} & ACC & Both & Neither\\
\midrule
pósta & ʽpost (online)ʼ & \textbf{96,0} & 2,0 & 0,5 & 1,5\\
gúffa & ʽeat greedilyʼ & \textbf{87,1} & 8,0 & 2,7 & 2,2\\
sjera & ʽshare (online)ʼ & \textbf{81,3} & 6,0 & 0,7 & 12,0\\
blasta & ʽplay loudly, blastʼ & \textbf{76,6} & 12,4 & 3,0 & 8,0\\
átsorsa & ʽoutsourceʼ & \textbf{64,2} & 11,2 & 5,5 & 19,1\\
\lspbottomrule
\end{tabularx}
\end{table}

The test sentences with the top three verbs in \tabref{tab:jonsson:1} are shown in \REF{ex:jonsson:2} below:


\ea%2
    \label{ex:jonsson:2}

\ea
\gll  Ertu  búin  að  dánlóda  nýju  myndinni  með  Ryan  Gosling?\\
   are.you  done  to  download  new.\textsc{dat}  the.movie.\textsc{dat}  with  Ryan  Gosling?\\
\glt `Have you downloaded the new movie with Ryan Gosling?'
 

\ex
\gll   Ég  held  að  ég  verði  að  droppa  þessu  námskeiði.\\
 I  think  that  I  must  to  drop  this.\textsc{dat}  course.\textsc{dat}\\
\glt `I think that I must drop this course.'
 

\ex
\gll   Djöfull  er  bossinn  duglegur  að  dömpa  á  þig  verkefnum.\\
 bloody  is  the.boss  relentless  to  dump  on  you  tasks.\textsc{dat}\\
\glt `How relentlessly the boss dumps tasks on you!'
\z
\z

The motion verbs \textit{dánlóda}, \textit{droppa} and \textit{dömpa} can be replaced here by the dative verbs \textit{hlaða} \textit{niður} ʽdownloadʼ, \textit{sleppa} ʽrelease, skipʼ and \textit{demba} ʽdump, pourʼ, respectively, without any change in meaning.\textstyleFootnoteSymbol{} Hence, it is impossible to determine if the datives in (2a--c) are due to isolate attraction or caused motion. The same applies to \textit{brodkasta}, a verb of emission which has a translational substitute in the dative verb \textit{sjónvarpa} ʽbroadcastʼ. However, the dative object with \textit{sneika} and \textit{sjera} is presumably due to isolate attraction by \textit{lauma} ʽsneakʼ and \textit{deila} ʽshare, divideʼ, both of which take a dative object.

Other verbs in Tables 1 and 2 do not have a translational substitute taking a dative object in the traditional vocabulary of Icelandic, e.g. \textit{installa}, \textit{pósta}, and \textit{gúffa}. All these verbs encode motion of the object, although not in a literal sense, except perhaps \textit{gúffa}. The relevant test sentences are shown in \REF{ex:jonsson:3}: 


\ea%3
    \label{ex:jonsson:3}
\ea\label{ex:jonsson:3a}
\gll   Þú   þarft   að   installa   Office   pakkanum.\\
     you   need   to   install   Office   the.package.\textsc{dat}\\
\glt `You need to install the Office package.'
 
\ex\label{ex:jonsson:3b}
\gll   Helga  póstaði  ótrúlega  skemmtilegu  myndbandi  á  vegginn  minn  áðan.\\
 Helga   posted   incredibly   entertaining.\textsc{dat}   video.\textsc{dat}   on   the.wall   my   just\\
\glt `Helga just posted an incredibly funny video on my wall.'
 

\ex\label{ex:jonsson:3c}
\gll   Af hverju eru allir farnir að gúffa í sig chiafræjum?\\
 from what are all started to shovel in \textsc{refl} chia.seeds.\textsc{dat}\\
\glt `Why has everybody started to eat chia seeds like crazy?'
\z
\z

The sense of motion is quite clear with \textit{pósta} since the meaning can be paraphrased roughly as ʽplace (text, picture, video etc.) on a website to make available to othersʼ. Matters are more complicated with \textit{installa} because this verb describes the process of getting a software program ready for use and that does not involve movement in any obvious sense. However, since programs are usually downloaded from the internet before they are installed, we think that native speakers see \textit{installa} as a process that includes downloading from the internet. This is supported by the fact that a directional PP like \textit{á} \textit{tölvuna} \textit{þína} ‘to your computer’ can be added in \REF{ex:jonsson:3a} to express the final location of the program. Hence, the object of \textit{installa} gets dative case just like the object of \textit{dánlóda}. 

The verb \textit{gúffa} is obligatorily accompanied by the directional preposition \textit{í} ʽinʼ plus a simple reflexive bound by the subject. Thus, it seems that the verb itself encodes caused motion whereas the directional PP denotes where the food ends up. Examples like \REF{ex:jonsson:3c} describe putting food quickly and/or greedily into the mouth but the food is not necessarily consumed. This is shown in \REF{ex:jonsson:4} below, which is not a contradiction in our judgment: 

 
\ea%4
    \label{ex:jonsson:4}
\gll     Hann   gúffaði   í   sig   kökum   en   skyrpti   þeim   út   í    laumi.\\
  he   shovelled   in   \textsc{refl}   cakes.\textsc{dat}   but   spat   them\textsc{.dat}   out   in   secret\\
\glt `He ate cookies like crazy but spat them out secretly.'
\z

This is not possible with ingestion verbs like \textit{éta} ʽeatʼ or \textit{borða} ʽeatʼ, both of which take an accusative object. Unlike \textit{gúffa}, these verbs encode consumption of food but not movement into the mouth. Of course, a sentence like \REF{ex:jonsson:3c} would generally be understood as saying that people eat a lot of chia seeds but this is through real world knowledge as it is not customary to put food into one‘s mouth without eating it. The contrast between \textit{gúffa} and \textit{éta} or \textit{borða} suggests that motion vs. consumption of food may be the critical factor determining object case with verbs of ingestion, but this will have to be an issue for future investigation.

The verbs that still require some comment are \textit{flexa}, \textit{átsorsa}, \textit{slumma} and \textit{blasta}. The verb \textit{flexa} means to throw money around to show off so the sense of motion is quite clear. The same is true of \textit{átsorsa} which typically involves moving a task from one company to another. The verb \textit{blasta} denotes sound emission and emission of all kinds is a type of ballistic motion \citep{Jónsson2013a}. Finally, \textit{slumma} is clearly a verb of ballistic motion so only dative is possible (see \citealt{Jónsson2013a} for more examples and discussion of similar verbs).

\subsection{Accusative objects} %3.3 /
\label{sec:jonsson:3.3}

Some verbs in the study received a significantly higher score for accusative than dative. These verbs are listed in the following table:  

\begin{table}
{\caption{\label{tab:jonsson:3}Verbs taking an accusative object} }
\begin{tabularx}{\textwidth}{Xlrrrr} 
\lsptoprule
Verb & Gloss & DAT & \textbf{ACC} & Both & Neither\\
\midrule 
fiffa & ʽfix (illegally)ʼ & 1,5 & \textbf{94,5} & 0,0 & 4,0\\
gúgla & ʽgoogleʼ & 4,6 & \textbf{93,6} & 0,5 & 1,3\\
krakka & ʽunlock, crackʼ & 1,3 & \textbf{86,2} & 2,3 & 10,2\\
gólfa & ʽpress to the floorʼ & 3,5 & \textbf{74,9} & 0,7 & 20,9\\
skrína & ʽscreen, keep an eye onʼ & 1,3 & \textbf{74,1} & 0,0 & 24,6\\
gramma & ʽput on Instagramʼ & 8,5 & \textbf{66,9} & 3,0 & 21,6\\
jáa & ʽsearch for on ja.isʼ & 7,1 & \textbf{58,8} & 0,3 & 33,8\\
offa & ʽturn off, shockʼ & 9,4 & \textbf{58,0} & 0,5 & 32,1\\
domma & ʽdominateʼ & 8,5 & \textbf{52,5} & 0,0 & 39,0\\
\lspbottomrule
\end{tabularx}
\end{table}

For most of these verbs, it is intuitively clear that the direct object does not undergo motion in any sense. Consider e.g. the following test examples of the verbs \textit{krakka}, \textit{offa} and \textit{fiffa}:

 
\ea%5
    \label{ex:jonsson:5} 
\ea
\gll  Geta  þeir  krakkað  hvaða  síma  sem  er?\\
   can  they  hack  any  phone.\textsc{acc}  which  is\\
\glt `Can they hack any phone whatsoever?'
 
\ex
\gll   Þetta  attitude  offaði  mig   alveg.\\
 this  attitude  turned.off  me.\textsc{acc}   completely\\
   \glt`This attitude shocked me completely.'
 

\ex
\gll   Þau  lentu  í  peningavandræðum  og  byrjuðu  að  fiffa  bókhaldið.\\
 they  landed  in  money.trouble  and  started  to  fix  the.book-keeping.\textsc{acc}\\
\glt `They got into financial difficulties and started to fiddle with the numbers.'
\z
\z

The verbs \textit{gramma} and \textit{gólfa} stand out in \tabref{tab:jonsson:3} because they seem to express motion of the object. The test examples with these verbs are provided in \REF{ex:jonsson:6}:


 
\ea%6
    \label{ex:jonsson:6}
    \ea
\gll  Hann  gólfaði  bensíngjöfina  þegar  hann  var  kominn  út  á  hraðbrautina.\\
   he  pushed.down  the.foot.pedal.\textsc{acc}  when  he  was  come  out  to  the.highway\\
   \glt `He started to speed when he entered the highway.' 

\ex
\gll   Er  einhver  búinn  að  gramma  nýja  tíuþúsundkallinn?\\
 is  someone  done  to  instagram  new  10.000.krónur.bill.\textsc{acc}\\
\glt `Has someone put the new 10.000 krónur bill on Instagram?'
\z
\z


These verbs are crucially different from the dative verb \textit{pósta} discussed in \sectref{sec:jonsson:3.2} in that they name the final location of the object. By contrast, \textit{pósta} does not specify the destination of the moved file and thus is compatible with a directional PP, as in \REF{ex:jonsson:3b}. The verb \textit{gólfa} is derived from the noun \textit{gólf} ʽfloorʼ and the meaning is literally ʽpush to the floorʼ and \textit{gramma} derives from the noun \textit{Instagram} and means ʽput on Instagramʼ. Hence, the final location of the object is encoded rather than movement to that location. Verbs of this kind are referred to as pocket verb by \citet{Levin1993} and they all take an accusative object in Icelandic, e.g. \textit{axla} ʽshoulderʼ, \textit{bóka} ʽbookʼ, \textit{fangelsa} ʽimprisonʼ, \textit{hýsa} ʽhouseʼ, \textit{jarða} ʽburyʼ, \textit{ramma} ʽframeʼ and \textit{slíðra} ʽsheatheʼ. 

\section{Case variation}  %4. /
\label{sec:jonsson:4}

Some verbs in the present study displayed significant variation between accusative and dative. Under our hypothesis, case variation is expected whenever a verb is semantically ambiguous in a way that is linked to caused motion or the existence of a translational substitute taking a dative object. However, as we will see, this does not necessarily entail a difference in truth conditions.

For convenience, the verbs examined here will be referred to as DAT/ACC-verbs. The discussion of these verbs is divided into two subsections below, montransitive verbs and ditransitive verbs, since they give rise to somewhat different issues.

\subsection{Monotransitive verbs} %4.1 /
\label{sec:jonsson:4.1}

The following table lists monotransitive DAT/ACC-verbs in the two surveys. As can be seen here, the dative outscored the accusative with six verbs but the reverse preference was found with four verbs: 

\begin{table}
{\caption{\label{tab:jonsson:4}Monotransitive verbs taking both dative and accusative object} }
\begin{tabularx}{\textwidth}{XXrrrr}
\lsptoprule
Verb & Gloss & \textbf{DAT} & ACC & Both & Neither\\
\midrule
bleima & ʽblameʼ & \textbf{48,8} & 32,3 & 1,2 & 17,7\\
krassa & ʽcause to crash, ruinʼ & \textbf{48,6} & 27,0 & 2,3 & 22,1\\
neimdroppa & ʽnamedropʼ & \textbf{42,5} & 30,9 & 3,2 & 23,4\\
mæna & ‘collect, mine’ & \textbf{41,5} & 17,9 & 2,5 & 38,1\\
syngja & ‘tell (a secret); sing’ & \textbf{36,1} & 15,2 & 1,5 & 47,2\\
farta & ‘drive fast’ & \textbf{34,1} & 13,0 & 0,5 & 52,4\\
rippa & ʽcopy (illegally)ʼ & 19,3 & \textbf{59,0} & 1,8 & 19,9\\
hannesa & ʽsteal (a text)ʼ & 19,1 & \textbf{51,9} & 3,3 & 25,7\\
peista & ʽpasteʼ & 44,8 & \textbf{47,5} & 4,7 & 3,0\\
bekka & ʽlift in a bench pressʼ & 32,1 & \textbf{38,6} & 7,2 & 22,1\\
\lspbottomrule
\end{tabularx}
\end{table}

All the DAT/ACC-verbs listed here, except \textit{peista}, scored over 15\% for the last option (neither) and this reflects the low frequency of these verbs. Arguably, infrequent novel verbs have not been used enough to acquire an established meaning across speakers. As a result, they may have different intuitions about the meaning of these verbs, including the presence or absence of the factors that license a dative object. Admittedly, our data on the meaning of monotransitive DAT/ACC-verbs for different speakers is rather limited and our remarks below will inevitably be somewhat speculative. Still, we hope to show that these verbs are ambiguous in ways which affects object case, unlike the verbs discussed in \sectref{sec:jonsson:3} and listed in \tabref{tab:jonsson:3}--\tabref{tab:jonsson:4}. 

Under our analysis, the dative variant with DAT/ACC-verbs that do not express caused motion must be due to isolate attraction. Speakers that select a dative object with \textit{bleima}, \textit{krassa} and \textit{mæna} do so because they see the dative verbs \textit{kenna} \textit{um} ʽblameʼ, \textit{rústa} ʽruinʼ, and \textit{safna} ʽcollectʼ as translational substitutes. As for \textit{rippa} and \textit{hannesa}, these verbs have a translational substitute in the dative verb \textit{stela} ʽstealʼ for some speakers. For other speakers, these two verbs denote copying without stealing, in which case \textit{stela} is not a translational substitute and consequently the object must be accusative. 

The verbs \textit{neimdroppa}, \textit{peista} and \textit{bekka} are among the DAT/ACC-verbs for which the dative variant is licensed by caused motion. The test examples with these verbs are shown in \REF{ex:jonsson:7}:

 
\ea%7
\label{ex:jonsson:7}
\ea\label{ex:jonsson:7a} 
\gll  Hún  byrjaði  strax  að  neimdroppa  einhverjum  böndum  sem  hún  hafði  djammað  með.\\
   she  started  right.away  to  namedrop  some.\textsc{dat}  bands.\textsc{dat}  which  she  had  partied  with\\
\glt { }
 
 
\ex\label{ex:jonsson:7b}
\gll   Hún  byrjaði  strax  að  neimdroppa  einhver  bönd  sem  hún  hafði  djammað  með.\\
 she  started  right.away  to  namedrop  some.\textsc{acc}  bands.\textsc{acc}  which  she  had  partied  with\\
 \glt `She started immediately to namedrop bands she had partied with.'
 
 
\ex\label{ex:jonsson:7c}
\gll   Tölvan  frýs  alltaf  þegar  ég  reyni  að  peista  myndinni  í  Word.  \\
 the.computer  freezes  always  when  I  try  to  paste  the.picture.\textsc{dat}  into  Word  \\
\glt { }
  
\ex\label{ex:jonsson:7d}
\gll   Tölvan  frýs  alltaf  þegar  ég  reyni  að  peista  myndina  í  Word.\\
 the.computer  freezes  always  when  I  try  to  paste  the.picture.\textsc{acc}  into  Word\\
`The computer always freezes when I try to paste the picture into a Word document.'%\todo[inline]{translation missing}
 

\ex\label{ex:jonsson:7e}
\gll   Þessi  gella  getur  bekkað  150  kílóum/kíló.\\
 this  chick  can  bench  150  kilos.\textsc{dat/acc}\\
\glt `This chick can bench 150 kilos.'
\z
\z


For some speakers, \textit{neimdroppa} is more or less synonymous with the accusative verbs \textit{nefna} ʽmentionʼ and \textit{telja} \textit{upp} ʽrecount, listʼ. As expected, only accusative is possible in this sense. For other speakers,  \textit{neimdroppa} means to mention something in a way that is similar to dropping, i.e. in a sneaky way as to show off by mentioning something or someone famous. This use is associated with a dative object. Thus, the variation between accusative and dative boils down to the presence or absence of caused motion in a metaphorical sense as part of the lexical semantics of \textit{neimdroppa}.

The case variation with \textit{peista} does not correlate with any obvious truth conditional difference between the two variants. Still, it is clear that the object must be dative if \textit{peista} is interpreted as a verb of motion in the sense of moving a piece of text or a picture from one file to another or within the same file. Alternatively, if \textit{peista} encodes the resulting attachment rather than motion, only accusative is possible. In the latter case, \textit{peista} is very much like the accusative verb \textit{líma} ʽglueʼ. For discussion of other similar examples of case variation, see \citet{Jónsson2013a}.

The verb \textit{bekka} takes a dative object if it encodes motion of the object, as reflected by the gloss ʽlift in a bench pressʼ. In that sense, \textit{bekka} is similar to the dative verb \textit{lyfta} ʻliftʼ. Still, \textit{lyfta} is not a translational substitute in \REF{ex:jonsson:7e} because replacing \textit{bekka} by \textit{lyfta} would yield a slightly different claim. The accusative variant may be due to the fact that \textit{bekka} in \REF{ex:jonsson:7e} is not only about moving a weight in a specified direction but also exerting great physical force against gravity. The verb \textit{bekka} can also be used with objects that do not undergo movement, e.g. \textit{bekka} \textit{heimsmet} (literally ʻbench a world recordʼ), in which case only accusative is possible. 

That leaves us with \textit{farta} and \textit{syngja}. These verbs had the highest score of all the DAT/ACC-verbs for the last option (neither), indicating that many native speakers were not familiar with these verbs in the relevant meaning. The verbs were tested in the following examples: 


\ea
    \label{ex:jonsson:8}
\ea%8
\gll  Þótt  þetta  sé  hálfgerður  dótabíll  er  ekkert  leiðinlegt  að  farta  honum/hann\\
   although  this  is  halfmade  toycar  is  not  boring  to  {drive.fast}  him.\textsc{dat/acc}\\
\glt `Although this is a kind of a toycar, it is fun to speed.'

\ex  \label{ex:jonsson:8b}
\gll   Hann  var  ekki  lengi  að  syngja  þessu/þetta  að  lögreglunni...  \\
 he  was  not  long  to  sing  this.\textsc{dat/acc}  to  the.police  \\
\glt `I did not take him long to tell the police the whole story.'
\z
\z

The dative variant with \textit{farta} encodes caused motion of a vehicle but the accusative is more difficult to explain. Perhaps it signals that the agent steps on the accelerator so that the car produces a sound similar to farting. This does not necessarily involve caused motion because this sound can be produced even if the car is not moving, e.g. if it is stuck in snow. 

In its basic sense, \textit{syngja} ʽsingʼ is a performance verb which takes an accusative object like all other such verbs in Icelandic, e.g. \textit{blístra} ʽwhistleʼ, \textit{flytja} ʽperformʼ, \textit{leika} ʽplayʼ, \textit{lesa} ʽreadʼ, \textit{raula} ʽhumʼ, \textit{spila} ʽplayʼ, \textit{tóna} ʽchantʼ and \textit{þylja} ʽreciteʼ. This basic meaning may have lead some speakers to chose accusative with \textit{syngja} in \REF{ex:jonsson:8b}. However, \textit{syngja} describes a manner of speaking in \REF{ex:jonsson:8b} and all such verbs take a dative object in Icelandic if they express the exchange of information. These verbs include \textit{blaðra} ʽbabbleʼ,  \textit{gaspra} ʽbabbleʼ, \textit{hreyta} ʽtoss (words)ʼ, \textit{hvísla} ʽwhisperʼ, \textit{kjafta} ʻtell (a secret)ʼ, \textit{muldra} ʽmumbleʼ and \textit{stynja} \textit{upp} ʽmoanʼ. Thus, it can be argued that \textit{syngja} in \REF{ex:jonsson:8b} encodes motion of the message conveyed to the police. 

\subsection{Ditransitive verbs} %4.2 /
\label{sec:jonsson:4.2}

Three ditransitive verbs were tested in the present study and they all displayed considerable variation between accusative and dative with the direct object. The participants were not asked about the indirect object since dative is the only possibility there for new verbs. As shown in \tabref{tab:jonsson:5}, the ditransitive verbs had virtually the same acceptance rate for both cases:

\begin{table}
{\caption{\label{tab:jonsson:5}Ditransitive verbs taking both dative and accusative object}}
\begin{tabularx}{\textwidth}{XXrrrr}
\lsptoprule
Verb & Gloss & DAT & ACC & Both & Neither\\
\midrule
græja & ʽprocure; take care ofʼ & \textbf{40,6} & 37,1 & 1,7 & 20,6\\
smessa & ʽsend by smsʼ & \textbf{36,9} & 34,1 & 4,6 & 24,4\\
meila & ʽe-mailʼ & 36,3 & \textbf{39,8} & 3,5 & 20,4\\
\lspbottomrule
\end{tabularx}
\end{table}

Verbs taking a dative indirect object and an accusative direct object (DAT-ACC verbs) constitute by far the biggest class of ditransitive verbs in Icelandic (see \citealt{ZaenenMalingThráinsson1985} and \citealt{Jónsson2000}). This class also includes most of the canonical ditransitive verbs in Icelandic, e.g. \textit{gefa} ʽgiveʼ, \textit{lána} ʽlendʼ, \textit{rétta} ʽpassʼ, \textit{segja} ʽtellʼ, \textit{selja} ‘sell’, \textit{senda} ʽsendʼ and \textit{sýna} ʽshowʼ. The DAT-DAT class is much smaller and contains only a handful of typical ditransitive verbs, including \textit{lofa} ʽpromiseʼ, \textit{skila} ʽreturnʼ and \textit{úthluta} ʽallotʼ. 

In view of this, one would expect new ditransitive verbs to exhibit only DAT-ACC, unless the verb in question has a translational substitute with DAT-DAT. However, as discussed in more detail below, the DAT-DAT class relates to caused motion in a way that is similar to what we have already shown for monotransitive verbs. This class is also theoretically interesting in that the double dative strongly suggests two different sources for the two datives, e.g. an applicative head for the indirect object and some other functional head for the direct object. 

We will start our discussion with \textit{græja} because it is more straightforward than the other two verbs. The relevant test examples are shown in \REF{ex:jonsson:9} below:

 
\ea%9
    \label{ex:jonsson:9}  
\ea
\gll  Þú  græjar  þér  bara  útilegudrasli  ef  þú  átt  það  ekki\\
   you  procure  you.\textsc{dat}  just  camping.stuff.\textsc{dat}  if  you  own  it  not\\

 
 
\ex
\gll   Þú  græjar  þér  bara  útilegudrasl  ef  þú  átt  það  ekki\\
 you  procure  you.\textsc{dat}  just  camping.stuff.\textsc{acc}  if  you  own  it  not\\
 \glt `You just get yourself camping stuff if you don‘t have it.'
\z
\z

For \textit{græja}, the double dative is due to the fact that this verb has, at least for some speakers, a translational substitute in the DAT-DAT verb \textit{redda} ʻprocure, take care ofʼ. In that sense, \textit{græja} indicates that something was obtained in a casual or hurried way. Speakers selecting DAT-ACC understand \textit{græja} presumably more like \textit{útvega} ʻprocureʼ, a DAT-ACC verb which has a more general meaning than \textit{redda} because it is completely neutral with respect to how the direct object is procured.

The test examples for the verbs \textit{meila} and \textit{smessa} are given in \REF{ex:jonsson:10}:

\ea%10
    \label{ex:jonsson:10}  
\ea
\gll  Gætirðu  meilað  mér  þessu/þetta  sem  fyrst?\\
   could.you  e-mail  me.\textsc{dat}  this.\textsc{dat/acc}  as  first\\
\glt `Could you e-mail this to me as soon as possible?' 

\ex
\gll   Geturðu  ekki  bara  smessað  honum  reikningsnúmerinu  okkar?\\
 can.you  not  just  sms  him.\textsc{dat}  the.account.number.\textsc{dat}  our\\


\ex
\gll   Geturðu  ekki  bara  smessað  honum  reikningsnúmerið  okkar?\\
 can.you  not  just  sms  him.\textsc{dat}  the.account.number.\textsc{acc}  our\\
\glt `Can‘t you just send him our account number by sms?' 
\z
\z
 
The verbs \textit{meila} and \textit{smessa} are verbs of instrument of communication and have no translational substitutes taking a dative object.  \citet{RappaportHovavLevin2008} claim that verbs of instrument of communication in English encode caused motion and the same is true for Icelandic. Both \textit{meila} and \textit{smessa} entail that the direct object changes location in electronic space, although it need not reach its intended goal (see \citealt{Beavers2011} on \textit{e-mail}). These verbs also encode caused possession as the indirect object must be capable of possession and thus cannot be a location. This is a standard diagnostic to show that the double object construction in English encodes caused possession (see \citealt{Green1974} and much subsequent work). Thus, the examples in (11a--b) are ungrammatical unless \textit{Berlin} refers to the people working in an office in Berlin: 

 
\ea%11
    \label{ex:jonsson:11} 
\ea
\gll  *Gætirðu  meilað  Berlín  þessu/þetta  sem  fyrst?\\
   \hspaceThis{*}could.you  e-mail  Berlin.\textsc{dat}  this.\textsc{dat/acc}  as  first\\
\glt `Could you e-mail Berlin this as soon as possible?' 

\ex
\gll   *Geturðu  ekki  smessað  Berlín  númerinu/númerið?\\
 \hspaceThis{*}can.you  not  sms  Berlin.\textsc{dat}  the.number.\textsc{dat/acc}\\
\glt `Can‘t you send Berlin the number by sms?'
\z
\z

This ambiguity means that native speakers are faced with two options when using \textit{meila} and \textit{smessa} as double object verbs, to treat them as DAT-DAT verbs encoding caused motion or DAT-ACC verbs encoding caused possession, apparently without any difference in truth conditions. 

The intended goal of verbs of instrument of communication can be expressed not only as a dative DP but also as a PP headed by the preposition \textit{til} ʻtoʼ \citep[128--132]{Barðdal2008} but this does not effect the case variation with the direct object:

 
\ea%12
    \label{ex:jonsson:12}  
\ea
\gll  Gætirðu  meilað  þessu/þetta  til  mín?\\
   could.you  e-mail  this.\textsc{dat/acc}  to  me.\textsc{gen}\\
\glt `Could you e-mail this to me?' 

\ex
\gll   Geturðu  smessað  númerinu/númerið  til  hennar?\\
 can.you  sms  the.number.\textsc{dat/acc}  to  her.\textsc{gen}\\
\glt `Can you send her the number by sms?'
\z
\z

This shows that \textit{meila} and \textit{smessa} encode caused motion in \REF{ex:jonsson:12} because only such verbs allow the goal to be expressed in a PP headed by \textit{til} in Icelandic. However, caused possession is also encoded in examples like \REF{ex:jonsson:12} because the goal must be capable of possession:

 
\ea%13
    \label{ex:jonsson:13}  
\ea
\gll  *Gætirðu  meilað  þessu/þetta  til  Berlínar?\\
   \hspaceThis{*}could.you  e-mail  this.\textsc{dat/acc}  to  Berlin.\textsc{gen}\\
\glt `Could you e-mail this to Berlin?' 

\ex
\gll   *Geturðu  smessað  númerinu/númerið  til  Berlínar?\\
 \hspaceThis{*}can.you  sms  the.number.\textsc{dat/acc}  to  Berlin.\textsc{gen}\\
\glt `Can you send the number by sms to Berlin?'
\z
\z

In view of the discussion above, one remaining issue is why the traditional motion verb \textit{senda} ʻsendaʼ always takes an accusative direct object. While we cannot provide a definitive answer here, this may have to do with the fact that (a) this verb lacks a manner component and (b) it does not entail motion that starts with the agent of the action. For instance, a sentence like \textit{Jón} \textit{sendi} \textit{Maríu} \textit{bók} (ʻJohn sent Mary a bookʼ) may describe a situation where Jón orders a book from an internet company that delivers the book directly to Mary (see also \citealt{Beavers2011} on \textit{send} in English). Thus, the verb \textit{senda} appears to be more about causing something to reach some person or place in any conceivable way rather than motion per se. 

\section{Conclusions}
\label{sec:jonsson:5}

The results from the two large-scale surveys discussed in this paper show that a novel transitive verb in Icelandic takes a dative object if it (a) encodes some kind of caused motion of the object referent, or (b) has a translational substitute that takes a dative object. If neither (a) nor (b) holds, the object gets the default accusative case. 

It is usually rather straightforward to determine if condition (b) holds and our discussion of such cases has indeed been rather brief. It is more difficult to argue that caused motion licenses a dative object. Crucially, the concept of caused motion has to be understood very broadly to include not only movement of concrete objects but also various abstract objects, including electronic files or messages. 

Some of the novel verbs discussed here vary between dative and accusative object. This applies to some monotransitive verbs as well as the three ditransitive verbs tested. Under our analysis, this is expected if the relevant verb is semantically ambiguous such that the dative variant encodes caused motion or has a translational substitute taking a dative object. As argued in \sectref{sec:jonsson:4}, the predictions of our analysis are borne out although some questions remain concerning the meaning of some verbs for individual speakers. 
\section*{Acknowledgements}

We wish to thank two anonymous reviewers for constructive feedback on an earlier version of this paper. The usual disclaimers apply. This study was financially supported by a grant from the Icelandic Research Fund (Rannís). 
\sloppy\printbibliography[heading=subbibliography,notkeyword=this]\end{document}

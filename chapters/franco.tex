\documentclass[output=paper,colorlinks,citecolor=brown]{./langscibook} 
\author{Ludovico Franco\affiliation{}\lastand 
Paolo Lorusso\affiliation{}}
\title{Aspectual datives (and instrumentals)}
 
 
\abstract{Change the  abstract in chapters/myname.tex}
\maketitle

\begin{document}
 

\textbf{\textit{Abstract}}. \textit{Dative adpositions}\textbf{ }\textit{instantiate part-whole/inclusion (${\subseteq}$) relations that holds between the goal and the direct object in the thematic grids of ditransitives. We assume that the same primitive part-whole relation is found: i) when the dative adposition is used in locative contexts; ii) with genitive adpositions, as shown by the widespread genitive/dative syncretism across natural languages. Instrumental inflections/adpositions are also the instantiation of the same primitive part-whole relation, but they denote the reverse with respect to genitives/datives (${\supseteq}$). We describe progressive aspectual constructions involving adpositions, crosslinguistically. We propose that the dative adpositions found in progressive periphrases are the lexicalization of the same basic ‘part-whole/inclusion’ content: the part-whole relation does not hold between argumental/thematic material but between two events, one event being the time of reference which is ‘part of’ the time-frame of a second embedded event/set of events. The variation in the adpositions found with the Italian aspectual periphrases is accounted for in the terms of the ‘direction’ (${\subseteq}$) vs. (${\supseteq}$) of the inclusion primitive predicate that implies different interpretations:  progressive vs. prospective aspect, respectively.} 

\textbf{\textit{Keywords}}: \textit{dative, instrumental, aspect, progressive, prospective.}

\begin{itemize}
\item \section{Introduction: background and aims}
\end{itemize}

In recent work, \citet{ManziniSavoia2011}, \citet{ManziniFranco2016}, Franco \& Manzini (\citeyear{FrancoManzini2017Gen, FrancoManzini2017Ins}) propose that dative morphemes are part-whole/inclusion predicates (cf. Belvin \& den \citealt{Dikken1997}), notated (${\subseteq}$), whose basic context of occurrence can be illustrated for English \textit{to} in \REF{ex:key:1}.

\ea%1
    \label{ex:key:1}
    \gll\\
        \\
    \glt
    \z

           a.  I gave the books \textbf{to} Peter

b.   [\textsubscript{VP} gave [\textsubscript{PredP} the books [[\textsubscript{${\subseteq}$} \textbf{to}] Peter ]]]]

Following \citet{Kayne1984}, \citet{Pesetsky1995}, \citet{BeckJohnson2004}, \citet{Harley2002}, among others, we can assume that in \REF{ex:key:1} a possession/part-whole/inclusion relation holds between the dative (\textit{Peter}) and the theme of the ditransitive verb (\textit{the books}). 

\citet{ManziniSavoia2011}, Franco \& Manzini (2016, \citeyear{FrancoManzini2017Gen}) ascribe the same (${\subseteq}$) content to genitives. Consider English in \REF{ex:key:2a}. The \textit{of} preposition (or the \textit{’s} genitive ending) introduces a possession relation between the argument it selects, namely \textit{the woman} (the possessor), and the head of the DP, namely (\textit{the}) \textit{children} (the possessum). The content of the \textit{’s} case or the \textit{of} preposition is the same part/whole elementary predicate ${\subseteq}$ assumed for datives. Thus, in \REF{ex:key:2b} (${\subseteq}$) takes as its internal argument the sister DP (the possessor) and as its external argument the head N/D (the possessum) – saying that ‘the children’ is in the domain of inclusion of ‘the woman’. 

\ea%2
    \label{ex:key:2}
    \gll\\
        \\
    \glt
    \z

           a. The woman\textbf{’s} children/the children \textbf{of} the woman 

b. [\textsubscript{DP} the children [\textsubscript{PP${\subseteq}$} \textbf{of} the woman]] 

\citet{ManziniSavoia2011} argue that the widespread genitive/dative syncretism (e.g. in Romanian as in \REF{ex:key:3}) precisely corresponds to such a common lexicalization. This approach is not incompatible with languages like English with two separate lexicalizations for ‘to’ (dative) and ‘of’ (genitive). Simply genitive ‘of’ is specialized for DP-embedding of (${\subseteq}$) and dative ‘to’ for sentential embedding of (${\subseteq}$).\footnote{The part-whole (\textrm{${\subseteq}$}) proposal for genitives and datives has been further articulated in \citet{ManziniFranco2016, FrancoManzini2017Gen} in order to account for the fact that formally identical genitive/dative DPs display different interpretive behaviours – as well as for the fact that cross-linguistically, syntactico-semantic differences may result in different lexicalization pattern. For instance, while with Goal datives the (\textrm{${\subseteq}$}) relator establishes a relation between two arguments (namely the goal and the theme), with experience datives the (\textrm{${\subseteq}$}) relator introduces relation between an argument (experiencer) and an event (the VP) (cf. \citealt[230-231]{ManziniFranco2016}). This is in line with the Applicative literature (cf. \citep{Pylkkänen2008}, which assumes that the same Appl head (externalized by dative/oblique) can be attached to different points in the syntactic tree (High Appl \textit{vs}. Low Appl heads)} 

\ea%3
    \label{ex:key:3}
    \gll\\
        \\
    \glt
    \z

          a.   (I)-l   am   dat   băieț-i-l-\textbf{or}/     fet-e-l-\textbf{or}     him.it   I.have given   boy-mpl-def-obl/  girl-fpl-def-obl


‘I gave it to the boys/ girls’

b. pahar-ul     băieț -i-l-\textbf{or} /    fet-e-l-\textbf{or} 

glass-msg.def   boy-mpl-def-obl/  girl-fpl-def-obl 

‘the glass of the boys/ girls’ 

\citet{FrancoManzini2017b} extend the part-whole proposal to the other oblique item, most likely to occur as a case inflection in natural languages \citep{Caha2009}, namely the instrumental; in English the core lexicalization of the instrumental is by the adposition \textit{with.} We employ here the cover term ‘instrumental’ for all the semantic values that can be rendered with \textit{with-}like morpheme (cf. \citealt{StolzStrohUrdze2006}). Our starting point is the observation made by \citet{Levinson2011} that possession relations may be realized by \textit{with}, as illustrated in \REF{ex:key:4}. The relation in \REF{ex:key:4} is reversed with respect to that in \REF{ex:key:1}-(2), since the preposition \textit{with} embeds the \textit{possessum}, while the \textit{possessor} is the head of the DP.

\ea%4
    \label{ex:key:4}
    \gll\\
        \\
    \glt
    \z

           The woman with the children

\citet{FrancoManzini2017b} show that instrumental inflections/adpositions precisely denote the reverse relation with respect to genitives/datives, by which the possessum, rather than the possessor is in the oblique case. For instrumentals they therefore adopt the (${\supseteq}$) content and label, as illustrated in \REF{ex:key:5}. What \REF{ex:key:5} basically says is that the complement of with (‘the children’) is the possessum (a part) of the possessor (the whole) ‘the woman’.

\ea%5
    \label{ex:key:5}
    \gll\\
        \\
    \glt
    \z

           [\textsubscript{DP} the woman [\textsubscript{PP(${\supseteq}$)} with the children]]

They further claim that \textit{with}{}-type morphemes provide very elementary means of attaching (i.e. including) extra participants (themes, initiators, etc.) (in)to events (VP or vP predicates, cf. fn. 1) – with specialized interpretations derived by pragmatic enrichment (contextual, encyclopaedic) at the C-I interface, and extend the proposal to account for the observation that the instrumentals can be employed cross-linguistically in triadic verb constructions alternating with datives,\footnote{\citet{FrancoManzini2017b} also account for dative/instrumental syncretism (eventually including DOM objects), arguing that the inclusion predicate (\textrm{${\subseteq}$}) corresponding to ‘to’ or dative case and its reverse (\textrm{${\supseteq}$}), corresponding to ‘with’ or instrumental case, may reduce to an even more primitive content capable of conveying inclusion in either direction (cf. \sectref{sec:key:3}).}  as illustrated in \REF{ex:key:6}-(7) respectively with English and Persian examples. 

\ea%6
    \label{ex:key:6}
    \gll\\
        \\
    \glt
    \z

            a. He presented his pictures \textbf{to} the museum    [dative]

b. He presented the museum \textbf{with} his pictures  [instrumental]

\ea%7
    \label{ex:key:7}
    \gll\\
        \\
    \glt
    \z

           a.   Pesar   sang-ro   \textbf{be}   sag   zad     [dative]

boy   stone-DOM   to   dog   hit.pst.3sg 

‘The boy hit the dog with the stone’ 

  b.   Pesar   sag-ro     \textbf{ba}   sang   zad [instrumental]

  boy   dog-\textsc{dom}  with   stone   hit.pst.3sg 

  ‘The boy hit the dog with the stone’

In this paper, we focus on the adpositional morphemes surfacing in aspectual periphrases in Italian and beyond. We precisely concentrate on imperfective/progressive periphrases. Our main claim is that the ‘dative’ morpheme in \REF{ex:key:8}, which happens to be involved in the encoding of progressive aspect in many Romance varieties \citep{ManziniLorussoSavoia2017} and beyond (e.g. \citealt{Jóhannsdóttir2012} for Icelandic) lexicalizes the same basic ‘part-whole/inclusion’ content illustrated above. Notice that also dative morphemes introducing modal periphrases have been analysed as inclusion/part-whole relational devices in the recent literature (cf. \citealt{BjorkmanCowper2016}, Tsedryk, this volume). 

Following \citet{BerwickChomsky2011}, we take the lexicon to be the locus of externalization, pairing syntactico-semantic and phonological content: we assume a steady (${\subseteq}$) signature for all the occurrences of the ‘dative’ \textit{a} (to, at) adposition of Italian. In \REF{ex:key:8}, basically, we might say that a (${\subseteq}$) part/whole relation hold of event pairs, saying that one event is ‘part of’ (or \textit{a stage of,} cf. \citealt{Landman1992}) of a second event – or rather a set of events/an event type. Specifically, we may say that the event which is introduced within the matrix (finite) verb phrase is anchored to the time of reference (or viewpoint, cf. \citealt{Comrie1976}, or the utterance time, cf. Higginbotham, 2009) and is ‘part of’ the embedded event introduced by the (${\subseteq}$) relator. 

\ea%8
    \label{ex:key:8}
    \gll\\
        \\
    \glt
    \z

           a.  Gianni sta/è \textbf{a} studiare

‘Gianni is studying’

b.  [\textsubscript{IP/TP} Gianni è [(${\subseteq}$) \textbf{a} [\textsubscript{VP} studiare ]]]

This study is not aimed at providing any sort of formal semantic characterization of progressive aspect: rather, it is limited to a morphosyntactic account of the occurrences of (${\subseteq}$) relators in aspectual periphrases. However, we must note that the idea of a part-whole rendering for progressives is far from being new. \citet[16]{Comrie1976} argues that: ‘perfectivity indicates the view of a situation as a single whole (…) while the imperfective pays essential attention to the internal structure of the situation’. Comrie’s approach pays attention to the internal temporal structure of the event, proposing that, in a sense, the perfective–imperfective contrast can be accounted for in terms of a  whole vs. structured time-frame of the event which in our terms, can be described as an whole \textit{vs}. part–whole contrast. \citet{Bach1986} further argues that a progressive operator in the verbal domain is the counterpart of the partitive operator in the nominal domain, both instantiating a part-whole/sub-set relation. \citet{Filip1999} is even more radical in claiming that: ‘the semantic core of many, possibly all, aspectual systems can be characterized in terms of the basic mereological notions ‘part’ and ‘whole’’ \citep[158]{Filip1999}. Given this, we think that translating a part-whole relational content for (progressive) aspect into morphosyntax is a welcome result.  

This quite trivial claim has at least two non-trivial consequences. First, the idea of a part-whole syntax for progressives stands against the widespread idea (both within the typological and theoretical literature) that progressives are cross-linguistically realized in the form of a locative predication (\citealt{MateuAmadas1999, BybeePerkinsPagliuca1994, DemirdacheUribe-Etxebarria1997}).  Second, the idea of an aspectual (${\subseteq}$) relator seems \textit{prima facie} to be inadequate to consistently represent progressives in Romance. There are, in fact, Romance languages where no locative/dative preposition is found and the most common morphosyntactic ‘progressive’ device is the ‘\textsc{be plus} gerund’ periphrasis, as illustrated in \REF{ex:key:9} for Italian and Spanish.

\ea%9
    \label{ex:key:9}
    \gll\\
        \\
    \glt
    \z

           a.  Gianni sta studiando         \textit{Italian}      b.   Juan esta estudiando~      \textit{Spanish}

We aim to show that the encoding of progressive aspectual relations by means of adpositional devices does not rely on a primitive locative content of the semantics they express (and of their mapping into syntax). Rather, we will show that adposition-based aspectual periphrases share a primitive relation of ‘inclusion’ (the same relation which is at work with dative/genitives) of an event within a set of events or between the reference time and the time-frame of an event/set of events. We will substantiate this claim with a set of cross-linguistic examples in which the expression of progressive meaning relies on \textit{with}{}-like adpositions and HAVE predicates, which -\textit{contra} previous assumptions (\citealt{Freeze1992}, Den \citealt{Dikken1998})- seem to have a \textit{bona fide} non-locative value, as demonstrated in \citet{Levinson2011}. We will then provide a morphosyntactic analysis of Italian progressive periphrases, assuming that gerunds encode a covert (${\subseteq}$) operator which is compatible with  a prepositional value (\citealt{Gallego2010}, \citealt{Franco2015}). We will further show that the (${\subseteq}$)/(${\supseteq}$) divide in the oblique case systems of natural languages put forward by \citet{FrancoManzini2017b} for the encoding of argumental/thematic material is relevant also within the aspectual domain.

\begin{itemize}
\item \section{Non-locative progressives periphrases (with datives and beyond)}
\end{itemize}

Cross-linguistically, the same material can be recruited from the lexicon to encode argumental and aspectual relation among syntactic constituents. A case in point is the dative adposition \textit{a} in a full set of Romance varieties, which, for instance, happens to have a role also in the encoding of progressives, as illustrated in \REF{ex:key:10}, with Italian examples.

\ea%10
    \label{ex:key:10}
    \gll\\
        \\
    \glt
    \z

           a.  Gianni ha dato un libro \textbf{a} Maria    \textit{dative}

    ‘Gianni gave a book to Maria’

b.  Gianni è \textbf{a} lavorare         \textit{progressive}

  ‘Gianni is working’

In a number of typological and theoretical studies progressive aspect has been linked to locative constructions (\citealt{BybeeEtAl1994, MateuAmadas1999, DemirdacheUribe-Etxebarria1997}).  This is \textit{prima facie} a reasonable characterization also for Italian, given that, for instance, the goal of motion is commonly expressed by the same \textit{a} preposition, as in \REF{ex:key:11}.

\ea%11
    \label{ex:key:11}
    \gll\\
        \\
    \glt
    \z

           Gianni va \textbf{a} casa

  ‘Gianni goes (to) home’

\citet[129-130]{BybeeEtAl1994} write: “The majority of progressive forms in our database derive from expressions involving locative elements (...). The locative notion may be expressed either in the verbal auxiliary employed or in the use of postpositions or prepositions indicating location —‘at’, ‘in’, or ‘on’. The verbal auxiliary may derive from a specific postural verb (…), or it may express the notion of being in a location without reference to a specific posture but meaning only ‘be at’, ‘stay’, or, more specifically, ‘live’ or ‘reside’”.

Actually, this characterization for progressives appears to be too restrictive. A more general part-whole characterization devoid of locative endowments (at least for adpositions) seems more appropriate, once we consider a wider set of cross-linguistic data. Indeed, \textit{with}{}-like morphemes, which happen to encode possession but not location (cf. \citealt{Levinson2011}) and \textsc{have} predicates\footnote{\citet{Levinson2011}, arguing against locative approaches to possession, convincingly shows that a non-locative approach to HAVE is superior to locative accounts in explaining possession in Germanic languages and accounting for the variation in preposition incorporation (cf. \citealt{Kayne1993}, \citealt{Harley2002}) within Germanic (and beyond).} (which are not listed among the ‘locative’ auxiliaries in \citeauthor{BybeePerkinsPagliuca1994}’s sample), are recruited to encode progressives in various natural languages. In our term, such evidence shows that not only dative-like (${\subseteq}$) morphemes, as illustrated in \REF{ex:key:10}, but also instrumental-like (${\supseteq}$) relators can be employed to convey a progressive interpretation. We discuss this issue in some details in section 3, specifically devoted to Romance aspectual periphrases. 

Here, we concentrate on cross-linguistic data, relying on the exhaustive typological survey provided in \citet{Cinque2017} (who lists up to twenty different strategies unrelated to locatives employed to encode progressives among natural languages), illustrating a set of aspectual periphrases not involving locative constructions. 

For instance, there are many languages which employ a ‘be with’ strategy to encode progressive meaning. The \textit{with} adposition introduces an infinitive form of the lexical verb. This progressive periphrasis is widespread among African languages (cf. \citealt{Cinque2017}:556). Such periphrasis is actually similar to the Romance one illustrated in \REF{ex:key:8} and \REF{ex:key:10}, except for the relator selected from the lexicon (\textit{to} vs. \textit{with}).

\ea%12
    \label{ex:key:12}
    \gll\\
        \\
    \glt
    \z

           wó  \textbf{t$\varepsilon$}   na   jo   dandù.

3pl   with   inf   eat   honey

‘They are eating honey.’     \textit{Baka} (Kilian-\citealt{Hatz1992}:29)

\ea%13
    \label{ex:key:13}
    \gll\\
        \\
    \glt
    \z

           tu   li   \textbf{l’}   oku-lya 

we  be  with  inf-eat

‘We are eating’  \textit{Umbundu}   (\citealt{HeineKuteva2002}:83)

 \REF{ex:key:14}  ní.dí.   \textbf{na}.kuzà.ta

I.am  with.work.inf     

‘I am working’     \textit{Lunda} (Bantu; \citealt{Kawasha2003}:194)

In a number of Iranian languages, progressive aspect is encoded through a \textsc{have} + lexical verb periphrasis \citep[556]{Cinque2017}, as illustrated in \REF{ex:key:15} for Persian. Note that both verbs are inflected and agree with the external argument. This pattern is reminiscent of the one illustrated in \citet{ManziniLorussoSavoia2017} for Southern Italian varieties, in which the ‘dative’ \textit{a} introduces finite complements, as illustrated in \REF{ex:key:16} for Conversano (Apulia). Actually, the adpositional relator does not surface in all Southern Italian varieties, as shown in \REF{ex:key:17} for Monteparano (Apulia). We may posit a silent adpositional relator \citep{Kayne2003} both for Persian and the Monteparano dialect. As we have seen, \textsc{have} verbs are characterized with a general ‘inclusion’ content (cf. fn. 3), that \citet{ManziniFranco2016} assume to be analogous to \textit{with}{}-like (${\supseteq}$) morphemes.

\ea%15
    \label{ex:key:15}
    \gll\\
        \\
    \glt
    \z

           Ali dar\textbf{e} mikhor\textbf{e}/ (Man)   dar\textbf{am} mikhor\textbf{am}   \textit{Persian}

Ali has eat.3sg/(I)   have.1sg   eat.1sg

‘Ali is eating’/I’m eating’

\ea%16
    \label{ex:key:16}
    \gll\\
        \\
    \glt
    \z

          u stek   \textbf{a}   ffattsə /u   ste   \textbf{a}  ffeʃə   \textit{Conversano} 

it.cl   stay.1sg to   do.1sg / it.cl   stay.3sg   to do.3sg

‘I am doing it’/‘He/she is doing it’

\ea%17
    \label{ex:key:17}
    \gll\\
        \\
    \glt
    \z

          lu   ʃtɔ   ccamu         \textit{Monteparano} 

him.cl stay.1sg   call.1sg 

‘I am calling him’      

Quite interestingly, a pattern involving a \textsc{have/hold} verb periphrasis for progressive is present also in Italo-Romance, as illustrated in \REF{ex:key:18}-(19) for Abruzzi-Molise dialects \citep[555]{Cinque2017}. Again the (dative) relator may be overt \REF{ex:key:18} or not \REF{ex:key:19} (this time with infinitive lexical verbs, showing that the finiteness of the embedded lexical verb is actually independent from the overt presence of the relator). 

\ea%18
    \label{ex:key:18}
    \gll\\
        \\
    \glt
    \z

             Təném   \textbf{a}   mmagná  \citep[133]{Rohlfs1969}

we.hold   to   eat.inf      

‘We are eating’ 

\ea%19
    \label{ex:key:19}
    \gll\\
        \\
    \glt
    \z

            té     ppjjove     \citep[266]{Ledgeway2016} 

it.holds   rain.inf

‘It is raining’

Thus, in spite of the fact that many languages adopt ‘locative metaphors’ to encode progressive, the data introduced above suggest that a more general (${\subseteq}$)/(${\supseteq}$) inclusion/part-whole content instantiate the relation between events and event properties that a part of the formal semantics literature, briefly reviewed in section 1, identifies with progressive aspect.\textsuperscript{} What holds of examples like \REF{ex:key:16} and \REF{ex:key:18} including an overt relator, also holds of ‘bare’ finite embeddings - for instance with the Apulian variety of Monteparano in \REF{ex:key:17} or Persian \REF{ex:key:15} - or bare infinitive embedding as in \REF{ex:key:19}, if the content of the progressive (i.e. part/whole) is given in virtue of the selection of an abstract preposition \textit{à la} Kayne. 

Following \citet{ManziniSavoia2011}, Franco \& Manzini (\citeyear{FrancoManzini2017Gen, FrancoManzini2017Ins}), we see no reason why spatial meanings should be primitive with respect to meanings connected to relations between events or between events and their participants, suggesting that it is in fact spatial relations that may be conceived as specialization of all-purpose relations (‘contains’/‘is part of’) when a location is involved. 

The \textit{with} adposition introduced in \REF{ex:key:12}-(14) has the interesting property of expressing no spatial relation at all \citep{Levinson2011} – as does the genitive preposition \textit{of} considered in \sectref{sec:key:1}, assumed to express the same (${\subseteq}$) content of datives.\footnote{The locative semantics found with progressives is an instantiation of a more general part-whole relation, which is also called also by Belvin \& den \citet[170]{Dikken1997} \textit{zonal inclusion}, meaning that all locative relations can be reduced to a primitive part-whole relation with the \textit{figure/locatum} as the \textit{part} and the \textit{ground/location} as the \textit{whole}. The non primitive status of locative can be accounted for by the fact that while locative adpositions alternate with non-locative one, the non-locative adpositions such as \textit{of} are not found in alternation with locative adpositions. For example in English the instrumental adposition \textit{with} alternate with locative prepositions (\textit{on /against})(i-iii) or with the dative/locative \textit{to} (iv). (i) a.   John sprayed the paint on the wall. b.   John sprayed the wall with paint.(ii)   a.   John embroidered peonies on the jacket. b.   John embroidered the jacket with peonies.(iii)  a.   John hit the fence with a stick. b.   John hit a stick against the fence.(iv)   a.   He presented the museum with his pictures.b.    He presented his pictures to the museum.} 

The Italian preposition \textit{da}, which does also have locative meaning, makes an interesting case study, illustrated in some details in Franco \& Manzini (\citeyear{FrancoManzini2017Gen, FrancoManzini2017Ins}). In Romance, the lexicalization of (spatial) adpositions seems to vary according to whether their object, i.e. the Ground in a Figure-Ground configuration \citep{Svenonius2006}, is a high-ranked or low-ranked referent (\citealt{Fábregas2015} on Spanish). In Italian, with inanimate referents, state and motion-to are lexicalized by \textit{a} ‘at, to’ or \textit{in}, as in \REF{ex:key:20a}, and motion from is lexicalized by \textit{da}, as in \REF{ex:key:20b}. However in \REF{ex:key:20c} it can be seen that state, motion-to and motion-from with human referents are all lexicalized by the \textit{da} preposition. 

\ea%20
    \label{ex:key:20}
    \gll\\
        \\
    \glt
    \z

           a.   Sono/vado   \textbf{in/a}   casa.  

‘I am at home/in the house’/‘I go home/into the house.’ 

b.   Vengo \textbf{da}   casa. 

‘I come from home.’ 

c.   Sono/vado/esco \textbf{dal} parrucchiere. 

‘I am at/I go to/I come from the hairdresser.’

Crucially, directionality and other specifications of location that are spatially salient are missing from \textit{da}’s core denotation – or its compatibility with the different locative predicates in \REF{ex:key:20c} could not be explained. Given the ability for \textit{da} to play any locative role with human referents, the natural conclusion is that locative meaning derives neither from the intrinsic content of \textit{da}, not of course from that of its complement (a human referent) – but from the locative nature of the stative/directional predicate. A reasonable characterization for the oblique morpheme \textit{da} in Italian is again that of a general relator involving a part-whole predicate, devoid of any intrinsic locative content.
 

\begin{itemize}
\item \section{Datives (and instrumentals) in Italian progressive/prospective periphrases}
\end{itemize}

At this point, we want to show that also intra-linguistically we may have variation concerning the relator(s) recruited from the lexicon to encode aspectual (progressive) periphrases. We will take Italian as a case study. We have seen in section 1 that, in Italian, a progressive interpretation can be rendered either with a ‘be/stay + dative preposition + infinitive’ schema \REF{ex:key:8a} or a ‘stay + gerund’ \REF{ex:key:9a} schema (cf. \citealt{Bertinetto2000}). 

Interestingly, the gerund periphrasis in Italian is able to encode not only a progressive meaning, but also a prospective one. Indeed, progressive interpretation is somewhat conditioned by the Aktionsart of the verbal item. Following \citegen{Vendler1967} canonical typology, we may say that (at least usually) progressive interpretation is available with \textit{activities} (e.g. ‘John is working’) and \textit{accomplishments} (e.g. ‘John is drawing a square’), while it is not readily available with \textit{states} (e.g. \#John is knowing the answer). With \textit{achievements} things are less clear-cut. Indeed, as noted in \citet[538]{Cinque2017} with achievements that have preparatory stages (e.g. ‘the plane is landing’, ‘John is leaving’, etc.): “Progressive aspect appears to apply to the stages that precede the final achievement thus resulting in a Prospective aspect interpretation”. In Italian, the prospective aspect interpretation triggered by achievement verbs can be rendered with the same (progressive) ‘stay+gerund’ periphrasis, as illustrated in \REF{ex:key:22}.

\ea%22
    \label{ex:key:22}
    \gll\\
        \\
    \glt
    \z

          a.  L’aereo sta atterando     

    ‘The plane is landing’      

b.  Il bambino sta nascendo  

  ‘The baby is being born’

\textit{prospective aspect}   \textit{[achievements]}

Nevertheless, the ‘be/stay + (dative) preposition + infinitive’ verb periphrasis, readily available for ‘progressive’ activities and accomplishments, is not able to encode prospective aspect. Indeed, Italian resorts to a different relator, the adposition \textit{per}, to render prospectives, as illustrated in \REF{ex:key:23}, matching the examples in \REF{ex:key:22}.

\ea%23
    \label{ex:key:23}
    \gll\\
        \\
    \glt
    \z

          a.  l’aereo sta \textbf{per}/*ad atterrare

    ‘The plane is landing’  

b.  Il bambino sta \textbf{per}/*a nascere

  ‘The baby is about to be born’

\citet{FrancoManzini2017b} ascribe to the Italian adposition \textit{per} the same ‘instrumental’ (${\supseteq}$) content expressed by the \textit{con} (with) morpheme, based (among others) on the evidence that \textit{con} and \textit{per} are both able to lexicalize causers, as in \REF{ex:key:24}. Following their insight, it is possible to assume that the (${\supseteq}$) relation between the \textit{con/per} phrase and the VP event in \REF{ex:key:24} yields inclusion in an event/concomitance with it. In a sense, \REF{ex:key:24} is paraphrasable as something like: “The government raised taxes and the crisis was part of its acting to raise them.” (cf. \citealt{FrancoManzini2017b}: 8-9).

\ea%24
    \label{ex:key:24}
    \gll\\
        \\
    \glt
    \z

          Il pericolo di conflitto aumentò \textbf{con}/\textbf{per} il golpe

  ‘The danger of a confrontation increased with/for the coup’

Actually, the same general relation (causation, in this case), may have more than one lexicalization in a given language. Though Italian \textit{con} can express cause, there is no doubt that causation is also expressed, by a different preposition, namely \textit{per}. The closest rendering of \textit{per} in English is \textit{for}, which expresses both purpose (‘they do it for financial gain’) and causation (‘he died for the want of food’), as Italian \textit{per} does. It seems that \textit{per} relates two events through the same basic (${\supseteq}$) operator that we have postulated for \textit{with} morphemes (see \citealt{FrancoManzini2017b}:26-27 for further evidence connecting \textit{for} and \textit{with} in Romance).

In order to conceptually account for the (${\subseteq}$)/(${\supseteq}$) split in the encoding of Progressive \textit{vs}. Prospective aspect, we may start from Jespersen’s (1924:277) insight that Progressive aspect is “a temporal frame encompassing some reference time”. Progressive aspect indeed seems to refer to an event which takes place at a certain time point (or interval) which is related to the reference/utterance time and at the same time is  ‘contained within’ the natural unfolding/time-frame of a more general event (cf. \citealt{Dowty1979}, \citealt{Higginbotham2004}, among others).\footnote{This semantics of progressive is obtained through the analysis of \citet{Higginbotham2009}, \citet{Parson1989} \citet{Landman1992} among others, which proposes that a progressive sentence requires for its truth that the event in question \textit{holds}, not that it \textit{culminates}. The event holds at the utterance/reference time. In the case of progressives in the past, the past auxiliary expresses a time which is previous to the utterance time \citep{Higginbotham2009}. That is, \textit{Mary is eating an apple} is true if the actual event realizes sufficiently (holds) much of the type of event (temporal frame) of \textit{Mary’s eating an apple}: so the actual event is a subset of the type event of \textit{Mary eating an appl}e since Mary may not have finished to eat the apple. For a more detailed analysis of the semantic of progressives for this type of constructions see Manzini at al. \REF{ex:key:2017}.} 

With achievement verbs the temporal frame encompassing the event is very narrow (i.e. punctual), so that they can be perceived as (partially) ‘included’ by the (more extended) time of reference, giving rise to a prospective interpretation. With activities or accomplishments, the event includes the time of reference (interpreted as a point in time) as its part. In other words, achievements are somewhat ‘momentaneous’ and cannot have subintervals, so that the progressive cannot pick up a (point in) time within the event.\footnote{As suggested by \citet{Rothstein2004}, if the achievement is coerced to being an accomplishment, it is possible to assume that the progressive picks up a time immediately preceding the culmination of the event.} 

In present terms, we may assume that the time of reference/utterance is a superset (${\supseteq}$) of the temporal frame of the event when we render prospective aspect, while it is a subset (${\subseteq}$) of the temporal frame of the event whenever we render a progressive interpretation.

From a morphosyntactic viewpoint, when we consider the Italian ‘be/stay + ‘oblique’ adposition + infinitive verb’ periphrasis, there is no difference in the encoding of prospective \textit{vs}. progressive aspect, except for the different relator  (${\subseteq}$) \textit{vs}. (${\supseteq}$) selected from the lexicon.\footnote{Languages vary in the lexical tools (e.g. aspectual periphrases) they employ to convey (different) aspectual flavours. French and Romanian employ axial parts/relational nouns \citep{Svenonius2006} to encode progressive meaning (e.g. French \textit{être en train de}+infinite, Romanian \textit{a fi în curs de a}+infinite); Italian can also encode prospective meaning in a similar vein (e.g. \textit{essere sul punto di+}infinite).  In Icelandic the progressive periphrasis can be employed to convey a terminative/cessative value (e.g. \textit{Ég var að borða,} both: ‘I was eating/I just finished eating’, cf. \citealt{Jóhannsdóttir2012}). In Japanese the same aspectual marker \textit{{}-te i-} can refer to either progressive or resultative meaning \citep{Shirai1998}. It is a likely scenario that these various interpretations (both intra and cross-linguistically) based on a given morphosyntactic template are derived by pragmatic enrichment at the C-I interface. The same can be said of the (\textrm{${\supseteq}$}) based African periphrases illustrated in \REF{ex:key:12}-(14).} 

Standardly assuming that the auxiliary moves to fill the Inflectional projection (\citealt{ManziniLorussoSavoia2017} and references cited there), we can provide the rough representation in \REF{ex:key:25} and \REF{ex:key:26}, respectively for the examples in \REF{ex:key:21a} and \REF{ex:key:23a}. \REF{ex:key:25} basically says that the reference time (as represented in the tensed matrix clause) is ‘part of’ the time frame of the (embedded) event, where the operator (${\subseteq}$) ‘sub-set’  is instantiated by the dative adposition \textit{a}, while \REF{ex:key:26} says that the reference time spans (i.e. include) the (punctual) time frame depicted by the event, where the operator (${\supseteq}$) ‘super-set’ is lexicalized by the \textit{per} adposition.

%%[Warning: Draw object ignored]
%%[Warning: Draw object ignored]
\ea%25
    \label{ex:key:25}
    \gll\\
        \\
    \glt
    \z

                                    IP

%%[Warning: Draw object ignored]
%%[Warning: Draw object ignored]
                  DP 

             Gianni          I                       VP

%%[Warning: Draw object ignored]
%%[Warning: Draw object ignored]
                                sta

                                            V                         (${\subseteq}$)P

%%[Warning: Draw object ignored]
%%[Warning: Draw object ignored]
                                           sta

                                                           (${\subseteq}$)                    VP

%%[Warning: Draw object ignored]
%%[Warning: Draw object ignored]
%%[Warning: Draw object ignored]
                                                             a                    

                                                                                 lavorare

%%[Warning: Draw object ignored]
%%[Warning: Draw object ignored]
\ea%26
    \label{ex:key:26}
    \gll\\
        \\
    \glt
    \z

                                    IP

%%[Warning: Draw object ignored]
%%[Warning: Draw object ignored]
                  DP 

             l’aereo          I                       VP

%%[Warning: Draw object ignored]
%%[Warning: Draw object ignored]
                                 sta

                                            V                         (${\supseteq}$)P

%%[Warning: Draw object ignored]
%%[Warning: Draw object ignored]
                                           sta

                                                           (${\supseteq}$)                    VP

%%[Warning: Draw object ignored]
%%[Warning: Draw object ignored]
%%[Warning: Draw object ignored]
                                                           per                    

                                                                                 atterrare

At this point, we still have to explain why the ‘stay + gerund periphrasis’ is able to encode both progressive and prospective aspect, and how such device can be related, from a morphosyntactic viewpoint, to our ‘part-whole’ model of aspectual periphrases.

We follow Gallego (2010, cf. \citealt{Mateu2002}, \citealt{Franco2015}) in assuming that Romance gerunds incorporate an adposition, namely the –\textit{ndo} morpheme is an inflectional counterpart of the prepositions which embed infinitive complements in the examples above. Consider the minimal pair below, involving a (${\subseteq}$) relator (cf. also \citealt{Casalicchio2013}, from which the example \REF{ex:key:27} is taken). 

\ea%27
    \label{ex:key:27}
    \gll\\
        \\
    \glt
    \z

           a.  \textbf{\textit{A}} ben guardare si nota la differenza

b.  Guarda\textbf{ndo} bene si nota la differenza

both: "If one looks well, he notices the difference’.

Quite interestingly, gerunds often happen to express the (${\supseteq}$) content that we have ascribed to \textit{with} and \textit{for} morpheme.\footnote{Note that according to \citealt{FrancoManzini2017b} the (\textrm{${\supseteq}$}) relation between a \textit{with/for} phrase and a vP/VP event precisely yields inclusion in an event/concomitance with it.} Consider the minimal pairs below, with an ‘instrument’ \REF{ex:key:28} and a ‘purpose’ \REF{ex:key:29} flavour.

\ea%28
    \label{ex:key:28}
    \gll\\
        \\
    \glt
    \z

           a. Il dottore ha curato il paziente somministrando un antibiotico

‘The doctor cured the patient administering an antibiotic’

b.Il dottore ha curato il paziente \textbf{con} la somministrazione di un antibiotico

‘The doctor cured the patient with the administration of an antibiotic’

\ea%29
    \label{ex:key:29}
    \gll\\
        \\
    \glt
    \z

          a.  Gianni lo dice scherzando

b.  Gianni lo dice \textbf{per} scherzo

both: “Gianni says that as a joke”

Given this evidence, we can assume that the gerund inflection in Italian is able to encode both (${\subseteq}$) and (${\supseteq}$) contents. More specifically, we hypothesize that the –\textit{ndo} inflection does not differentiate between the two specular ‘inclusion’ relations, instantiating an all-purpose oblique, spanning from datives to instrumentals (cf. \citealt{FrancoManzini2017}: 24-28, for relevant data from Kristang and Southern Italian dialects). This explains why the ‘stay + gerund’ periphrasis is able to encode both progressive and prospective aspect, always bearing in mind that the aspectual interpretations depends on the \textit{aktionsart} of the verbs that enter in the aspectual constructions (i.e. achievements vs accomplishments, see (22-23)) . We roughly schematize our proposal in structures \REF{ex:key:30}–(31), for \REF{ex:key:21b} and \REF{ex:key:22a}, respectively. These structures crucially prospect a lexical entry for \textit{–ndo}, where this element is associated with both (${\subseteq}$) and (${\supseteq}$) content.

\ea%30
    \label{ex:key:30}
    \gll\\
        \\
    \glt
    \z

                                                IP

%%[Warning: Draw object ignored]
%%[Warning: Draw object ignored]
                               

%%[Warning: Draw object ignored]
%%[Warning: Draw object ignored]
                            DP 

                          Gianni                                    …

                                             I                          VP

%%[Warning: Draw object ignored]
%%[Warning: Draw object ignored]
                                           sta

                                                          V                         …

%%[Warning: Draw object ignored]
%%[Warning: Draw object ignored]
                                                   

                                              V                      (${\supseteq}$)

                                           lavora                {}-ndo   

\ea%31
    \label{ex:key:31}
    \gll\\
        \\
    \glt
    \z

                                                IP

%%[Warning: Draw object ignored]
%%[Warning: Draw object ignored]
                               

%%[Warning: Draw object ignored]
%%[Warning: Draw object ignored]
                            DP 

                          l’aereo                                    …

                                             I                          VP

%%[Warning: Draw object ignored]
%%[Warning: Draw object ignored]
                                           sta

                                                          V                          …

%%[Warning: Draw object ignored]
%%[Warning: Draw object ignored]
                                                   

                                              V                      (${\subseteq}$)

                                           atterra                {}-ndo   

\textbf{4.} \textbf{Conclusion} 

In this paper, we have addressed the morphosyntactic status of the adpositional morphemes surfacing in aspectual periphrases in Italian and beyond. We have shown that adposition-based aspectual periphrases share a primitive relation of ‘part-whole/inclusion’ (the same (${\subseteq}$) relation which is at work with datives/genitives) of an event within a set of events or, alternatively, between the reference time and the time-frame of an event/set of events. We have supported this claim with a series of cross-linguistic examples in which the expression of progressive meaning relies on \textit{with}{}-like adpositions and \textsc{have} predicates, which seem to have a clear non-locative value \citep{Levinson2011}. We have provided a morphosyntactic analysis of Italian progressive periphrases, assuming that gerunds encode an inflectional ‘inclusion’ relator which is compatible with  a prepositional value. We have finally argued that the (${\subseteq}$)/(${\supseteq}$) distinction advanced by \citet{FrancoManzini2017b} for the encoding of argumental/thematic material, happens to be relevant also in the realm of aspectual periphrases.

\textbf{Acknowledgements}

We thank two anonymous reviewers for their comments and criticism. We also thank Greta Mazzaggio and Michelangelo Zaccarello for the proofreading of the Chapter. The usual disclaimers apply.  The authors contribute equally to this work. Ludovico Franco takes responsibility for \sectref{sec:key:2} and 3 and Paolo Lorusso for \sectref{sec:key:1.}

\begin{verbatim}%%move bib entries to  localbibliography.bib

@article{Bach1986,
	author = {Bach, Emmon},
	journal = {\textit{Linguistics and Philosophy}},
	pages = {5–16},
	title = {The algebra of events},
	volume = {9},
	year = {1986}
}%Franco

@book{Bertinetto2000,
	address = {In \textit{Tense and Aspect in the Languages of Europe}, ed. by Östen Dahl, 559–604. Berlin},
	author = {Bertinetto, Pier Marco},
	publisher = {Mouton de Gruyter},
	title = {The progressive in {Romance}, as compared with {English}},
	year = {2000}
}%Franco

@book{BerwickChomsky2011,
	address = {In \textit{The Biolinguistic Enterprise}, ed. by Anna Maria Di Sciullo and Cedric Boeckx, 19-41. Oxford},
	author = {Berwick, Robert, and Noam Chomsky},
	publisher = {OUP},
	title = {The Biolinguistic Program: {{T}}he Current State of Its Evolution and Development},
	year = {2011}
}%Franco

@book{BybeePerkinsPagliuca1994,
	address = {\textit{The Evolution of Grammar.} Chicago},
	author = {Bybee, Joan, Revere Perkins, and William Pagliuca},
	publisher = {The University of Chicago Press},
	year = {1994}
}%Franco

@thesis{Caha2009,
	author = {Caha, Pavel},
	school = {University of Tromsø},
	title = {The nanosyntax of case},
	type = {Dissertation},
	year = {2009}
}%Franco

@thesis{Casalicchio2013,
    note = {Casalicchio, Jan 2013. Pseudorelative, gerundi e infiniti nelle varietà romanze: affintà (solo) superficiali e corrispondenze strutturali. Università di Padova PhD dissertation},
    year = {2013}
}%Franco

@article{Cinque2017,
	author = {Cinque, Guglielmo},
	journal = {Language and Linguistics},
	pages = {521--576},
	title = {On the status of functional categories (heads and phrases)},
	volume = {18},
	year = {2017}
}%Franco

@book{Comrie1976,
	address = {Cambridge},
	author = {Comrie, Bernard},
	publisher = {Cambridge University Press},
	title = {Aspect},
	year = {1976}
}%Franco

@incollection{DemirdacheUribe-Etxebarria1997,
	address = {In \textit{Step by Step: Essays on Minimalist Syntax in Honor of Howard Lasnik}, ed. by Martin David Michaels Roger and Juan Uriagereka, 157-186. Cambridge, MA},
	author = {Demirdache, Hamida, and Miriam Uribe-Etxebarria},
	publisher = {MIT Press},
	title = {The Primitives of temporal Relations},
	year = {1997}
}%Franco

@incollection{DenDikken1998,
	address = {In \textit{Possessors, Predicates and Movement in the Determiner Phrase}, ed. by Artemis Alexiadou and Chris Wilder, 177-214. Amsterdam},
	author = {Den Dikken, Marcel},
	publisher = {Benjamins},
	title = {Predicate inversion in {DP}},
	year = {1998}
}%Franco

@book{Dowty1979,
    note = {Dowty, David. 1979.{\enspace}\textit{Word Meaning and Montague Grammar}. Dordrecht: Kluwer}
}%Franco

@article{Fábregas2015,
	author = {Fábregas, Antonio},
	journal = {\textit{Revue Romane}},
	pages = {163–190},
	sortname = {Fabregas, Antonio},
	title = {Direccionales con \textit{con} y Marcado Diferencial de Objeto},
	volume = {50},
	year = {2015}
}%Franco

@book{Filip1999,
	address = {New York},
	author = {Filip, Hana},
	publisher = {Garland},
	title = {\textit{Aspect, Situation Types and Noun Phrase Semantics}},
	year = {1999}
}%Franco

@article{Franco2015,
	author = {Franco, Ludovico},
	journal = {\textit{Probus}},
	pages = {271--306},
	title = {The {Morphosyntax} of Adverbs of the Carpone/i type in (Old and Modern) {Italian}},
	volume = {27},
	year = {2015}
}%Franco

@article{Freeze1992,
	author = {Freeze, Ray},
	journal = {\textit{Language}},
	pages = {553–595},
	title = {Existentials and other locatives},
	volume = {68},
	year = {1992}
}%Franco

@article{Gallego2010,
	author = {Gallego, Ángel},
	journal = {\textit{Catalan Journal of Linguistics}},
	pages = {79–102},
	sortname = {Gallego, Angel},
	title = {On the prepositional nature of non-finite verbs},
	volume = {9},
	year = {2010}
}%Franco

@book{HeineKuteva2002,
	address = {Cambridge},
	author = {Heine, Bernd, and Tania Kuteva},
	publisher = {CUP},
	title = {{\enspace}\textit{World Lexicon of Grammaticalization}},
	year = {2002}
}%Franco%

@incollection{Higginbotham2004,
	address = {In \textit{The Syntax of Time}, ed. by Jacqueline Guéron \& Jacqueline Lecarme, 329–358. Cambridge, MA},
	author = {Higginbotham, James},
	publisher = {MIT Press},
	title = {{\enspace}The {English} progressive},
	year = {2004}
}%Franco

@book{Higginbotham2009,
	address = {\textit{Oxford Studies in Theoretical Linguistics}. Oxford},
	author = {Higginbotham, James},
	publisher = {Oxford University Press},
	title = {\textit{Tense, Aspect, and Indexicality}},
	year = {2009}
}%Franco

@book{Jespersen1924,
	address = {London},
	author = {Jespersen, Otto},
	publisher = {Allen \& Unwin Ltd},
	title = {{\enspace}\textit{The Philosophy of Grammar}},
	year = {1924}
}%Franco


@thesis{Jóhannsdóttir2011,
	author = {Jóhannsdóttir, Kristín},
	school = {University of British Columbia},
	sortname = {Johannsdottir, Kristin},
	title = {Aspects of the Progressive in {English} and {Icelandic}},
	type = {Dissertation},
	year = {2011}
}%Franco

@thesis{Kawasha2003,
	author = {Kawasha, Boniface Kaumba},
	school = {University of Oregon},
	title = {Lunda Grammar},
	type = {Dissertation},
	year = {2003}
}%Franco

@article{Kayne1993,
	author = {Kayne, Richard},
	journal = {\textit{Studia Linguistica}},
	pages = {3--31},
	title = {Toward a Modular theory of Auxiliary Selection},
	volume = {47},
	year = {1993}
}%Franco

@book{Kayne2003,
	address = {In \textit{Grammar in Focus: Festschrift for Christer Platzack Vol II}, ed. by L.-O. Delsing, C. Falk, G. Josefsson and H. Á. Sigurðsson, 209–226. Lund},
	author = {Kayne, Richard},
	publisher = {Wallin \& Dalholm},
	title = {{\enspace}Silent years, silent hours},
	year = {2003}
}%Franco

@book{Kilian-Hatz1992,
	address = {Der Komitativ im Baka},
	author = {Kilian-Hatz, Christa},
	note = {Universität zu Köln MA thesis.},
	publisher = {Eine Fallstudie zur Grammatikalisierung},
	year = {1992}
}%Franco

@article{Landman1992,
	author = {Landman, Fred},
	journal = {\textit{Natural Language Semantics}},
	pages = {1--32},
	title = {The Progressive},
	volume = {1},
	year = {1992}
}%Franco

@book{Ledgeway2016,
	address = {In \textit{The Oxford Guide to Romance Languages}, ed. by Adam Ledgeway and Martin Maiden, 246–269. Oxford},
	author = {Ledgeway, Adam},
	publisher = {OUP},
	title = {{\enspace}The dialects of southern {Italy}},
	year = {2016}
}%Franco

@article{Levinson2011,
	author = {Levinson, Lisa},
	journal = {\textit{Syntax}},
	pages = {355–393},
	title = {Possessive \textit{with} in {Germanic}: {{H}}ave and the role of P},
	volume = {14},
	year = {2011}
}%Franco

@article{ManziniSavoia2011,
	author = {Manzini, M. Rita, and Leonardo Savoia},
	journal = {\textit{Linguistic Variation}},
	pages = {76–120},
	title = {Reducing ‘case’ to denotational primitives},
	volume = {11},
	year = {2011}
}%Franco

@article{ManziniLorussoSavoia2017,
	author = {Manzini, M. Rita, Paolo Lorusso and Leonardo M. Savoia},
	journal = {\textit{a}/bare finite complements in Southern Italian varieties: Mono-clausal or bi-clausal syntax? \textit{QULSO}},
	pages = {11--59},
	volume = {3},
	year = {2017}
}%Franco

@article{MateuAmadas1999,
	author = {Mateu, Jaume, and Laia Amadas},
	journal = {\textit{Catalan Working Papers in Linguistics}},
	pages = {159--174},
	title = {Extended Argument Structure: {{P}}rogressive as Unaccusative},
	volume = {7},
	year = {1999}
}%Franco

@book{Mateu2002,
	address = {Argument Structure},
	author = {Mateu, Jaume},
	note = {Universitat Autònoma de Barcelona PhD dissertation.},
	publisher = {Relational construal at the Syntax-Semantic Interface},
	year = {2002}
}%Franco

@book{Parsons1989,
	address = {\textit{Linguistics and Philosophy} 12 \REF{ex:key:2}},
	author = {Parsons, Terence},
	publisher = {213-241},
	title = {The Progressive in {{E}nglish}: {{{E}}}vents, States and Processes},
	year = {1989}
}%Franco

@book{Rothstein2004,
	address = {Blackwell},
	author = {Rothstein, Susan},
	publisher = {Oxford},
	title = {\textit{Structuring events}},
	year = {2004}
}%Franco

@article{Shirai1998,
	author = {Shirai, Yasuhiro},
	journal = {\textit{Studies in Language}},
	pages = {661–692},
	title = {Where the progressive and resultative meet: {{I}}mperfective aspect in {Japanese}, {Korean}, {Chinese}, and {English}},
	volume = {22},
	year = {1998}
}%Franco

@book{StolzStrohUrdze2006,
	address = {Berlin},
	author = {Stolz, Thomas, Cornelia Stroh and Aina Urdze},
	publisher = {De Gruyter},
	title = {On comitatives and related categories},
	year = {2006}
}%Franco

@article{Svenonius2006,
	author = {Svenonius, Peter},
	journal = {Nordlyd},
	pages = {1--22},
	title = {The Emergence of Axial Parts},
	volume = {33},
	year = {2006}
}%Franco

@book{Vendler1967,
    note = {Vendler, Zeno. 1967. \textit{Linguistics in Philosophy}. Ithaca, NY: Cornell Uni},
    year = {1967}
}%Franco

\end{verbatim}
\sloppy\printbibliography[heading=subbibliography,notkeyword=this]
\end{document}

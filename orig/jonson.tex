% This file was converted to LaTeX by Writer2LaTeX ver. 1.4
% see http://writer2latex.sourceforge.net for more info
\documentclass[12pt]{article}
\usepackage[utf8]{inputenc}
\usepackage[T1]{fontenc}
\usepackage[english]{babel}
\usepackage{amsmath}
\usepackage{amssymb,amsfonts,textcomp}
\usepackage{array}
\usepackage{supertabular}
\usepackage{hhline}
\usepackage{hyperref}
\hypersetup{colorlinks=true, linkcolor=blue, citecolor=blue, filecolor=blue, urlcolor=blue}
\newcommand\textsubscript[1]{\ensuremath{{}_{\text{#1}}}}
% Text styles
\newcommand\textstyleFootnoteSymbol[1]{{\fontsize{6.5pt}{7.8pt}\selectfont #1}}
\makeatletter
\newcommand\arraybslash{\let\\\@arraycr}
\makeatother
\raggedbottom
% Paragraph styles
\renewcommand\familydefault{\rmdefault}
\newenvironment{styleStandard}{\setlength\leftskip{0cm}\setlength\rightskip{0cm plus 1fil}\setlength\parindent{0cm}\setlength\parfillskip{0pt plus 1fil}\setlength\parskip{0in plus 1pt}\writerlistparindent\writerlistleftskip\leavevmode\normalfont\normalsize\writerlistlabel\ignorespaces}{\unskip\vspace{0.111in plus 0.0111in}\par}
\newenvironment{stylelsAbstract}{\setlength\leftskip{0.5in}\setlength\rightskip{0.5in}\setlength\parindent{0in}\setlength\parfillskip{0pt plus 1fil}\setlength\parskip{0in plus 1pt}\writerlistparindent\writerlistleftskip\leavevmode\normalfont\normalsize\itshape\writerlistlabel\ignorespaces}{\unskip\vspace{0.111in plus 0.0111in}\par}
\newenvironment{stylelsSectioni}{\setlength\leftskip{0cm}\setlength\rightskip{0cm plus 1fil}\setlength\parindent{0cm}\setlength\parfillskip{0pt plus 1fil}\setlength\parskip{0.1665in plus 0.016649999in}\writerlistparindent\writerlistleftskip\leavevmode\normalfont\normalsize\fontsize{18pt}{21.6pt}\selectfont\bfseries\writerlistlabel\ignorespaces}{\unskip\vspace{0.0835in plus 0.00835in}\par}
\newenvironment{stylelsSectionii}{\setlength\leftskip{0.1972in}\setlength\rightskip{0in plus 1fil}\setlength\parindent{-0.1972in}\setlength\parfillskip{0pt plus 1fil}\setlength\parskip{0.222in plus 0.0222in}\writerlistparindent\writerlistleftskip\leavevmode\normalfont\normalsize\fontsize{16pt}{19.2pt}\selectfont\bfseries\writerlistlabel\ignorespaces}{\unskip\vspace{0.0835in plus 0.00835in}\par}
\newenvironment{styleFootnote}{\setlength\leftskip{0.2354in}\setlength\rightskip{0in plus 1fil}\setlength\parindent{-0.2354in}\setlength\parfillskip{0pt plus 1fil}\setlength\parskip{0in plus 1pt}\writerlistparindent\writerlistleftskip\leavevmode\normalfont\normalsize\fontsize{10pt}{12.0pt}\selectfont\writerlistlabel\ignorespaces}{\unskip\vspace{0.111in plus 0.0111in}\par}
% List styles
\newcommand\writerlistleftskip{}
\newcommand\writerlistparindent{}
\newcommand\writerlistlabel{}
\newcommand\writerlistremovelabel{\aftergroup\let\aftergroup\writerlistparindent\aftergroup\relax\aftergroup\let\aftergroup\writerlistlabel\aftergroup\relax}
\setlength\tabcolsep{1mm}
\renewcommand\arraystretch{1.3}
\title{}
\author{Anna}
\date{2019-07-23}
\begin{document}
\title{Dative objects with novel verbs in Icelandic }
\maketitle

\begin{styleStandard}
\textbf{Jóhannes Gísli Jónsson \& Rannveig Thórarinsdóttir}
\end{styleStandard}

\begin{styleStandard}
\textbf{University of Iceland}
\end{styleStandard}

\begin{stylelsAbstract}
Abstract. This paper discusses the results of two online surveys testing object case with novel verbs in Icelandic. The results show that a novel transitive verb takes a dative direct object if the verb (a) encodes some kind of motion of the object referent, or (b) has a translational substitute that takes a dative object. If neither (a) nor (b) holds, the object gets the default accusative case. Thus, caused motion plays a major role in the licensing of dative case with direct objects in Icelandic.
\end{stylelsAbstract}

\begin{stylelsSectioni}
1. Introduction
\end{stylelsSectioni}

\begin{styleStandard}
Dative case with direct objects in Icelandic has been widely discussed in the linguistic literature (see e.g. Yip, Maling \& Jackendoff 1987, Barðdal 2001, 2008, Svenonius 2002, Maling 2002, and Jónsson 2013a). The central issue is the degree to which the dative is semantically predictable. As discussed by Maling (2002), verbs with dative objects are found in various verb classes in Icelandic, most of which also include verbs with accusative objects. Thus, it appears that dative is predictable only in a broad sense. However, it can be shown that dative objects are fully predictable in at least three closely related classes, verbs of ballistic motion (Svenonius 2002) verbs of emission (Maling 2002, Jónsson 2013a) and pour verbs (Jónsson 2013a).
\end{styleStandard}

\begin{styleStandard}
One way of probing the semantics of dative objects in Icelandic is to examine novel transitive verbs since these verbs should reflect the regular aspects of dative case assignment. Indeed, the fact that new verbs never take genitive objects (Jónsson \& Eythórsson 2011) suggests that they cannot assign truly idiosyncratic case. However, with the exception of Barðdal (2001, 2008), new verbs with dative objects have not been a central concern in the literature on the Icelandic case system. 
\end{styleStandard}

\begin{styleStandard}
We report here on the results of two online surveys testing object case with verbs that have become part of colloquial Icelandic in the last few decades (see Thórarinsdóttir 2015). The results show that a novel transitive verb takes a dative direct object if the verb (a) encodes motion of the object referent, or (b) has a translational substitute that takes a dative object. We will refer to (b) as isolate attraction, following Barðdal (2001), and take the term translational substitute to mean an established verb taking a dative object that can replace the new verb semantically. If neither (a) nor (b) applies, the object gets accusative case, the default case for direct objects in Icelandic. This holds not only for verbs selecting one object case but also for verbs displaying variation between dative and accusative. This means that some verbs may be ambiguous in whether they encode caused motion or not. Note, however, that case variation in Icelandic may also be purely formal and not reflect any semantic distinction between the variants (see Jónsson 2013b); for discussion of formal case variation in Romance, see Ledgeway et al. (this volume) and Royo (this volume).
\end{styleStandard}

\begin{styleStandard}
The strong link between caused motion and dative objects in Icelandic has often been discussed, e.g. by Barðdal (2001, 2008), Svenonius (2002), Maling (2002), and Jónsson (2013a). Our proposal is new in that caused motion is argued to be the crucial meaning component of new dative verbs in Icelandic that are not licensed by isolate attraction. That isolate attraction plays a role independent of caused motion is shown by novel verbs like \textit{dílíta} [2BD?]delete electronically[2BC?], which does not express any motion of the object. This verb takes a dative object just like its translational substitute \textit{eyða} [2BD?]delete, spend, waste[2BC?], which has a broader meaning than \textit{dílíta}. Further examples of isolate attraction will be discussed in 3.2 below.
\end{styleStandard}

\begin{styleStandard}
Since there are only two ways in which a novel verb can get a dative object and both of them are quite restricted, our proposal makes strong predictions about dative objects with novel verbs in Icelandic. As discussed in sections 3 and 4, these predictions are borne out by the data from the two online surveys. Importantly and in clear contrast to Barðdal (2001, 2008), we do not allow for the possibility that novel verbs take a dative object if they are attracted to specific classes of dative verbs with a similar meaning. Thus, the data from the two surveys will be accounted for without any recourse to this possibility although various subclasses of verbs taking the same object case will be mentioned in our discussion.
\end{styleStandard}

\begin{stylelsSectioni}
2. Background
\end{stylelsSectioni}

\begin{styleStandard}
There is a fundamental unity to all dative objects in Icelandic in that dative is preserved under passivization. In this respect, dative differs sharply from accusative (Zaenen, Maling \& Thráinsson 1985). Case preservation in passives applies equally to datives that are completely predictable, such as dative recipients or benefactives with ditransitive verbs (Jónsson 2000), and datives that are idiosyncratically associated with some monotransitive verbs. This latter type is exemplified by verbs like \textit{anna} [2BD?]meet (demand), have time for[2BC?], \textit{gleyma} ‘forget’, \textit{stríða} ‘tease’,\textit{ treysta} [2BD?]trust[2BC?] and \textit{unna} ‘love’. This contrast between dative and accusative shows that dative is not a structural case in Icelandic, at least not in the same sense as accusative (see Thráinsson 2007:181-192 and references cited there).
\end{styleStandard}

\begin{styleStandard}
A further difference is that accusative is not associated with any specific semantics as transitive verbs of all kinds take accusative objects in Icelandic. In fact, as shown the by the ECM construction, accusative can even be assigned to a DP that is not an argument of the relevant verb. Although it has been observed that certain sublasses of transitive verbs in Icelandic only allow accusative objects (Jónsson 2013a), this is best understood as a constraint on the assignment of dative (and genitive) case. In sections 3 and 4, some verb classes that systematically exclude dative or genitive objects will be mentioned but this should not be taken to mean that accusative is semantically determined in these classes.
\end{styleStandard}

\begin{styleStandard}
Despite the differences between accusative and dative discussed above, it has become fairly common in recent years to link both these cases to functional heads in the extended vP. Thus, the Icelandic dative is often associated with an applicative head inside VoiceP/vP. Wood (2015:128-138) argues that this is correct for indirect objects but generally not for direct objects. His arguments are based e.g. on the fact that dative is preserved with indirect objects but not direct objects under suffixation of the “middle” suffix –\textit{st} in Icelandic. His proposal is that direct object datives are licensed by a functional head that he labels v\textsc{\textsubscript{dat}}, following Svenonius (2006). The results discussed in sections 3 and 4 suggest that there is a functional head that licenses dative objects with verbs that express caused motion. Diachronic evidence from Faroese points in the same direction since dative has systematically disappeared with all such verbs but is preserved with various other monotransitive verbs in Faroese (Jónsson 2009). This diachronic development can be interpreted as the loss of the relevant functional head in the history of Faroese.
\end{styleStandard}

\begin{styleStandard}
Svenonius (2002) shows that verbs of ballistic motion like \textit{kasta} [2BD?]throw[2BC?] always take a dative object in Icelandic. With these verbs, the agent applies force to cause an object to move but the motion of the object continues after the agent has done his/her part. Svenonius (2002) claims that dative objects in Icelandic are found with verbs where the subevent associated with the agent does not completely overlap temporally with the subevent associated with the theme object. This is correct as a one-way generalization as every verb that complies with it takes a dative object in Icelandic but it is not immediately obvious how far this generalization extends beyond verbs of ballistic motion. We cannot discuss this issue fully here, but it seems to us that it also comprises emission verbs, pour verbs and many of the verbs tested in the online surveys relating to information technology and expressing motion from one electronic location to another. 
\end{styleStandard}

\begin{styleStandard}
Another complicating factor is that dative case is found with various verbs of motion that involve complete temporal overlap of the two subevents associated with the agent and the theme. Thus, verbs of accompanied or directed motion may take a dative object (cf. \textit{drösla} ‘move (with difficulty)’, \textit{lyfta} ‘lift, raise’, \textit{smeygja} ‘slip, slide’, and \textit{ýta} ‘push’) or an accusative object (cf. \textit{bera} ‘carry’, \textit{draga} ‘pull’, \textit{hækka} ‘raise’, and \textit{lækka} ‘lower’). However, the data discussed in section 3 and 4 suggest that dative objects with novel verbs are licensed by caused motion, irrespective of any subclassification of the relevant verbs. Hence, it appears that dative is in the process of being generalized to all transitive motion verbs in Icelandic (Barðdal 2008).
\end{styleStandard}

\begin{styleStandard}
The theoretical literature on motion verbs across languages is very much focused on intransitive verbs like \textit{run} and \textit{dance} and there is no standard definition of caused motion that we are aware of. Still, this does not turn out to be much of a problem for our purposes. As we will see, the crucial issue is to distinguish verbs that encode caused motion of the direct object from verbs where caused motion is not encoded but rather inferred from world knowledge. It is only novel verbs in the former class that take a dative object, i.e. if isolate attraction does not play a role.
\end{styleStandard}

\begin{stylelsSectioni}
3. The results
\end{stylelsSectioni}

\begin{styleStandard}
In the following two sections, the results of a large-scale study of direct object case with 40 novel verbs in Icelandic will be discussed. These verbs have become part of the Icelandic lexicon in the last few decades, mainly as borrowings from English or Danish but some as native neologisms. Most of these verbs are highly colloquial and not often found in writing, especially the loanverbs, but as far as we know this has no effect on object case. 
\end{styleStandard}

\begin{stylelsSectionii}
3.1 The two surveys
\end{stylelsSectionii}

\begin{styleStandard}
The study to be discussed here consisted of two online surveys, with 393 and 402 participants, respectively (see Thórarinsdóttir 2015 for details). Each survey featured 50 sentences, 20 sentences testing object case with novel verbs and 30 fillers. For every sentence, the participants were asked to select four options presented to them in this order: (a) the accusative form of the direct object, (b) the dative form of the direct object, (c) both forms accepted, (d) neither form accepted. Option (d) was selected quite often, especially with verbs of low frequency, presumably because some of the participants were not familiar with these verbs. By contrast, very few opted for (c), even with verbs where we suspect that many speakers allow both accusative and dative.
\end{styleStandard}

\begin{styleStandard}
The verbs tested in the two surveys are listed below. The glosses are based on the relevant test sentences in the surveys. 
\end{styleStandard}

\begin{flushleft}
\tablefirsthead{}
\tablehead{}
\tabletail{}
\tablelasttail{}
\begin{supertabular}{m{0.31505984in}m{0.23655984in}m{3.9566598in}}
(1) &
a. &
Verbs in the first survey: 

brodkasta [2BD?]broadcast[2BC?], dánlóda [2BD?]download[2BC?], droppa [2BD?]quit, drop[2BC?], drulla [2BD?]get, put[2BC?], dömpa [2BD?]dump[2BC?], farta [2BD?]drive fast[2BC?], flexa [2BD?]show off with, throw around[2BC?], gúgla [2BD?]google[2BC?], hannesa [2BD?]steal (a text)[2BC?], installa [2BD?]install (a program)[2BC?], jáa [2BD?]search for on ja.is[2BC?], jinxa [2BD?]put a curse on[2BC?], krakka [2BD?]unlock, crack[2BC?], krassa [2BD?]cause to crash[2BC?], offa [2BD?]turn off, shock[2BC?], rippa [2BD?]copy (illegally)[2BC?], slaka [2BD?]pass[2BC?], slumma [2BD?]kick (a ball)[2BC?], smessa [2BD?]send by sms[2BC?], sneika [2BD?]sneak[2BC?]\\
\end{supertabular}
\end{flushleft}
\begin{flushleft}
\tablefirsthead{}
\tablehead{}
\tabletail{}
\tablelasttail{}
\begin{supertabular}{m{0.31505984in}m{0.23655984in}m{3.9566598in}}
 &
b. &
Verbs in the second survey: 

átsorsa [2BD?]outsource[2BC?], bekka [2BD?]lift in bench press[2BC?], blasta [2BD?]play loudly, blast[2BC?], bleima [2BD?]blame[2BC?], domma [2BD?]dominate[2BC?], fiffa [2BD?]fix (illegally)[2BC?], gólfa [2BD?]press (the pedal) to the floor[2BC?], gramma [2BD?]put on Instagram[2BC?], græja [2BD?]procure[2BC?], gúffa [2BD?]eat greedily[2BC?], kikka [2BD?]kick, hit[2BC?], meila [2BD?]e-mail[2BC?], mæna [2BD?]collect, mine[2BC?], neimdroppa [2BD?]namedrop[2BC?], peista [2BD?]paste[2BC?], pósta [2BD?]post (online)[2BC?], sjera [2BD?]share (online)[2BC?], skrína [2BD?]screen, keep an eye on[2BC?], skúbba [2BD?]be the first to tell (a piece of news)[2BC?], syngja [2BD?]tell (a secret)[2BC?]\\
\end{supertabular}
\end{flushleft}
\begin{styleStandard}
Five verbs are not included in the following discussion here, either because the relevant test sentences allowed for too many possibilities for their semantic interpretation (\textit{jinxa}, \textit{kikka}) or because it can be argued that they are not really new \ (\textit{drulla}, \textit{slaka}, \textit{skúbba}). 
\end{styleStandard}

\begin{styleFootnote}
The two surveys were designed to test our hypothesis that dative case with novel transitive verbs in Icelandic is licensed by two factors: (a) caused motion of the object referent, or (b) a translational substitute taking a dative object (isolate attraction). The verbs were selected so that they would fall into three groups of roughly the same size: (a) verbs taking a dative object, (b) verbs taking an accusative object, and (c) verbs displaying variation between dative and accusative. A random selection of novel verbs would have produced a less balanced sample in view of Barðdal’s (2008:78-79) study of 107 novel verbs in Icelandic where accusative outscored dative by a ratio of approximately 2:1. Note that no effort was made to include verbs from all the subclasses of the dative verbs discussed by Maling (2002) as the right verbs would also have been hard to find and this would have required a much bigger study.
\end{styleFootnote}

\begin{styleStandard}
The novel verbs tested in the study can be divided into three classes: (a) verbs that strongly favour dative, (b) verbs that strongly favor accusative, and (c) verbs that vary between dative and accusative object. For concreteness, classes (a) and (b) were defined such that the preferred case was selected at least five times more often than the other case. Verbs from the first two classes are discussed in 3.2 and 3.3 below but variation between dative and accusastive is the topic of section 4.
\end{styleStandard}

\begin{stylelsSectionii}
3.2 Dative objects
\end{stylelsSectionii}

\begin{styleStandard}
Many verbs in the current study showed a strong preference for a dative object. This is true of the following verbs in the first survey: 
\end{styleStandard}

\begin{flushleft}
\tablefirsthead{}
\tablehead{}
\tabletail{}
\tablelasttail{}
\begin{supertabular}{|m{0.71915984in}m{1.6052599in}m{0.42405984in}m{0.41365984in}m{0.48725984in}m{0.6025598in}}
\hline
\multicolumn{1}{|m{0.71915984in}|}{{\bfseries Table 1:}} &
\multicolumn{5}{m{3.8477597in}|}{{\bfseries Verbs taking a dative object in survey 1}}\\
Verb &
Gloss &
\textbf{DAT} &
ACC &
Both &
Neither\\
dánlóda &
[2BD?]download[2BC?] &
\textbf{93,1} &
4,1 &
0,5 &
2,3\\
droppa &
[2BD?]quit, drop[2BC?] &
\textbf{90,6} &
1,5 &
0,3 &
7,6\\
dömpa &
[2BD?]dump[2BC?] &
\textbf{87,8} &
0,5 &
1,3 &
10,4\\
installa &
[2BD?]install[2BC?] &
\textbf{85,8} &
6,6 &
2,5 &
5,1\\
brodkasta &
[2BD?]broadcast[2BC?] &
\textbf{85,2} &
2,8 &
1,0 &
11,0\\
sneika &
[2BD?]sneak[2BC?] &
\textbf{55,0} &
5,1 &
1,5 &
38,4\\
flexa &
[2BD?]throw around[2BC?] &
\textbf{54,7} &
8,7 &
1,0 &
35,6\\\hline
slumma &
‘kick (a ball)’ &
\textbf{47,8} &
8,4 &
3,8 &
40,0\\\hline
\end{supertabular}
\end{flushleft}
\begin{styleStandard}
Although the acceptance rate for dative ranges from 47,8\% to 93,1\%, the dative was chosen at least five times more often than the accusative for every verb here. There were also significant differences with respect to the last option (neither), with high frequency verbs like \textit{dánlóda}, \textit{droppa}, and \textit{installa} scoring below 8\% but verbs of low frequency like \textit{sneika}, \textit{flexa} and \textit{slumma} scoring above 35\%. We take this to show that the lowest scoring verbs were the most familiar to the participants and vice versa. The same trend was also evident in other tables in this paper.
\end{styleStandard}

\begin{styleStandard}
As discussed in more detail below, all the verbs listed in Table 1 encode some kind of motion of the object referent. This is also true of all the verbs in the second survey that showed a clear preference for a dative object:
\end{styleStandard}

\begin{flushleft}
\tablefirsthead{}
\tablehead{}
\tabletail{}
\tablelasttail{}
\begin{supertabular}{|m{0.70805985in}m{1.3386599in}m{0.51155984in}m{0.51295984in}m{0.51155984in}|m{0.70805985in}|}
\hline
\multicolumn{1}{|m{0.70805985in}|}{{\bfseries Table 2:}} &
\multicolumn{5}{m{3.89776in}|}{{\bfseries Verbs taking a dative object in survey 2}}\\
Verb &
Gloss &
\textbf{DAT} &
ACC &
Both &
Neither\\
pósta &
[2BD?]post (online)[2BC?] &
\textbf{96,0} &
2,0 &
0,5 &
1,5\\
gúffa &
[2BD?]eat greedily[2BC?] &
\textbf{87,1} &
8,0 &
2,7 &
2,2\\
sjera &
[2BD?]share (online)[2BC?] &
\textbf{81,3} &
6,0 &
0,7 &
12,0\\
blasta &
[2BD?]play loudly, blast[2BC?] &
\textbf{76,6} &
12,4 &
3,0 &
8,0\\\hline
átsorsa &
[2BD?]outsource[2BC?] &
\textbf{64,2} &
11,2 &
5,5 &
19,1\\\hline
\end{supertabular}
\end{flushleft}
\begin{styleStandard}
The test sentences with the top three verbs in Table 1 are shown in (2) below:
\end{styleStandard}

\begin{flushleft}
\tablefirsthead{}
\tablehead{}
\tabletail{}
\tablelasttail{}
\begin{supertabular}{m{0.31505984in}m{0.23655984in}m{0.39345986in}m{0.27545986in}m{0.108059846in}m{0.5511598in}m{0.47405985in}m{0.5906598in}m{0.23095986in}m{0.31505984in}m{0.6913598in}}
(2) &
a. &
Ertu &
búin &
að &
dánlóda &
nýju &
myndinni &
með &
Ryan &
Gosling?\\
 &
 &
are.you &
done &
to &
download &
new.\textsc{dat} &
movie.\textsc{dat} &
with &
Ryan &
Gosling?\\
 &
 &
\multicolumn{9}{m{4.26016in}}{[2BD?]Have you downloaded the new movie with Ryan Gosling?[2BC?]}\\
\end{supertabular}
\end{flushleft}
\begin{flushleft}
\tablefirsthead{}
\tablehead{}
\tabletail{}
\tablelasttail{}
\begin{supertabular}{m{0.31365985in}m{0.23655984in}m{0.21155986in}m{0.35315984in}m{0.27405986in}m{0.19625986in}m{0.35385984in}m{0.19485986in}m{0.47195986in}m{0.5497598in}m{0.8205598in}}
 &
b. &
Ég &
held &
að &
ég &
verði &
að &
droppa &
þessu &
námskeiði\\
 &
 &
I &
think &
that &
I &
must &
to &
drop &
this.\textsc{dat} &
course.\textsc{dat}\\
 &
 &
\multicolumn{9}{m{4.0559597in}}{[2BD?]I think that I must drop this course.[2BC?]}\\
\end{supertabular}
\end{flushleft}
\begin{flushleft}
\tablefirsthead{}
\tablehead{}
\tabletail{}
\tablelasttail{}
\begin{supertabular}{m{0.31505984in}m{0.23655984in}m{0.43305984in}m{0.11845984in}m{0.43305984in}m{0.51155984in}m{0.15805984in}m{0.39275986in}m{0.15805984in}m{0.21635985in}m{0.9045598in}}
 &
c. &
Djöfull &
er &
bossinn &
duglegur &
að &
dömpa &
á &
þig &
verkefnum\\
 &
 &
bloody &
is &
the.boss &
relentless &
to &
dump &
on &
you &
tasks.\textsc{dat}\\
 &
 &
\multicolumn{9}{m{3.9558597in}}{[2BD?]How relentlessly the boss dumps tasks on you![2BC?]}\\
\end{supertabular}
\end{flushleft}
\begin{styleStandard}
The motion verbs \textit{dánlóda}, \textit{droppa} and \textit{dömpa} can be replaced here by the dative verbs \textit{hlaða niður }[2BD?]download[2BC?], \textit{sleppa} [2BD?]release, skip[2BC?] and \textit{demba }[2BD?]dump, pour[2BC?], respectively, without any change in meaning.\textstyleFootnoteSymbol{ }Hence, it is impossible to determine if the datives in (2a-c) are due to isolate attraction or caused motion. The same applies to \textit{brodkasta}, a verb of emission which has a translational substitute in the dative verb \textit{sjónvarpa }[2BD?]broadcast[2BC?]. However, the dative object with \textit{sneika} and \textit{sjera }is presumably due\textit{ }to isolate attraction by \textit{lauma} [2BD?]sneak[2BC?] and \textit{deila }[2BD?]share, divide[2BC?], both of which take a dative object.
\end{styleStandard}

\begin{styleStandard}
Other verbs in Tables 1 and 2 do not have a translational substitute taking a dative object in the traditional vocabulary of Icelandic, e.g. \textit{installa},\textit{ pósta}, and \textit{gúffa}. All these verbs encode motion of the object, although not in a literal sense, except perhaps \textit{gúffa}. The relevant test sentences are shown in (3): 
\end{styleStandard}

\begin{flushleft}
\tablefirsthead{}
\tablehead{}
\tabletail{}
\tablelasttail{}
\begin{supertabular}{m{0.31505984in}m{0.23655984in}m{0.27545986in}m{0.31505984in}m{0.15805984in}m{0.47195986in}m{0.45315984in}m{1.8886598in}}
(3) &
a. &
Þú &
þarft &
að &
installa &
Office &
pakkanum\\
 &
 &
you &
need &
to &
install &
Office &
the.package.\textsc{dat}\\
 &
 &
\multicolumn{6}{m{3.95606in}}{[2BD?]You need to install the Office package.[2BC?]}\\
\end{supertabular}
\end{flushleft}
\begin{flushleft}
\tablefirsthead{}
\tablehead{}
\tabletail{}
\tablelasttail{}
\begin{supertabular}{m{0.31505984in}m{0.23655984in}m{0.27335986in}m{0.31295985in}m{0.43165985in}m{0.7462598in}m{0.51015985in}m{0.11565984in}m{0.35455984in}m{0.23725986in}m{0.34415984in}}
 &
b. &
Helga &
póstaði &
ótrúlega &
skemmtilegu &
myndbandi &
á &
vegginn &
minn &
áðan\\
 &
 &
Helga &
posted &
incredibly &
entertaining.\textsc{dat} &
video.\textsc{dat} &
on &
the.wall &
my &
just\\
 &
 &
\multicolumn{9}{m{3.9559603in}}{[2BD?]Helga just posted an incredibly funny video on my wall.[2BC?]}\\
\end{supertabular}
\end{flushleft}
\begin{flushleft}
\tablefirsthead{}
\tablehead{}
\tabletail{}
\tablelasttail{}
\begin{supertabular}{m{0.31505984in}m{0.23655984in}m{0.5615598in}m{0.21435985in}m{0.25525984in}m{0.35595986in}m{0.14275983in}m{0.40945986in}m{0.12335984in}m{0.31295985in}m{0.9504598in}}
 &
c. &
Af hverju &
eru &
allir &
farnir &
að &
gúffa &
í &
sig &
chiafræjum?\\
 &
 &
why &
are &
all &
started &
to &
shovel &
in &
\textsc{refl} &
chia.seeds.\textsc{dat}\\
 &
 &
\multicolumn{9}{m{3.95606in}}{[2BD?]Why has everybody started to eat chia seeds like crazy?[2BC?]}\\
\end{supertabular}
\end{flushleft}
\begin{styleStandard}
The sense of motion is quite clear with \textit{pósta} since the meaning can be paraphrased roughly as [2BD?]place (text, picture, video etc.) on a website to make available to others[2BC?]. Matters are more complicated with \textit{installa} because this verb describes the process of getting a software program ready for use and that does not involve movement in any obvious sense. However, since programs are usually downloaded from the internet before they are installed, we think that native speakers see \textit{installa }as a process that includes downloading from the internet. This is supported by the fact that a directional PP like \textit{á tölvuna þína }‘to your computer’ can be added in (3a) to express the final location of the program. Hence, the object of \textit{installa} gets dative case just like the object of \textit{dánlóda}. 
\end{styleStandard}

\begin{styleStandard}
The verb \textit{gúffa} is obligatorily accompanied by the directional preposition \textit{í} [2BD?]in[2BC?] plus a simple reflexive bound by the subject. Thus, it seems that the verb itself encodes caused motion whereas the directional PP denotes where the food ends up. Examples like (3c) describe putting food quickly and/or greedily into the mouth but the food is not necessarily consumed. This is shown in (4) below, which is not a contradiction in our judgment: 
\end{styleStandard}

\begin{flushleft}
\tablefirsthead{}
\tablehead{}
\tabletail{}
\tablelasttail{}
\begin{supertabular}{m{0.24005985in}m{0.26155984in}m{0.53025985in}m{0.12335984in}m{0.26155984in}m{0.5441598in}m{0.18655986in}m{0.39345986in}m{0.5344598in}m{0.18655986in}m{0.12545985in}m{0.43025985in}m{-0.062040158in}}
(4) &
Hann &
gúffaði &
í &
sig &
kökum &
en &
skyrpti &
þeim &
út &
í &
\multicolumn{2}{m{0.44695982in}}{laumi}\\
 &
he &
shovelled &
in &
\textsc{refl} &
cakes.\textsc{dat} &
but &
spat &
them\textsc{.dat} &
out &
in &
\multicolumn{2}{m{0.44695982in}}{secret}\\
 &
\multicolumn{11}{m{4.36506in}}{[2BD?]He ate cookies like crazy but spat them out secretly.[2BC?]} &
\\
\end{supertabular}
\end{flushleft}
\begin{styleStandard}
This is not possible with ingestion verbs like \textit{éta} [2BD?]eat[2BC?]\textit{ }or \textit{borða} [2BD?]eat[2BC?], both of which take an accusative object. Unlike \textit{gúffa}, these verbs encode consumption of food but not movement into the mouth. Of course, a sentence like (3c) would generally be understood as saying that people eat a lot of chia seeds but this is through real world knowledge as it is not customary to put food into one‘s mouth without eating it. The contrast between \textit{gúffa} and \textit{éta} or \textit{borða} suggests that motion vs. consumption of food may be the critical factor determining object case with verbs of ingestion, but this will have to be an issue for future investigation.
\end{styleStandard}

\begin{styleStandard}
The verbs that still require some comment are \textit{flexa}, \textit{átsorsa},\textit{ slumma} and \textit{blasta}. The verb \textit{flexa} means to throw money around to show off so the sense of motion is quite clear. The same is true of \textit{átsorsa} which typically involves moving a task from one company to another. The verb \textit{blasta} denotes sound emission and emission of all kinds is a type of ballistic motion (Jónsson 2013a). Finally, \textit{slumma} is clearly a verb of ballistic motion so only dative is possible (see Jónsson 2013a for more examples and discussion of similar verbs).
\end{styleStandard}

\begin{stylelsSectionii}
3.3 Accusative objects
\end{stylelsSectionii}

\begin{styleStandard}
Some verbs in the study received a significantly higher score for accusative than dative. These verbs are listed in the following table: \ 
\end{styleStandard}

\begin{flushleft}
\tablefirsthead{}
\tablehead{}
\tabletail{}
\tablelasttail{}
\begin{supertabular}{|m{0.70805985in}m{1.6934599in}m{0.43305984in}m{0.47265986in}m{0.43375984in}|m{0.5900598in}|}
\hline
\multicolumn{1}{|m{0.70805985in}|}{{\bfseries Table 3:}} &
\multicolumn{5}{m{3.9379597in}|}{{\bfseries Verbs taking an accusative object }}\\
Verb &
Gloss &
DAT &
\textbf{ACC} &
Both &
Neither\\
fiffa &
[2BD?]fix (illegally)[2BC?] &
1,5 &
\textbf{94,5} &
0,0 &
4,0\\
gúgla &
[2BD?]google[2BC?] &
4,6 &
\textbf{93,6} &
0,5 &
1,3\\
krakka &
[2BD?]unlock, crack[2BC?] &
1,3 &
\textbf{86,2} &
2,3 &
10,2\\
gólfa &
[2BD?]press to the floor[2BC?] &
3,5 &
\textbf{74,9} &
0,7 &
20,9\\
skrína &
[2BD?]screen, keep an eye on[2BC?] &
1,3 &
\textbf{74,1} &
0,0 &
24,6\\
gramma &
[2BD?]put on Instagram[2BC?] &
8,5 &
\textbf{66,9} &
3,0 &
21,6\\
jáa &
[2BD?]search for on ja.is[2BC?] &
7,1 &
\textbf{58,8} &
0,3 &
33,8\\
offa &
[2BD?]turn off, shock[2BC?] &
9,4 &
\textbf{58,0} &
0,5 &
32,1\\\hline
domma &
[2BD?]dominate[2BC?] &
8,5 &
\textbf{52,5} &
0,0 &
39,0\\\hline
\end{supertabular}
\end{flushleft}
\begin{styleStandard}
For most of these verbs, it is intuitively clear that the direct object does not undergo motion in any sense. Consider e.g. the following test examples of the verbs \textit{krakka}, \textit{offa} and \textit{fiffa}:
\end{styleStandard}

\begin{flushleft}
\tablefirsthead{}
\tablehead{}
\tabletail{}
\tablelasttail{}
\begin{supertabular}{m{0.30875984in}m{0.23095986in}m{0.31365985in}m{0.31435984in}m{0.5490598in}m{0.41225985in}m{0.7462598in}m{0.41225985in}m{0.7476598in}}
(5) &
a. &
Geta &
þeir &
krakkað &
hvaða &
síma &
sem &
er?\\
 &
 &
can &
they &
hack &
any &
phone.\textsc{acc} &
which &
is\\
 &
 &
\multicolumn{7}{m{3.96796in}}{[2BD?]Can they hack any phone whatsoever?[2BC?]}\\
\end{supertabular}
\end{flushleft}
\begin{flushleft}
\tablefirsthead{}
\tablehead{}
\tabletail{}
\tablelasttail{}
\begin{supertabular}{m{0.31435984in}m{0.23655984in}m{0.31435984in}m{0.51155984in}m{0.6698598in}m{0.51155984in}m{1.6351599in}m{-0.04054016in}}
 &
b. &
Þetta &
attitude &
offaði &
mig &
\multicolumn{2}{m{1.6733599in}}{alveg}\\
 &
 &
this &
attitude &
turned.off &
me.\textsc{acc} &
\multicolumn{2}{m{1.6733599in}}{completely}\\
 &
 &
\multicolumn{5}{m{3.95746in}}{[2BD?]This attitude shocked me completely.[2BC?]} &
\\
\end{supertabular}
\end{flushleft}
\begin{flushleft}
\tablefirsthead{}
\tablehead{}
\tabletail{}
\tablelasttail{}
\begin{supertabular}{m{0.31365985in}m{0.23795986in}m{0.20185986in}m{0.31505984in}m{0.07335984in}m{0.9844598in}m{0.15735984in}m{0.35455984in}m{0.117759846in}m{0.19695985in}m{1.0872599in}}
 &
c. &
Þau &
lentu &
í &
peningavandræðum &
og &
byrjuðu &
að &
fiffa &
bókhaldið\\
 &
 &
they &
landed &
in &
money.trouble &
and &
started &
to &
fix &
the.book-keeping.\textsc{acc}\\
 &
 &
\multicolumn{9}{m{4.11856in}}{[2BD?]They got into financial difficulties and started to fiddle with the numbers.[2BC?]}\\
\end{supertabular}
\end{flushleft}
\begin{styleStandard}
The\textbf{ }verbs\textbf{ }\textit{gramma} and \textit{gólfa} stand out in Table 3 because they seem to express motion of the object. The test examples with these verbs are provided in (6):
\end{styleStandard}

\begin{flushleft}
\tablefirsthead{}
\tablehead{}
\tabletail{}
\tablelasttail{}
\begin{supertabular}{m{0.31505984in}m{0.23655984in}m{0.23655984in}m{0.5906598in}m{0.82685983in}m{0.22885984in}m{0.20665985in}m{0.13375986in}m{0.29975986in}m{0.117759846in}m{0.07885984in}m{0.5400598in}}
(6) &
a. &
Hann &
gólfaði &
bensíngjöfina &
þegar &
hann &
var &
kominn &
út  &
á &
hraðbrautina\\
 &
 &
he &
pushed.down &
the.foot.pedal.\textsc{acc} &
when &
he &
was &
come &
out &
to &
the.highway\\
 &
 &
\multicolumn{10}{m{3.96846in}}{[2BD?]He started to speed when he entered the highway.[2BC?]}\\
\end{supertabular}
\end{flushleft}
\begin{flushleft}
\tablefirsthead{}
\tablehead{}
\tabletail{}
\tablelasttail{}
\begin{supertabular}{m{0.31505984in}m{0.23585984in}m{0.13445985in}m{0.51155984in}m{0.35525984in}m{0.15665984in}m{0.5511598in}m{0.23655984in}m{1.6504599in}}
 &
b. &
Er &
einhver &
búinn &
að &
gramma &
nýja &
tíuþúsundkallinn?\\
 &
 &
is &
someone &
done &
to &
instagram &
new &
10.000.krónur.bill.\textsc{acc}\\
 &
 &
\multicolumn{7}{m{4.0685596in}}{[2BD?]Has someone put the new 10.000 krónur bill on Instagram?[2BC?]}\\
\end{supertabular}
\end{flushleft}
\begin{styleStandard}
These verbs are crucially different from the dative verb \textit{pósta} \ discussed in 3.2 in that they name the final location of the object. By contrast, \textit{pósta} does not specify the destination of the moved file and thus is compatible with a directional PP, as in (3b). The verb \textit{gólfa} is derived from the noun \textit{gólf} [2BD?]floor[2BC?] and the meaning is literally [2BD?]push to the floor[2BC?] and \textit{gramma} derives from the noun \textit{Instagram} and means [2BD?]put on Instagram[2BC?]. Hence, the final location of the object is encoded rather than movement to that location. Verbs of this kind are referred to as pocket verb by Levin (1993) and they all take an accusative object in Icelandic, e.g. \textit{axla} [2BD?]shoulder[2BC?],\textit{ bóka} [2BD?]book[2BC?], \textit{fangelsa} [2BD?]imprison[2BC?], \textit{hýsa} [2BD?]house[2BC?], \textit{jarða} [2BD?]bury[2BC?], \textit{ramma} [2BD?]frame[2BC?] and \textit{slíðra} [2BD?]sheathe[2BC?]. 
\end{styleStandard}

\begin{stylelsSectioni}
4. Case variation 
\end{stylelsSectioni}

\begin{styleStandard}
Some verbs in the present study displayed significant variation between accusative and dative. Under our hypothesis, case variation is expected whenever a verb is semantically ambiguous in a way that is linked to caused motion or the existence of a translational substitute taking a dative object. However, as we will see, this does not necessarily entail a difference in truth conditions.
\end{styleStandard}

\begin{styleStandard}
For convenience, the verbs examined here will be referred to as DAT/ACC-verbs. The discussion of these verbs is divided into two subsections below, montransitive verbs and ditransitive verbs, since they give rise to somewhat different issues.
\end{styleStandard}

\begin{stylelsSectionii}
4.1 Monotransitive verbs
\end{stylelsSectionii}

\begin{styleStandard}
The following table lists monotransitive DAT/ACC-verbs in the two surveys. As can be seen here, the dative outscored the accusative with six verbs but the reverse preference was found with four verbs: 
\end{styleStandard}

\begin{flushleft}
\tablefirsthead{}
\tablehead{}
\tabletail{}
\tablelasttail{}
\begin{supertabular}{|m{0.86635983in}m{1.4969599in}m{0.43305984in}m{0.47265986in}m{0.43305984in}|m{0.5906598in}|}
\hline
\multicolumn{1}{|m{0.86635983in}|}{{\bfseries Table 4:}} &
\multicolumn{5}{m{3.74136in}|}{{\bfseries Monotransitive verbs taking both dative and accusative object }}\\
Verb &
Gloss &
\textbf{DAT} &
ACC &
Both &
Neither\\
bleima &
[2BD?]blame[2BC?] &
\textbf{48,8} &
32,3 &
1,2 &
17,7\\
krassa &
[2BD?]cause to crash, ruin[2BC?] &
\textbf{48,6} &
27,0 &
2,3 &
22,1\\
neimdroppa &
[2BD?]namedrop[2BC?] &
\textbf{42,5} &
30,9 &
3,2 &
23,4\\
mæna &
‘collect, mine’ &
\textbf{41,5} &
17,9 &
2,5 &
38,1\\
syngja &
‘tell (a secret); sing’  &
\textbf{36,1} &
15,2 &
1,5 &
47,2\\
farta &
‘drive fast’ &
\textbf{34,1} &
13,0 &
0,5 &
52,4\\
rippa &
[2BD?]copy (illegally)[2BC?] &
19,3 &
\textbf{59,0} &
1,8 &
19,9\\
hannesa &
[2BD?]steal (a text)[2BC?] &
19,1 &
\textbf{51,9} &
3,3 &
25,7\\
peista &
[2BD?]paste[2BC?] &
44,8 &
\textbf{47,5} &
4,7 &
3,0\\\hline
bekka &
[2BD?]lift in a bench press[2BC?] &
32,1 &
\textbf{38,6} &
7,2 &
22,1\\\hline
\end{supertabular}
\end{flushleft}
\begin{styleStandard}
All the DAT/ACC-verbs listed here, except \textit{peista}, scored over 15\% for the last option (neither) and this reflects the low frequency of these verbs. Arguably, infrequent novel verbs have not been used enough to acquire an established meaning across speakers. As a result, they may have different intuitions about the meaning of these verbs, including the presence or absence of the factors that license a dative object. Admittedly, our data on the meaning of monotransitive DAT/ACC-verbs for different speakers is rather limited and our remarks below will inevitably be somewhat speculative. Still, we hope to show that these verbs are ambiguous in ways which affects object case, unlike the verbs discussed in section 3 and listed in Tables 1-3. 
\end{styleStandard}

\begin{styleStandard}
Under our analysis, the dative variant with DAT/ACC-verbs that do not express caused motion must be due to isolate attraction. Speakers that select a dative object with \textit{bleima}, \textit{krassa} and \textit{mæna} do so because they see the dative verbs \textit{kenna um} [2BD?]blame[2BC?], \textit{rústa} [2BD?]ruin[2BC?], and \textit{safna} [2BD?]collect[2BC?] as translational substitutes. As for \textit{rippa} and \textit{hannesa}, these verbs have a translational substitute in the dative verb \textit{stela} [2BD?]steal[2BC?] for some speakers. For other speakers, these two verbs denote copying without stealing, in which case \textit{stela} is not a translational substitute and consequently the object must be accusative. 
\end{styleStandard}

\begin{styleStandard}
The verbs \textit{neimdroppa}, \textit{peista} and \textit{bekka} are among the DAT/ACC-verbs for which the dative variant is licensed by caused motion. The test examples with these verbs are shown in (7):
\end{styleStandard}

\begin{flushleft}
\tablefirsthead{}
\tablehead{}
\tabletail{}
\tablelasttail{}
\begin{supertabular}{m{0.31295985in}m{0.23725986in}m{0.17265984in}m{0.28095984in}m{0.39345986in}m{0.101859845in}m{0.48935983in}m{0.40315986in}m{0.36845985in}m{0.19835985in}m{0.15875983in}m{0.17195985in}m{0.32335985in}m{0.21295986in}}
(7) &
a. &
Hún &
byrjaði &
strax &
að &
neimdroppa &
einhverjum &
böndum &
sem &
hún &
hafði &
djammað &
með\\
 &
 &
she &
started &
right.away &
to &
namedrop &
some.\textsc{dat} &
bands.\textsc{dat} &
which &
she &
had &
partied &
with\\
 &
 &
\multicolumn{12}{m{4.14146in}}{[2BD?]She started immediately to namedrop bands she had partied with.[2BC?]}\\
\end{supertabular}
\end{flushleft}
\begin{flushleft}
\tablefirsthead{}
\tablehead{}
\tabletail{}
\tablelasttail{}
\begin{supertabular}{m{0.31365985in}m{0.23655984in}m{0.17265984in}m{0.28095984in}m{0.39345986in}m{0.101859845in}m{0.48935983in}m{0.39345986in}m{0.36845985in}m{0.19835985in}m{0.15875983in}m{0.17195985in}m{0.32265985in}m{0.21365985in}}
 &
b. &
Hún &
byrjaði &
strax &
að &
neimdroppa &
einhver &
bönd &
sem &
hún &
hafði &
djammað &
með\\
 &
 &
she &
started &
right.away &
to &
namedrop &
some.\textsc{acc} &
bands.\textsc{acc} &
which &
she &
had &
partied &
with\\
\end{supertabular}
\end{flushleft}
\begin{flushleft}
\tablefirsthead{}
\tablehead{}
\tabletail{}
\tablelasttail{}
\begin{supertabular}{m{0.31435984in}m{0.23725986in}m{0.5906598in}m{0.31435984in}m{0.27615985in}m{0.23655984in}m{0.11845984in}m{0.21635985in}m{0.11845984in}m{0.23655984in}m{0.6691598in}m{0.15805984in}m{0.32815984in}m{0.12265984in}}
 &
c. &
Tölvan &
frýs &
alltaf &
þegar &
ég &
reyni &
að &
peista &
myndinni &
í &
Word &
\\
 &
 &
the.computer &
freezes &
always &
when &
I &
try &
to &
paste &
the.picture.\textsc{dat} &
into &
Word &
\\
 &
 &
\multicolumn{12}{m{4.2517595in}}{[2BD?]The computer always freezes when I try to paste the picture into a Word document.[2BC?]}\\
\end{supertabular}
\end{flushleft}
\begin{flushleft}
\tablefirsthead{}
\tablehead{}
\tabletail{}
\tablelasttail{}
\begin{supertabular}{m{0.31435984in}m{0.23655984in}m{0.5712598in}m{0.29695985in}m{0.27335986in}m{0.22405985in}m{0.09695984in}m{0.20455986in}m{0.09625984in}m{0.23095986in}m{0.70045984in}m{0.14625984in}m{0.28095984in}}
 &
d. &
Tölvan &
frýs &
alltaf &
þegar &
ég &
reyni &
að &
peista &
myndina &
í &
Word\\
 &
 &
the.computer &
freezes &
always &
when &
I &
try &
to &
paste &
the.picture.\textsc{ACC} &
into &
Word\\
\end{supertabular}
\end{flushleft}
\begin{flushleft}
\tablefirsthead{}
\tablehead{}
\tabletail{}
\tablelasttail{}
\begin{supertabular}{m{0.30185986in}m{0.23725986in}m{0.35525984in}m{0.35525984in}m{0.35455984in}m{0.51155984in}m{0.30665985in}m{1.6913599in}}
 &
e. &
Þessi &
gella &
getur &
bekkað &
150 &
kílóum/kíló\\
 &
 &
this &
chick &
can &
bench &
150 &
kilos.\textsc{dat/acc}\\
 &
 &
\multicolumn{6}{m{3.96836in}}{[2BD?]This chick can bench 150 kilos.[2BC?]}\\
\end{supertabular}
\end{flushleft}
\begin{styleStandard}
For some speakers, \textit{neimdroppa} is more or less synonymous with the accusative verbs \textit{nefna} [2BD?]mention[2BC?] and \textit{telja upp }[2BD?]recount, list[2BC?]. As expected, only accusative is possible in this sense. For other speakers, \textit{\ neimdroppa} means to mention something in a way that is similar to dropping, i.e. in a sneaky way as to show off by mentioning something or someone famous. This use is associated with a dative object. Thus, the variation between accusative and dative boils down to the presence or absence of caused motion in a metaphorical sense as part of the lexical semantics of \textit{neimdroppa}.
\end{styleStandard}

\begin{styleStandard}
The case variation with \textit{peista} does not correlate with any obvious truth conditional difference between the two variants. Still, it is clear that the object must be dative if \textit{peista} is interpreted as a verb of motion in the sense of moving a piece of text or a picture from one file to another or within the same file. Alternatively, if \textit{peista} encodes the resulting attachment rather than motion, only accusative is possible. In the latter case, \textit{peista} is very much like the accusative verb \textit{líma} [2BD?]glue[2BC?]. For discussion of other similar examples of case variation, see Jónsson (2013a).
\end{styleStandard}

\begin{styleStandard}
The verb \textit{bekka} takes a dative object if it encodes motion of the object, as reflected by the gloss [2BD?]lift in a bench press[2BC?]. In that sense, \textit{bekka} is similar to the dative verb \textit{lyfta} [2BB?]lift[2BC?]. Still, \textit{lyfta} is not a translational substitute in (7e) because replacing \textit{bekka} by \textit{lyfta} would yield a slightly different claim. The accusative variant may be due to the fact that \textit{bekka} in (7e) is not only about moving a weight in a specified direction but also exerting great physical force against gravity. The verb \textit{bekka} can also be used with objects that do not undergo movement, e.g. \textit{bekka heimsmet} (literally [2BB?]bench a world record[2BC?]), in which case only accusative is possible. 
\end{styleStandard}

\begin{styleStandard}
That leaves us with \textit{farta} and \textit{syngja}. These verbs had the highest score of all the DAT/ACC-verbs for the last option (neither), indicating that many native speakers were not familiar with these verbs in the relevant meaning. The verbs were tested in the following examples: 
\end{styleStandard}

\begin{flushleft}
\tablefirsthead{}
\tablehead{}
\tabletail{}
\tablelasttail{}
\begin{supertabular}{m{0.31505984in}m{0.23655984in}m{0.39345986in}m{0.19695985in}m{0.085159846in}m{0.43305984in}m{0.31565985in}m{0.09695984in}m{0.27615985in}m{0.39345986in}m{0.07885984in}m{0.23725986in}m{0.6691598in}}
(8) &
a. &
Þótt &
þetta &
sé &
hálfgerður  &
dótabíll &
er &
ekkert &
leiðinlegt &
að &
farta &
honum/hann\\
 &
 &
although &
this &
is &
halfmade &
toycar &
is &
not &
boring &
to &
drive &
him.\textsc{dat/acc}\\
 &
 &
\multicolumn{11}{m{3.9635596in}}{[2BD?]Although this is a kind of a toycar, it is fun to speed.[2BC?]}\\
\end{supertabular}
\end{flushleft}
\begin{flushleft}
\tablefirsthead{}
\tablehead{}
\tabletail{}
\tablelasttail{}
\begin{supertabular}{m{0.31505984in}m{0.23655984in}m{0.30735984in}m{0.23655984in}m{0.23725986in}m{0.27615985in}m{0.15735984in}m{0.35455984in}m{0.7483598in}m{0.14835984in}m{0.82335985in}m{-0.041940156in}}
 &
b. &
Hann &
var &
ekki &
lengi &
að &
syngja &
þessu/þetta &
að &
lögreglunni... &
\\
 &
 &
he &
was &
not &
long &
to &
sing &
this.\textsc{dat/acc} &
to &
the.police &
\\
 &
 &
\multicolumn{10}{m{3.95606in}}{[2BD?]I did not take him long to tell the police the whole story.[2BC?]}\\
\end{supertabular}
\end{flushleft}
\begin{styleStandard}
The dative variant with \textit{farta} encodes caused motion of a vehicle but the accusative is more difficult to explain. Perhaps it signals that the agent steps on the accelerator so that the car produces a sound similar to farting. This does not necessarily involve caused motion because this sound can be produced even if the car is not moving, e.g. if it is stuck in snow. 
\end{styleStandard}

\begin{styleStandard}
In its basic sense, \textit{syngja} [2BD?]sing[2BC?] is a performance verb which takes an accusative object like all other such verbs in Icelandic, e.g. \textit{blístra} [2BD?]whistle[2BC?], \textit{flytja} [2BD?]perform[2BC?], \textit{leika} [2BD?]play[2BC?], \textit{lesa} [2BD?]read[2BC?],\textit{ raula} [2BD?]hum[2BC?], \textit{spila} [2BD?]play[2BC?], \textit{tóna} [2BD?]chant[2BC?] and \textit{þylja} [2BD?]rattle[2BC?]. This basic meaning may have lead some speakers to chose accusative with \textit{syngja} in (8b). However, \textit{syngja} describes a manner of speaking in (8b) and all such verbs take a dative object in Icelandic if they express the exchange of information. These verbs include \textit{blaðra} [2BD?]babble[2BC?], \ \textit{gaspra} [2BD?]babble[2BC?],\textit{ hreyta} [2BD?]toss (words)[2BC?],\textit{ hvísla} [2BD?]whisper[2BC?], \textit{kjafta} [2BB?]tell (a secret)[2BC?], \textit{muldra} [2BD?]mumble[2BC?] and \textit{stynja upp} [2BD?]moan[2BC?]. Thus, it can be argued that \textit{syngja} in (8b) encodes motion of the message conveyed to the police. 
\end{styleStandard}

\begin{stylelsSectionii}
4.2 Ditransitive verbs
\end{stylelsSectionii}

\begin{styleStandard}
Three ditransitive verbs were tested in the present study and they all displayed considerable variation between accusative and dative with the direct object. The participants were not asked about the indirect object since dative is the only possibility there for new verbs. As shown in Table 5, the ditransitive verbs had virtually the same acceptance rate for both cases:
\end{styleStandard}

\begin{flushleft}
\tablefirsthead{}
\tablehead{}
\tabletail{}
\tablelasttail{}
\begin{supertabular}{|m{0.6691598in}m{1.5754598in}m{0.43305984in}m{0.41365984in}m{0.41295984in}|m{0.7872598in}|}
\hline
\multicolumn{1}{|m{0.6691598in}|}{{\bfseries Table 5:}} &
\multicolumn{5}{m{3.9373598in}|}{{\bfseries Ditransitive verbs taking both dative and accusative object}}\\
Verb &
Gloss &
DAT &
ACC &
Both &
Neither\\
græja &
[2BD?]procure; take care of[2BC?] &
\textbf{40,6} &
37,1 &
1,7 &
20,6\\
smessa &
[2BD?]send by sms[2BC?] &
\textbf{36,9} &
34,1 &
4,6 &
24,4\\\hline
meila &
[2BD?]e-mail[2BC?] &
36,3 &
\textbf{39,8} &
3,5 &
20,4\\\hline
\end{supertabular}
\end{flushleft}
\begin{styleStandard}
Verbs taking a dative indirect object and an accusative direct object (DAT-ACC verbs) constitute by far the biggest class of ditransitive verbs in Icelandic (see Zaenen, Maling \& Thráinsson 1985 and Jónsson 2000). This class also includes most of the canonical ditransitive verbs in Icelandic, e.g. \textit{gefa} [2BD?]give[2BC?], \textit{lána} [2BD?]lend[2BC?],\textit{ rétta} [2BD?]pass[2BC?],\textit{ segja} [2BD?]tell[2BC?], \textit{selja} ‘sell’, \textit{senda} [2BD?]send[2BC?] and \textit{sýna} [2BD?]show[2BC?]. The DAT-DAT class is much smaller and contains only a handful of typical ditransitive verbs, including \textit{lofa} [2BD?]promise[2BC?], \textit{skila} [2BD?]return[2BC?] and \textit{úthluta} [2BD?]allot[2BC?]. 
\end{styleStandard}

\begin{styleStandard}
In view of this, one would expect new ditransitive verbs to exhibit only DAT-ACC, unless the verb in question has a translational substitute with DAT-DAT. However, as discussed in more detail below, the DAT-DAT class relates to caused motion in a way that is similar to what we have already shown for monotransitive verbs. This class is also theoretically interesting in that the double dative strongly suggests two different sources for the two datives, e.g. an applicative head for the indirect object and some other functional head for the direct object. 
\end{styleStandard}

\begin{styleStandard}
We will start our discussion with \textit{græja} because it is more straightforward than the other two verbs. The relevant test examples are shown in (9) below:
\end{styleStandard}

\begin{flushleft}
\tablefirsthead{}
\tablehead{}
\tabletail{}
\tablelasttail{}
\begin{supertabular}{m{0.31505984in}m{0.23655984in}m{0.18445987in}m{0.38305986in}m{0.43305984in}m{0.20595986in}m{0.90525985in}m{0.12405985in}m{0.21635985in}m{0.21565986in}m{0.21635985in}m{0.36295986in}}
(9) &
a. &
Þú &
græjar &
þér &
bara &
útilegudrasli &
ef &
þú &
átt &
það &
ekki\\
 &
 &
you &
procure &
you.\textsc{dat} &
just &
camping.stuff.\textsc{dat} &
if &
you &
own &
it &
not\\
 &
 &
\multicolumn{10}{m{3.9558601in}}{[2BD?]You just get yourself camping stuff if you don‘t have it.[2BC?]}\\
\end{supertabular}
\end{flushleft}
\begin{flushleft}
\tablefirsthead{}
\tablehead{}
\tabletail{}
\tablelasttail{}
\begin{supertabular}{m{0.31505984in}m{0.23655984in}m{0.18445987in}m{0.38305986in}m{0.43305984in}m{0.20595986in}m{0.90525985in}m{0.12405985in}m{0.21635985in}m{0.21565986in}m{0.21635985in}m{0.36295986in}}
 &
b. &
Þú &
græjar &
þér &
bara &
útilegudrasl &
ef &
þú &
átt &
það &
ekki\\
 &
 &
you &
procure &
you.\textsc{dat} &
just &
camping.stuff.\textsc{acc} &
if &
you &
own &
it &
not\\
\end{supertabular}
\end{flushleft}
\begin{styleStandard}
For \textit{græja}, the double dative is due to the fact that this verb has, at least for some speakers, a translational substitute in the DAT-DAT verb \textit{redda} [2BB?]procure, take care of[2BC?]. In that sense, \textit{græja} indicates that something was obtained in a casual or hurried way. Speakers selecting DAT-ACC understand \textit{græja} presumably more like \textit{útvega} [2BB?]procure[2BC?], a DAT-ACC verb which has a more general meaning than \textit{redda} because it is completely neutral with respect to how the direct object is procured.
\end{styleStandard}

\begin{styleStandard}
The test examples for the verbs \textit{meila} and \textit{smessa} are given in (10):
\end{styleStandard}

\begin{flushleft}
\tablefirsthead{}
\tablehead{}
\tabletail{}
\tablelasttail{}
\begin{supertabular}{m{0.31365985in}m{0.23655984in}m{0.6691598in}m{0.43305984in}m{0.51155984in}m{0.90525985in}m{0.27615985in}m{0.7684598in}}
(10) &
a. &
Gætirðu &
meilað &
mér &
þessu/þetta &
sem &
fyrst?\\
 &
 &
could.you &
e-mail &
me.\textsc{dat} &
this.\textsc{dat/acc} &
as &
first\\
 &
 &
\multicolumn{6}{m{3.9573598in}}{[2BD?]Could you e-mail this to me as soon as possible?[2BC?]}\\
\end{supertabular}
\end{flushleft}
\begin{flushleft}
\tablefirsthead{}
\tablehead{}
\tabletail{}
\tablelasttail{}
\begin{supertabular}{m{0.31505984in}m{0.23655984in}m{0.45385984in}m{0.22125986in}m{0.21985984in}m{0.44415984in}m{0.45255986in}m{1.3427598in}m{0.34905985in}}
 &
b. &
Geturðu &
ekki &
bara &
smessað &
honum &
reikningsnúmerinu &
okkar?\\
 &
 &
can.you &
not &
just &
sms &
him.\textsc{dat} &
the.account.number.\textsc{dat} &
our\\
 &
 &
\multicolumn{7}{m{3.9559598in}}{[2BD?]Can‘t you just send him our account number by sms?[2BC?]}\\
\end{supertabular}
\end{flushleft}
\begin{flushleft}
\tablefirsthead{}
\tablehead{}
\tabletail{}
\tablelasttail{}
\begin{supertabular}{m{0.31505984in}m{0.23655984in}m{0.45385984in}m{0.22125986in}m{0.21915984in}m{0.44415984in}m{0.45385984in}m{1.3775599in}m{0.34905985in}}
 &
c. &
Geturðu &
ekki &
bara &
smessað &
honum &
reikningsnúmerið &
okkar?\\
 &
 &
can.you &
not &
just &
sms &
him.\textsc{dat} &
the.account.number.\textsc{acc} &
our\\
\end{supertabular}
\end{flushleft}
\begin{styleStandard}
The verbs \textit{meila} and \textit{smessa} are verbs of instrument of communication and have no translational substitutes taking a dative object. Rappaport Hovav \& Levin (2008) claim that verbs of instrument of communication in English encode caused motion and the same is true for Icelandic. Both \textit{meila} and \textit{smessa} entail that the direct object changes location in electronic space, although it need not reach its intended goal (see Beavers 2011 on \textit{e-mail}). These verbs also encode caused possession as the indirect object must be capable of possession and thus cannot be a location. This is a standard diagnostic to show that the double object construction in English encodes caused possession (see Green 1974 and much subsequent work). Thus, the examples in (11a-b) are ungrammatical unless \textit{Berlin} refers to the people working in an office in Berlin: 
\end{styleStandard}

\begin{flushleft}
\tablefirsthead{}
\tablehead{}
\tabletail{}
\tablelasttail{}
\begin{supertabular}{m{0.31435984in}m{0.23655984in}m{0.7483598in}m{0.44345984in}m{0.71985984in}m{0.8684598in}m{0.25185984in}m{0.5344598in}}
(11)  &
a. &
*Gætirðu &
meilað &
Berlín &
þessu/þetta &
sem &
fyrst?\\
 &
 &
\ \ could.you &
e-mail &
Berlin.\textsc{dat} &
this.\textsc{dat/acc} &
as &
first\\
 &
 &
\multicolumn{6}{m{3.9601598in}}{[2BD?]Could you e-mail Berlin this as soon as possible?[2BC?]}\\
\end{supertabular}
\end{flushleft}
\begin{flushleft}
\tablefirsthead{}
\tablehead{}
\tabletail{}
\tablelasttail{}
\begin{supertabular}{m{0.31505984in}m{0.23655984in}m{0.6295598in}m{0.27545986in}m{0.5518598in}m{0.7219598in}m{1.4622599in}}
 &
b. &
*Geturðu &
ekki &
smessað &
Berlín &
númerinu/númerið?\\
 &
 &
\ \ can.you &
not &
sms &
Berlin.\textsc{dat} &
the.number.\textsc{dat/acc}\\
 &
 &
\multicolumn{5}{m{3.95606in}}{[2BD?]Can‘t you send Berlin the number by sms?[2BC?]}\\
\end{supertabular}
\end{flushleft}
\begin{styleStandard}
This ambiguity means that native speakers are faced with two options when using \textit{meila} and \textit{smessa} as double object verbs, to treat them as DAT-DAT verbs encoding caused motion or DAT-ACC verbs encoding caused possession, apparently without any difference in truth conditions. 
\end{styleStandard}

\begin{styleStandard}
The intended goal of verbs of instrument of communication can be expressed not only as a dative DP but also as a PP headed by the preposition \textit{til} [2BB?]to[2BC?] (Barðdal 2008:128-132) but this does not effect the case variation with the direct object:
\end{styleStandard}

\begin{flushleft}
\tablefirsthead{}
\tablehead{}
\tabletail{}
\tablelasttail{}
\begin{supertabular}{m{0.31435984in}m{0.23725986in}m{0.68865985in}m{0.44275984in}m{0.90595984in}m{0.14345986in}m{1.5761598in}}
(12)  &
a. &
Gætirðu &
meilað &
þessu/þetta &
til &
mín?\\
 &
 &
could.you &
e-mail &
this.\textsc{dat/acc} &
to &
me.\textsc{gen}\\
 &
 &
\multicolumn{5}{m{4.07196in}}{[2BD?]Could you e-mail this to me?[2BC?]}\\
\end{supertabular}
\end{flushleft}
\begin{flushleft}
\tablefirsthead{}
\tablehead{}
\tabletail{}
\tablelasttail{}
\begin{supertabular}{m{0.31365985in}m{0.23655984in}m{0.5525598in}m{0.5511598in}m{1.4170599in}m{0.15805984in}m{1.0622599in}}
 &
b. &
Geturðu &
smessað &
númerinu/númerið &
til &
hennar?\\
 &
 &
can.you &
sms &
the.number.\textsc{dat/acc} &
to &
her.\textsc{gen}\\
 &
 &
\multicolumn{5}{m{4.05606in}}{[2BD?]Can you send her the number by sms?[2BC?]}\\
\end{supertabular}
\end{flushleft}
\begin{styleStandard}
This shows that \textit{meila} and \textit{smessa} encode caused motion in (12) because only such verbs allow the goal to be expressed in a PP headed by \textit{til} in Icelandic. However, caused possession is also encoded in examples like (12) because the goal must be capable of possession:
\end{styleStandard}

\begin{flushleft}
\tablefirsthead{}
\tablehead{}
\tabletail{}
\tablelasttail{}
\begin{supertabular}{m{0.31365985in}m{0.23795986in}m{0.7872598in}m{0.44345984in}m{0.9045598in}m{0.14345986in}m{1.4781599in}}
(13)  &
a. &
*Gætirðu &
meilað &
þessu/þetta &
til &
Berlínar?\\
 &
 &
\ \ could.you &
e-mail &
this.\textsc{dat/acc} &
to &
Berlin.\textsc{gen}\\
 &
 &
\multicolumn{5}{m{4.07186in}}{[2BD?]Could you e-mail this to Berlin?[2BC?]}\\
\end{supertabular}
\end{flushleft}
\begin{flushleft}
\tablefirsthead{}
\tablehead{}
\tabletail{}
\tablelasttail{}
\begin{supertabular}{m{0.31435984in}m{0.23655984in}m{0.6698598in}m{0.5511598in}m{1.4177599in}m{0.15805984in}m{0.9434598in}}
 &
b. &
*Geturðu &
smessað &
númerinu/númerið &
til &
Berlínar?\\
 &
 &
\ \ can.you &
sms &
the.number.\textsc{dat/acc} &
to &
Berlin.\textsc{gen}\\
 &
 &
\multicolumn{5}{m{4.0552597in}}{[2BD?]Can you send the number by sms to Berlin?[2BC?]}\\
\end{supertabular}
\end{flushleft}
\begin{styleStandard}
In view of the discussion above, one remaining issue is why the traditional motion verb \textit{senda} [2BB?]senda[2BC?] always takes an accusative direct object. While we cannot provide a definitive answer here, this may have to do with the fact that (a) this verb lacks a manner component and (b) it does not entail motion that starts with the agent of the action. For instance, a sentence like \textit{Jón sendi Maríu bók} ([2BB?]John sent Mary a book[2BC?]) may describe a situation where Jón orders a book from an internet company that delivers the book directly to Mary (see also Beavers 2011 on \textit{send} in English). Thus, the verb \textit{senda} appears to be more about causing something to reach some person or place in any conceivable way rather than motion per se. 
\end{styleStandard}

\begin{stylelsSectioni}
Conclusions
\end{stylelsSectioni}

\begin{styleStandard}
The results from the two large-scale surveys discussed in this paper show that a novel transitive verb in Icelandic takes a dative object if it (a) encodes some kind of caused motion of the object referent, or (b) has a translational substitute that takes a dative object. If neither (a) nor (b) holds, the object gets the default accusative case. 
\end{styleStandard}

\begin{styleStandard}
It is usually rather straightforward to determine if condition (b) holds and our discussion of such cases has indeed been rather brief. It is more difficult to argue that caused motion licenses a dative object. Crucially, the concept of caused motion has to be understood very broadly to include not only movement of concrete objects but also various abstract objects, including electronic files or messages. 
\end{styleStandard}

\begin{styleStandard}
Some of the novel verbs discussed here vary between dative and accusative object. This applies to some monotransitive verbs as well as the three ditransitive verbs tested. Under our analysis, this is expected if the relevant verb is semantically ambiguous such that the dative variant encodes caused motion or has a translational substitute taking a dative object. As argued in section 4, the predictions of our analysis are borne out although some questions remain concerning the meaning of some verbs for individual speakers. 
\end{styleStandard}

\begin{stylelsSectioni}
Acknowledgements
\end{stylelsSectioni}

\begin{styleStandard}
We wish to thank two anonymous reviewers for constructive feedback on an earlier version of this paper. The usual disclaimers apply. This study was financially supported by a grant from the Icelandic Research Fund (Rannís).
\end{styleStandard}

\begin{stylelsSectioni}
References
\end{stylelsSectioni}

\begin{styleStandard}
Barðdal, Jóhanna. 2001. \textit{Case in Icelandic – A synchronic, diachronic and comparative approach}. Lund: University of Lund dissertation. 
\end{styleStandard}

\begin{styleStandard}
Barðdal, Jóhanna. 2008. \textit{Productivity: Evidence from case and argument structure in Icelandic}. Amsterdam: John Benjamins Publishing. 
\end{styleStandard}

\begin{styleStandard}
Beavers, John. 2011. An aspectual analysis of ditransitive verbs of caused possession in English. \textit{Journal of Semantics }28. 1–54.
\end{styleStandard}

\begin{styleStandard}
Green, Georgia M. 1974. \textit{Semantics and syntactic regularity}. Bloomington, IN: Indiana University Press.
\end{styleStandard}

\begin{styleStandard}
Jóhannsdóttir, Kristín M. 1996. Á sögnum verður sjaldnast skortur: Afleiðslusagnir og innlimunarsagnir í íslensku. [Derived and incorporating verbs in Icelandic.] Reykjavík: University of Iceland MA-thesis. 
\end{styleStandard}

\begin{styleStandard}
Jónsson, Jóhannes Gísli. 2000. Case and double objects in Icelandic. \textit{Leeds Working Papers in Linguistics and Phonetics }8. 71-94.
\end{styleStandard}

\begin{styleStandard}
Jónsson, Jóhannes Gísli. 2009. Verb classes and dative objects in Insular Scandinavian. In Jóhanna Barðdal \& Shobhana Chelliah (eds.),\textit{ The role of semantic, pragmatic and discourse factors in the development of case}, 203-224. Amsterdam/Philadelphia: John Benjamins.
\end{styleStandard}

\begin{styleStandard}
Jónsson, Jóhannes Gísli. 2013a. Dative versus accusative and the nature of inherent case. In Beatriz Fernández \& Ricardo Etxepare (eds.), \textit{Variation in datives. A microcomparative perspective}, 144-160. Oxford: Oxford University Press. 
\end{styleStandard}

\begin{styleStandard}
Jónsson, Jóhannes Gísli. 2013b. Two types of case variation. \textit{Nordic Journal of Linguistics} 36(1). 5-25.
\end{styleStandard}

\begin{styleStandard}
Jónsson, Jóhannes Gísli \& Thórhallur Eythórsson. 2011. Structured exceptions and case selection in Insular Scandinavian. In Horst Simon \& Heike Wiese (eds.),\textit{ Expecting the unexpected:} \textit{Exceptions in the grammar}, 213-242. Berlin: Mouton de Gruyter.
\end{styleStandard}

\begin{styleStandard}
Levin, Beth. 1993. \textit{English verb classes and alternations}. \textit{A preliminary investigation}. Chicago: The University of Chicago Press.
\end{styleStandard}

\begin{styleStandard}
Maling, Joan. 2002. Það rignir þágufalli á Íslandi: Sagnir sem stjórna þágufalli á andlagi sínu. [Verbs that govern dative case in Icelandic.] \textit{Íslenskt mál og almenn málfræði} 24. 31-106. 
\end{styleStandard}

\begin{styleStandard}
Rappaport Hovav, Malka \& Beth Levin. 2008. The English Dative Alternation: The case for verb sensitivity. \textit{Journal of Linguistics }44. 129–167.
\end{styleStandard}

\begin{styleStandard}
Svenonius, Peter. 2002. Icelandic case and the structure of events. \textit{The Journal of Comparative Germanic Linguistics} 5. 197-225. 
\end{styleStandard}

\begin{styleStandard}
Svenonius, Peter. 2006. Case alternations and the Icelandic passive and middle. Unpublished manuscript, University of Tromsø.
\end{styleStandard}

\begin{styleStandard}
Wood, Jim. 2015. \textit{Icelandic morphosyntax and argument structure}. Dordrecht: Springer.
\end{styleStandard}

\begin{styleStandard}
Yip, Maling, and Ray Jackendoff. 1987. Case in Tiers. \textit{Language} 63. 217–50.
\end{styleStandard}

\begin{styleStandard}
Thórarinsdóttir, Rannveig B. 2015. Fallstjórn slangursagna [The case government of slang verbs.] Reykjavík: University of Iceland B.A.-thesis.
\end{styleStandard}

\begin{styleStandard}
Thráinsson, Höskuldur. 2007. \textit{The syntax of Icelandic}. Cambridge: Cambridge University Press.
\end{styleStandard}

\begin{styleStandard}
Zaenen, Annie, Joan Maling \& Höskuldur Thráinsson. 1985. Case and grammatical functions: the Icelandic passive. \textit{Natural Language \& Linguistic Theory }3. 441–483.
\end{styleStandard}

\end{document}

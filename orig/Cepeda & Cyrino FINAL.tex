\documentclass[[usenames,dvipsnames,output=paper
,modfonts
,nonflat]{langsci/langscibook} 
\usepackage{tikz}
\usepackage{tikz-qtree}
\definecolor{light-gray}{gray}{0.5}
\bibliography{localbibliography} 
% add all extra packages you need to load to this file  
\usepackage{tabularx} 
\usepackage{url} 
\urlstyle{same}

\usepackage{listings}
\lstset{basicstyle=\ttfamily,tabsize=2,breaklines=true}


%%%%%%%%%%%%%%%%%%%%%%%%%%%%%%%%%%%%%%%%%%%%%%%%%%%%
%%%                                              %%%
%%%           Examples                           %%%
%%%                                              %%%
%%%%%%%%%%%%%%%%%%%%%%%%%%%%%%%%%%%%%%%%%%%%%%%%%%%% 
%% to add additional information to the right of examples, uncomment the following line
% \usepackage{jambox}
%% if you want the source line of examples to be in italics, uncomment the following line
% \renewcommand{\exfont}{\itshape}
\usepackage{langsci-optional}
\usepackage{./langsci/styles/langsci-gb4e}
\usepackage{./langsci/styles/langsci-lgr}
\usepackage{pgfplots,pgfplotstable}

\definecolor{lsDOIGray}{cmyk}{0,0,0,0.45}

\usepackage{xassoccnt}
\newcounter{realpage}
\DeclareAssociatedCounters{page}{realpage}
\AtBeginDocument{%
  \stepcounter{realpage}
}


 



 

%% hyphenation points for line breaks
%% Normally, automatic hyphenation in LaTeX is very good
%% If a word is mis-hyphenated, add it to this file
%%
%% add information to TeX file before \begin{document} with:
%% %% hyphenation points for line breaks
%% Normally, automatic hyphenation in LaTeX is very good
%% If a word is mis-hyphenated, add it to this file
%%
%% add information to TeX file before \begin{document} with:
%% %% hyphenation points for line breaks
%% Normally, automatic hyphenation in LaTeX is very good
%% If a word is mis-hyphenated, add it to this file
%%
%% add information to TeX file before \begin{document} with:
%% \include{localhyphenation}
\hyphenation{
affri-ca-te
affri-ca-tes
Tarra-go-na
Vio-le-ta
Jacken-doff
clit-ics
Giar-di-ni
Mor-fo-sin-tas-si
mi-ni-mis-ta
nor-ma-li-tza-ció
Caus-ees
an-a-phor-ic
caus-a-tive
caus-a-tives
Mar-antz
ac-cu-sa-tive
Ma-no-les-sou
phe-nom-e-non
Holm-berg
}

\hyphenation{
affri-ca-te
affri-ca-tes
Tarra-go-na
Vio-le-ta
Jacken-doff
clit-ics
Giar-di-ni
Mor-fo-sin-tas-si
mi-ni-mis-ta
nor-ma-li-tza-ció
Caus-ees
an-a-phor-ic
caus-a-tive
caus-a-tives
Mar-antz
ac-cu-sa-tive
Ma-no-les-sou
phe-nom-e-non
Holm-berg
}

\hyphenation{
affri-ca-te
affri-ca-tes
Tarra-go-na
Vio-le-ta
Jacken-doff
clit-ics
Giar-di-ni
Mor-fo-sin-tas-si
mi-ni-mis-ta
nor-ma-li-tza-ció
Caus-ees
an-a-phor-ic
caus-a-tive
caus-a-tives
Mar-antz
ac-cu-sa-tive
Ma-no-les-sou
phe-nom-e-non
Holm-berg
}

\newcommand{\appref}[1]{Appendix \ref{#1}}
\newcommand{\fnref}[1]{Footnote \ref{#1}} 

\newenvironment{langscibars}{\begin{axis}[ybar,xtick=data, xticklabels from table={\mydata}{pos}, 
        width  = \textwidth,
	height = .3\textheight,
    	nodes near coords, 
	xtick=data,
	x tick label style={},  
	ymin=0,
	cycle list name=langscicolors
        ]}{\end{axis}}
        
\newcommand{\langscibar}[1]{\addplot+ table [x=i, y=#1] {\mydata};\addlegendentry{#1};}

\newcommand{\langscidata}[1]{\pgfplotstableread{#1}\mydata;}

\makeatletter
\let\thetitle\@title
\let\theauthor\@author 
\makeatother

\newcommand{\togglepaper}[1][0]{ 
%   \bibliography{../localbibliography}
  \papernote{\scriptsize\normalfont
    \theauthor.
    \thetitle. 
    To appear in: 
    Change Volume Editor \& in localcommands.tex 
    Change volume title in localcommands.tex
    Berlin: Language Science Press. [preliminary page numbering]
  }
  \pagenumbering{roman}
  \setcounter{chapter}{#1}
  \addtocounter{chapter}{-1}
}
\newcommand{\orcid}[1]{}
 

\title{Putting objects in order:\\
Asymmetrical relations in Spanish and Portuguese ditransitives
} 
\author{%
Paola Cépeda\affiliation{Stony Brook University / Pontificia Universidad Católica del Perú}\lastand 
Sonia Cyrino\affiliation{University of Campinas}
}
% \chapterDOI{} %will be filled in at production

% \epigram{}

\abstract{
Spanish, European Portuguese, and Brazilian Portuguese allow two possible linear orders for the direct (DO) and indirect object (IO) in ditransitives: DO>IO and IO>DO. The goal of this paper is twofold. First, we show that the arguments supporting a Double Object Construction (DOC) in these languages are inconclusive on both semantic and structural grounds. Accordingly, we claim that there is no DOC in these three languages. Second, we provide evidence that DO>IO and IO>DO are derivationally related. We show that DO>IO is the base order and that IO>DO is the result of an information structure operation, the latter order being possible only when IO conveys given information in the discourse and occupies the specifier of a low-periphery TopP. We offer a unified analysis that contributes to a comparative understanding of ditransitives in Romance.
}

\begin{document}

\maketitle
\section{Introduction} 
Spanish, European Portuguese (EP), and Brazilian Portuguese (BP) allow two possible linear orders for the direct (DO) and indirect object (IO) in ditransitive constructions: DO may precede or follow IO; that is, both DO>IO and IO>DO are possible. Examples are offered in (\ref{ex01}) for Spanish and (\ref{ex02}) for EP and BP (with the preposition \textit{a} and \textit{para}, respectively):

\ea \label{ex01}
	\ea[]{Spanish DO>IO\\
		\gll Olga (le) dio [$_{DO}$ una manzana] [$_{IO}$ a  Mario]. \\
			Olga \hspaceThis{(}\textsc{cl} gave \hspaceThis{[$_{DO}$} an apple \hspaceThis{[$_{IO}$} to Mario \\
		\glt ‘Olga gave an apple to Mario.' }\label{ex01a}
	\ex[]{Spanish IO>DO\\
		\gll Olga (le) dio [$_{IO}$ a  Mario] [$_{DO}$ una manzana]. \\
			Olga \hspaceThis{(}\textsc{cl} gave \hspaceThis{[$_{IO}$} to Mario \hspaceThis{[$_{DO}$} an   apple\\
		\glt ‘Olga gave an apple to Mario.’}\label{ex01b}
	\z
\z

\ea \label{ex02}
	\ea[]{EP/BP DO>IO\footnote{BP, unlike EP and Spanish, does not use the preposition \textit{a} in dative constructions. On the loss of the preposition \textit{a} and the syntax of \textit{para} in BP, see Calindro (this volume).}\\
		\gll A Olga  deu [$_{DO}$ uma maçã] [$_{IO}$ a/para  o  Mario].\\
			the Olga gave  \hspaceThis{[$_{DO}$} an   apple \hspaceThis{[$_{IO}$} to the Mario \\
		\glt ‘Olga gave an apple to Mario.’}\label{ex02a}
	\ex[]{EP/BP IO>DO\\
		\gll A Olga  deu [$_{IO}$ a/para  o  Mario] [$_{DO}$ uma maçã]. \\
			the Olga gave \hspaceThis{[$_{IO}$} to the Mario \hspaceThis{[$_{DO}$} an   apple \\
		\glt ‘Olga gave an apple to Mario.’}\label{ex02b}
	\z
\z

For these three languages, there is a debate in the literature on the availability of a Double Object Construction (DOC), similar to the configuration found in English. Larson (1988, 2014) argues that English ditransitive verbs such as \textit{give} allow both a Prepositional Phrase dative configuration (PP-dative), as in (\ref{ex03a}), and a DOC configuration, as in (\ref{ex03b}), and that these two configurations are derivationally related.\footnote{Other derivational accounts for the relationship between (\ref{ex03a}) and (\ref{ex03b}) in English have been presented in the literature. For an interesting review of arguments, see Rappaport-Levin \& Hovav (2008) and Hallman (2015). It is worth noting that the generalizations we arrive at in this paper hold independently of these theoretical positions, since we argue that there is no construction such as (\ref{ex03b}) in Spanish or Portuguese.} 

\ea \label{ex03}
	\ea[]{English PP-dative\\
		Olga gave [$_{DP}$ an apple] [$_{PP}$ to Mario].}\label{ex03a}
	\ex[]{English DOC\\
		Olga gave [$_{DP}$ Mario] [$_{DP}$ an apple].}\label{ex03b}
	\z
\z

Demonte (1995), Bleam (2003), Cuervo (2003, 2010), a.o., have claimed that, when the IO-doubling clitic appears in Spanish sentences such as those in (\ref{ex01}), the sentences resemble the English DOC. In contrast, the clitic-less ditransitive corresponds, in their view, to a PP-dative. It has also been claimed that the basic order in this kind of constructions is IO>DO. For Portuguese sentences such as those in (\ref{ex02}), Torres Morais \& Salles (2010) have claimed that the order IO>DO is equivalent to the English DOC.

In this paper, we investigate whether \textit{give}-type verbs in Spanish and Portuguese exhibit the kind of derivational relation they show in English. After analyzing the arguments that have been used to support the existence of DOC in these languages, we claim that there is no DOC in either Spanish, EP, or BP, and that the different linear orders for DO and IO are derivationally related. Our unified analysis aims to contribute to a better understanding of ditransitives in Romance, a topic that has been scarcely analyzed comparatively (except for Pineda 2016).

The paper is structured as follows. In section \ref{section2}, we analyze the arguments used to support a DOC approach for Spanish, EP, and BP, and propose that there is no conclusive evidence in favor of a DOC in these languages. In section \ref{section3}, we argue that the IO>DO order is strictly related to information structure. We offer our conclusions in the last section.

\section{The asymmetry of DO and IO in Spanish and Portuguese}\label{section2}
In this section, we examine the syntactic and semantic arguments supporting a DOC approach for Spanish, BP and EP (as defended by Demonte 1995, Cuervo 2003, Torres Morais \& Salles 2010, a.o.). We claim that these arguments are not conclusive, as DO and IO have asymmetrical properties regardless of their linear order.

\subsection{DO>IO and IO>DO are derivationally related.}\label{section2.1}
For English, Harley (1995) proposed decomposing verbal units into a \textsc{cause} and another abstract element, either \textsc{loc(ation)} or \textsc{have}. The order DO>IO corresponds to \textsc{cause} + \textsc{loc}, whereas IO>DO corresponds to  \textsc{cause} + \textsc{have}. Therefore, these two orders correlate with two independent structures. Examples in (\ref{ex04}) and (\ref{ex05}) are adapted from Harley (1995).

\ea \label{ex04}
	\ea[]{DO>IO (= \textsc{cause} + \textsc{loc})\\
		Olga gave an apple to Mario.}\label{ex04a}
	\protectedex{
	\ex[]{ }\label{ex04b}
		\begin{tikzpicture}[sibling distance=2pt, level distance=27pt]
		\Tree
			[.\textit{v}P
			[.Olga ]
			[.\textit{v}$'$ 
			[.\textit{v} \textsc{cause} ]
			[.PP
			[.{an apple} ]
			[.P$'$ 
			[.P \textsc{loc} ]
			[.Mario ] ] ] ] ]
		\end{tikzpicture}}
	\z
\z

\ea \label{ex05}
	\ea[]{IO>DO (= \textsc{cause} + \textsc{have})\\
		Olga gave Mario an apple.}\label{ex05a}
	\ex[]{ }\label{ex05b}
		\begin{tikzpicture}[sibling distance=2pt, level distance=27pt]
		\Tree
			[.\textit{v}P
			[.Olga ]
			[.\textit{v}$'$ 
			[.\textit{v} \textsc{cause} ]
			[.PP
			[.Mario ]
			[.P$'$ 
			[.P \textsc{have} ]
			[.{an apple} ] ] ] ] ]
		\end{tikzpicture}
	\z
\z

Harley’s independent structures have been applied to the analysis of Romance ditransitives (Bleam 2003, Costa 2009, Brito 2014, 2015). The central argument used has been based on the non-compositionality of idiomatic expressions. Let us consider Brito’s (2014, 2015) analysis as an example of this approach.

When discussing EP ditransitives, Brito (2014, 2015) concludes that there is no English-like DOC in EP and the DO>IO and IO>DO orders correspond to the different underlying structures in (\ref{ex06}).

\protectedex{
\ea \label{ex06}
	\ea[]{DO>IO}\label{ex06a}
		\begin{tikzpicture}[sibling distance=2pt, level distance=27pt]
		\Tree
			[.VP [.V$'$
			[.V ]
			[.VP 
			[.DO ]
			[.V$'$
			[.V ] 
			[.IO ] ] ] ] ]
		\end{tikzpicture}
	\ex[]{IO>DO}\label{ex06b}
		\begin{tikzpicture}[sibling distance=2pt, level distance=27pt]
		\Tree
			[.VP [.V$'$
			[.V ]
			[.VP 
			[.IO ]
			[.V$'$
			[.V ] 
			[.DO ] ] ] ] ]
		\end{tikzpicture}
	\z
\z}

Using idiomatic expressions to support her claim, Brito (2014) argues that certain idioms have a necessarily strict order since the idiomatic meaning is lost when the order is reversed. Thus, the idiomatic reading in (\ref{ex07a}), \textit{dar pérolas aos porcos} ‘give something valuable to someone who does not appreciate it’ usually appears as DO>IO (\ref{ex06a}), while the idiomatic reading in (\ref{ex08a}), \textit{dar a Deus o que o diabo não quis} ‘pass as a good person after a sinful life’ is related to IO>DO (\ref{ex06b}).

\ea \label{ex07}
	\ea[]{EP idiomatic DO>IO\\
		\gll A  Olga deu [$_{DO}$ pérolas] [$_{IO}$ aos porcos].\\
			the Olga gave  \hspaceThis{[$_{DO}$} pearls \hspaceThis{[$_{IO}$} to.the pigs \\
		\glt ‘Olga cast pearls before swine.’}\label{ex07a}
	\ex[]{EP non-idiomatic IO>DO\\
		\gll A  Olga deu [$_{IO}$ aos porcos] [$_{DO}$ pérolas]. \\
			the Olga gave \hspaceThis{[$_{IO}$} to.the pigs \hspaceThis{[$_{DO}$} pearls\\
		\glt ‘Olga gave pearls to the pigs.’}\label{ex07b}
	\z
\z

\protectedex{
\ea \label{ex08}
	\ea[]{EP idiomatic IO>DO\\
		\gll Dar [$_{IO}$ a  Deus] [$_{DO}$ o que o diabo não quis].\\
			give \hspaceThis{[$_{IO}$} to God \hspaceThis{[$_{DO}$} the what the devil  not  wanted\\
		\glt ‘to pass as virtuous despite an immoral past.’}\label{ex08a}
	\ex[]{EP non-idiomatic DO>IO\\
		\gll Dar [$_{DO}$ o que o diabo não quis] [$_{IO}$ a  Deus]. \\
			give \hspaceThis{[$_{DO}$} the what the devil  not  wanted \hspaceThis{[$_{IO}$} to God\\
		\glt ‘to give God what the Devil did not want.’}\label{ex08b}
	\z
\z}

In the three languages, some idioms seem to have the form V+DO, with IO in sentence-final position (as in (7a) for EP) and many times as an empty slot to be filled. For example, Spanish \textit{dar lata a alguien} ‘give someone a hard time’ and BP \textit{dar canja a alguém} ‘make things easy for someone’ have IO slots filled by \textit{a/para Olga}, respectively, in (9).

\ea \label{ex09}
	\ea[]{Spanish\\
		\gll Mario (le) está dando lata a  Olga.\\
			Mario \hspaceThis{(}\textsc{cl} is giving tin.can to Olga\\
		\glt ‘Mario is giving Olga a hard time.’}\label{ex09a}
	\ex[]{BP\\
		\gll O Mario está dando canja para a Olga.\\
			the Mario is    giving chicken.broth  to   the Olga\\
		\glt ‘Mario is making things easy for Olga.’}\label{ex09b}
	\z
\z

Sentences like (\ref{ex09}) have been used as an argument to claim that V+DO must form a constituent and, therefore, IO must be generated higher than DO (Bleam 2003). However, Larson (2014, 2017) argues convincingly that idiomatic expressions are not a conclusive argument for the existence of two independent structures, let alone for DOC.

First, the so-called \textit{idiomatic reading} is in fact compositional: the objects always receive specific meanings. Larson (2017) shows that speakers can interpret the alleged idiomatic reading in a phrase even in isolation. He finds support for this in the dictionary entries. For instance, the English sentence \textit{Olga gave Mario a kick} can be interpreted as ‘Olga gave Mario some feeling of excitement’. But this meaning is exactly what Larson finds in the dictionary entry for \textit{kick}:

\ea \label{ex10}
	\textbf{kick} n... \textbf{5} \textit{Slang} a feeling of pleasurable stimulation. (\textit{AHDEL})\\
		(Larson 2017:406)
\z

The same analysis can be applied to Spanish and Portuguese. The examples in (\ref{ex11}) suggest that the Spanish and Portuguese sentences in (\ref{ex09}) are really non-idiomatic since \textit{lata} and \textit{canja} can be interpreted as ‘bothersome situation’ (\ref{ex11a}) and ‘easy situation’ (\ref{ex11b}), respectively, even without the presence of the verb.

\ea \label{ex11}
	\ea[]{Spanish\\
		\gll ¡Esto es una lata!\\
			\hspaceThis{¡}this is an  annoyance\\
		\glt ‘This is annoying!’}\label{ex11a}
	\ex[]{BP\\
		\gll Isto é  uma canja!\\
			this is  an    ease\\
		\glt ‘This is easy!’}\label{ex11b}
	\z
\z

This shows that the so-called idiomatic expressions appear to be fully compositional. Therefore, in ditransitive structures, DO and the verb do not necessarily form a constituent that excludes IO. Even if we are persuaded that DO>IO and IO>DO are not derivationally related, idiomatic expressions cannot be used as a core argument for that claim. But are DO>IO and IO>DO really not related? In what follows, we argue that they are.

May (1977) shows that quantifier scope ambiguities offer relevant information about sentence structure. For instance, the sentences in (\ref{ex12}) and (\ref{ex13}) both contain two quantifiers: the universal \textit{every} (represented as $\forall$) and the existential \textit{a} (represented as $\exists$). For each sentence, we show the surface scope (the reading in which the scope of the quantifiers follows the superficial order of the constituents) and the inverse scope (the reading that results from inverting the linear order of the quantifiers):

\ea \label{ex12} Every ambassador visited a country.
	\ea[]{Surface scope: $\forall$ > $\exists$\\
		For every ambassador, there is a (potentially different) country that she/he visited.}
	\ex[]{Inverse scope: $\exists$ > $\forall$ \\
		There is one country that every ambassador visited.}
	\z
\z

\ea \label{ex13} An ambassador visited every country.
	\ea[]{Surface scope: $\exists$ > $\forall$\\
		There is one ambassador that visited every country.}
	\ex[]{Inverse scope: $\forall$ > $\exists$\\
		For every country, there is a (potentially different) ambassador that visited it.}
	\z
\z

We focus on linear $\exists$ > $\forall$ sentences like (\ref{ex13}) to test inverse scope (see Larson 2014). English is a fluid scope language since it typically allows quantified arguments in simple sentences to be read with varying scopes. However, in some constructions, scope seems \textit{frozen} in its surface order (i.e., the inverse scope is not possible). For instance, whereas (\ref{ex14a}) is scopally ambiguous, (\ref{ex14b}) is not because the scope has frozen.

\ea \label{ex14}
	\ea[]{English $\exists$ > $\forall$, $\forall$ > $\exists$\\
		The President assigned [a country] [to every ambassador].}\label{ex14a}
	\ex[]{English $\exists$ > $\forall$, *$\forall$ > $\exists$\\
		The President assigned [an ambassador] [every country].}\label{ex14b}
	\z
\z

We find the same asymmetries in Spanish and Portuguese ditransitives with \textit{give}-type verbs. When DO contains an existential quantifier (DO$_∃$), IO contains a universal quantifier (IO$_∀$), and the order is DO$_∃$>IO$_∀$, the sentence is scopally ambiguous: it has both a surface and an inverse scope reading. In contrast, when DO contains a universal quantifier (DO$_∀$), IO contains an existential quantifier (IO$_∃$), and the order is IO$_∃$>DO$_∀$, the scope in the sentence is frozen: no inverse scope reading is allowed. BP examples are provided in (\ref{ex15}).

\ea \label{ex15}
	\ea[]{BP DO$_∃$ IO$_∀$: $\exists$ > $\forall$, $\forall$ > $\exists$\\
		\gll A Olga deu [$_{DO}$ um presente] [$_{IO}$ para todos os  alunos]\\
			the Olga gave \hspaceThis{[$_{DO}$} a    gift \hspaceThis{[$_{IO}$} to    every the students\\
		\glt ‘Olga gave a gift to every student.’}\label{ex15a}
	\ex[]{BP IO$_∃$  DO$_∀$: $\exists$ > $\forall$, *$\forall$ > $\exists$\\
		\gll A Olga deu [$_{IO}$ para um aluno] [$_{DO}$ todos os presentes]\\
			the Olga gave \hspaceThis{[$_{IO}$} to a student \hspaceThis{[$_{DO}$} every the gifts\\
		\glt ‘Olga gave a student every gift.’}\label{ex15b}
	\z
\z

Sentence (\ref{ex15a}), DO$_∃$ IO$_∀$, has two possible readings. Its surface scope reading is that there is one gift that Olga gave to every student. Its inverse scope reading is that, for every student, there is a (potentially different) gift that Olga gave to them. In contrast, sentence (\ref{ex15b}), IO$_∃$  DO$_∀$, can only be interpreted with a surface scope reading: there is one student to whom Olga gave every present. The inverse scope is not possible, which means that it has frozen.

Antonyuk (2015, this volume) proposes a theory of scope freezing based on overt movement. Scope freezing occurs when a quantifier raises over another to a c-commanding position as a result of a single instance of movement. We use scope freezing as a diagnostic tool for observing the argument structure of ditransitives. Whereas sentences with no instances of object movement must be scopally ambiguous, sentences in which one object has moved over the other must be interpreted in scope freezing terms.

The interpretation of the sentences in (\ref{ex15}) suggests that they have different structures. Based on the possible scope ambiguity for DO>IO, we claim that there has been no object movement in (\ref{ex15a}). Conversely, in (\ref{ex15b}), based on the frozen scope of IO>DO, IO must have moved from a lower position to a higher one crossing over DO. The same scope asymmetry is also found in EP and Spanish. In the latter, the presence/absence of a dative clitic does not play any role in altering the scope relations between two co-occurring quantifiers. We return to the dative clitic’s role in section \ref{section2.2}.

This scope asymmetry strongly indicates that DO>IO and IO>DO must be related and that the base order is DO>IO, as proposed by Larson (1988, 2014). IO>DO must be derived by movement.\footnote{Comparable freezing facts and sensitivity to the different orders of DO and IO have been used to argue for an IO>DO base order in Germanic languages, DO>IO being the result of scrambling (see Abraham 1986, Choi 1996, Bacovcin 2017, a.o.). For reasons of space, we leave a discussion of this proposal for future work.}

\subsection{There is no DOC in Spanish or Portuguese.}\label{section2.2}
As already mentioned, scholars such as Demonte (1995), Bleam (2003), Cuervo (2003, 2010), a.o., claim that the presence of the dative clitic in Spanish indicates a DOC. In this section, we show that the presence of the clitic does not support a DOC analysis for Spanish, EP, and BP. Although we refer to examples by Demonte (1995), our discussion also applies to other scholars’ work, as they use Demonte (1995) as the base of their proposals. In addition, we show that the impossibility of passivization suggests against a DOC analysis for these three languages.

Demonte (1995) argues that only with the presence of the clitic can an anaphoric or possessive DO appear higher than an IO. To support her claim, she finds a contrast between (\ref{ex16a})/(\ref{ex17a}), without a clitic, and (\ref{ex16b})/(\ref{ex17b}), with a clitic, respectively (examples based on Demonte):

\protectedex{
\ea \label{ex16} Spanish
	\ea[*]{
		\gll El tratamiento devolvió [$_{DO}$ a sí  misma] [$_{IO}$ a Olga].\\
			the therapy      gave-back \hspaceThis{[$_{DO}$} to her self \hspaceThis{[$_{IO}$} to Olga\\
		\glt Intended: ‘The therapy helped Olga to be herself again.’}\label{ex16a}
	\ex[]{
		\gll El tratamiento \textbf{le} devolvió [$_{DO}$ la estima  de sí  misma] [$_{IO}$ a Olga].\\
			the therapy \textsc{cl} gave-back \hspaceThis{[$_{DO}$} the esteem of her self \hspaceThis{[$_{IO}$} to Olga\\
		\glt ‘The therapy gave Olga her self-esteem.’}\label{ex16b}
	\z
\z}

\ea \label{ex17} Spanish
	\ea[]{
		\gll La  profesora entregó [$_{DO}$ su$_i$ dibujo] [$_{IO}$ a  cada  niño$_i$].\\
			the teacher gave-back \hspaceThis{[$_{DO}$} his/her drawing \hspaceThis{[$_{IO}$} to each child\\
		\glt ‘The teacher gave each child their drawing.’\\
			(* for Demonte)}\label{ex17a}
	\ex[]{
		\gll  La  profesora \textbf{le} entregó [$_{DO}$ su$_i$ dibujo] [$_{IO}$ a  cada  niño$_i$].\\
			the teacher \textsc{cl}  gave-back \hspaceThis{[$_{DO}$} his/her drawing \hspaceThis{[$_{IO}$} to each child\\
		\glt ‘The teacher gave each child their drawing.’}\label{ex17b}
	\z
\z

However, the grammaticality differences offered by Demonte are not informative of the underlying structure of ditransitive constructions. First, the grammaticality difference of the sentences in (\ref{ex16}) is not an effect of the presence of the dative clitic, as the same difference arises when adding the clitic to (\ref{ex16a}) or removing it from (\ref{ex16b}) (also noted by Pineda 2013). Rather, the contrast arises from the different internal structure of the DO DPs: $[_{DO}$ \textit{a sí misma}] ‘herself’ in (\ref{ex16a}), and $[_{DO}$ \textit{la estima de sí misma}] ‘her self-esteem’ in (\ref{ex16b}). The grammaticality of (\ref{ex16b}) is due to the deeper structural position of the anaphor.

Second, against Demonte’s (1995) judgment, we consider (\ref{ex17a}) unquestionably grammatical (so does Pineda 2013). Thus, there is no real grammaticality differences between (\ref{ex17a}) and (\ref{ex17b}). The grammaticality effects remain the same regardless of the presence or absence of the dative clitic for both sentences. We conclude that the dative clitic in Spanish does not play any role in determining the structural position of DO or IO.

But does the presence of a clitic inform about a DOC? When analyzing English ditransitives, Oehrle (1976) claimed that DO>IO sentences such as (\ref{ex18a}) and IO>DO sentences such as (\ref{ex18b}) have a different interpretation in terms of \textit{possession entailment}. Oehrle says that the English DOC entails that there is a successful transfer or change of possession, either literally or symbolically. Therefore, by uttering (\ref{ex18a}), the speaker does not have any commitment to whether Mario actually learned Quechua. In contrast, only in (\ref{ex18b}) is there a possession entailment: Mario was transferred knowledge and, therefore, he did in fact learn Quechua.

\ea \label{ex18} English
	\ea[]{Olga taught Quechua to Mario.}\label{ex18a}
	\ex[]{Olga taught Mario Quechua.}\label{ex18b}
	\z
\z

Demonte (1995) assumes Oehrle’s analysis for English to be directly applicable for Spanish sentences depending on the absence or presence of a clitic. She differentiates between sentences with and without a clitic and argues that the presence of the clitic assures a possession entailment. To test this claim, we analyze the sentences in (\ref{ex19a}) and (\ref{ex19b}) (adapted from Demonte). We think that these sentences are ideal to test whether the presence of the clitic plays a role in conveying a transfer of possession, because they do not contain a \textit{give}-type verb in the main clause. If the transfer of possession is a property of the clitic, then the sentence containing a clitic must entail a transfer of possession. However, as we show, the presence of the clitic does not generate a possession entailment.

Sentence (\ref{ex19a}) contains no clitic in the main clause and includes a \textit{para}-phrase (‘for’). The fact that the main clause can be continued by \textit{que luego le dio a Mario} ‘which she later gave to Mario’ is interpreted by Demonte as a suggestion that there is no transfer of possession because there is no clitic supporting that interpretation. Sentence (\ref{ex19b}) contains the clitic \textit{le} in the main clause and an \textit{a}-phrase (‘to’). Demonte adds a double question mark to the continuation \textit{que luego le dio a Mario} under the assumption that the presence of the clitic conveys a clear transfer of possession. In other words, she assumes that in (\ref{ex19b}) the cake is now in the possession of Olga, so it cannot be further transferred to Mario.

\ea \label{ex19} Spanish
	\ea[]{
		\gll Hizo [una torta] [\textbf{para} Olga] (que luego le   dio   a  Mario).\\
			made \hspaceThis{[}a cake \hspaceThis{[}for Olga \hspaceThis{(}that  later \textsc{cl} gave to Mario\\
		\glt ‘She made a cake for Olga (which she later gave to Mario).’}\label{ex19a}
	\ex[]{
		\gll  \textbf{Le}  hizo [una torta] [\textbf{a}  Olga] (que luego le  dio    a  Mario).\\
			\textsc{cl} made \hspaceThis{[}a cake \hspaceThis{[}to Olga \hspaceThis{(}that  later \textsc{cl} gave to Mario\\
		\glt ‘She made a cake for Olga (which she later gave to Mario).’\\
			(?? for Demonte)}\label{ex19b}
	\z
\z

However, the semantics proposed by Demonte for these sentences is not accurate. In both (\ref{ex19a}) and (\ref{ex19b}), the transfer of possession is not an \textit{entailment}, but an \textit{implicature}. An implicature is an inference that may not hold in the context of other information and, thus, can be canceled. Entailments cannot be canceled. \textit{Hacerle una torta a Olga} ‘making a cake for Olga’ does not entail that Olga is in the possession of the cake, which suggests that the clitic is not playing any role in conveying transfer of possession. Rather, the continuation \textit{que luego le dio a Mario} in both (\ref{ex19a}) and (\ref{ex19b}) cancels the inference that Olga is in the possession of the cake, which makes this inference an implicature. Note that (\ref{ex19b}) is not judged ungrammatical by Demonte. Since the presence of the clitic does not generate a possession entailment, its presence or absence does not change the meaning of the sentence. The presence of the clitic does not support a DOC analysis.

A further argument against a DOC analysis is passivization. English DOCs are able to passivize the argument generated in the IO position. Larson (1988) explains that passivization and the PP-dative/DOC alternation are related processes, since passives advance an object to a subject position, while DOCs advance an indirect object to a direct object position. IO passivization is shown in (\ref{ex20}), where \textit{Mario} was generated as an IO, even though it appears occupying the subject position after spell-out.

\ea \label{ex20} English\\
	Mario was given an apple.
\z

However, IO passivization is simply not allowed in Spanish, EP, or BP. The examples in (\ref{ex21}) show the impossibility of the counterparts of (\ref{ex20}) in these three languages. Note that the presence of the dative clitic in Spanish does not improve the grammaticality of the sentence (\ref{ex21b}).

\ea \label{ex21}
	\ea[]{EP/BP\\
		\gll *O Mario  foi   dado uma maçã.\\
			\hspaceThis{*}the Mario was given an    apple\\
		\glt Intended: ‘Mario was given an apple.’}\label{ex21a}
	\ex[]{Spanish\\
		\gll  *Mario (le) fue dado  una manzana.\\
			\hspaceThis{*}Mario \hspaceThis{(}\textsc{cl} was given an apple\\
		\glt Intended: ‘Mario was given an apple.’}\label{ex21b}
	\z
\z

The absence of IO passivization in Spanish, EP, and BP has been largely overlooked as if it did not offer any insights for these languages. But, if IO passivization is not possible, then we need to assume that IO in IO>DO is not occupying any object position (Larson 2014), even though its linear order may suggest differently. We return to this issue in section \ref{section3}. For now, it is safe to say that, if IO is not occupying an object position when it precedes DO, then it is not accurate to claim that IO>DO is a DOC.

We conclude that the claim that there is DOC in Spanish, EP, and BP does not have support in the data, and there is no solid semantic or structural evidence for a DOC in these three languages. 

\section{The order of objects and information structure}\label{section3}
We have claimed that there is no DOC in Spanish, EP, or BP and that the base structure is DO>IO in these three languages. In this section, we propose that information structure shapes the IO>DO configuration in these languages.

\subsection{The distribution of DO>IO and IO>DO}\label{section3.1}
In Romance languages, given information (i.e., information assumed or already supplied in the context) appears early in the sentence and does not carry sentential stress, whereas new information (i.e., information introduced for the first time in an interchange) typically occurs sentence-finally and receives a special intonation (Zubizarreta 1988). When the whole sentence conveys new information, its linear order follows the default, unmarked structure.

The informationally unmarked order for ditransitives is DO>IO. This is the default order for answering a general question with no topic-comment structure, such as ‘what happened?’. Observe the BP example in (\ref{ex22}), which is an answer to the question \textit{O que aconteceu?} ‘What happened?’

\ea \label{ex22}
	\gll A   Olga deu [$_{DO}$ uma maçã] [$_{IO}$ para o   Mario].\\
			the Olga gave \hspaceThis{[$_{DO}$} an apple \hspaceThis{[$_{IO}$} to the Mario\\
	\glt ‘Olga gave an apple to Mario.’
\z

In (\ref{ex22}), the whole sentence conveys new information in the discourse. DO>IO is the only appropriate order to answer the question, which offers support to the claim that this is the base structure for ditransitives. The same generalization applies to both EP and Spanish.

Besides, following the general pattern for Romance, DO>IO is also the unique answer to the question ‘to whom?’, which asks for IO. Since IO encodes new information in the answer to such a question, it appears in final position. These effects are found in Spanish, EP, and BP. Therefore, the sentence in (\ref{ex22}) can also be the answer to the question \textit{A quem deu a Olga uma maçã?} ‘Who did Olga give an apple to?’. As we discuss in the next subsection, although the answers to ‘what happened?’ and ‘to whom’? are linearly identical, they certainly differ structurally.

So DO>IO is the default order when the whole sentence is the new information and when IO conveys new information. In contrast, the IO>DO order is more constrained. First, IO>DO appears when DO encodes new information, as the answer to the question ‘what?’ and, as is regular in Romance, occurs in final position. An example appears in (\ref{ex23}) in BP, which is the answer to the question \textit{O que a Olga deu para o Mario?} ‘What did Olga give to Mario?’

\ea \label{ex23}
	\gll A   Olga deu [$_{IO}$ para o   Mario] [$_{DO}$ uma maçã].\\
			the Olga gave \hspaceThis{[$_{IO}$} to the Mario  \hspaceThis{[$_{DO}$} an apple\\
	\glt ‘Olga gave an apple to Mario.’
\z

Second, IO>DO also appears when DO is heavy, that is, when it is either a long or complex constituent. Previous corpus and theoretical studies in Romance (Beavers \& Nishida 2010 for Spanish, Brito 2014 for EP, Mioto 2003 for BP) show that it is expected to find a heavy DO in final position.\footnote{IO>DO is also found in non-Romance languages with a heavy DO. In the following English examples (adapted from Larson 2014), (\ref{exib}) is not a DOC as IO contains the preposition ‘to’:\ea \label{exi} English IO>DO
		\ea[?/\#]{Olga gave [$_{DO}$ a reason not to accept the job] [$_{IO}$ to Mario].}\label{exib}
		\ex[]{Olga gave [$_{IO}$ to Mario] [$_{DO}$ a reason not to accept the job].}\label{exib}
	\z
\z} Examples in (24) show a contrast for EP.

\ea \label{ex24}
	\ea[?/\#]{
		\gll A Olga deu [$_{DO}$  três razões para não aceitar o trabalho] [$_{IO}$ ao Mario]\\
			the Olga gave \hspaceThis{[$_{DO}$} three reasons to not accept the job \hspaceThis{[$_{IO}$} to.the Mario\\
		\glt Intended: ‘Olga gave Mario three reasons not to accept the job.’}\label{ex24a}
	\ex[]{
		\gll A Olga deu [$_{IO}$ ao Mario] [$_{DO}$ três razões para não aceitar o trabalho]\\
			the Olga gave \hspaceThis{[$_{IO}$} to.the Mario \hspaceThis{[$_{DO}$} three reasons to not accept the job\\
		\glt ‘Olga gave Mario three reasons not to accept the job.’}\label{ex24b}
	\z
\z

For cases like (\ref{ex23}) and (\ref{ex24}), IO>DO is the most natural order. IO>DO is, therefore, the result of a discourse related configuration that affects the basic order of the arguments of ditransitives. From these facts, we conclude that the IO>DO order should be explained in terms of information structure.

\subsection{A low left periphery}\label{section3.2}
Belletti (2004) argues that the verb phrase is endowed with a fully-fledged periphery of discourse related structural positions, in parallel with the high left periphery. Her seminal work has been successfully developed in the recent literature (Mioto 2003, Quarezemin 2005, Jiménez-Fernández 2009, a.o.) and is relevant for us to explain the IO>DO order in Spanish, EP, and BP. We propose a low left periphery that minimally contains a Topic Phrase (TopP), a Focus Phrase (FocP), and the verbal domain (\textit{v}P), as shown in (\ref{ex25}). TopP and FocP are motivated by the discursive processes that change the order of sentence constituents.

\ea \label{ex25}
	[TopP [FocP [\textit{v}P ]]]
\z

We propose that IO>DO is possible when IO occupies TopP and DO occupies FocP. Consider again the BP IO>DO answer in (\ref{ex23}), repeated below as (\ref{ex26}).

\ea \label{ex26}
	\gll A   Olga deu [$_{IO}$ para o   Mario] [$_{DO}$ uma maçã].\\
			the Olga gave \hspaceThis{[$_{IO}$} to the Mario  \hspaceThis{[$_{DO}$} an apple\\
	\glt ‘Olga gave an apple to Mario.’
\z

As argued in section \ref{section2.1}, the DO \textit{uma maçã} ‘an apple’ in (\ref{ex26}) is generated higher than the IO \textit{para o Mario} ‘to Mario’. This base order is altered by two movement operations, which are motivated by information structure properties in the low left periphery. First, since DO encodes new information by offering the exact answer to the question ‘what?’, it moves from its verb-internal position to the low FocP. This movement is not surprising as answers to questions are associated with focus (Rooth 1992). Second, since IO offers given information, it moves from its initial position to the low TopP, crossing over both DO’s base position and its landing site. Additionally, the complex V+\textit{v} has moved to Tense, as is the general case in Romance. A structure for (\ref{ex26}) is shown in (\ref{ex27}). The arrows mark the movements towards the low left periphery.

\protectedex{
\ea \label{ex27}
	\begin{tikzpicture}[sibling distance=2pt, level distance=27pt, scale=.65]
	\Tree
	[.TP
	[.{\textbf{A Olga}} ]
	[.T$'$
	  [.T
	      [.T ]
	      [.{\textit v}
		[.{\textit v} ]
		[.{\textbf{deu}} ] ] ]
	  [.TopP
	      [.\node(a){\textbf{para o Mario}}; ]
	      [.Top$'$
		[.Top ]
		[.FocP
		    [.\node(x){\textbf {uma ma\c{c}\~a}}; ]
		    [.Foc$'$
			[.Foc ]
			[.{\textit v}P
				[.\textcolor{light-gray}{A Olga} ]
				[.{\textit v}$'$
					[.\textcolor{light-gray}{\textit v}
						[.\textcolor{light-gray}{\textit v} ]
						[.\textcolor{light-gray}{deu} ] ]
					[.VP
						[.\node(y){\textcolor{light-gray}{uma ma\c{c}\~a}}; ]
						[.V$'$
							[.\textcolor{light-gray}{deu} ]
							[.\node(b){\textcolor{light-gray}{PP}};
								[.\textcolor{light-gray}{para} ]
								[.\textcolor{light-gray}{o Mario} ] ] ] ] ] ] ] ] ] ] ] ]
	\draw[semithick, <-] (x) to [bend right=60] (y.south);
	\draw[semithick, <-] (a) to [bend right=60] (b.west);
	\end{tikzpicture}
\z}

As for the DO>IO order in sentences such as (22), repeated below as (\ref{ex28}), the syntactic structure depends on the kind of discourse-related information it conveys. When the whole sentence is the new information, the low left periphery does not host any constituent and we could safely say that both DO and IO remain in situ, as in (\ref{ex28a}). In contrast, when only IO conveys the new information, both DO and IO move to the specifier of TopP and FocP in the low left periphery, respectively, as shown in (\ref{ex28b}).

\ea \label{ex28}
	\gll A   Olga deu $[_{DO}$ uma maçã] $[_{IO}$ para o   Mario].\\
			the Olga gave \hspaceThis{$[_{DO}$} an apple \hspaceThis{$[_{IO}$} to the Mario\\
	\glt ‘Olga gave an apple to Mario.’
		\ea[]{[$_{TP}$ A Olga  T+\textit{v}+deu [$_{\textit{v}P}$ \textcolor{light-gray}{<A Olga> <\textit{v}+deu>} [$_{VP}$ [uma maçã] \textcolor{light-gray}{<deu>} [para o Mario] ] ] ]}\label{ex28a}
		\ex[]{[$_{TP}$ A Olga T+\textit{v}+deu [$_{TopP}$ [uma maçã]  Top [$_{FocP}$ [para o Mario]  Foc  [$_{\textit{v}P}$ \textcolor{light-gray}{<A Olga> <\textit{v}+deu>} [$_{VP}$ \textcolor{light-gray}{<uma maçã> <deu> <para o Mario>} ] ] ] ] ]}\label{ex28b}
	\z
\z

The analyses we have proposed for IO>DO and DO>IO apply equally to BP, EP, and Spanish. Our proposal can account for the fact that it is possible to find an IO>DO order in Spanish and Portuguese, which is derivationally related to the basic order DO>IO, without assuming a DOC construction for these languages.\footnote{For a formal proposal on the role of information structure features, see Cépeda \& Cyrino (2017).}

\section{Conclusions}\label{section4}
In this paper, we have dismissed the arguments supporting a DOC approach for Spanish and Portuguese while showing that there are no DOCs in these three languages. We have proposed that the internal argument structure of ditransitives is based on a DO>IO order. The IO>DO order is a derived configuration, which we have explained in terms of movement to a low left periphery with discourse-related positions available. Our comparative approach unifies the analysis of ditransitives in Spanish, EP and BP.

%\section*{Abbreviations}

\section*{Acknowledgements}
Both authors thank Richard Larson, the participants at the Syntax Seminar at Stony Brook University (Fall 2015), the audiences at the International Workshop on Dative Structures in Barcelona (Spring 2017), at the 47th Linguistic Symposium on Romance Languages at the University of Delaware (Spring 2017), and at the Department of Linguistics at Stony Brook University (Spring 2017), as well as two anonymous reviewers. Both authors also express their gratitude to Russell Tanenbaum, who proofread the document. The first author thanks the support of the American Council of Learned Societies (Mellon/ACLS Dissertation Completion Fellowship 2017-2018). The second author thanks the support of CNPq–Conselho Nacional de Desenvolvimento Científico e Tecnológico, Brazil, (Grant 303742/2013-5), FAPESP- Fundação de Amparo à Pesquisa do Estado de São Paulo, Brazil, (Grants 2012/06078-9 and 2014/17477-7), and Ministerio de Economía y Competitividad, España (FFI2014/52015-P).

%\section*{References}
%\begin{list}{}%
%{\leftmargin=2em \itemindent=-2em}\small{
%\item Abraham, Werner. 1986. Word order in the middle field of the German sentence. In Werner Abraham \& Sjaak de Meij (eds.), \textit{Topic, focus and configurationality: Papers from the 6th Groningen Grammar Talks}, 15-38. Amsterdam: John Benjamins.
%\item Antonyuk, Svitlana. 2015. \textit{Quantifier scope and scope freezing in Russian}. Ph.D. dissertation, Stony Brook University.
%\item Bacovcin, Hezekiah Akiva. 2017. \textit{Parameterising Germanic ditransitive variation: A historical-comparative study}. Ph.D. dissertation, University of Pennsylvania.
%\item Beavers, John \& Chiyo Nishida. 2010. The Spanish dative alternation revisited. In Sonia Colina, Antxon Olarrea \& Ana Maria Carvalho (eds.), \textit{Romance linguistics 2009: Selected papers from the 39th Linguistic Symposium on Romance Languages (LSRL)}, 217-30. Amsterdam: John Benjamins.
%\item Belletti, Adriana. 2004. Aspects of the low IP area. In Luigi Rizzi (ed.), \textit{The structure of CP and IP. The cartography of syntactic structures}, 16-51. Oxford: OUP.
%\item Bleam, Tonia. 2003. Properties of double object construction in Spanish. In Rafael Nunez-Cedeno, Luis Lopez \& Richard Cameron (eds.), \textit{A Romance perspective on language knowledge and use: Selected papers from the 31st Linguistic Symposium on Romance Languages (LRSL)}, 233-252. Amsterdam: John Benjamins.
%\item Brito, Ana Maria. 2015. Two base generated structures for ditransitives in European Portuguese. \textit{Oslo Studies in Language} 7(1): 337-357.
%\item Brito, Ana Maria. 2014. As construções ditransitivas revisitadas. Alternância dativa em Português Europeu? \textit{Textos selecionados do XXIX Encontro Nacional da Associação Portuguesa de Linguística}: 103-119.
%\item Cépeda, Paola \& Sonia Cyrino. 2017. \textit{A point of order. Object asymmetries in Spanish and Portuguese ditransitives}. Manuscript. Stony Brook University.
%\item Choi, Hye-Won. 1996. \textit{Optimizing structure in context: Scrambling and information structure}. Ph.D. dissertation, Rutgers University.
%\item Costa, João. 2009. A focus-binding conspiracy. Left-to-right merge, scrambling and binary structure in European Portuguese. In Jeroen van Craenenbroeck (ed.), \textit{Alternatives to cartography}, 87-108. Berlin: De Gruyter Mouton.
%\item Cuervo, María Cristina. 2010. Two types of (apparently) ditransitive light verb constructions. In Karlos Arregi, Zsuzsanna Fagyal, Silvina A. Montrul \& Annie Tremblay (eds.), \textit{Romance linguistics 2008: Interactions in Romance. Selected papers from the 38th Linguistic Symposium on Romance Languages (LRSL)}, 139-155. Amsterdam: John Benjamins.
%\item Cuervo, María Cristina. 2003. \textit{Datives at Large}. Ph.D. dissertation, MIT. 
%\item Demonte, Violeta. 1995. Dative alternation in Spanish. \textit{Probus} 7: 5-30.
%\item Hallman, Peter. 2015. Syntactic neutralization in double object constructions. \textit{Linguistic Inquiry} 46(3): 389-424.
%\item Harley, Heidi. 1995. If you have, you can give. \textit{Proceedings of the 15th West Coast Conference on Formal Linguistics}, 193-207. Stanford: CSLI.
%\item Jiménez-Fernández, Ángel. 2009. The low periphery of Double Object Constructions in English and Spanish. \textit{Philologia Hispalensis} 23: 179-200.
%\item Larson, Richard. 2017. On “dative idioms” in English. \textit{Linguistic Inquiry} 48(3): 389-426.
%\item Larson, Richard. 2014. \textit{On shell structure}. New York: Routledge.
%\item Larson, Richard. 1988. On the double object construction. \textit{Linguistic Inquiry} 19: 335-392.
%\item May, Robert. 1977. \textit{The grammar of quantification}. Ph.D. dissertation, MIT. 
%\item Mioto, Carlos. 2003. Focalização e quantificação. \textit{Revista Letras} 61: 169-189.
%\item Oerhle, Richard. 1976. \textit{The grammatical status of the English dative alternation}. Ph.D. dissertation, MIT. 
%\item Pineda, Anna. 2016. \textit{Les fronteres de la (in)transitivitat. Estudi dels aplicatius en llengües romàniques i basc}. Barcelona: Institut d’Estudis Món Juïc.
%\item Pineda, Anna. 2013. Double object constructions and dative/accusative alternations in Spanish and Catalan: A unified account. \textit{Borealis–An International Journal of Hispanic Linguistics} 2: 57-115.
%\item Quarezemin, Sandra. 2005. \textit{A focalização do sujeito no PB}. M.A. thesis, Universidade Federal de Santa Catarina. 
%\item Rappaport-Hovav, Malka \& Beth Levin. 2008. The English dative alternation: The case for verb sensitivity. \textit{Journal of Linguistics} 44: 126-167.
%\item Rooth, Mats. 1992. A theory of focus interpretation. \textit{Natural Language Semantics} 1: 75-116.
%\item Torres Morais, Maria Aparecida \& Helena M.L. Salles. 2010. Parametric change in the grammatical encoding of indirect objects in Brazilian Portuguese. \textit{Probus} 22: 181-209.
%\item Zubizarreta, Maria Luisa. 1998. \textit{Prosody, focus, and word order}. Cambridge, MA: MIT Press.}
%\end{list}

{\sloppy
\printbibliography[heading=subbibliography,notkeyword=this]
}
\end{document}